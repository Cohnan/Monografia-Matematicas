\documentclass[12pt]{report}

\usepackage[utf8]{inputenc}
\usepackage[marginparwidth = 2cm]{geometry}

\usepackage{amsmath, amsthm, amssymb}

% My Codes
\usepackage[pdftex,dvipsnames]{xcolor}
%\usepackage[dvipsnames]{xcolor}

%
% \newcommand{\ytext}[1]{\textcolor{yellow}{#1}}
% \newcommand{\otext}[1]{\textcolor{orange}{#1}}
% \newcommand{\rtext}[1]{\textcolor{red}{#1}}
% \newcommand{\lbtext}[1]{\textcolor{cyan}{#1}}
% \newcommand{\dbtext}[1]{\textcolor{blue}{#1}}
% \newcommand{\ptext}[1]{\textcolor{Plum}{#1}}
% \newcommand{\lgtext}[1]{\textcolor{LimeGreen}{#1}}
% \newcommand{\dgtext}[1]{\textcolor{OliveGreen}{#1}}

\newcommand{\ytext}[1]{\textcolor{black}{#1}}
\newcommand{\otext}[1]{\textcolor{black}{#1}}
\newcommand{\rtext}[1]{{\it #1}}
\newcommand{\lbtext}[1]{\textcolor{black}{#1}}
\newcommand{\dbtext}[1]{\textcolor{black}{#1}}
\newcommand{\ptext}[1]{\textcolor{black}{#1}}
\newcommand{\lgtext}[1]{\textcolor{black}{#1}}
\newcommand{\dgtext}[1]{\textcolor{black}{#1}}

% \newcommand{\ybox}[1]{\colorbox{yellow}{#1}}
% \newcommand{\obox}[1]{\colorbox{orange}{#1}}
% \newcommand{\rbox}[1]{\colorbox{Salmon}{#1}}
% \newcommand{\lbbox}[1]{\colorbox{SkyBlue}{#1}}
% \newcommand{\dbbox}[1]{\colorbox{NavyBlue}{#1}}
% \newcommand{\pbox}[1]{\colorbox{Plum}{#1}}
% \newcommand{\lgbox}[1]{\colorbox{LimeGreen}{#1}}
% \newcommand{\dgbox}[1]{\colorbox{OliveGreen}{#1}}

\usepackage{xargs}                      % Use more than one optional parameter in a new
%\usepackage{xargs}                      % Use more than one optional parameter in a new
%\usepackage[pdftex,dvipsnames]{xcolor}  % Coloured text etc.

%
%\usepackage[colorinlistoftodos,prependcaption,textsize=tiny]{todonotes}
\usepackage{todonotes}

\newcommandx{\unsure}[2][1=]{\todo[linecolor=red,backgroundcolor=red!25,bordercolor=red,#1]{#2}}
\newcommandx{\change}[2][1=]{\todo[linecolor=blue,backgroundcolor=blue!25,bordercolor=blue,#1]{#2}}
\newcommandx{\complete}[2][1=]{\todo[linecolor=pink,backgroundcolor=pink!25,bordercolor=blue,#1]{#2}}
\newcommandx{\info}[2][1=]{\todo[linecolor=OliveGreen,backgroundcolor=OliveGreen!25,bordercolor=OliveGreen,#1]{#2}}
\newcommandx{\improvement}[2][1=]{\todo[linecolor=Plum,backgroundcolor=Plum!25,bordercolor=Plum,#1]{#2}}
\newcommandx{\thiswillnotshow}[2][1=]{\todo[disable,#1]{#2}}
%

\title{Resumen Marathe}
\date{February 2020}
\author{Sebastian Camilo Puerto}

\begin{document}
\maketitle

%%%%%%%%%%%%%%%%%%%%%%%%%%%%%%%%%%%%%%%%%%%%%%%%%%%%%%%%%%%%%%%%%%%%%%%%%%%%%%
%%%%%%%%%%%%%%%%%%%%%%%%%%%%%%%%%%%%%%%%%%%%%%%%%%%%%%%%%%%%%%%%%%%%%%%%%%%%%%
%%%%%%%%%%%%%%%%%%%%%%%%%%%%%%%%%%%%%%%%%%%%%%%%%%%%%%%%%%%%%%%%%%%%%%%%%%%%%%
\chapter*{6. Theory of Fields, I: Classical}
%%%%%%%%%%%%%%%%%%%%%%%%%%%%%%%%%%%%%%%%%%%%%%%%%%%%%%%%%%%%%%%%%%%%%%%%%%%%%%
\section{Introduction}

%%%%%%%%%%%%%%%%%%%%%%%%%%%%%%%%%%%%%%%%%%%%%%%%%%%%%%%%%%%%%%%%%%%%%%%%%%%%%%
\section{Physical Background}

\subsection{Important Facts}
\begin{itemize}
    \item The principal theoretical tools for studying elementary particles and their interactios are probided by (quantum) gauge theories. Apparently Ch 7 discusses problem of quantization
    
    \item \rtext{Maxwell's EM theory} is the simples example of a gauge theory: $U(1)$ gauge theory over Minkowski, trivial principal bundle. No need to talk about matter fields yet.
    \begin{itemize}
        \item The possible vector potentials are obtained from one another by a gauge transformation by a $\psi \in C^\infty(M)$: \[A \mapsto A^\psi = A + d\psi\]: the infinitesimal change $d\psi$ corresponds to a local change of scale: this is the origin of the term \emph{gauge invariance} introduced by Weyl.
        \begin{itemize}
            \item Weyl sought to incorportate the EM field into the geometric structures associated to the space-time manifold (so not just introducing $F$ by hand to a spacetime manifold) AS arising from local scale invariance.
        \end{itemize}
        
        \item However, the correct\improvement{what is incorrect about it} way to interpret this invariance is as a gauge transformation by a $g_\psi = e^{i \psi} \in  C^\infty(M, U(1))$: \[iA \mapsto iA^g = g^{-1} (iA) g + g^{-1} dg.\]
        \begin{itemize}
            \item This doesn't achieve the greatness desired by Weyl of giving rise naturally to a unification of EM and gravitation, but does allow to see \emph{Classical EM theory as a Gauge theory}.
        \end{itemize}
    \end{itemize}
    
    \item \rtext{Yang Mills equations: differential equations that have to be satisfied BY THE GAUGE POTENTIAl, perhaps when it is in interaction with a matter/source field: they are part or the Euler-Lagrange equations for some action which depends on both the matter field AND the connection}: differential equation for the \rbox{correct connection}; other part of the Euler-Lagrange equations give the equations of motion of the particle field, e.g. Dirac equation, Schrodinger, Klein-Gordon, etc.
    \begin{itemize}
        \item The original Yang-Mills equations: group $SU(2)$, equations for the vector potential $b_\mu$ of \emph{isotopic spin} in \emph{interaction with} a field $\psi$ of isotopic spin $1/2$.
        
        \item \dbbox{They were not seen useful immediately as they predicted massless gauge partiles}.
        
        \item \rtext{The Yang-Mills equations may be thought of as a matrix-valued generalization of the equations for the classical vector potential of Maxwell theory}. (In that case we get the wave equation/Klein gordon equatio form $m=0$ if source free, or $\square A = \mu_0 J$ if coupled with matter field)
    \end{itemize}
    
    \item There is not yet a generally acceptedmathematical theory of quantiation of gauge fields.
\end{itemize}

%%%%%%%%%%%%%%%%%%%%%%%%%%%%%%%%%%%%%%%%
\subsection{Medium Important Details}
\begin{itemize}
    \item Another approach toa  unified treatment of EM and gravitatioanl fields leads to the Kaluza-Klein theory
\end{itemize}
%%%%%%%%%%%%%%%%%%%%%%%%%%%%%%%%%%%%%%%%%%%%%%%%%%%%%%%%%%%%%%%%%%%%%%%%%%%%%%
\section{Gauge Fields}

%%%%%%%%%%%%%%%%%%%%%%%%%%%%%%%%%%%%%%%%
\subsection{Important Facts}
\begin{itemize}
    \item The Dirac monopole quantization condition corresponds to the classification of principal $U(1)$-bundles over $S^2$.
    
    \item A suitable Sobolev completion of the gauge group $\mathcal G$ is a Hilbert Lie group, whose Lie algebra is a Sobolev completion of $\mathcal{LG}$, which is a Banach Lie algebra.
    \begin{itemize}
        \item The Lie group has $3$ equivalent formulations: base preserving, equivariant automorphisms of $P$; equivariant funtions on $P$ to $G$; sections of $P \times G / G = Ad P$ under adjoint action.
        
        \item The gauge algebra is isomorphic as a Lie algebra to the equivariant sections on $ad P = P \times \mathfrak g / G$.
    \end{itemize}
\end{itemize}

%%%%%%%%%%%%%%%%%%%%%%%%%%%%%%%%%%%%%%%%%%%%%%%%%%%%%%%%%%%%%%%%%%%%%%%%%%%%%%
\section{The Space of Gauge Potentials}

%%%%%%%%%%%%%%%%%%%%%%%%%%%%%%%%%%%%%%%%%%%%%%%%%%%%%%%%%%%%%%%%%%%%%%%%%%%%%%%%
\subsection{Important Facts}
\begin{itemize}
    \item \emph{The topology and geometry of the space of gauge connections has significance for all physical theories and especially for the problem of quantization of gauge theories}.
    
    \item The space of connections $\mathcal A$ is an affine space with associated vector space $\Lambda^1(M, ad P)$ (basic?), thus the tangent space at each $\omega \in \mathcal A$ is isomorphic to $\Lambda^1(M, ad P)$.
    
    \item \lbtext{Moduli space of gauge potentials}: space of connections modulo the action of the gauge group: $\mathcal O := \mathcal{A/G}$
    
    
\end{itemize}

%%%%%%%%%%%%%%%%%%%%%%%%%%%%%%%%%%%%%%%%%%%%%%%%%%%%%%%%%%%%%%%%%%%%%%%%%%%%%%%%
\subsection{Medium Important Details}
\begin{itemize}
    \item .Many fibrations are introduced similar to the the fibration of the moduli space over.... Their topology is studied.
\end{itemize}

%%%%%%%%%%%%%%%%%%%%%%%%%%%%%%%%%%%%%%%%%%%%%%%%%%%%%%%%%%%%%%%%%%%%%%%%%%%%%%
\section{Gribov Ambiguity}

\subsection{Important Facts}
\begin{itemize}
    \item \rtext{In the Feynman integral approach to quantization, one must integrate these gauge invariant functions over the orbit/moduli space $\mathcal O$ to avoid the infinite contributions coming form gauge equivalent fields}.
    \begin{itemize}
        \item Gauge fixing: the procedure of choosing one connection in $\mathcal A$ for each equivalence class in $\mathcal O$. ``Faddeev-Popov approach''
        
        \item \lbbox{The Gribov ambiguity} is the problem of the non-existence of a global gauge/not being able to gauge fix. \rtext{It is present in all physically relevant cases}.
        
        \item However ways around it are defined, like constructing local gauges and weigting them to define the path integral.
    \end{itemize}
\end{itemize}

%%%%%%%%%%%%%%%%%%%%%%%%%%%%%%%%%%%%%%%%%%%%%%%%%%%%%%%%%%%%%%%%%%%%%%%%%%%%%%
\section{Matter Fields}

%%%%%%%%%%%%%%%%%%%%%%%%%%%%%%%%%%%%%%%%%%%%%%%%%%%%%%%%%%%%%%%%%%%%%%%%%%%%%%
\section{Gravitational Field Equations}
%%%%%%%%%%%%%%%%%%%%%%%%%%%%%%%%%%%%%%%%%%%%%%%%%%%%%%%%%%%%%%%%%%%%%%%%%%%%%%
\section{Geometrization Conjecture and Gravity}


\chapter*{7. Field Theories II: Quantum and Topological}

Apparently the problems of quantization are discussed here. Something to do with the topology of the spaces of connection, probably.

\end{document}