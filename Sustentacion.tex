\documentclass{beamer}

%%% Packages
\usepackage{amssymb, amsmath, amsthm}
\usepackage[english]{babel}
\usepackage[utf8]{inputenc}
\usepackage{biblatex} \addbibresource{bibliografia.bib}

% TODO notes package
\usepackage{xargs}                  % Use more than one optional parameter in a new command
%\usepackage[pdftex,dvipsnames]{xcolor}
\input{tools/todoCode}

% Colored text and boxes with my color conventions for highlighting
%\usepackage[dvipsnames]{xcolor}
%\usepackage[dvipsnames]{xcolor}

%
% \newcommand{\ytext}[1]{\textcolor{yellow}{#1}}
% \newcommand{\otext}[1]{\textcolor{orange}{#1}}
% \newcommand{\rtext}[1]{\textcolor{red}{#1}}
% \newcommand{\lbtext}[1]{\textcolor{cyan}{#1}}
% \newcommand{\dbtext}[1]{\textcolor{blue}{#1}}
% \newcommand{\ptext}[1]{\textcolor{Plum}{#1}}
% \newcommand{\lgtext}[1]{\textcolor{LimeGreen}{#1}}
% \newcommand{\dgtext}[1]{\textcolor{OliveGreen}{#1}}

\newcommand{\ytext}[1]{\textcolor{black}{#1}}
\newcommand{\otext}[1]{\textcolor{black}{#1}}
\newcommand{\rtext}[1]{\textcolor{black}{#1}}
\newcommand{\lbtext}[1]{\textcolor{black}{#1}}
\newcommand{\dbtext}[1]{\textcolor{black}{#1}}
\newcommand{\ptext}[1]{\textcolor{black}{#1}}
\newcommand{\lgtext}[1]{\textcolor{black}{#1}}
\newcommand{\dgtext}[1]{\textcolor{black}{#1}}

% \newcommand{\ybox}[1]{\colorbox{yellow}{#1}}
% \newcommand{\obox}[1]{\colorbox{orange}{#1}}
% \newcommand{\rbox}[1]{\colorbox{Salmon}{#1}}
% \newcommand{\lbbox}[1]{\colorbox{SkyBlue}{#1}}
% \newcommand{\dbbox}[1]{\colorbox{NavyBlue}{#1}}
% \newcommand{\pbox}[1]{\colorbox{Plum}{#1}}
% \newcommand{\lgbox}[1]{\colorbox{LimeGreen}{#1}}
% \newcommand{\dgbox}[1]{\colorbox{OliveGreen}{#1}}

% Math symbols
\usepackage{xparse}

%\usepackage{amssymb,amsmath,amsthm}
%\usepackage{xparse}

%%% Common symbols redifined
\let\oldepsilon\epsilon
\renewcommand{\epsilon}{\varepsilon}
\renewcommand{\varepsilon}{\oldepsilon}

\let\oldphi\phi
\renewcommand{\phi}{\varphi}
\renewcommand{\varphi}{\oldphi}

\newcommand{\emty}{\varnothing}
\newcommand{\varempty}{\nothing}

%%% Common Sets
\newcommand{\bb}[1]{\ensuremath{\mathbb{#1}} }
\newcommand{\ZZ}{\ensuremath{\mathbb{Z}} }
\newcommand{\NN}{\ensuremath{\mathbb{N}} }
\newcommand{\QQ}{\ensuremath{\mathbb{Q}} }
\newcommand{\RR}{\ensuremath{\mathbb{R}} }
\newcommand{\CC}{\ensuremath{\mathbb{C}} }


\newcommand{\iss}{\cong}  % Isomorphism symbol, here it is the one with a tilde

%%%%%%%%%%%%%% Sets

\newcommand{\set}[1]{\ensuremath{\left\{ #1 \right\}}} % Set function, simply puts nice left and right braces
\newcommand{\st}{\ensuremath{\ |\ }} % Such that symbol, TODO improve

\newcommand{\psubset}{\ensuremath{\subset}}
\renewcommand{\subset}{\ensuremath{\subseteq}} % Subset symbol, in this case it is the subset or equal to symbol

\newcommand{\bunion}{\bigcup}
%\newcommand{\biun}{\bun^{\infty}}
%\newcommand{\bfun}[3][n]{\bun_{#1 = #2}^{#3}}
\newcommand{\union}{\cup}
%\newcommand{\siun}{\sun^{\infty}}
%\newcommand{\sfun}[3][n]{\sun_{#1 = #2}^{#3}}

\newcommand{\binter}{\bigcap}
%\newcommand{\biinter}{\binter^{\infty}}
%\newcommand{\bfifter}[3][n]{\binter_{#1 = #2}^{#3}}
\newcommand{\inter}{\cap}
%\newcommand{\sininter}{\sinter^{\infty}}
%\newcommand{\sfinter}[3][n]{\sinter_{#1 = #2}^{#3}}

%%%%%% Calculus

% Integrals

%% Single Integrals
\NewDocumentCommand \integ {s O{} O{} m o}
{
	\IfBooleanTF{#1}{\oint}{\int}_{#2}^{#3} #4 %
	\IfNoValueF {#5} {\mathrm{d} #5}
}

%% Derivatives

% Normal derivative
% Example: \der[n]{f}{x}[x_0]
\NewDocumentCommand \der {O{} m m o}
{
	\frac{\mathrm{d}^{#1} #2}{\mathrm{d} {#3}^{#1}}%
	\IfNoValueF{#4} {\biggr|_{#4}}
}

%% Partial Derivatives

% With respect to one variable
% Example: \pder[n]{f}{y}[\pthvars[x_0][y_0][z_0]][(x, z)
\NewDocumentCommand \pder {O{} m m O{} o}
{
    \ensuremath{
	\IfNoValueTF {#5}
	{
		\frac{\partial^{#1} #2}{\partial {#3}^{#1}} #4
	}
	{
		\left(%
		\frac{\partial^{#1} #2}{\partial {#3}^{#1}}%
		\right)_{#5}  #4
	}
	}
}

% With respecto to two variables
% Example: \twpder{g}{y}{x}[(x_0, y_0)]
\NewDocumentCommand \twpder {m m m O{}}
{
	\frac{\partial^2 #1}{\partial #2 \partial #3} #4
}

\newcommand{\abs}[1]{\left\lvert #1 \right\rvert}
\newcommand{\norm}[1]{\left\lVert #1 \right\rVert}

% Physics symbols (vectors, units)
%\usepackage{tikz}
%\input{tools/physics_macros}

% Theorem environments
%%Version of October 8, 2016

%\usepackage{amsthm}

\theoremstyle{definition} %To avoid the annoying italics all the time, and to not sloppily redefine all of them 

\newtheorem{theo}{Theorem}[section]  %numbered according to section environment, so in section to it restarts as 2.1 
\newtheorem{prop}{Proposition}[section]  %numbered according to section environment, so in section to it restarts as 2.1 
\newtheorem{lemma}[theo]{Lemma}     %numbering shared with theorem 
\newtheorem{defn}{Definition}[section]   
\newtheorem{coro}{Corollary}[theo]


\theoremstyle{remark} 
\newtheorem*{remark}{Remark} 
 
%\let\oldtheo\theo 
%\renewcommand{\theo}{\oldtheo\normalfont}  
%  
%\let\olddefn\defn  
%\renewcommand{\defn}{\olddefn\normalfont}  
%  
%\let\oldlemma\lemma  
%\renewcommand{\lemma}{\oldlemma\normalfont}  
%  
%\let\oldcoro\coro  
%\renewcommand{\coro}{\oldcoro\normalfont}

%%%%%%%% ``Example'' environment, very basic, doesnt work with itemize
\theoremstyle{definition}

\newtheorem*{exmp}{Example}

\usepackage{tikz-cd} %To do Commutative Diagrams

% Extras
\newcommand{\linea}{\rule{10cm}{1mm}}
\newcommand{\lin}{\rule{5cm}{0.4mm}}
\usepackage{ulem}
%\renewcommand{\theequation}{\thechapter.\arabic{equation}} % To label Ch 0 equations as 0.1, 0.2, etc

% %%%%%%%
% \newenvironment{dedication}
%   {\clearpage           % we want a new page
%   \thispagestyle{empty}% no header and footer
%   \vspace*{\stretch{1}}% some space at the top 
%   \raggedleft          % flush to the right margin
%   }
%   {\par % end the paragraph
%   \vspace{\stretch{3}} % space at bottom is three times that at the top
%   \clearpage           % finish off the page
%   }

%%%%%%%
\graphicspath{ {images/} }

%%%%%%%
% \theoremstyle{definition}
% \newtheorem{example}{Example}[section]
% \newtheorem{definition}[example]{Definition}
% \newtheorem{theorem}[example]{Theorem}
% \newtheorem{proposition}[example]{Proposition}
% \newtheorem{lemma}[example]{Lemma}
% \newtheorem{corollary}[example]{Corollary}
% \newtheorem{notation}[example]{Notation}
% \newtheorem{remark}[example]{Remark}

\usepackage{environ}

\NewEnviron{eqnsplit}{%
  \begin{equation}
  \begin{split}
    \BODY
  \end{split}
  \end{equation}
}

\NewEnviron{eqnsplit*}{%
  \begin{equation*}
  \begin{split}
    \BODY
  \end{split}
  \end{equation*}
}

% Compile document without proofs
\usepackage{comment}

%\excludecomment{proof}
%%%%%%%

\newcommand{\vphi}{\varphi}
\newcommand{\vepsilon}{\varepsilon}
\newcommand{\conj}[1]{\overline{#1}}

\DeclareMathOperator{\HH}{\bb H}
\DeclareMathOperator{\FF}{\bb F}
\DeclareMathOperator{\diffeo}{\cong}
\DeclareMathOperator{\tr}{tr}

\DeclareMathOperator{\Diff}{\textit{Diff}}
\DeclareMathOperator{\Hom}{\textit{Hom}}
\DeclareMathOperator{\End}{\textit{End}}
\DeclareMathOperator{\Alt}{\textit{Alt}}
\DeclareMathOperator{\Sym}{\textit{Sym}}


\newcommand{\algeb}[1]{\mathfrak{#1}}
\newcommand{\alg}[1]{\mathfrak{#1}}
\newcommand{\oid}[1]{\ensuremath{\mathfrak{#1}}}
\newcommand{\ppal}[1]{\ensuremath{\mathcal{#1}}}

\newcommand{\sect}[1]{\ensuremath{\bm{#1}}} % Boldface that affects greek letters
\newcommand{\sectalg}[1]{\sect{\alg{#1}}}
\newcommand{\sectoid}[1]{\sect{\oid{#1}}}
\newcommand{\sectppal}[1]{\sect{\ppal{#1}}}

\newcommand{\cl}[1]{\ensuremath{\left\langle #1 \right\rangle}}
\newcommand{\upsect}[1]{\overline{\sect{#1}}}
\newcommand{\up}[1]{\overline{#1}}
\newcommand{\downsect}[1]{\cl{\sect{#1}}}
\newcommand{\down}[1]{\cl{#1}}

\let\oldtilde\tilde
\renewcommand{\tilde}{\widetilde}
\newcommand{\stilde}[1]{\tilde{\sect{#1}}}

\DeclareMathOperator{\comp}{\circ}

%%%%%
% Choose how your presentation looks.
%
% For more themes, color themes and font themes, see:
% http://deic.uab.es/~iblanes/beamer_gallery/index_by_theme.html
%
\mode<presentation>
{
  \usetheme{Darmstadt}      % or try Darmstadt, Madrid, Warsaw, ...
  \usecolortheme{beaver}%seagull} % or try albatross, beaver, crane, ...
  \usefonttheme{default}  % or try serif, structurebold, ...
  \setbeamertemplate{navigation symbols}{}
  \setbeamertemplate{caption}[numbered]
} 

% Show number of slides at the bottom
\addtobeamertemplate{navigation symbols}{}{%
    \usebeamerfont{footline}%
    \usebeamercolor[fg]{footline}%
    \hspace{1em}%
    \insertframenumber/\inserttotalframenumber
}

%%%%%% Show TOC before a new Section
\AtBeginSection[]
{
    \begin{frame}[noframenumbering]{Table of Contents}
        \tableofcontents[currentsection]
    \end{frame}
}

% Show TOC before a new SubSection
\AtBeginSubsection[]
{
    \begin{frame}[noframenumbering]{Table of Contents}
        \tableofcontents[currentsection,currentsubsection]
    \end{frame}
}
% Reduce size of TOC
\AtBeginDocument{
  %\addtocontents{toc}{\tiny}
  %\addtocontents{subsection in toc}{\tiny}
}

%\setbeamerfont{subsection in toc}{size=\tiny}

%%%%%% Make paragraphs start with no indentation and leave spaces between paragraphs
% \setlength{\parindent}{0em}
 \setlength{\parskip}{1em}

% \renewcommand\englishhyphenmins{22}
% \usepackage{microtype}

%%%%%% Document Information
\title[l]{Gauge Theories on Transitive Lie Algebroids}
\author{Sebastian Puerto}
\institute{Universidad de los Andes\\ Proyecto de Grado}
\date{December 18, 2020}

%%%%%%%%%%%%%%%%%%%%%%%%%%%%%%%%%%%%%%%%%%%%%%%%%%%%%%%%%%%%%%
%%%%%%%%%%%%%%%%%%%%%%%%%%%%%%%%%%%%%%%%%%%%%%%%%%%%%%%%%%%%%%
%%%%%%%%%%%%%%%%%%%%%%%%%%%%%%%%%%%%%%%%%%%%%%%%%%%%%%%%%%%%%%
%%%%%%%%%%%%%%%%%%%%%%%%%%%%%%%%%%%%%%%%%%%%%%%%%%%%%%%%%%%%%%
\begin{document}

\begin{frame}[noframenumbering]
  \titlepage
\end{frame}

\begin{frame}{The Complete Framework} % % % % % % % % % % % % % % % % % % %
An \emph{orientable} \emph{transitive Lie algebroid} $A$, over a base manifold $M$, equipped with a \emph{metric} $\hat g \equiv (g, h, \tilde \nabla)$, with an \emph{adjoint Lie algebroid} $L$ on which $h$ is a \emph{Killing metric}, and a vector bundle $E$ on which there is a \emph{representation} $\phi: A \to \alg D(E)$ and a metric $h^E$ \emph{compatible with $\phi$}.

Then, the gauge theory is determined by its action functional: given a \emph{connection form} $\hat \omega$ on $A$ with \emph{curvature} $\hat R$, and a \emph{matter field} $\mu \in \Gamma(E)$, the \emph{action} is:
\begin{eqnsplit*}
\mathcal S[\hat \omega, \hat \mu] &= \int_A h(\hat R, *\hat R) + \int_A h^E(\hat \nabla^E \mu, \hat \nabla^E\mu)\\
    &= \int_M [ \mathcal L_{gauge}[\hat \omega] + \mathcal L_{matter}[\hat \omega, \mu]] ,
\end{eqnsplit*}
where $\hat \nabla^E$ is the \emph{$A$-connection on $E$ produced by the connection} $\hat \omega$ given the representation $\phi$ \cite{Fournel2013}.

\end{frame}

% Uncomment these lines for an automatically generated outline.
\begin{frame}[noframenumbering]{Outline}
  \tableofcontents
\end{frame}



\begin{frame}{Conventions}

    - $A$ is Lie algebroid with anchor $a$ and bracket $[\cdot, \cdot]$
    \\- $p:E \to M$ is a vector bundle with typical fiber $V$. Let $\{e_u\}_{u = 1, \dots, t}$ denote a basis of $E$, dual to $\{\tilde e^u\}$.
    \\- $\alg g$ denotes a Lie algebra. Let $\{E_a\}_{a, \dots }$ be a basis, dual to $\{\epsilon^a\}$.
    \\- $G \to P \to M$ denotes a principal bundle, and $\text{Lie}(G) = \alg g$.
    \\- $X \in TM$ or $X \in \Gamma(TM)$, depending on context.
    \\- $\eta, \theta$ either in $\alg g$ or in $C^\infty(M, \alg g)$, or in a LAB, or in a section of a LAB.
    \\- $0 \rightarrow L \rightarrow A \xrightarrow{a} TM \rightarrow 0$ is transitive Lie algebroid sequence.
    \\- $\phi: A \to \alg D(E)$ is a representation, with vertical part $\phi_L$.
    
    \end{frame}



%%%%%%%%%%%%%%%%%%%%%%%%%%%%%%%%%%%%%%%%%%%%%%%%%%%%%%%%%%%%%%
%%%%%%%%%%%%%%%%%%%%%%%%%%%%%%%%%%%%%%%%%%%%%%%%%%%%%%%%%%%%%%
\section{Lie Algebroid Theory}

\begin{frame}{Lie algebroid}

   - Lie algebroid\cite{Mackenzie2005} $A$: vector bundle over $M$, with \emph{anchor} $a:A \to TM$ and \emph{bracket} $[\cdot, \cdot]_A: \Gamma(A) \times \Gamma(A) \to \Gamma(A)$.
    \begin{equation}
    \begin{tikzcd}[ampersand replacement=\&, column sep=small]
    (A, [\cdot, \cdot]_A) \arrow{r}{a} \arrow{dr}{q} \& (TM, [\cdot, \cdot]) \arrow{d}{\pi}\\
    \& M.
    \end{tikzcd}
    \end{equation}
    - \textit{Examples}: $TM$, Lie Algebra Bun. (LAB), Involutive Distributions.
    
    - \emph{Transitive Lie algebroid}: $a$ is fiberwise surjective. Then $L \cong ker(a)$ is LAB, and induces:
    \begin{equation*}
        0 \to L \xrightarrow{j} A \xrightarrow{a} TM \to 0.
    \end{equation*}
    \\
    - \emph{Trivial Lie Algebroid} (TLA): Given Lie algebra $\alg g$, $TM \oplus (M \times \alg g)$. Bracket: $[X, \eta] := X(\eta)$, for $X \in \Gamma(M)$, $\eta \in C^\infty(M, \alg g)$.
    % \begin{equation}
    % \begin{tikzcd}[ampersand replacement=\&, column sep=small]
    % (TM \oplus (M \times \alg g), [\cdot, \cdot]) \arrow{r}{pr_1} \arrow{dr}{\pi \oplus p_1} \& (TM, [\cdot, \cdot]) \arrow{d}{\pi}\\
    % \& M.
    % \end{tikzcd}
    % \end{equation}
    
\end{frame}



\begin{frame}{Examples}
    \begin{equation*}
        0 \to M \times \alg g \to TM \times \alg g \to TM \to 0
    \end{equation*}
    
    - \emph{Atiyah Lie algebroid} of principal bundle $G \to P \to M$. Notice the short exact sequence of Lie algebras and $C^\infty(M)$-modules:
    \begin{equation*}
        0 \to C^\infty_G(P, \alg g) \xrightarrow{\overline{j}} \Gamma^G(TP) \xrightarrow{\pi_*} \Gamma(TM) \to 0,
    \end{equation*}
    induces the sequence of Lie algebroids:
    \begin{equation*}
        0 \to P \times \alg g/G \xrightarrow{j} TP/G \xrightarrow{\pi_*^G} TM \to 0,
    \end{equation*}
    
    - \emph{Derivations Lie algebroid}: A \emph{derivation} $D: \Gamma(E) \to \Gamma(E)$ is such that:
    $%\[
        D(f \mu) - f D( \mu) = a(D)(f) \mu
    $%\] 
    , for some $a(D) \in \Gamma(TM)$. Then
    \begin{equation*}
        0 \rightarrow \End(E) \rightarrow \alg D(E) \xrightarrow{a} TM \rightarrow 0.
    \end{equation*}
    If $E = M \times V$ trivial v.b., then $\alg D(E) = TM \times \alg{gl}(V)$.
    
\end{frame}



\begin{frame}{Lie Algebroid Atlas of transitive L.A.}
    
    - \emph{Morphism} $\phi$. Between TLAs is $\phi(X \oplus \eta) = X \oplus (\omega(X) + \phi_L(\eta))$.
    
    - \emph{Representation on E}: morphism $\phi: A \to \alg D(E)$.
    Of $TP/G$ on associated v.b. $E = P \times V / G$: $\tilde{\phi(X)( \mu)}:= \overline{X}(\tilde{ \mu}) \in C^\infty_G(P, V).$
    
    - \emph{Algebroid Atlas}: $\{(U_i, \psi_i: U_i \times \alg g \to L|_{U_i}, \nabla^{0, i}: TU_i \to A|_{U_i})\}_{i \in I}$ such that the following is isomorphism of Lie algebroids
    \begin{align}
        S_i: TU_i \times \alg g &\to A|_{U_i} & X \oplus \eta &\mapsto \nabla^{0,i}_X \oplus j \psi_i(\eta).
    \end{align}
    Change of coordinates:
    \begin{align}
        S_j(X \oplus \eta) = S_i(X \oplus [\alpha^i_j(\eta) + \chi^i_j(X)])
    \end{align}
    - For $TP/G$, if $\sigma_i$ is local section and $g_{ij}$ transition map for $P$:
    \begin{align*}
        S_i(X \oplus \eta)&= \cl{\sigma_{i*}(X) + \der{t}[t = 0]\sigma_i(m) \exp{(-t \eta)}},\\
         \alpha^i_j(\eta) &= g_{ij} \eta {g_{ij}}^{-1},&
         \chi^i_j(X) &= g_{ij} dg^{-1}_{ij}.
    \end{align*}
    
\end{frame}




\begin{frame}{Examples}
    - \emph{Group induced representation}: representation $\phi:TP/G \to $ $\alg D(P \times V/G)$ locally looks like:
    \begin{equation}
        \phi_i(X \oplus \eta) \psi = X(\psi) + \eta \cdot \psi.
    \end{equation}
    
    \textit{Examples}:
    $U_S = S^n \setminus NP$, $U_N = S^n \setminus SP$ atlas of spheres $S^n$. $S^1$-bundles over $S^2$ characterized by transition functions:
    \begin{align*}
        g^k_{NS}: U_{NS} &\subset S^2 \to S^1,&
    E(\phi, \theta) &\mapsto e^{i k \theta}, \text{   then}
    \end{align*}
    \begin{align*}
        \alpha^N_S &= Id = \alpha^S_N, & \chi^N_S &= -ikd\theta,&
     \chi^S_N &= +ik d\theta.
    \end{align*}
    Hence, the following change of local trivialization holds:
\begin{align*}
    S^S_N(\partial_\phi) &= \partial_\phi, &  S^S_N(\partial_\theta), &= \partial_\theta \oplus ik, & S^S_N(i) &= i.
\end{align*}
\end{frame}


%%%%%%%%%%%%%%%%%%%%%%%%%%%%%%%%%%%%%%%%%%%%%%%%%%%%%%%%%%%%%%
%%%%%%%%%%%%%%%%%%%%%%%%%%%%%%%%%%%%%%%%%%%%%%%%%%%%%%%%%%%%%%
\section{Differential Forms}

\begin{frame}
    \textit{Why?}\cite{Fournel2011, Lazzarini2012} \textbf{Connection form} formulation of principal bundle connection will be generalized. Moreover, forms are what can be \textbf{integrated}, so to define the Lagrangian and action functional.
    
    - \emph{$E$-valued differential forms on $A$}: $\omega \in \Omega_U^p(A, E)$ is an alternating v.b. map $\omega : A|_U \otimes \cdots \otimes A|_U \to E|_U$. $\Omega^\bullet(A, E)$ is $C^\infty(M)$-module.\\
    \textit{E.g.:} $\Omega^\bullet(TM)$, $\Omega^\bullet(TP, P \times \alg g)$, $\Omega^\bullet(A)$, $\Omega^\bullet(A, L)$, $\Omega^\bullet(A, \End(E))$.
    
    - We need: \\
    1. \emph{Local trivializations}: forms on $\Omega^\bullet(TU_i \times \alg g, U_i \times V)$, $\omega_i = \sum_{r + s = p} (\omega_i^\epsilon)_{\mu_1 \cdots \mu_r, a_1 \cdots a_s}\wedge dx^{\mu_1} \wedge \cdots \wedge dx^{\mu_r} \wedge \epsilon^{a_1} \wedge \cdots \wedge \epsilon^{a_s}$.\\
    2. A \emph{connection form on $A$}: $\hat \omega \in \Omega^1(A,L)$ with \emph{curvature form} $\hat R = \hat d \hat \omega + \frac{1}{2} \hat \omega \wedge^{[,]} \hat \omega$.
    
    - To define $\hat d_\phi$: representation $\phi$ needed $\to$ Lie bracket $\to \hat d_\phi^2 = 0$: 
    $\hat d_\phi \omega(\oid X_1, \dots, \oid X_{p+1}) = \sum_{i=1}^{p+1} (-1)^{i+1} \phi(\oid X_i)\cdot \omega(\oid X_1, \cdots, \overset{\vee}{i}, \cdots, \oid X_{p+1})$ \\
    $+ \sum_{1 \leq i < j \leq p+1} (-1)^{i+j}\omega([\oid X_i, \oid X_j], \oid X_1, \cdots, \overset{\vee}{i}, \cdots, \overset{\vee}{j}, \cdots, \oid X_{p+1})
    $.

\end{frame}

\begin{frame}{Background}
    
    - \emph{Wedge product}: For $\Omega^\bullet(A)$ with $\Omega^\bullet(A)$ $|$ $\Omega^\bullet(A)$ with $\Omega^\bullet(A,E)$ $|$ \& $\Omega^\bullet(A,E)$ with $\Omega^\bullet(A,E)$ if $(E, \bullet)$ is AB and \emph{$\phi$ compatible with $\bullet$}: \quad $(\omega \wedge \eta)(\oid X_1, \dots, \oid X_{p+q}) :=$ 
    $\frac{1}{p!q!} \sum_{\sigma \in S_{p+q}} (-1)^{\sigma} \omega(\oid X_{\sigma(1)}, \cdots, \oid X_{\sigma(p)}) \bullet \eta(\oid X_{\sigma(p+1)}, \cdots, \oid X_{\sigma(p+q)}).$
    
    Compatibility: $\hat d_\phi(\omega \wedge \eta) = (\hat d_\phi\omega)\wedge \eta + (-1)^{|\omega|} \omega \wedge (\hat d_\phi\eta)$.
        
    - \textbf{Theorem}: 
    \begin{itemize}
        \item $(\Omega^\bullet(A), \wedge, \hat d_A)$ is differential graded commutative algebra;
        
        \item $(\Omega^\bullet(A, E), \hat d_\phi)$ is differential complex over $(\Omega^\bullet(A), \wedge)$ and satisfies graded Leibniz;
        
        \item $(\Omega^\bullet(A, E), \wedge, \hat d_{\phi})$ is differential graded Lie algebra if $(E, [\cdot, \cdot])$ is LAB and $\phi$ is compatible with $[\cdot, \cdot]$. 
        
    \end{itemize}
    
    - TP/G
    \end{frame}

\begin{frame}{Local Trivialization}
    Local trivialization of forms:
    
   $\omega_i : (TU_i \times \alg g)^p \xrightarrow{S_i} (A|_{U_i})^p \xrightarrow{\omega} E|_{U_i} \xrightarrow{\beta_i^{-1}} U_i \times V$
    
    - \textbf{Theorem}: The trivializing map $\cdot_i$ is an isomorphism of differential complexes. If $\Omega(A, E)$ is DGA, it also respects the wedge product.
    
%     - \begin{theorem}\label{theoremHatAlphaRespectsDifferentialAndWedgeIsomorphismOfModulesAndAlgebra}
% $\hat \beta^i_j: \Omega_{U_{ij}}^q(TU_{ij}\times \alg g, U_{ij}\times V) \to \Omega_{U_{ij}}^q(TU_{ij}\times \alg g, U_{ij}\times V)$ is an isomorphism of differential graded modules. Furthermore, if they are DGAs, it is also isomorphism of DGAs.
% \end{theorem}

- Change of trivialization: if $\beta_i$ is trivialization of $E$, $\hat \beta^i_j(\omega_j) = \omega_i = \beta^i_j \comp \omega_j \comp S_i^j$. For $\Omega^\bullet (A)$ and $\Omega^\bullet(A, L)$ we call it $\alpha^i_j$.

\end{frame}

\begin{frame}{Examples}

Local triv. of an element in $\Omega^1(TP^k/S^1, P^k \times i\RR/S^1)$ over $U_S$
\begin{equation}
    \hat \omega_S = i\hat \omega^\epsilon_{S; 1}(\phi, \theta) dx^1 + i\hat \omega^\epsilon_{S; 2}(\phi, \theta) dx^2 + i\hat \omega^\epsilon_{S; i}(\phi, \theta) Im;
\end{equation}
Since $U_S$ covers all but one point, we can evaluate $\hat \alpha^N_S(\hat \omega|_{U_{SN}})$ to find the complete family of triv., if $\Omega \in \Omega^1(TP^k/S^1, P^k \times i\RR/S^1)$ exists.  $\hat \omega_S$ is the local trivialization of a global form. For example: derivatives $\hat \omega_{S; 1}^\epsilon(\vec y = 0)^{(n)}=0$, $n = 0,1,2$ and $\hat \omega_{S; i}^\epsilon(\vec y)^{(m)} = 0$, $m= 0,1$.

Differential: $(\hat d \omega)_S = \hat d \omega_S$:
\begin{equation*}
    \hat d \hat \omega_S = \hat d (i\omega^\epsilon_{S;\mu}) \wedge dx^\mu + i  \omega^\epsilon_{S;\mu} d(dx^\mu) + \hat d(i \hat \omega^\epsilon_{S;i}) \wedge Im + \omega^\epsilon_{S;i} \hat d_{TU_S \times i\RR} Im;
\end{equation*}
using that $i\RR$ is commutative: $\hat d_{TU_S \times i\RR} Im = 0$ and $\hat d (i \omega^\epsilon_{S;\cdot}) = i d\omega^\epsilon_{S;\cdot}$, hence
\begin{equation}
    \hat d \omega_S = i(\partial_1 \omega^\epsilon_{S;2} - \partial_2 \omega^\epsilon_{S;1}) dx^1 \wedge dx^2 + i \partial_1 \omega^\epsilon_{S;i} dx^1 \wedge Im + i \partial_2 \omega^\epsilon_{S;2} dx^2 \wedge Im
\end{equation}


\end{frame}


%%%%%%%%%%%%%%%%%%%%%%%%%%%%%%%%%%%%%%%%%%%%%%%%%%%%%%%%%%%%%%
%%%%%%%%%%%%%%%%%%%%%%%%%%%%%%%%%%%%%%%%%%%%%%%%%%%%%%%%%%%%%%
\section{Connections and $A$-Connections}

\begin{frame}{Connections on $A$}
    %In traditional gauge theories, connections are introduced on the principal bundle, also called gauge potentials, to induce covariant derivatives on the representation vector bundles. It is this kind of covariant derivative what guarantees that the equations of motion of matter fields, i.e. sections, are preserved under a change of gauge (local).
    In traditional gauge theories, gauge potentials are introduced to induce covariant derivatives on matter fields.
    The same will happen here: connections on $A$ will induce a generalized version of covariant derivatives on representation vector bundles, called $A$-connections.
    
    - \emph{Ordinary connection on $A$}: $\nabla: TM \to A$ section of $a$. Equiv. to $\omega \in \Omega^1(A, L)$ such that $\omega \comp j = -1_L$. \todo{horizontality, lift} E.g. connections on $P$ and $E$.
    
    - \emph{Connection form on $A$} $\hat \omega \in \Omega^1(A, L)$. $\tau = \hat \omega \comp j + 1_L \in End(L)$.
    
    Equivalent to anchor preserving v.b. map: $\hat \nabla_{\oid X} = \oid X + j \comp \hat \omega(\oid X)$.
    
    - \emph{Curvature Form}\todo{measure Lie morphism}: $\hat R := \hat d \hat \omega + \frac{1}{2} \hat \omega \wedge^{[,]} \hat \omega \in \Omega^2(A, L)$. \\Bianchi Identity: $\hat d \hat R + \hat \omega \wedge^{[,]} \hat R = 0$.\\ - \emph{Algebraic curvature of $\tau$}: $R_\tau(\eta, \theta) := [\tau(\eta), \tau(\theta)] - \tau[\eta, \theta]$. 
\end{frame}



\begin{frame}{A-Connection and Local Trivializations}
    - \emph{$A$-Connection on $E$}: $\hat \nabla^E: A \to \alg D(E)$ anchor preserving v.b. map. E.g. covariant derivatives on $E$.
    
    - Given $\phi$, equivalent to $\hat \omega^E \in \Omega^1(A, End(E))$, with representation $\tilde \phi$, $\tilde \phi(\oid X) = [\phi(\oid X), \cdot] \in End(E)$: $\hat \nabla^E_{\oid X} = \phi(\oid X) + j\comp \hat \omega^E(\oid X)$.
    
    - \emph{Curvature}\todo{measure Lie}: $\hat R^E = \hat d_{E} \hat \omega^E + \frac{1}{2} \hat \omega^E \wedge^{[,]} \hat \omega^E$. Satisfies Bianchi.
    
    - \emph{Produced $A$-connection}: given $\hat \omega$, $\hat \nabla^{E, \hat \omega} := \phi \comp \hat \nabla$, $\hat \omega^E = \phi_L \comp \hat \omega$
    
    \textit{Local Trivializations:}
    
    - $\hat \omega_i = A_i \oplus (-\epsilon + \tau_i)$ and $\hat \nabla^i_{X \oplus \eta} = X \oplus (A_i(X) - \tau_i(\eta))$.
    
    - $\hat \nabla^{E, \hat \omega}_{X \oplus \eta} = X \oplus (B_i(X) + \phi_L \comp A_i(X) - \phi_L \comp \tau(\eta))$. \todo{$2$ extra terms}
    
\end{frame}



\begin{frame}{Examples}
    - Group produced $A$-connection: $\{E_a\}$ basis of $\alg g$. $\hat \nabla^{E, \hat \omega, i}_{X \oplus \eta} \psi = X(\psi) + i A_i^b(X) E_b \cdot \psi - \tau^b_a \eta^a E_b \cdot \psi$.
    
    Example $TP^k/S^2$ over $U_S$:
    
    $$\hat \omega_S = i\hat \omega^\epsilon_{S; 1} dx^1 + i\hat \omega^\epsilon_{S; 2} dx^2 - Im + i \tilde \tau Im,$$ where $\tilde \tau = \hat \omega^\epsilon_{S; i} + 1$.
    
    Curvature:
    $$\hat R_S = \hat d \hat \omega_S 
        = i(\partial_{1} \hat \omega^\epsilon_{S;2} - \partial_{2} \hat \omega^\epsilon_{S; 1}) dx^1 \wedge dx^2 + i \partial_{1} \tilde \tau dx^1 \wedge Im +  i \partial_{2} \tilde \tau dx^2 \wedge Im$$
\end{frame}

%%%%%%%%%%%%%%%%%%%%%%%%%%%%%%%%%%%%%%%%%%%%%%%%%%%%%%%%%%%%%%
%%%%%%%%%%%%%%%%%%%%%%%%%%%%%%%%%%%%%%%%%%%%%%%%%%%%%%%%%%%%%%
\section{Metrics and Integration}

\begin{frame}{Metric on Representation Vector Bundles}
    
    Yang-Mills-Higgs Action: $ S[\omega, \psi]=\int_M ||R||^2 + ||\hat \nabla^{E, \omega} \psi||^2$
    
    - \emph{(Degenerate) metric on $E$}: $h^E: E \otimes E \to M \times \RR$ symmetric.
    
    $h^E: \Omega^p(A, E) \otimes \Omega^q(A, E) \to \Omega^{p+q}(A)$ by
    \begin{multline*}
        h^E(\omega, \eta)(\oid X_1, \dots, \oid X_{p+q}) :=\\ \frac{1}{p!q!} \sum_{\sigma \in Sym_{p+q}} 
        h^E\left( \omega(\oid X_{\sigma(1)}, \dots, \oid X_{\sigma(p)}),  \eta(\oid X_{\sigma(p+1)}, \dots, \oid X_{\sigma_{p+q}}) \right);
    \end{multline*}
    in trivialization: $h^E|_{U_i}(\omega, \eta) = h^{E,i}_{uv}\,\omega^u \wedge \eta^v$.
    
    - \emph{$h^E$ compatible with $\phi$}: $h^E(\phi_L(\eta) \mu_1, \mu_2) + h^E(\mu_1, \phi_L(\eta)\mu_2) = 0$.
    
    If $E = L$, it is called \emph{a Killing inner metric on $A$}.
\end{frame}



\begin{frame}{Metric on $A$}
    - Metric $\hat g$ on $A$: equivalent to $(g, h, \tilde \nabla)$ by $$\hat g(\oid X, \oid Y) = g(a(\oid X), a(\oid Y)) + h(\alg a(\oid X), \alg a(\oid Y)).$$
    
    - Then subbundles $\{\tilde \nabla_X\} =: \mathcal H A$ and $jL$ are orthogonal. Locally $\{\tilde \nabla^i_{\mu}\}_\mu$ with $\{-E_a\}_a$ are basis, hance a local frame of $A$.
    
    It is dual to $\{dx^\mu, \alg a^a\}$, where $\alg a =  E_a \alg a^a$ is form associated to $\tilde \nabla$.
    
    Then $\hat g^i = g^i_{\mu \nu} dx^\mu \otimes dx^\nu + h^i_{ab} \alg a^a \wedge \alg a^b$, and any form $\beta \in \Omega^p(A, E)$ has decomposition:
    $
    \beta_i = \sum_{r + s = p} \emph{\left(\beta_i \right)_{\mu_1 \cdots \mu_r, a_1 \cdots a_s}} \, dx^{\mu_1} \wedge \cdots \wedge dx^{\mu_r} \wedge \alg a_i^{a_1} \wedge \cdots \wedge \alg a_i^{a_s}.
    $
    
\end{frame}



\begin{frame}{Integration of scalar valued forms}
    
    - $L$ orientable, then $\omega_{h, \alg a}^i := (-1)^n \sqrt{|h^i|} \alg a_i^{1} \wedge \cdots \wedge \alg a_i^{n} $ is the trivialization of the \emph{inner volume form of $A$}.
    
    \emph{Inner integration}: write arbitrary form $\beta_i = \beta^{M}_i \wedge \omega_{h, \alg a} + \beta_i^R$. Then $\int_{inner} \beta := \beta^M$.
    
    - $A$ orientable, \emph{volume form of $A$} $\omega^{Vol}= \sqrt{|g^i|} dx^1 \wedge \cdots \wedge dx^m \omega_{h, a}$.
    \\\emph{Integration over $A$}: 
    $\int_A \beta := \int_M \int_{inner} \beta$.
    
    - $g^{-1}_{h^E}: \Omega^p(A, E) \otimes \Omega^p(A, E) \to C^\infty(M)$ defined by
    $\emph{\hat g^i_{h}{}^{-1} (\alpha_i, \beta_i)} 
    := \sum_{r+s = p} \frac{1}{p!}  \boxed{g^{\mu_1 \nu_1} \cdots g^{\mu_r \nu_r} h^{a_1 b_1} \cdots h^{a_s b_s} h^E_{uv} \alpha^u_{\mu_1 \dots \mu_r a_1 \dots a_s}} \beta^v_{\nu_1 \dots \nu_r b_1 \dots  b_s}$
    
    - Hodge-$*$:$\Omega^p(A, E) \to \Omega^{m+n-p}(A, E)$: $h_E(\alpha, \emph{*\beta}) = g^{-1}_{h^E}(\alpha, \beta) \omega^{Vol}$
    
     - $\emph{(\alpha, \beta)} := \int_A h^E(\alpha, *\beta)$.
\end{frame}



\begin{frame}{Example}
    - If $K$ is the Killing form of semisimple $\alg g$, given $c \in C^\infty(M)$ never $0$: $\emph{h^c(\cl{p, \eta}, \cl{p, \theta})}:= c(m) K(\cl{p, \eta}, \cl{p, \theta})$ is Killing metric on $P \times \alg g/G$.
    
    On $P^k \times i\RR/S^1$ all Killing m. are $h^c(\cl{m, ia}, \cl{m, ib}) := c(m)ab$.
    
    Inner volume form: $\omega^S_{h^c, \alg a} = -\sqrt{|c|} (\tilde \omega^\epsilon_{S; 1}dx^1 + \tilde \omega^\epsilon_{S; 2}dx^2 - Im)$.
    
    - All c. irreps. of $S^1$ induce reps. of $i\RR$ $\pi^h(ir)(z) = irh\,z$, $h \in \ZZ$.
    
    Each of these induces representation $\phi^{k, h}: TP^k/S^1 \to E^k$.
    
    They are compatible with well defined metrics $c^E h^E : E^k \otimes E^k \to S^2 \times \RR$, where $c^E \in C^\infty(M)$ never $0$ and $h^E$ is standard $\RR^2$ metric.
\end{frame}

%%%%%%%%%%%%%%%%%%%%%%%%%%%%%%%%%%%%%%%%%%%%%%%%%%%%%%%%%%%%%%
%%%%%%%%%%%%%%%%%%%%%%%%%%%%%%%%%%%%%%%%%%%%%%%%%%%%%%%%%%%%%%
\section{Gauge Theory}

\begin{frame}{Gauge Transformations}
    Traditional infinitesimal gauge action of $\eta  \in \Gamma(P \times \alg g/G)$:
    \begin{align*}
        -\eta \cdot & \mu, &
        d\eta &+ w \wedge \eta, &
        - \eta \wedge  R;
    \end{align*}
    
    - \emph{Gauge algebra of $A$}: $\Gamma(L)$.
    
    - Gauge action of $\eta \in \Gamma(L)$ on $A$:
    \begin{align}
        \hat d \eta + \hat \omega \wedge \eta && 
        \hat R \wedge \eta; &&
        \hat \nabla^\eta := \hat \nabla + [\hat \nabla, j\eta]&&
        \tau^\eta := \tau + [\tau, \eta]
    \end{align}
    
    - Gauge action of $\eta \in \Gamma(L)$ on $E$:
    \begin{align}
        \mu^\eta := \mu - \phi_L(\eta)\mu, && \hat \nabla^{E, \eta} := \hat \nabla^E + [\hat \nabla^E, j\phi_L(\eta)] && \hat R^E \wedge \eta
    \end{align}
    $
        \hat \omega^{E, \eta} := \hat \omega^E + \hat d_E(\phi_L(\eta)) + \hat \omega^E \wedge \phi_L(\eta))
    $
    
\end{frame}



\begin{frame}{Gauge Action}
    - \emph{Gauge Lagrangian density:} $\mathcal L_{gauge}[\hat \omega] := \int_{inner} h(\hat R, *\hat R)$,
    
    - \emph{Gauge action}: $\mathcal S_{gauge}[\hat \omega] := (\hat R, \hat R)$; invariant under infinitesimal gauge transformations of $\hat \omega$ if $h$ is Killing.
    
    New induced ordinary connection: $\omega := \hat \omega + \tau \comp \tilde {\alg a}$.
    
    - $\hat F := R - \tau \comp \tilde R \in \Omega^2(TM, L)$\\
    - $(\mathcal D_X \tau)(\eta) := [\nabla_X, \tau(\eta)] - \tau[\tilde \nabla_X, \eta]\in \Omega^1(TM, End(L))$\\
    - $[a^* \mathcal D \tau, \tilde \omega] (\oid X, \oid Y) := (\mathcal D_{a(\oid X)})(\tilde \omega \oid Y) - (\mathcal D_{a(\oid Y)})(\tilde \omega \oid X)$
    
    -$(\hat F_i)^a_{\mu \nu} = R^a_{\mu \nu} - \tau^a_b \tilde R^b_{\mu \nu}$
    -$(\mathcal D \tau_i)^b_{\mu, a} = \partial_\mu (\tau_i)^b_a + (A_i)^c_\mu (\tau_i)^d_c C^b_{cd} - (\tilde A_i)_\mu^d C^c_{da} (\tau_i)^b_c$
    - $(W_i)^c_{ab} = (\tau_i)^b_a (\tau_i)^e_b C^c_{de} - C^d_{ab}(\tau_i)^c_d$.

    
    

%\textbf{Proposition}: The gauge Lagrangian density $\mathcal L_{gauge}$ is invariant under infinitesimal gauge transformations of the connection $\hat \omega$.% with respect to any $\eta \in \Gamma(L)$, up to first order terms in $\eta$.
\end{frame}


\begin{frame}
    \begin{multline*}
        \to \hat R = a^* \hat F - [a^* \mathcal D \tau, \tilde \omega] + \tilde \omega ^* R_\tau\\
        = (\hat F_i)_{\mu \nu}^a E_a dx^\mu \wedge dx^\nu + 
                    (\mathcal D \tau_i)^b_{\mu, a} E_b  dx^\mu \wedge (\alg a_i)^a +
                    (W_i)^c_{ab} E_c (\alg a_i)^a \wedge (\alg a_i)^b
    \end{multline*}
    \begin{eqnsplit*}
        \mathcal L_{gauge}& = 
        [g_i^{\mu_1 \nu_1} g^{\mu_2 \nu_2} h_{cd} (\hat F_i)^c_{\mu_1 \mu_2} (\hat F_i)^d_{\nu_1 \nu_2} 
        +  
        g^{\mu \nu} h^{a b} h_{cd} (\mathcal D \tau_i)^c_{\mu, a} (\mathcal D \tau_i)^d_{\nu, b} \\
        &\hfill 
        + h^{a_1 b_1} g^{a_2 b_2} h_{cd} (W_i)^c_{a_1 a_2} (W_i)^d_{b_1 b_2}] \sqrt{|g_i|}dx^1 \wedge \cdots \wedge dx^m \\
        &= \left[(\hat F_i)^c_{\mu \nu} (\hat F_i)_c^{\mu \nu} 
        + (\mathcal D \tau)^c_{\mu, a} (\mathcal D \tau)_c^{\mu, a} 
        + (W_i)^c_{ab} (W_i)_c^{ab}\right] vol_M.
\end{eqnsplit*}
\end{frame}

\begin{frame}{Matter Action}
    - \emph{Matter Lagrangian density:} $\mathcal L_{matter}[\mu, \hat \omega] := \int_{inner} h^E(\hat \nabla^E \mu, *\hat \nabla^E \mu)$,
    
    - \emph{Matter action}: $\mathcal S_{gauge}[\hat \omega] := (\hat \nabla^E \mu, \hat \nabla^E \mu)$; invariant under infinitesimal gauge transformations of $\hat \omega$ if $h^E$ is compatible with $\phi$.
    
    - $\hat \nabla^E \mu = a^* \phi(\nabla)\mu - (\phi_L(\tau) \mu)\comp \tilde \omega,$
    \begin{eqnsplit*}
        a^* \phi(\nabla)\mu_i &= [\partial_\nu (\mu_i)^u + (B_i)^u_{\nu, v} (\mu_i)^v + (\phi_{L, i})^u_{b, v} (A_i)_\nu^b (\mu_i)^v ]e_u dx^\nu\\
        (\phi_L(\tau) \mu)\comp \tilde \omega_i &= [(\phi_{L, i})_{b, v}^u \tau^b_a \,(\mu_i)^v] e_u \alg a_i^a,
    \end{eqnsplit*}
    \begin{multline*}
         \mathcal S_{matter} = (a^* \phi(\nabla)\mu, a^* \phi(\nabla)\mu) 
        + ((\phi_L(\tau) \mu)\comp \tilde \omega, (\phi_L(\tau) \mu)\comp \tilde \omega)\\
        = [g^{\nu_1 \nu_2} h^E_{u_1 u_2}  (a^* \phi(\nabla)\mu_i)_{\nu_1}^{u_1}(a^* \phi(\nabla)\mu_i)_{\nu_2}^{u_2}\\
           + h^{a_1 a_2} h^E_{u_1 u_2} ((\phi_L(\tau) \mu)\comp \tilde \omega_i)_{a_1}^{u_1} ((\phi_L(\tau) \mu)\comp \tilde \omega_i)_{a_2}^{u_2}] \sqrt{|g_i|}dx^1 \wedge \cdots \wedge dx^m
    \end{multline*}
\end{frame}


\begin{frame}{Example}
    $\hat R_S = i(\partial_{1} \hat \omega^\epsilon_{S;2} - \partial_{2} \hat \omega^\epsilon_{S; 1} + i \partial_1 \tilde \tau \tilde \omega_{S; 2} - i \partial_2 \tilde \tau \tilde \omega_{S; 1}) dx^1 \wedge dx^2 
    - i \partial_{1} \tilde \tau dx^1 \wedge \alg a^1 
    -  i \partial_{2} \tilde \tau dx^2 \wedge \alg a^2.$
    
    \begin{multline}
    \mathcal L_{gauge}[\hat \omega_S, \tilde \tau] = [-4(g_S^{12})^2 c (\partial_{1} \hat \omega^\epsilon_{S;2} - \partial_{2} \hat \omega^\epsilon_{S;1} + i \partial_1 \tilde \tau \tilde \omega_{S; 2} - i \partial_2 \tilde \tau \tilde \omega_{S; 1})^2 \\ + 
        g_S^{11} c^2 (\partial_1 \tilde \tau)^2 + 2g_S^{12} c^2 (\partial_1 \tilde \tau)(\partial_2 \tilde \tau) \\ + g_S^{22} c^2 (\partial_2 \tilde \tau)^2
        +0] \sqrt{|g_S|}dx^1 \wedge dx^2.
\end{multline}

\begin{multline}
    \mathcal L_{matter}[\hat \omega_S, \tilde \tau, \psi_i] =
    [g_S^{\mu \nu} c^E (\partial_\mu \psi_S + h \omega_{S; \mu}  \psi_S)\cdot(\partial_\nu \psi_S + h \omega_{S; \nu}  \psi_S) \\
    \\+ c(i \tilde \tau h \psi_S)^2] \sqrt{|g_i|}dx^1 \wedge  dx^2,
\end{multline}
\end{frame}

%%%%%%%%%%%%%%%%%%%%%%%%%%%%%%%%%%%%%%%%%%%%%%%%%%%%%%%%%%%%%%
%%%%%%%%%%%%%%%%%%%%%%%%%%%%%%%%%%%%%%%%%%%%%%%%%%%%%%%%%%%%%%
\section{Conclusions}

\begin{frame}{Conclusions}
- Two easy generalizations: $\tau$ and other algebroids

- No need for new tools like Spontaneous Symmetry Breaking

- Background connections now becomes relevant, so there are multiple $A$ fields in the equations.
\end{frame}


\begin{frame}{Further Work}
    - $\tau$ fields fixed to reproduce the Glashow-Weinberg-Salam theory for electroweak interactions.
    
    - Gauge group
    
    - Quantization
    
    - Transformation laws under gauge transformations
    
    - Simple Lie algebroids different to Atiyah Lie algebroids
    
\end{frame}

\printbibliography[title=References]

\end{document}
