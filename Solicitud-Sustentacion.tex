\documentclass{article}
\usepackage[centering]{geometry}
\usepackage[spanish]{babel}

\usepackage{amsmath,amssymb,amsthm}
\usepackage{mathtools}
\usepackage{lipsum}

\setlength{\parskip}{0.5em} % PARECE QUE NO ES USUAL DEJAR ESTE ESPACIO

\usepackage{sansmath}
\renewcommand*\familydefault{\sfdefault}

\newcommand{\mD}{\mathcal D}
\newcommand{\mA}{\mathcal A}
\newcommand\define[1]{\emph{#1}}

\usepackage{xcolor}
\newcommand{\commentbox}[1]{\marginpar{\textcolor{blue}{#1}} }
\newcommand{\comment}[1]{{\bf\color{blue} #1}}
\newcommand{\commentr}[1]{{\bf\color{magenta} #1}}
\DeclareMathOperator{\range}{Rango}
\newcommand\inner[2]{\langle #1, #2 \rangle}


\begin{document}

\begin{center}
   \bf\Large Solicitud de jurado y fecha para la sustentaci\'on de\\
   mi proyecto de grado 2020-20
\end{center}
\bigskip
\bigskip
\bigskip

\noindent
{\bf Estudiante:} SEBASTIAN CAMILO PUERTO GALINDO\\
{\bf T\'itulo del trabajo:} GAUGE THEORIES ON TRANSITIVE LIE ALGEBROIDS\\
{\bf Asesor:} ALEXANDER CARDONA GUÍO\\
%{\bf Co-asesor:} NOMBRES APELLIDOS del coasesor si el trabajo cuenta con uno.
%{\color{blue}\bf Quitar esta l\'inea si no tiene co-asesor. Agragar m\'as l\'ineas si ten\'ia m\'as co-asesores.}
\bigskip


\noindent
{\bf Jurado propuesto:} ANDRÉS FERNANDO REYES LEGA, DEPTO. DE FÍSICA, U. DE LOS ANDES\\
{\bf Fecha y hora propuesta:} 18 DICIEMBRE 2020, 10:00A.M.
\bigskip

\noindent
% {\bf Requerimientos especiales:} VIDEO BEAM, CONEXI\'ON A INTERNET, SKYPE, PREFERENCIA DE SAL\'ON, ETC.
% {\bf Requerimientos especiales:} 
% \bigskip


\smallskip

\noindent
{\bf Resumen del trabajo:}
%\smallskip


\noindent
%{\color{blue}\bf 
%Resumen de su trabajo de grado (1-3 p\'aginas).
%NO puede ser una copia de su propuesta.
%}

La dinámica de las partículas elementales y sus interacciones son modeladas por teorías gauge cuantizadas, un tipo de teorías de campos. Éstas incluyen a la Electrodinámica Cuántica, a la Interacción Electrodébil, a la Cromodinámica Cuántica y el Modelo Estándar de la Física de Partículas, la cual describe a tres de las cuatro fuerzas fundamentales conocidas y clasifica a todas las partículas elementales conocidas. El modelo matemático usualmente utilizado para la formulación de las teorías gauge, previo a su cuantización, es el de la geometría diferencial de haces principales y haces vectoriales. En este proyecto se estudia una posible generalización de la formulación de teorías gauge en el lenguaje de algebroides de Lie transitivos a través de funcionales de acción, de la cual la formulación estándar se deriva como el caso particular en que el algebroide subyacente es el algebroide de Lie de Atiyah asociado a un haz principal. La generalización presentada introduce naturalmente nuevos campos $\tau$ que inducen en el lagrangiano nuevos términos de acople de estos campos $\tau$ tanto con los campos gauge, como con los campos de materia, sin tener que recurrir a mecanismos externos a la teoría como de rompiento de simetría para el surgimiento de términos de masa de los campos.

En el primer capítulo se repasan los fundamentos sobre álgebroides de Lie
%sobre una variedad M
, haces vectoriales $A$ con un campo de corchetes de Lie $[\,,\,]$ y un morfismo de haces $a:A \to TM$ llamado ancla. Se hace especial énfasis en los algebroides de Lie transitivos, aquellos cuya ancla es sobreyectiva fibra a fibra, y que por lo tanto son suma directa del haz tangente del espacio base con un haz de álgebras de Lie, llamado un algebroide de Lie adjunto. El ejemplo que motiva el estudio de estos algebroides es el algebroide de Lie de Atiyah $TP/G$ asociado a un haz un principal $P$ con grupo de estructura $G$, el cual tiene como algebroide adjunto a $P \times \mathfrak g/G$ donde $\mathfrak g = \text{Lie}(G)$; las conexiones de un haz principal están en correspondencia biyectiva con los morfismos de haz vectorial entre $TP/G$ y $P \times \mathfrak g/G$. El algebroide de Lie transitivo de derivaciones $\mathfrak D(E)$ de un haz vectorial $E$ permite la expresión del concepto de conexión en $E$, y de la generalización de la noción de haz vectorial asociado a un haz principal a través del concepto de representación $\phi: A \to \mathfrak D(E)$ de un algebroide de Lie $A$ en $E$. Una propiedad de los algebroides de Lie transitivos que facilita su manipulación, especialmente para quienes están familiarizados con el punto de vista de las teorías gauge utilizado en la literatura de la física, es el hecho de que localmente son isomorfos a los algebroides de Lie triviales, de la forma $TM\oplus (M \times \mathfrak g)$ con $M$ una variedad, lo cual permite la descripción de los algebroides de Lie transitivos a partir de descripciones locales en abiertos $U_i$ y dos funciones de pegado en cada intersección no vacía de abiertos $U_{ij}$; para $TP/G$, con $G$ un grupo de Lie matricial, estas dos funciones son las familiares $\alpha^i_j = g^{-1}_{ij}\cdot \, \cdot g_{ij}$ y $\chi^i_j = g_{ij}^{-1} dg_{ij}$, con $g_{ij}$ la función de transición de $P$.

En el capítulo $2$ se definen espacios de formas diferenciales sobre los algebroides de Lie, dando las bases para la definición en los siguientes capítulos de las formas de conexión e integración en algebroides de Lie transitivos. Dada una representación $\phi: A \to \mathfrak D(E)$ de un algebroide de Lie $A$ sobre un haz vectorial $E$, una forma diferencial de grado $n \in \mathbb Z_{\geq 0}$, elemento de $\Omega^\bullet(A, E)$, es un morfismo de haces multilineal alternante que toma valores en $A$ y retorna valores en $E$. El álgebra diferencial graduada $(\Omega^\bullet(A), \wedge, \hat d_A)$ de formas diferenciales con valores escalares, que incluirá a la forma de volumen de ciertos algebroides, permitirá definir a $(\Omega^\bullet(A, E), \hat d_\phi)$ como un complejo diferencial sobre el anillo graduado $(\Omega^\bullet(A), \wedge)$. Cuando $E$ es además un haz de álgebras, el producto $\bullet$ en las fibras de $E$ induce un producto $\wedge^\bullet$ entre las formas diferenciales con valores en $E$, haciendo de $(\Omega^\bullet(A, E), \wedge^\bullet, \hat d_\phi)$ un álgebra diferencial graduada, y en particular en álgebra de Lie diferencial graduada cuando $E$ es un haz de álgebras de Lie, como lo son el haz $\text{End}(E)$ o los haces adjuntos de un algebroide de Lie transitivo. Cuando $A$ es transitivo, la trivialización local del algebroide da lugar a la localización de las formas diferenciales, donde esta trivialización es una operación que respeta tanto a los diferenciales como a los productos cuña. En cada trivialización local sobre $U$ abierto de la variedad base que también trivialice a $E$, el diferencial toma la forma $\hat d_\phi = d + s'$, donde $d$ es el diferencial usual de $TU$ y $s'$ es un diferencial asociado al diferencial de Chevalley-Eilenbert en el álgebra exterior del álgebra de Lie, fibra del haz adjunto a $A$.

Tanto las conexiones de un haz principal $P$, como las conexiones de un haz vectorial $E$ pueden ser vistas como secciones del ancla de los algebroides de Lie transitivos asociados, también llamadas conexiones ordinarias. En el capítulo $3$ se definen dos nociones de conexión dado un algebroide de Lie transitivo $A$, cada una generalizando una de los dos tipos de conexión previamente mencionadas. Generalizando a las conexiones principales, el conjunto de formas de conexión generalizadas en $A$ es $\Omega^1(A, L)$, donde $L$ es cualquier algebroide de Lie adjunto a $A$; cada una tiene asociada un endomorfismo $\nabla: A \to A$ de haces, que será una proyección horizontal si la conexión es ordinaria, y la $2$-forma de curvatura expresa la falla de $\nabla$ de respetar el corchete. Generalizando a las conexiones, o derivadas covariantes, de haces vectoriales se definen las $A$-conexiones de $E$ como morfismos de haces vectoriales $\hat \nabla^E: A \to \mathfrak D(E)$ que respetan el ancla, con su curvatura definida como la falla de $\hat \nabla^E$ de respetar el corchete. Así como una conexión de un haz principal define una derivada covariante en los haces vectoriales asociados, sobre los haces vectoriales $E$ de representación de un algebroide de Lie transitivo $A$ tenemos el concepto, central en este proyecto, de la $A$-conexión en $E$ producida por una conexión generalizada $\hat \omega$ en $A$; si la representación es $\phi: A \to \mathfrak D(E)$, la $A$-conexión producida tiene la descomposición $\hat \nabla^E = \phi + \hat \omega^E$ con $\hat \omega^E = \phi \circ \hat \omega \in \Omega^1(A, \text{End}(E))$. En una vecindad que trivialice a $A$, la forma de conexión tiene la trivialización $\hat \omega_{loc} = A \oplus (-\mathbb 1 + \tau)$ con $A \in \Omega^1(U)\otimes \mathfrak g$ y $\tau \in C^\infty(U, \text{End}(\mathfrak g))$, donde $\tau = 0$ si y solo si la conexión es ordinaria. Si $U$ además trivializa a $E$ con fibra $V$ y hay coordenadas ${x^\mu}$, una $A$-conexión producida por $\hat \omega$ se descompone en dos tipos de derivaciones de $C^\infty(U, V)$: tangentes $\hat \nabla^{E, loc}_{\partial_\mu} = \partial_\mu + \alpha_\mu + A^b_\mu \phi^{loc}(E_b)$ y verticales $\hat \nabla^{E, loc}_{E_a} = - \tau^b_a \phi^{loc} (E_b)$, donde $E_a, E_b$ son elementos de una base de $\mathfrak g$ y $\alpha$ es una $1$-forma de Maurer-Cartan que proviene de la trivialización de la representación $\phi$; en particular, los grados de libertad adicionales en que se pueden tomar ``derivadas covariantes'' se convierten en un acople de las secciones de $E$ con campos $\tau^b_a$ que no están presentes en el formalismo usual de las teorías gauge.

Los últimos ingredientes para construir el funcional de acción de una teoría gauge son métricas en los haces involucrados e integrales sobre algebroides de Lie transitivos. Una métrica $\hat g$ en el haz vectorial $A$ es equivalente a una tripla $(g, h, \nabla)$, donde $g$ es una métrica en la variedad base, $h$ es una métrica en un algebroide de Lie adjunto $L$ y $\nabla$ es una conexión ordinaria que determina el isomorfismo de haces $A \cong TM \oplus L$, con $1$-forma asociada $\mathfrak a$. Si $L$ es orientable, y suponiendo que $\{E_a\}_{a = 1, \dots, n}$ es una base de la fibra típica $\mathfrak g$ de $L$, en una vecindad $U_i$ que trivialice a $A$, $\mathfrak a$ se descompone como $\mathfrak a_i^a E_a$ y la forma $(-1)^{n}\sqrt{|h_i|} \mathfrak a_i^1 \wedge \cdots \mathfrak a_i^{n}$ se transforma correctamente entre vecindades, definiendo una forma global $\omega_{h,\mathfrak a} \in \Omega^{n}(A)$ llamada la forma de volumen interna, la cual permite definir la integración interna $\int_{inner}$ de formas arbitrarias en $A$ como la operación que extrae el factor que acompaña a $\omega_{h,\mathfrak a}$. Si $A$ es orientable, la integración sobre $A$ de formas con valores escalares $\int_A$ se define como la composición de la integración interna con la integración en la variedad base; asociada a esta integración está la forma de volumen en $A$ $\omega^{Vol}$, con trivialización local sobre $U_i$ igual a $(-1)^{n}\sqrt{|h_i| |g_i|} dx^1\wedge\cdots \wedge dx^m \wedge \mathfrak a_i^1 \wedge \cdots \mathfrak a_i^{n}$, si $\{x^\mu\}_{\mu = 1, \dots, m}$ son coordenadas en $U_i$. Dada una métrica $h^E$ en un haz vectorial $E$ de representación de $A$, se induce el producto de formas $h^E: \Omega^p(A, E)\otimes \Omega^q(A, E) \to \Omega^{p+q}(A)$ y la métrica inversa $\hat g^{-1}_{h^E}: \Omega^p(A, E)\otimes \Omega^p(A, E) \to C^\infty(M, \mathbb R)$ para todo $p, q \in \mathbb Z_{\geq 0}$, a partir de las cuales se define el operador Hodge-$*$ de una forma $E$-valuada $\beta$ como aquel para el que se satisface $h_E(\alpha, *\beta) = \hat g_{h^E}^{-1}(\alpha, \beta) \omega^{Vol}$ para toda forma $\alpha$. El operador Hodge-$*$ permite la definición de un producto interno entre formas homogéneas $\alpha, \beta \in \Omega^p(A, E)$ como $(\alpha, \beta) = \int_A h^E(\alpha, *\beta)$, del cual el funcional de acción es ejemplo.

Finalmente, una teoría gauge se define en el capítulo $5$ a partir de un algebroide de Lie transitivo $A$ orientable con métrica $\hat g \equiv (g, h, \tilde \nabla)$ y un haz vectorial $E$ de representación de $A$ con métrica $h^E$, a través del funcional de acción $\mathcal S = \mathcal S_{gauge} + \mathcal S_{matter}$ de la teoría. El funcional gauge $\mathcal S_{gauge}$ se aplica a una conexión generalizada en $A$ $\hat \omega$ con curvatura $\hat R$, y se define como $S_\text{gauge}[\hat \omega] := (\hat R, \hat R)$. La segunda parte del funcional de acción, el funcional de materia $S_{matter}[\hat \omega, \mu] = (\hat \nabla^E \mu, \hat \nabla^E \mu)$ se aplica a campos de materia $\mu$, i.e. secciones del haz vectorial de representación, donde $\hat \nabla^E$ es la $A$-conexión en $E$ producida por $\hat \omega$. Cada una de estos funcionales es una integral sobre $A$, y truncando esta integral a nivel de la integral interna se obtienen las densidades lagrangianas $\mathcal L_{gauge}$ y $\mathcal L_{matter}$ correspondientes. El funcional de acción de una teoría gauge es invariante ante transformaciones gauge infinitesimales, siempre y cuando las métricas $h$ y $h^E$ satisfagan una condición de compatibilidad con la representación correspondiente, donde se entiende a $\Gamma(L)$ como el álgebra de transformaciones gauge infinitesimales de la teoría, con $L$ un algebroide adjunto a $A$. Un elemento $\eta \in \Gamma(L)$ actúa sobre la conexión generando $\hat \omega^\eta = \hat \omega + \hat d\eta + \hat \omega \wedge \eta$, y sobre la $A$-conexión generando $\hat \nabla^{E, \eta} = \hat \nabla^E + [\hat \nabla^E, \eta]$. Finalmente, el funcional de acción tiene una descomposición de la forma $\mathcal S[\hat \omega, \mu] = (a^* \hat F, a^* \hat F) + ((a^* \mathcal D \tau) \circ \tilde \omega, (a^* \mathcal D \tau) \circ \tilde \omega) + (R_\tau \circ \tilde \omega, R_\tau \circ \tilde \omega) + (a^* \phi(\nabla)\cdot \mu, a^* \phi(\nabla)\cdot \mu) + ((\phi_L(\tau)\mu)\circ \tilde \omega, (\phi_L(\tau)\mu)\circ \tilde \omega)$; el segundo, tercer y quinto término desaparecen cuando $\hat \omega$ es una conexión ordinaria, i.e. cuando su $\tau$ asociado es $0$, dejando términos análogos a los de una teoría gauge usual; a partir del acople con $\tau$, el segundo término será un término de masa para los campos gauge, el cuarto término lo será para los campos de materia, y el tercer término describe la dinámica de los campos $\tau$.



\vspace{\fill}


\parbox{.4\textwidth}{%
\underline{\hspace{7cm}}\\
\raggedright SEBASTIAN CAMILO PUERTO GALINDO
}
\bigskip
\bigskip
\bigskip


\parbox{.4\textwidth}{%
\underline{\hspace{7cm}}\\
\raggedright ALEXANDER CARDONA GUÍO (asesor)
}
\hspace*{\fill}
\parbox{.4\textwidth}{%
\underline{\hspace{7cm}}\\
\raggedright ANDRÉS FERNANDO REYES LEGA (jurado)
}

\end{document}