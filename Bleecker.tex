\documentclass[12pt]{report}

\usepackage[utf8]{inputenc}
\usepackage[textwidth = 16cm]{geometry}

\usepackage{amsmath, amsthm, amssymb}

% My Codes
\usepackage[dvipsnames]{xcolor}
%\usepackage[dvipsnames]{xcolor}

%
% \newcommand{\ytext}[1]{\textcolor{yellow}{#1}}
% \newcommand{\otext}[1]{\textcolor{orange}{#1}}
% \newcommand{\rtext}[1]{\textcolor{red}{#1}}
% \newcommand{\lbtext}[1]{\textcolor{cyan}{#1}}
% \newcommand{\dbtext}[1]{\textcolor{blue}{#1}}
% \newcommand{\ptext}[1]{\textcolor{Plum}{#1}}
% \newcommand{\lgtext}[1]{\textcolor{LimeGreen}{#1}}
% \newcommand{\dgtext}[1]{\textcolor{OliveGreen}{#1}}

\newcommand{\ytext}[1]{\textcolor{black}{#1}}
\newcommand{\otext}[1]{\textcolor{black}{#1}}
\newcommand{\rtext}[1]{{\it #1}}
\newcommand{\lbtext}[1]{\textcolor{black}{#1}}
\newcommand{\dbtext}[1]{\textcolor{black}{#1}}
\newcommand{\ptext}[1]{\textcolor{black}{#1}}
\newcommand{\lgtext}[1]{\textcolor{black}{#1}}
\newcommand{\dgtext}[1]{\textcolor{black}{#1}}

% \newcommand{\ybox}[1]{\colorbox{yellow}{#1}}
% \newcommand{\obox}[1]{\colorbox{orange}{#1}}
% \newcommand{\rbox}[1]{\colorbox{Salmon}{#1}}
% \newcommand{\lbbox}[1]{\colorbox{SkyBlue}{#1}}
% \newcommand{\dbbox}[1]{\colorbox{NavyBlue}{#1}}
% \newcommand{\pbox}[1]{\colorbox{Plum}{#1}}
% \newcommand{\lgbox}[1]{\colorbox{LimeGreen}{#1}}
% \newcommand{\dgbox}[1]{\colorbox{OliveGreen}{#1}}


\title{Resumen Bleecker}
\date{February 2020}
\author{Sebastian Camilo Puerto}

\begin{document}
\maketitle

%%%%%%%%%%%%%%%%%%%%%%%%%%%%%%%%%%%%%%%%%%%%%%%%%%%%%%%%%%%%%%%%%%%%%%%%%%%%%%
%%%%%%%%%%%%%%%%%%%%%%%%%%%%%%%%%%%%%%%%%%%%%%%%%%%%%%%%%%%%%%%%%%%%%%%%%%%%%%
%%%%%%%%%%%%%%%%%%%%%%%%%%%%%%%%%%%%%%%%%%%%%%%%%%%%%%%%%%%%%%%%%%%%%%%%%%%%%%
\chapter{Prologe}
%%%%%%%%%%%%%%%%%%%%%%%%%%%%%%%%%%%%%%%%%%%%%%%%%%%%%%%%%%%%%%%%%%%%%%%%%%%%%%
\section{High Level Summary}

%%%%%%%%%%%%%%%%%%%%%%%%%%%%%%%%%%%%%%%%%%%%%%%%%%%%%%%%%%%%%%%%%%%%%%%%%%%%%%
\section{Very Important Facts}

\begin{itemize}
    \item If we have a ``particle field''/wave function $\psi:M \to V$, where $V$ is vector field, \emph{there is an implicit choice of reference frame} (perhaps for zero-phase angle, or for axes of isospin)!!
    
    \item Thus, what we really have is a ``frame'' bundle $P$ and $\psi:P \to V$, where \rtext{$\psi(p)$ means the measurement of $V$ \emph{with respect to $p$}} (not talking simply about the frame bundle).

    \item \lbbox{A gauge}: \emph{a continuous choice of reference frame}: a local section of the principal bundle. 
    
    \item \rtext{Given a gauge, we may thus say, locally, $\psi(m)$, for $M$ in $P$}
    
    \item \rtext{Connections/Gauge potentials are needed to construct truly gauge invariant action densities from a wave function}, which are is required to have gauge fields (like the vector potential field).
    
    \item \rtext{Principal bundles and connections also allow to geometrize forces: JUST LIKE GRAVITY geometrizes gravity} by adding a ``'time'' dimension, the additional dimmensions of the principal bundle can be seen as the \rtext{``charge'' dimensions}. \rtext{A geodesic in the principal bundle} \small{spacetime metric in $M$ \otext{plus connection} induce metric in $P$} \rtext{projects to a 3d trajectory of a charged particle under the corresponding force!!}
    
    \item \rtext{GRAVITY MAY BE SEEN AS AN EXTERNAL GAUGE THEORY where the gauge symmetry is the change of the points in $M$ themselves: diffeomorphisms}. This is in contrast to the usual/internal gauge theorie, where the spacetime points are left intact and the variation occurs in the internal degrees of freedom.
    
    \item The action density is the scalar curvature. \rtext{Einstein Field equations are obtained by imposing the action density be invariant under changes of metric in $M$!!}, similarly \rtext{Yang-Mills equations arise from invariance under variation of gauge potential (connection)!!}
\end{itemize}

%%%%%%%%%%%%%%%%%%%%%%%%%%%%%%%%%%%%%%%%%%%%%%%%%%%%%%%%%%%%%%%%%%%%%%%%%%%%%%
%%%%%%%%%%%%%%%%%%%%%%%%%%%%%%%%%%%%%%%%%%%%%%%%%%%%%%%%%%%%%%%%%%%%%%%%%%%%%%
%%%%%%%%%%%%%%%%%%%%%%%%%%%%%%%%%%%%%%%%%%%%%%%%%%%%%%%%%%%%%%%%%%%%%%%%%%%%%%
\chapter{Principal Fiber Bundles and Connections}

%%%%%%%%%%%%%%%%%%%%%%%%%%%%%%%%%%%%%%%%%%%%%%%%%%%%%%%%%%%%%%%%%%%%%%%%%%%%%%
\section{Very Important Facts}

\begin{itemize}
    \item Geometrize forces by adding additional dimensions: principal bundles.
    
    \item \lbbox{Gauge potential/Gauge field}: local $1$-form
    
    \item Because $U(1)$ is abelian, the \lbtext{electromagnetic field strength relative to each gauge $\sigma:U \to P$} $F = \vec E \cdot d\vec r \wedge dt + \vec B \cdot d\vec \sigma$ has the same components for each gauge.
\end{itemize}
%%%%%%%%%%%%%%%%%%%%%%%%%%%%%%%%%%%%%%%%%%%%%%%%%%%%%%%%%%%%%%%%%%%%%%%%%%%%%%
\section{Medium Importance Facts}

%%%%%%%%%%%%%%%%%%%%%%%%%%%%%%%%%%%%%%%%%%%%%%%%%%%%%%%%%%%%%%%%%%%%%%%%%%%%%%
\section{More Detailed Summary}

%%%%%%%%%%%%%%%%%%%%%%%%%%%%%%%%%%%%%%%%%%%%%%%%%%%%%%%%%%%%%%%%%%%%%%%%%%%%%%
\section{Commentaries}

%%%%%%%%%%%%%%%%%%%%%%%%%%%%%%%%%%%%%%%%%%%%%%%%%%%%%%%%%%%%%%%%%%%%%%%%%%%%%%
\section{Doubts}

\begin{itemize}
    \item Why is the gauge potential a $1$-form? Just because locally they ``act'' by multiplication of the field?
\end{itemize}

%%%%%%%%%%%%%%%%%%%%%%%%%%%%%%%%%%%%%%%%%%%%%%%%%%%%%%%%%%%%%%%%%%%%%%%%%%%%%%
%%%%%%%%%%%%%%%%%%%%%%%%%%%%%%%%%%%%%%%%%%%%%%%%%%%%%%%%%%%%%%%%%%%%%%%%%%%%%%
%%%%%%%%%%%%%%%%%%%%%%%%%%%%%%%%%%%%%%%%%%%%%%%%%%%%%%%%%%%%%%%%%%%%%%%%%%%%%%
\chapter{Curvature}
%%%%%%%%%%%%%%%%%%%%%%%%%%%%%%%%%%%%%%%%%%%%%%%%%%%%%%%%%%%%%%%%%%%%%%%%%%%%%%
\section{High Level Summary}

%%%%%%%%%%%%%%%%%%%%%%%%%%%%%%%%%%%%%%%%%%%%%%%%%%%%%%%%%%%%%%%%%%%%%%%%%%%%%%
\section{Very Important Facts}

\begin{itemize}
    \item Locally, \lbbox{the field strength associated to $\omega_u$} is the local curvature (the pullback through the local section).
\end{itemize}

%%%%%%%%%%%%%%%%%%%%%%%%%%%%%%%%%%%%%%%%%%%%%%%%%%%%%%%%%%%%%%%%%%%%%%%%%%%%%%
\section{Medium Importance Facts}
\begin{itemize}
    \item $\Omega_v = g_{uv}^{-1} \Omega_u g_{uv}$
\end{itemize}
%%%%%%%%%%%%%%%%%%%%%%%%%%%%%%%%%%%%%%%%%%%%%%%%%%%%%%%%%%%%%%%%%%%%%%%%%%%%%%
\section{More Detailed Summary}

%%%%%%%%%%%%%%%%%%%%%%%%%%%%%%%%%%%%%%%%%%%%%%%%%%%%%%%%%%%%%%%%%%%%%%%%%%%%%%
\section{Commentaries}

%%%%%%%%%%%%%%%%%%%%%%%%%%%%%%%%%%%%%%%%%%%%%%%%%%%%%%%%%%%%%%%%%%%%%%%%%%%%%%
\section{Doubts}


%%%%%%%%%%%%%%%%%%%%%%%%%%%%%%%%%%%%%%%%%%%%%%%%%%%%%%%%%%%%%%%%%%%%%%%%%%%%%%
%%%%%%%%%%%%%%%%%%%%%%%%%%%%%%%%%%%%%%%%%%%%%%%%%%%%%%%%%%%%%%%%%%%%%%%%%%%%%%
\chapter{3}
%%%%%%%%%%%%%%%%%%%%%%%%%%%%%%%%%%%%%%%%%%%%%%%%%%%%%%%%%%%%%%%%%%%%%%%%%%%%%%
\section{High Level Summary}

%%%%%%%%%%%%%%%%%%%%%%%%%%%%%%%%%%%%%%%%%%%%%%%%%%%%%%%%%%%%%%%%%%%%%%%%%%%%%%
\section{Very Important Facts}

%%%%%%%%%%%%%%%%%%%%%%%%%%%%%%%%%%%%%%%%%%%%%%%%%%%%%%%%%%%%%%%%%%%%%%%%%%%%%%
\section{Medium Importance Facts}

%%%%%%%%%%%%%%%%%%%%%%%%%%%%%%%%%%%%%%%%%%%%%%%%%%%%%%%%%%%%%%%%%%%%%%%%%%%%%%
\section{More Detailed Summary}

%%%%%%%%%%%%%%%%%%%%%%%%%%%%%%%%%%%%%%%%%%%%%%%%%%%%%%%%%%%%%%%%%%%%%%%%%%%%%%
\section{Commentaries}

%%%%%%%%%%%%%%%%%%%%%%%%%%%%%%%%%%%%%%%%%%%%%%%%%%%%%%%%%%%%%%%%%%%%%%%%%%%%%%
\section{Doubts}

\end{document}