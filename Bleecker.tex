\documentclass[12pt]{report}

\usepackage[utf8]{inputenc}
\usepackage[marginparwidth = 2cm]{geometry}

\usepackage{amsmath, amsthm, amssymb}

% My Codes
\usepackage[pdftex,dvipsnames]{xcolor}
%\usepackage[dvipsnames]{xcolor}

%
% \newcommand{\ytext}[1]{\textcolor{yellow}{#1}}
% \newcommand{\otext}[1]{\textcolor{orange}{#1}}
% \newcommand{\rtext}[1]{\textcolor{red}{#1}}
% \newcommand{\lbtext}[1]{\textcolor{cyan}{#1}}
% \newcommand{\dbtext}[1]{\textcolor{blue}{#1}}
% \newcommand{\ptext}[1]{\textcolor{Plum}{#1}}
% \newcommand{\lgtext}[1]{\textcolor{LimeGreen}{#1}}
% \newcommand{\dgtext}[1]{\textcolor{OliveGreen}{#1}}

\newcommand{\ytext}[1]{\textcolor{black}{#1}}
\newcommand{\otext}[1]{\textcolor{black}{#1}}
\newcommand{\rtext}[1]{\textcolor{black}{#1}}
\newcommand{\lbtext}[1]{\textcolor{black}{#1}}
\newcommand{\dbtext}[1]{\textcolor{black}{#1}}
\newcommand{\ptext}[1]{\textcolor{black}{#1}}
\newcommand{\lgtext}[1]{\textcolor{black}{#1}}
\newcommand{\dgtext}[1]{\textcolor{black}{#1}}

% \newcommand{\ybox}[1]{\colorbox{yellow}{#1}}
% \newcommand{\obox}[1]{\colorbox{orange}{#1}}
% \newcommand{\rbox}[1]{\colorbox{Salmon}{#1}}
% \newcommand{\lbbox}[1]{\colorbox{SkyBlue}{#1}}
% \newcommand{\dbbox}[1]{\colorbox{NavyBlue}{#1}}
% \newcommand{\pbox}[1]{\colorbox{Plum}{#1}}
% \newcommand{\lgbox}[1]{\colorbox{LimeGreen}{#1}}
% \newcommand{\dgbox}[1]{\colorbox{OliveGreen}{#1}}

\usepackage{xargs}                      % Use more than one optional parameter in a new
\input{tools/todoCode}
%

\title{Resumen Bleecker}
\date{February 2020}
\author{Sebastian Camilo Puerto}

\begin{document}
\maketitle

\tableofcontents

%%%%%%%%%%%%%%%%%%%%%%%%%%%%%%%%%%%%%%%%%%%%%%%%%%%%%%%%%%%%%%%%%%%%%%%%%%%%%%
%%%%%%%%%%%%%%%%%%%%%%%%%%%%%%%%%%%%%%%%%%%%%%%%%%%%%%%%%%%%%%%%%%%%%%%%%%%%%%
%%%%%%%%%%%%%%%%%%%%%%%%%%%%%%%%%%%%%%%%%%%%%%%%%%%%%%%%%%%%%%%%%%%%%%%%%%%%%%
\chapter*{Prologe}
%%%%%%%%%%%%%%%%%%%%%%%%%%%%%%%%%%%%%%%%%%%%%%%%%%%%%%%%%%%%%%%%%%%%%%%%%%%%%%
\section{High Level Summary}

%%%%%%%%%%%%%%%%%%%%%%%%%%%%%%%%%%%%%%%%%%%%%%%%%%%%%%%%%%%%%%%%%%%%%%%%%%%%%%
\section{Very Important Facts}

\begin{itemize}
    \item If we have a ``particle field''/wave function $\psi:M \to V$, where $V$ is vector field, \emph{there is an implicit choice of reference frame} (perhaps for zero-phase angle, or for axes of isospin)!!
    
    \item Thus, what we really have is a ``frame'' bundle $P$ and $\psi:P \to V$, where \rtext{$\psi(p)$ means the measurement of $V$ \emph{with respect to $p$}} (not talking simply about the frame bundle).

    \item \lbbox{A gauge}: \emph{a continuous choice of reference frame}: a local section of the principal bundle. 
    
    \item \rtext{Given a gauge, we may thus say, locally, $\psi(m)$, for $M$ in $P$}
    
    \item \rtext{Connections/Gauge potentials are needed to construct truly gauge invariant action densities from a wave function}, which are is required to have gauge fields (like the vector potential field).
    
    \item \rtext{Principal bundles and connections also allow to geometrize forces: JUST LIKE GRAVITY geometrizes gravity} by adding a ``'time'' dimension, the additional dimmensions of the principal bundle can be seen as the \rtext{``charge'' dimensions}. \rtext{A geodesic in the principal bundle} \small{spacetime metric in $M$ \otext{plus connection} induce metric in $P$} \rtext{projects to a 3d trajectory of a charged particle under the corresponding force!!}
    
    \item \rtext{GRAVITY MAY BE SEEN AS AN EXTERNAL GAUGE THEORY where the gauge symmetry is the change of the points in $M$ themselves: diffeomorphisms}. This is in contrast to the usual/internal gauge theorie, where the spacetime points are left intact and the variation occurs in the internal degrees of freedom.
    
    \item The action density is the scalar curvature. \rtext{Einstein Field equations are obtained by imposing the action density be invariant under changes of metric in $M$!!}, similarly \rtext{Yang-Mills equations arise from invariance under variation of gauge potential (connection)!!}
\end{itemize}

%%%%%%%%%%%%%%%%%%%%%%%%%%%%%%%%%%%%%%%%%%%%%%%%%%%%%%%%%%%%%%%%%%%%%%%%%%%%%%
%%%%%%%%%%%%%%%%%%%%%%%%%%%%%%%%%%%%%%%%%%%%%%%%%%%%%%%%%%%%%%%%%%%%%%%%%%%%%%
%%%%%%%%%%%%%%%%%%%%%%%%%%%%%%%%%%%%%%%%%%%%%%%%%%%%%%%%%%%%%%%%%%%%%%%%%%%%%%
\chapter{Principal Fiber Bundles and Connections}

%%%%%%%%%%%%%%%%%%%%%%%%%%%%%%%%%%%%%%%%%%%%%%%%%%%%%%%%%%%%%%%%%%%%%%%%%%%%%%
\section{Very Important Facts}

\begin{itemize}
    \item Geometrize forces by adding additional dimensions: principal bundles.
    
    \item \lbbox{Gauge potential/Gauge field}: local $1$-form
    
    \item Because $U(1)$ is abelian, the \lbtext{electromagnetic field strength relative to each gauge $\sigma:U \to P$} $F = \vec E \cdot d\vec r \wedge dt + \vec B \cdot d\vec \sigma$ has the same components for each gauge.
\end{itemize}
%%%%%%%%%%%%%%%%%%%%%%%%%%%%%%%%%%%%%%%%%%%%%%%%%%%%%%%%%%%%%%%%%%%%%%%%%%%%%%
\section{Medium Importance Facts}

%%%%%%%%%%%%%%%%%%%%%%%%%%%%%%%%%%%%%%%%%%%%%%%%%%%%%%%%%%%%%%%%%%%%%%%%%%%%%%
\section{More Detailed Summary}

%%%%%%%%%%%%%%%%%%%%%%%%%%%%%%%%%%%%%%%%%%%%%%%%%%%%%%%%%%%%%%%%%%%%%%%%%%%%%%
\section{Commentaries}

%%%%%%%%%%%%%%%%%%%%%%%%%%%%%%%%%%%%%%%%%%%%%%%%%%%%%%%%%%%%%%%%%%%%%%%%%%%%%%
\section{Doubts}

\begin{itemize}
    \item Why is the gauge potential a $1$-form? Just because locally they ``act'' by multiplication of the field?
\end{itemize}

%%%%%%%%%%%%%%%%%%%%%%%%%%%%%%%%%%%%%%%%%%%%%%%%%%%%%%%%%%%%%%%%%%%%%%%%%%%%%%
%%%%%%%%%%%%%%%%%%%%%%%%%%%%%%%%%%%%%%%%%%%%%%%%%%%%%%%%%%%%%%%%%%%%%%%%%%%%%%
%%%%%%%%%%%%%%%%%%%%%%%%%%%%%%%%%%%%%%%%%%%%%%%%%%%%%%%%%%%%%%%%%%%%%%%%%%%%%%
\chapter{Curvature}
%%%%%%%%%%%%%%%%%%%%%%%%%%%%%%%%%%%%%%%%%%%%%%%%%%%%%%%%%%%%%%%%%%%%%%%%%%%%%%
\section{High Level Summary}

%%%%%%%%%%%%%%%%%%%%%%%%%%%%%%%%%%%%%%%%%%%%%%%%%%%%%%%%%%%%%%%%%%%%%%%%%%%%%%
\section{Very Important Facts}

\begin{itemize}
    \item Locally, \lbbox{the field strength associated to $\omega_u$} is the local curvature (the pullback through the local section).
    
    \item The curvature is the covariant derivative of the connection $1$-form with respect to itself.
    
    \item In EM, the structure group is the abelian $U(1)$, the $field strength$ is the same for every local trivialization as $F_V = \Omega_V = g^{-1}_{UV}\Omega_U g_{UV} = \Omega_u = F_U$ 
\end{itemize}

%%%%%%%%%%%%%%%%%%%%%%%%%%%%%%%%%%%%%%%%%%%%%%%%%%%%%%%%%%%%%%%%%%%%%%%%%%%%%%
\section{Medium Importance Facts}
\begin{itemize}
    \item $\Omega_v = g_{uv}^{-1} \Omega_u g_{uv}$
\end{itemize}
%%%%%%%%%%%%%%%%%%%%%%%%%%%%%%%%%%%%%%%%%%%%%%%%%%%%%%%%%%%%%%%%%%%%%%%%%%%%%%
\section{More Detailed Summary}

\begin{itemize}
    \item The Algebra of $\mathfrak g-$valued differential forms on $N$ is a gradded Lie algebra ($\Lambda(N, \mathfrak g)$).
    
    \item Given a connection 1-form $w$, the covariant derivative of $g$ valued $k$-forms, $D^\omega: \Lambda^k(P, \mathfrak g) \to \Lambda^{k+1}(P, \mathfrak g)$ is defined by $D^\omega \phi = (d\phi)^H$
    
    \item The curvature $\Omega^\omega$ of a connection $\omega$ is its covariant derivative with respect to itself.
    
    \item Structural equation $\Omega^\omega = D^\omega \omega = d\omega + \frac{1}{2}[\omega, \omega]$. If $G$ is a matrix Lie group, it may also be written as $\Omega\omega = d\omega + \omega \wedge \omega$ (and the wedge is simply matrix multiplication where the pointwise multiplication is the wedge).
    
    \item Bianchi Identity/Homogeneous Field equation: $D^\omega \Omega = 0$.
    
    \item Locally the curvature, called the \lbbox{field strength} associated to the gauge potential $\omega_u$ / \lbbox{local field strength} is the pullback by the local section. It can be written by the Structural equation (or its alternative if matrix group) applied to the local gauge potentials $\omega_U$
    
    \item The curvature transforms locally by $\Omega_V = ad_{g_{UV}^-1} \Omega_U$, if $G$ is matrix group: $\Omega_V = g^{-1}_{UV}\Omega_U g_{UV}$
    
\end{itemize}


%%%%%%%%%%%%%%%%%%%%%%%%%%%%%%%%%%%%%%%%%%%%%%%%%%%%%%%%%%%%%%%%%%%%%%%%%%%%%%
%%%%%%%%%%%%%%%%%%%%%%%%%%%%%%%%%%%%%%%%%%%%%%%%%%%%%%%%%%%%%%%%%%%%%%%%%%%%%%
%%%%%%%%%%%%%%%%%%%%%%%%%%%%%%%%%%%%%%%%%%%%%%%%%%%%%%%%%%%%%%%%%%%%%%%%%%%%%%
\chapter{Particle Fields, Lagrangians, Gauge Invariance}
%%%%%%%%%%%%%%%%%%%%%%%%%%%%%%%%%%%%%%%%%%%%%%%%%%%%%%%%%%%%%%%%%%%%%%%%%%%%%%
\section{High Level Summary}

%%%%%%%%%%%%%%%%%%%%%%%%%%%%%%%%%%%%%%%%%%%%%%%%%%%%%%%%%%%%%%%%%%%%%%%%%%%%%%
\section{Very Important Facts}
\begin{itemize}
    \item A particle field can be regarded as a section of a vector bundle associated to some PFB / as a vector-valued function on $P$ equivariant under the action of $G$.
    
    \item Examples include the Schrodinger wave function, the Klein-Gordon field, and the Dirac electron field.
    
    \item The ``correct'' particle field obeys differential equations given by minimum action.
    
    \item An action density $\mathfrak L_0(\psi)$ is obtained from a $G$-invariant Lagrangian $L(\psi, d\psi)$, but for it to be gauge invariant it is necessary to \lbbox{minimally couple} \rtext{the particle fields with the gauge fields}: changing the derivatives by Covariant derivatives. So, a good action density needs a connection and is of the form $\mathfrak L(\psi, \omega)(x) = L(p, \psi(p), D^\omega \psi_p)$
    
    \item The group of gauge transformations $GA(P)$ of a PFG is the subset of automorphisms of $P$ that keep you in the fiber and are equivariant. The action can be seen as a pointwise multiplicative action of $C(P, G) = \Gamma(\frac{P}{G}/G)$ (adjoint action of $G$ on $G$). $GA(P)$ and $GA(P)$ are antisomorphic as groups.
    
    \item $C(P, \mathfrak g) \equiv \frac{P \times \mathfrak g}{G}$ can be regarded as the Lie algebra of the Lie algebra of the gauge group. There is an exponential map that takes you from $C(P, \mathfrak g)$ to $C(P, G) = GA(P)$: \rtext{for $H \in C(P, \mathfrak g)$, $f = exp(tH) \in GA(P)$}. It is called \lbbox{the gauge algebra of a PFB},  or \lbbox{the algebra of infinitesimal gauge transformations}.
    
    \item We can explicitly the action of the gauge group (the pullback) on a connection and on its curvature (basic form). They are still connections and curvatures (basic form).
    
    \item Similarly, the infinitesimal gauge algebra acts on the connection and on the curvature, but this time taking the curvature to its ``tangent plane'': 
    \begin{itemize}
        \item The action of $H \in C(P, \mathfrak g)$ on a connection $\omega$ is $dH + [\omega, H] = D^\omega H$ $\in \bar \Lambda^1(P, \mathfrak g)$ (basic form)
        
        \item The action of $H \in C(P, \mathfrak g)$ on a basic $v$ valued $k-$form $\phi$ is $-H \cdot \phi(\cdot, \cdot) \in \Lambda^k(P, V)$.
    \end{itemize}
\end{itemize}

%%%%%%%%%%%%%%%%%%%%%%%%%%%%%%%%%%%%%%%%%%%%%%%%%%%%%%%%%%%%%%%%%%%%%%%%%%%%%%
\section{Medium Importance Facts}

%%%%%%%%%%%%%%%%%%%%%%%%%%%%%%%%%%%%%%%%%%%%%%%%%%%%%%%%%%%%%%%%%%%%%%%%%%%%%%
\section{More Detailed Summary}

%%%%%%%%%%%%%%%%%%%%%%%%%%%%%%%%%%%%%%%%%%%%%%%%%%%%%%%%%%%%%%%%%%%%%%%%%%%%%%
\section{Commentaries}

%%%%%%%%%%%%%%%%%%%%%%%%%%%%%%%%%%%%%%%%%%%%%%%%%%%%%%%%%%%%%%%%%%%%%%%%%%%%%%
\section{Doubts}

\end{document}