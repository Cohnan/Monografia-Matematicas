Fernandes remarks that it is thanks to the Lie braket that it is possible to make this kind of Cartan calculi.

Why is it necessary? What does this mean? What does it allow?

What is the difference, if any, with connection theory?

Relation between exterior derivative and usual covariant derivative:
    \begin{itemize}
    
    \item Finer than $d$: requiring $TM$ necessarily, because it is based on a connection $TM \to \mathcal D(E)$
    
    \item Coarser than $d$: does not require a representation, a connection doesn't have to be a homomorphism of Lie algebras.
        
    \end{itemize}

Why relevant to me:
    \begin{itemize}
        
    \item Vector calculus?
    
    \item What can be integrated. Inner integration generates usual (vector bundle valued) forms on the base manifold.
    
    \item Algebroid connections are ``normalized'' elements $ \omega \in \Omega^1(A, L)$, and their curvature is $d\omega \in \Omega^2(A, L)$. Generalized connections are precisely the elements of $\Omega^1(A, L)$.
    
    \item When talking about matter fields, a generalized connection $\hat \omega$ induces covariant derivatives $\hat \nabla^E: \Gamma(E) \to \mathbf{\Omega^1(A, E)}$, so that $\hat \nabla^E \phi \in \Omega^1(A, E)$.
    
    \item $\Omega^\bullet(A)$ is used to do cohomology on Lie algebroids
    
    \item In each $\Omega^p(A, L)$ the natural scalar product $(\omega, \eta) = \cl{\omega, \star \eta} = \int_A h(\omega, \star \eta)$, for $h$ in inner metric, is what will be used to define the gauge action:
        
        \begin{itemize}
            
        \item For $\hat \omega \in \Omega^1(A, L)$, define $S_{Gauge}[\hat \omega] = (\hat R, \hat R)$
        
        \item For $\phi \in \Gamma(E)$ and $\hat \omega$ as above, $S_{Matter}[\phi, \hat \omega] := (\hat \nabla^E \phi, \hat \nabla^E \phi)$
        
        \end{itemize}
        
    \end{itemize}

TODO\todo{check this}: the third ``important example'' introduced in Lazzarini is important to me? $A$ connections on $E$ are used to define finite gauge transformations, which somehow induce the infinitesimal gauge transformation definition on generalized connections?

%%%%%%%%%%%%%%%%%%%%%%%%%%%%%%%%%%%%%%%%%%%%%%%%%%%%%%%%%%%%%%%%%%%%%%%%%%%%%%%%%%%%
%%%%%%%%%%%%%%%%%%%%%%%%%%%%%%%%%%%%%%%%%%%%%%%%%%%%%%%%%%%%%%%%%%%%%%%%%%%%%%%%%%%%
%%%%%%%%%%%%%%%%%%%%%%%%%%%%%%%%%%%%%%%%%%%%%%%%%%%%%%%%%%%%%%%%%%%%%%%%%%%%%%%%%%%%
%%%%%%%%%%%%%%%%%%%%%%%%%%%%%%%%%%%%%%%%%%%%%%%%%%%%%%%%%%%%%%%%%%%%%%%%%%%%%%%%%%%%
\section{Differential Structures and Cartan Operations}

%%%%%%%%%%%%%%%%%%%%%%%%%%%%%%%%%%%%%%%%%%%%%%%%%%%%%%%%%%%%%%%%%%%%%%%%%%%%%%%%%%%%
%%%%%%%%%%%%%%%%%%%%%%%%%%%%%%%%%%%%%%%%%%%%%%%%%%%%%%%%%%%%%%%%%%%%%%%%%%%%%%%%%%%%
\subsection{Differential Algebras}

\begin{itemize}

\item Alt

\item $\Omega$

\item $d$

\item Respect further structure... of $L$?

\item Differential algebras? Differential calculi $(\Omega, d)$?

\end{itemize}

%%%%%%%%%%%%%%%%%%%%%%%%%%%%%%%%%%%%%%%%%%%%%%%%%%%%%%%%%%%%%%%%%%%%%%%%%%%%%%%%%%%%
%%%%%%%%%%%%%%%%%%%%%%%%%%%%%%%%%%%%%%%%%%%%%%%%%%%%%%%%%%%%%%%%%%%%%%%%%%%%%%%%%%%%
\subsection{Cartan Operations}

\begin{itemize}

\item $i, L$

\item Horizontal, invariant and basic forms.

\end{itemize}

%%%%%%%%%%%%%%%%%%%%%%%%%%%%%%%%%%%%%%%%%%%%%%%%%%%%%%%%%%%%%%%%%%%%%%%%%%%%%%%%%%%%
%%%%%%%%%%%%%%%%%%%%%%%%%%%%%%%%%%%%%%%%%%%%%%%%%%%%%%%%%%%%%%%%%%%%%%%%%%%%%%%%%%%%
%%%%%%%%%%%%%%%%%%%%%%%%%%%%%%%%%%%%%%%%%%%%%%%%%%%%%%%%%%%%%%%%%%%%%%%%%%%%%%%%%%%%
%%%%%%%%%%%%%%%%%%%%%%%%%%%%%%%%%%%%%%%%%%%%%%%%%%%%%%%%%%%%%%%%%%%%%%%%%%%%%%%%%%%%
\section{Trivial Lie Algebroid}

To understand better:
\begin{itemize}
    
\item $(\Omega(A), \hat d_A)$ as $(\Omega(M)\otimes \bigwedge g^*, d + s)$.

\item $(\Omega(A, L), \hat d)$ as $(\Omega(M)\otimes \bigwedge g^* \otimes \mathfrak g, d + s')$ : \[(\Omega_{TLA}(M, \mathfrak g), \hat d_{TLA})\]

\item Generic element in $\Omega^n$, given dual basis $\theta = \set{\theta^a}$ of a basis of $\alg g$ $\set{E_a}$
\begin{align*}
    \omega = \sum_{p + s = n} \omega^\theta_{\mu_1 \cdots a_1 \cdots a_s} dx^{\mu_1} wedge \cdots \theta^{a_s}
\end{align*}
    
\end{itemize}

%%%%%%%%%%%%%%%%%%%%%%%%%%%%%%%%%%%%%%%%%%%%%%%%%%%%%%%%%%%%%%%%%%%%%%%%%%%%%%%%%%%%
%%%%%%%%%%%%%%%%%%%%%%%%%%%%%%%%%%%%%%%%%%%%%%%%%%%%%%%%%%%%%%%%%%%%%%%%%%%%%%%%%%%%
%%%%%%%%%%%%%%%%%%%%%%%%%%%%%%%%%%%%%%%%%%%%%%%%%%%%%%%%%%%%%%%%%%%%%%%%%%%%%%%%%%%%
%%%%%%%%%%%%%%%%%%%%%%%%%%%%%%%%%%%%%%%%%%%%%%%%%%%%%%%%%%%%%%%%%%%%%%%%%%%%%%%%%%%%
\section{Local Description of (real and $L$ valued) forms on Transitive Lie Algebroids}

\begin{itemize}
    
\item Local trivialization of a form $\omega \mapsto \omega_{loc}$

\item $d_{TLA}\omega_{loc} = (d \omega)_{loc}$

\item $\hat \alpha^i_j$ for both real and $L$ valued forms.

\item Respects $d$. So $\hat \alpha$ is isomorphism of differential graded algebras.

\item $G^i_j$, matrix representation of $\hat \alpha^i_j$. Real valued, and $L$-valued:
    \begin{align}
        \omega^i_{\mu_1 \cdots a_1 \cdots a_s} &= G\cdots G \alpha^i_j(\omega^j_{\mu_1 \cdots b_1}) \\
        \omega^i_{\mu_1 \cdots a_1 \cdots a_s} &= G\cdots G \omega^j_{\mu_1 \cdots b_1}
    \end{align}

\item For Atiyah... for example, how does $G$ look. $\omega^i_{loc} = s_i^* \hat \omega$
    
\end{itemize}

%%%%%%%%%%%%%%%%%%%%%%%%%%%%%%%%%%%%%%%%%%%%%%%%%%%%%%%%%%%%%%%%%%%%%%%%%%%%%%%%%%%%
%%%%%%%%%%%%%%%%%%%%%%%%%%%%%%%%%%%%%%%%%%%%%%%%%%%%%%%%%%%%%%%%%%%%%%%%%%%%%%%%%%%%
%%%%%%%%%%%%%%%%%%%%%%%%%%%%%%%%%%%%%%%%%%%%%%%%%%%%%%%%%%%%%%%%%%%%%%%%%%%%%%%%%%%%
%%%%%%%%%%%%%%%%%%%%%%%%%%%%%%%%%%%%%%%%%%%%%%%%%%%%%%%%%%%%%%%%%%%%%%%%%%%%%%%%%%%%
\section{Gauge group, Infinitesimal Gauge transformations and Infinitesimal gauge action on forms}

(Lazzarini 2012) Perhaps this is not the best place, but try, specially to understand which are definitions and which are somehow forced definitions.

%%%%%%%%%%%%%%%%%%%%%%%%%%%%%%%%%%%%%%%%%%%%%%%%%%%%%%%%%%%%%%%%%%%%%%%%%%%%%%%%%%%%
%%%%%%%%%%%%%%%%%%%%%%%%%%%%%%%%%%%%%%%%%%%%%%%%%%%%%%%%%%%%%%%%%%%%%%%%%%%%%%%%%%%%
%%%%%%%%%%%%%%%%%%%%%%%%%%%%%%%%%%%%%%%%%%%%%%%%%%%%%%%%%%%%%%%%%%%%%%%%%%%%%%%%%%%%
%%%%%%%%%%%%%%%%%%%%%%%%%%%%%%%%%%%%%%%%%%%%%%%%%%%%%%%%%%%%%%%%%%%%%%%%%%%%%%%%%%%%
\section{Atiyah Lie Algebroid alternative Description?}

There seem to be $2$, related, alternative descriptions (Lazzarini2012):
    
    \begin{itemize}
        
    \item $(\Omega(A, L), \hat d)$ a $(\Omega_{TLA}(P, \mathfrak g)_{equ}, \hat d_{TLA})$,
    
    \item and $(R, Ad)$-equivariant forms in $(\Omega(P) \times \mathfrak g, d)$. Denoted by: \[ (\Omega_{Lie}(P, \mathfrak g), \hat d) \]
        
    \end{itemize}

