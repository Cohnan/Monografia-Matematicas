Fernandes remarks that it is thanks to the Lie braket that it is possible to make this kind of Cartan calculi.

Why is it necessary? What does this mean? What does it allow?

What is the difference, if any, with connection theory?

Relation between exterior derivative and usual covariant derivative:
    \begin{itemize}
    
    \item Finer than $d$: requiring $TM$ necessarily, because it is based on a connection $TM \to \mathcal D(E)$
    
    \item Coarser than $d$: does not require a representation, a connection doesn't have to be a homomorphism of Lie algebras.
        
    \end{itemize}

Why relevant to me:
    \begin{itemize}
        
    \item Vector calculus?
    
    \item What can be integrated. Inner integration generates usual (vector bundle valued) forms on the base manifold.
    
    \item Algebroid connections are ``normalized'' elements $ \omega \in \Omega^1(A, L)$, and their curvature is $d\omega \in \Omega^2(A, L)$. Generalized connections are precisely the elements of $\Omega^1(A, L)$.
    
    \item When talking about matter fields, a generalized connection $\hat \omega$ induces covariant derivatives $\hat \nabla^E: \Gamma(E) \to \mathbf{\Omega^1(A, E)}$, so that $\hat \nabla^E \phi \in \Omega^1(A, E)$.
    
    \item $\Omega^\bullet(A)$ is used to do cohomology on Lie algebroids
    
    \item In each $\Omega^p(A, L)$ the natural scalar product $(\omega, \eta) = \cl{\omega, \star \eta} = \int_A h(\omega, \star \eta)$, for $h$ in inner metric, is what will be used to define the gauge action:
        
        \begin{itemize}
            
        \item For $\hat \omega \in \Omega^1(A, L)$, define $S_{Gauge}[\hat \omega] = (\hat R, \hat R)$
        
        \item For $\phi \in \Gamma(E)$ and $\hat \omega$ as above, $S_{Matter}[\phi, \hat \omega] := (\hat \nabla^E \phi, \hat \nabla^E \phi)$
        
        \end{itemize}
        
    \end{itemize}

TODO\todo{check this}: the third ``important example'' introduced in Lazzarini is important to me? $A$ connections on $E$ are used to define finite gauge transformations, which somehow induce the infinitesimal gauge transformation definition on generalized connections? If not for this, $A$-connections, and so this examples, are of no relevance to me.

%%%%%%%%%%%%%%%%%%%%%%%%%%%%%%%%%%%%%%%%%%%%%%%%%%%%%%%%%%%%%%%%%%%%%%%%%%%%%%%%%%%%
%%%%%%%%%%%%%%%%%%%%%%%%%%%%%%%%%%%%%%%%%%%%%%%%%%%%%%%%%%%%%%%%%%%%%%%%%%%%%%%%%%%%
%%%%%%%%%%%%%%%%%%%%%%%%%%%%%%%%%%%%%%%%%%%%%%%%%%%%%%%%%%%%%%%%%%%%%%%%%%%%%%%%%%%%
%%%%%%%%%%%%%%%%%%%%%%%%%%%%%%%%%%%%%%%%%%%%%%%%%%%%%%%%%%%%%%%%%%%%%%%%%%%%%%%%%%%%
\section{Differential Structures and Cartan Operations}

%%%%%%%%%%%%%%%%%%%%%%%%%%%%%%%%%%%%%%%%%%%%%%%%%%%%%%%%%%%%%%%%%%%%%%%%%%%%%%%%%%%%
%%%%%%%%%%%%%%%%%%%%%%%%%%%%%%%%%%%%%%%%%%%%%%%%%%%%%%%%%%%%%%%%%%%%%%%%%%%%%%%%%%%%
\subsection{Differential Algebras}

\begin{itemize}

\item Alt

\item $\Omega$

\item $d$. Is antiderivation

\item $(\Omega, d)$ Graded differential algebras. Differential calculus (over a commutative algebra?) ?
    
    \begin{itemize}
    
    \item $(\Omega(A), \hat d_A)$ is graded commutative differential algebra
    
    \item $(\Omega(A, L), \hat d)$ is graded differential Lie algebra.
    
    \end{itemize}

\item Respect further structure... of $L$?

\end{itemize}

\linea 

Throughout this section, let $A$ be a Lie algebroid over the manifold $M$ with anchor $a$.

\begin{definition}
Let $E$ be a vector bundle over $M$. We now define some vector bundles over $M$.
    \begin{itemize}
    
    \item Let \emph{$\Alt^0(A, E)$}$:= E$. 
    
    \item For $n \in \ZZ_{\geq 1}$, let \emph{$\Alt^n(A, E)$} be the natural embedding, as alternating maps, of the (hom-)bundle $\Hom(\bigwedge^n A, E)$ into the vector bundle $\Hom(\bigoplus^n A, E)$ (as defined according to \ref{}) over $M$. Its fiber at $m \in M$ is, then, the vector space of alternating linear transformations $\{\omega: \underbrace{A_m \times \cdots \times A_m}_{\text{$n$ times}} \to E_m \st \text{$\omega$ is $\RR$-linear and alternating}\}$.
    
    \item Let \emph{$\Alt^\bullet(A, E)$}$:= \bigoplus^\infty_{n = 0} \Alt^n(A, E)$.
    % OJO: i can't call this forms, since forms are either C^\infty M multilinear, or complete vector bundle morphisms!
    
    %\item For $n \in \ZZ_{\geq 0}$, let \emph{$\Alt^n(A)$} and \emph{$Alt^\bullet (A)$} be the application of the previous definitions to the trivial vector bundle $E = M \times \RR$.
        
    \end{itemize}
    
\end{definition}

Notice that $\Alt^\bullet(A, E)$ is indeed a vector bundle, with finite dimensional fibers, since, if $U \subset M$ is a trivializing neighborhood of both $A$ and $E$ with fibers $V$ and $W$ respectively, $\Alt^\bullet(A, E)|_U \cong U \times Hom(V, \bigwedge^* W)$ is a trivial vector bundle.

\begin{definition}
Let $E$ be a vector bundle over $M$ on which there is a representation $\phi: A \to \alg D(E)$ of $A$. Let $U \subset M$.

    \begin{itemize}
    
    \item For $n \in \ZZ_{\geq 0}$, define the $C^\infty(U)$-module \emph{$\Omega_U^n(A, E)$}$:= \Gamma_U\Alt^n(A, E)$. Its elements are called \emph{local $E$-valued $n$-forms on $A$ over $U$}. If $U = M$, simply write \emph{$\Omega^n(A, E)$} and call its elements \emph{$E$-valued $n$-forms on $A$}.
    
    \item Define the $C^\infty(U)$-module $\emph{\Omega_U^\bullet(A, E)} := \bigoplus^\infty_{n = 0} \Omega_U^n(A, E)$. Its elements are simply called \emph{local $E$-valued forms on $A$ over $U$}. If $U = M$, simply write $\emph{\Omega^\bullet(A, E)}$ and call its elements \emph{$E$-valued forms on $A$.}
    
    %\item \item For $n \in \ZZ_{\geq 0}$, let \emph{$\Omega^n(A)$}$:= \Gamma\Alt^n(A)$. 
    
    %\item Define the $C^\infty(M)$-module $\emph{\Omega^\bullet(A)} := \bigoplus^\infty_{n = 0} \Omega^n(A)$.
        
    \end{itemize}
    
\end{definition}

\begin{remark}
Due to the bijective \dbtext{correspondence between $C^\infty(M)$-linear maps between local sections over $U$ of vector bundles and vector bundle morphisms resticted to $U$}\todo{is this a thing? I think it might be correct to leave the linearity to be over all of $M$}, $\Omega_U^n(A, E)$ may be considered the space of \dbtext{$C^\infty(M)$-multilinear} antisymmetric maps from $\Gamma_U(A)^n$ to $\Gamma_U(E) = \Omega_U^0(A, E)$. We will refer as \ptext{(local) $E$-valued $n$-forms on $A$} to both elements
\begin{align}
    \omega &: \underbrace{A|_U \oplus \cdots \oplus A|_U}_{\text{$n$ times}} \to E_U & \text{alternating, $\RR$-vector bundle morphism}\\
    \omega &: \underbrace{\Gamma_U(A) \times \cdots \times \Gamma_U(A)}_{\text{$n$ times}} \to \Gamma_U(E) & \text{alternating, $C^\infty(M)$-multilinear map}.
\end{align}
    
    %\item $\Omega^n(A)$ may be considered the space of $C^\infty(M)$-multilinear antisymmetric maps from $\Gamma(A)^n$ to $C^\infty(M)$.
    
\end{remark}

\lin



\begin{example}
On the trivial vector bundle $M \times \bb K$\todo{$A$ as a $\bb K$ vector bundle}, the associated space of forms on $A$ is denoted by $\lbtext{\lbtext{\Omega^\bullet(A)}}$, and it is called the \lbtext{space of scalar valued forms on $A$}. Notice that the module $\Omega^n(A) = \Gamma \bigwedge^n A^*$.

This is the fundamental example of forms on an algebroid $A$. It has been used to define a cohomology theory on Lie algebroids and \todo{complement}.
\end{example}

\begin{example}
Let $0 \to L \xrightarrow{j} A \xrightarrow{a} TM \to 0$ be a transitive Lie algebroid sequence.
The space $\lbtext{\Omega^\bullet(A, L)}$ of $L$-valued forms on $A$ will be an important example since (generalized) algebroid connections and their curvatures are particular cases of this type of forms on $A$.
\end{example}

\lin

\begin{theorem}\label{isomorphOmega}
    \begin{itemize}
    
    \item $\Alt^n(A, E)$ and $E \otimes \bigwedge^n A^*$ are isomorphic vector bundles.
    
    %\item $\Alt^n(A)$ and $\bigwedge^n A^*$ are isomorphic vector bundles.
    
    \item \rtext{$\Omega^n(A, E)$ and $\Gamma(E) \otimes \Omega^n(A)$ are isomorphic} $C^\infty(M)$-modules.
    
    \item If $U$ is trivializing neighborhood for both $A$ and $E$, $\rtext{\Omega_U^n(A, E)} \cong C^\infty(U, V) \otimes_{C^\infty(U)} \Gamma(U \times \bigwedge^n W^*) \rtext{\cong C^\infty(U, V) \otimes_{\bb K} \bigwedge^n W^*}$
    
    %\item $\Omega^n(A)$ and $\Gamma\Alt^nA^*$ are isomorphic $C^\infty(M)$-modules.
    
    \end{itemize}
\end{theorem}

\begin{proof}
$\Alt^n(A, E) \cong \Hom(\bigwedge^n A, E)$ by definition, and $\Hom(\bigwedge^n A, E) \cong E \otimes (\bigwedge^n A)* \cong E \otimes \bigwedge^n A^*$, but $\bigwedge^n A^* = \Omega^n(A)$.

Since $\Gamma$ is a strong monoidal function\todo{can I make this more clear?}, it distributes over the tensor product, and so the result follows from the first part of the theorem.
\end{proof}

\linea


\ptext{In the following, assume that $E$ is a vector bundle over $M$ on which there is a representation $\phi: A \to \alg D(E)$ of the Lie algebroid $A$ over $M$}.

\begin{definition}
    \begin{itemize}
    
    \item Differential complex:
    
    \item Differential module:
        
    \end{itemize}
\end{definition}

\begin{definition}
Let $E$ be a vector bundle over $M$ on which there is a representation $\phi: A \to \alg D(E)$ of $A$. Define \emph{the differential of $E$-valued forms on $A$} by the Koszul formula:
\begin{eqnsplit*}
\lbtext{\hat d_\phi}: \Omega^{\bullet}(A, E) &\to \Omega^{\bullet+1}(A, E)
\end{eqnsplit*}
\begin{multline}
(\hat d_\phi \omega)(\sectoid X_1, \dots, \sectoid X_{p+1}) = \sum_{i=1}^{p+1} (-1)^{i+1} \phi(\sectoid X_i)\cdot \omega(\sectoid X_1, \cdots, \overset{\vee}{\sectoid X_i}, \cdots, \sectoid X_{p+1}) \\
 + \sum_{1 \leq i < j \leq p+1} (-1)^{i+j}\omega([\sectoid X_i, \sectoid X_j], \sectoid X_1, \cdots, \overset{\vee}{\sectoid X_i}, \cdots, \overset{\vee}{\sectoid X_j}, \cdots, \sectoid X_{p+1})
\end{multline}
for any $p \in \ZZ_{\geq 0}$, $p$-form $\omega$, and any $\sectoid X_1, \dots, \sectoid X_{p+1} \in \Gamma(A)$.
\end{definition}

\begin{proposition}
 $(\Omega^\bullet(A, E), \hat d_\phi)$ is a differential module. In particular, it is a differential complex.
\end{proposition}

\begin{theorem}\label{theoDifferentialLocal}
$d_U$: Leibniz w.r.t. functions giben by the previous definition.\todo{complete}. \rtext{$\hat d_\phi$ is a local operator}
\end{theorem}

\begin{remark}
So 1., 2. .
\end{remark}

\lin

\begin{definition}\label{defnDiffGModule}
    Let $M = \oplus_{n = 0}^\infty M^n$ graded module over the graded ring $R = \oplus_{n = 0}^\infty R^n$, i.e. an $R$-module such that $R^p \cdot M^q \subset M^{p+q}$.
    
    \begin{itemize}
    
    \item An element $\omega \in N^n$ is called a \emph{homogeneous element of degree $n$}, for $n \in \ZZ_{\geq 0}$, and its degree is denoted by $|\omega|$.
    
    \item An \emph{antiderivation on $M$} is an $R$-linear map $d: M \to M$ such that, if $\omega$ is homogeneous, 
    \begin{align*}
        d(\omega \, \eta) = (d\omega)\tau + (-1)^{|\omega|} \omega (d\tau)
    \end{align*}
    
    \item An antiderivation $d: M \to M$ is \emph{of degree $m \in \ZZ$} if, for all homogeneous elements $\omega \in M$,
    \begin{align*}
        |d\omega| = |\omega| + m.
    \end{align*}
    If the antiderivation is of degree $1$ or $-1$, it is called a \emph{differential on $M$}.
    
    \item Let $M = \oplus_{n = 0}^\infty M^n$ be a graded module and $d: M \to M$ a differential on $M$. $(M, d)$ is called a \emph{differential graded module} if $d \circ d = 0$.
    
    \item Let $(M, d)$ and $(M', d')$ be differential graded modules over $R$. A \emph{morphism of differential graded modules} is a graded module morphism $\phi: M \to M'$, i.e. an $R$-linear map that sends homogeneous elements of degree $m \in \ZZ$ of $M$ into similar elements in $M'$, such that $d' \circ \phi = \phi \circ d$.
    
    \end{itemize}
    
\end{definition}

\begin{definition}
Define the $C^\infty(M)$-bilinear operation operation
\begin{align*}
    \lbtext{\wedge}: \Omega^\bullet(A) \times \Omega^\bullet(A, E) &\to \Omega^\bullet(A, E)
\end{align*}
for any $p, q \in \ZZ_{\geq 0}$,
through the linear extension on the homogeneous elements:
\begin{eqnsplit}
(\omega \wedge \eta)(\sectoid X_1, \dots, \sectoid X_{p+q}) &:= \\
\frac{1}{p!q!} \sum_{\sigma \in S_{p+q}} &(-1)^{\sigma} 
\underbrace{\omega(\oid X_{\sigma(1)}, \cdots, \oid X_{\sigma(p)})}_{\in \,\Gamma(M \times \RR)} 
\cdot 
\underbrace{\eta(\oid X_{\sigma(p+1)}, \cdots, \oid X_{\sigma(p+q)})}_{\in \, \Gamma(L)}
\end{eqnsplit}
for any real valued $p$-form $\omega$, $L$-valued $q$-form $\eta$ and any $\sectoid X_1, \dots, \sectoid X_{p+q} \in \Gamma(A)$.
\end{definition}

\begin{theorem}\label{theoFormsAreDiffGModule}
 \rtext{$(\Omega^\bullet(A, E), \hat d_\phi)$ is a graded differential module over $\Omega^\bullet(A)$}.
\end{theorem}

\begin{proof}
To see:

$\Omega^p(A) \wedge \Omega^q(A, E) \subset \Omega^{p+q}(A, E)$.
\end{proof}

\begin{example}
Let $A$ act by the trivial action $\phi^0: A \to \alg D(M \times \bb K)$, i.e. considering $\Gamma(M \times \RR)$ as $C^\infty(M)$ and letting the representation be the anchor. The associated differential on $\Omega^\bullet(A)$ will be denoted by $\hat d_A$, making of \lbtext{$(\Omega^\bullet(A), \hat d_A)$} a differential graded module
\end{example}

\begin{example}
Let $0 \to L \xrightarrow{j} A \xrightarrow{a} TM \to 0$ be a transitive Lie algebroid sequence.
The differential graded module associated to the adjoint action defined in \ref{defnAdjointAct}, $ad: A \to \alg D(L)$, $\oid X \mapsto j^{-1}[\oid X, \cdot]$, will be denoted by \lbtext{$(\Omega^\bullet(A, L), \hat d)$}.
\end{example}

\lin

\begin{definition} \label{defnDiffGAlgebra}
Let $\mathcal A = \oplus_{n = 0}^\infty \mathcal A^n$ be a graded algebra (over the field $\bb K$), and let $\cdot$ denote the product.
    \begin{itemize}
    
    \item An element $\omega \in \mathcal A^n$ is called a \emph{homogeneous element of degree $n$}, for $n \in \ZZ_{\geq 0}$, and its degree is denoted by $|\omega|$.
    
    \item An \emph{antiderivation on $\mathcal A$} is an $\bb K$-linear map $d: \mathcal A \to \mathcal A$ such that, if $\omega$ is homogeneous, 
    \begin{align*}
        d(\omega \bullet \eta) = (d\omega)\bullet\tau + (-1)^{|\omega|} \omega \bullet (d\tau)
    \end{align*}
    
    \item An antiderivation $d: \mathcal A \to \mathcal A$ is \emph{of degree $m \in \ZZ$} if, for all homogeneous elements $\omega \in \mathcal A$,
    \begin{align*}
        |d\omega| = |\omega| + m.
    \end{align*}
    If the antiderivation is of degree $1$ or $-1$, it is called a \emph{differential on $\mathcal A$}.
    
    \item Let $d: \mathcal A \to \mathcal A$ be a differential on $A$. Then the pair $(A, d)$ is called a \emph{differential graded algebra} if $d \circ d = 0$. 
    
    \item Let $(\mathcal A, \bullet, d)$ and $(\mathcal A', \bullet', d')$ be differential graded algebras. A \emph{morphism of differential graded algebras} is a graded algebra morphism $\phi: \mathcal A \to \mathcal A'$ such that $d' \circ \phi = \phi \circ d$.
    
    \end{itemize}
    
\end{definition}


\begin{definition}
Let $E$ be an algebra bundle over $M$, i.e. a vector bundle such that each fiber is an algebra and such that there exists an atlas compatible with the algebra multiplication, on which $A$ is represented. Define on $\Omega^\bullet(A, E)$ the $C^\infty(M)$-bilinear operation operation, called \emph{the wedge product of LAB-valued forms on $A$}
\begin{eqnsplit*}
\wedge : \Omega^\bullet(A, E) &\times \Omega^\bullet(A, E) \to \Omega\bullet(A, E)
\end{eqnsplit*}
as the linear extension of the maps, for any $p, q \in \ZZ_{\geq 0}$
\begin{eqnsplit*}
\wedge : \Omega^p(A, E) \times \Omega^q(A, E) &\to \Omega^{p+q}(A, E)
\end{eqnsplit*}
\begin{eqnsplit}
(\omega \wedge \eta)(\sectoid X_1, \dots, \sectoid X_{p+q}) &:= \\
\frac{1}{p!q!} \sum_{\sigma \in S_{p+q}} &(-1)^{\sigma} \omega(\oid X_{\sigma(1)}, \cdots, \oid X_{\sigma(p)}) \bullet \eta(\oid X_{\sigma(p+1)}, \cdots, \oid X_{\sigma(p+q)}).
\end{eqnsplit}
for any $p$-form $\omega$, $q$-form $\eta$ and any $\sectoid X_1, \dots, \sectoid X_{p+q} \in \Gamma(A)$, where $\bullet$ is the algebra multiplication induced on $\Gamma(E)$ by the fiberwise algebra multiplication in $E$.
\end{definition}

\begin{theorem} \label{theoFormsAreDiffGAlgebebra}
Let $E$ be an algebra bundle over $M$ on which there is a representation $\phi: A \to \alg D(E)$ of $A$. \rtext{Then $(\Omega^\bullet(A, E), \wedge, \hat d_\phi)$ is a differential graded algebra over $\Omega^\bullet(A)$.}
\end{theorem}

\begin{proof}
To see:

Associativity... just say it is, alhthough, is there an easy way to see it?

$d^2 = 0$. Apparently this follows from the fact that $\phi$ respects the Lie algebroid bracket.

Graded Leibniz
\end{proof}

\begin{example}
\rtext{$(\Omega^\bullet(A), \wedge, \hat d_A)$ is a graded-commutative differential graded algebra}.

The additional conditions is simply that $\alpha \beta = (-1)^{|x||y|}\beta \alpha$ for all homogeneous elements $\alpha, \beta$. This follows from $\bb K$ being commutative.
\end{example}

\linea 

{\color{gray}
The following instances of $(\Omega^\bullet(A, E), \hat d_\phi)$ will be important to us in future chapters.

\begin{theorem}
 NOT USEFUL TO US
$(\Omega^\bullet(A, L, \wedge, \hat d)$ is a Lie-graded differential graded algebra.
\end{theorem}
}

%%%%%%%%%%%%%%%%%%%%%%%%%%%%%%%%%%%%%%%%%%%%%%%%%%%%%%%%%%%%%%%%%%%%%%%%%%%%%%%%%%%%
%%%%%%%%%%%%%%%%%%%%%%%%%%%%%%%%%%%%%%%%%%%%%%%%%%%%%%%%%%%%%%%%%%%%%%%%%%%%%%%%%%%%
{\color{gray} This only seems to be useful if we want to define the alternative description of the Atiyah Lie algebroid found in Lazzarini2012

\subsection{Cartan Operations}

% \begin{itemize}

% \item $i, L$

% \item Horizontal, invariant and basic forms.

% \end{itemize}

\begin{definition}
Let $E$ be an algebra bundle over $M$, let $\phi: A \to \alg D(E)$ be a representation of $A$, and let $B$ be a Lie algebroid over $M$. A \emph{Cartan operation on $(\Omega^\bullet(A, E), \wedge, \hat d_\phi)$} is a triple $(B, i, L)$ such that: for any $\sectoid X \in B$, for any $p \geq 1$ there is a map $i_{\sectoid X}: \Omega^p(A, E) \to \Omega^{p-1}(A, E)$ such that, defining 
\begin{align}
L_{\sectoid X} := \hat d_\phi i_{\sectoid X} + i_{\sectoid X}\hat d_\phi,
\end{align}
the following relations hold for any $\sectoid X, \sectoid Y \in B$ and $f \in C^\infty(M)$:
\begin{align}
    i_{f \sectoid X} = f i_{\sectoid X} & i_{\sectoid X} i_{\sectoid Y} + i_{\sectoid Y} i_{\sectoid X} = 0 \\
    [L_{\sectoid X}, i_{\sectoid Y}] = i_{[\sectoid X, \sectoid Y]} & [L_{\sectoid X}, L_{\sectoid Y}] = L_{[\sectoid X, \sectoid Y]}
\end{align}
\end{definition}

\begin{example}
$B = A$ defines one.
\end{example}

\begin{example}
$B = L$ also defines one.
\end{example}

\begin{definition}
Let $(B, i, L)$ be a Cartan operation on $(\Omega^\bullet(A, E), \wedge, \hat d_\phi)$.
    
    \begin{itemize}
        
    \item Define $\Omega^\bullet(A, E)_{Hor}$ to be the graded subspace of \emph{horizontal elements in $\Omega^\bullet(A, E)$} characterized by 
    \[
    i_{\sectoid X} \omega = 0?
    \]
    
    \item Define $\Omega^\bullet(A, E)_{Inv}$ to be the graded subspace of \emph{invariant elements in $\Omega^\bullet(A, E)$} characterized by 
    \[
    L_{\sectoid X} \omega = \omega?
    \]
    
    \item Define $\Omega^\bullet(A, E)_{Basic} = \Omega^\bullet(A, E)_{Hor} \inter \Omega^\bullet(A, E)_{Inv}$ to be the graded subspace of \emph{basic or tensorial elements in $\Omega^\bullet(A, E)$}.
        
    \end{itemize}
    
\end{definition}

\begin{example}
Atiyah. The concept coincides? (See Mackenzie)
\end{example}}
%%%%%%%%%%%%%%%%%%%%%%%%%%%%%%%%%%%%%%%%%%%%%%%%%%%%%%%%%%%%%%%%%%%%%%%%%%%%%%%%%%%%
%%%%%%%%%%%%%%%%%%%%%%%%%%%%%%%%%%%%%%%%%%%%%%%%%%%%%%%%%%%%%%%%%%%%%%%%%%%%%%%%%%%%
%%%%%%%%%%%%%%%%%%%%%%%%%%%%%%%%%%%%%%%%%%%%%%%%%%%%%%%%%%%%%%%%%%%%%%%%%%%%%%%%%%%%
%%%%%%%%%%%%%%%%%%%%%%%%%%%%%%%%%%%%%%%%%%%%%%%%%%%%%%%%%%%%%%%%%%%%%%%%%%%%%%%%%%%%
\section{Trivial Lie Algebroid}

\begin{itemize}
    
\item $(\Omega(A), \hat d_A)$ as $(\Omega(M)\otimes \bigwedge g^*, d + s =: \lbtext{\hat d_{triv}})$.

\item $(\Omega(A, L), \hat d)$ as $(\Omega(M)\otimes \bigwedge g^* \otimes \mathfrak g, d + s')$ : \[(\Omega_{TLA}(M, \mathfrak g), \lbtext{\hat d_{TLA} =: \hat d})\]

\item Generic element in $\Omega^n$, given dual basis $\theta = \set{\theta^a}$ of a basis of $\alg g$ $\set{E_a}$
\begin{align*}
    \omega = \sum_{p + s = n} \omega^\theta_{\mu_1 \cdots a_1 \cdots a_s} dx^{\mu_1} \wedge \cdots \theta^{a_s}
\end{align*}
    
\end{itemize}

Let $A = TM \oplus (M \times \alg g)$ be a trivial Lie algebroid with fiber type $\alg g$.

\begin{proposition}\label{propIsoScalarFormsTLA}
The $C^\infty(M)$-module of $\bb K$-valued forms on the trivial Lie algebroid bundle $TM \times \alg g$ is isomorphic to $\Omega^\bullet(TM) \otimes_{\bb K} \bigwedge^\bullet \alg g^* = \Omega^\bullet(TM) \otimes_{C^\infty(M)} C^\infty(M, \bigwedge^\bullet \alg g^*)$. The isomorphism, defined by linear extension of
\begin{align*}
    \Omega^r(TM) \otimes C^\infty(U, \bigwedge^s \alg g^*) &\to \Omega^{r+s}(TM \times \alg g)\\
    \alpha \otimes \beta &\mapsto \tilde \alpha \wedge \tilde \beta
\end{align*} where, for any $\alpha \in \Omega^r(TM)$ $r, s \in \ZZ_{\geq 0}$:
\begin{align*}
    \tilde \alpha: (\Gamma(TM) \times C^\infty(M, \alg g)) \times \cdots (\Gamma(TM) \times C^\infty(M, \alg g)) &\to C^\infty(M) \\
    (X_1 \oplus \tilde \eta_1, \dots, X_r \oplus \tilde \eta_r) &\mapsto \alpha(X_1, \dots, X_r)
\end{align*} and for any $\beta \in C^\infty(M, \bigwedge^s \alg g^*)$:
\begin{align*}
    \tilde \beta: (\Gamma(TM) \times C^\infty(M, \alg g)) \times \cdots (\Gamma(TM) \times C^\infty(M, \alg g)) &\to C^\infty(M) \\
    (X_1 \oplus \tilde \eta_1, \dots, X_s \oplus \tilde \eta_s) &\mapsto \beta(\tilde \eta_1, \dots, \tilde \eta_s)
\end{align*}
is compatible with the grading, where the set of homogeneous elements of degree $p \in \ZZ_{\geq 0}$ are $\sum_{r + s = p} \Omega^r(TU) \otimes_{C^\infty(M)} C^\infty(M, \bigwedge^s \alg g^*)$.
\end{proposition}

\begin{remark}
The previous isomorphism between $\Omega^\bullet(TM \times \alg g)$ and $\Omega^\bullet(TM) \otimes_{\bb K} \bigwedge^\bullet \alg g^*$ can, and will, be used to define a $C^\infty(M)$-linear multiplication $\lbtext{\wedge}$ on $\Omega^\bullet(TM) \otimes_{\bb K} \bigwedge^\bullet \alg g^*$ which is compatible with the gradings. 

Furthermore, this isomorphisms allows us to see $\Omega^\bullet(TM)$ and $C^\infty(M, \alg g)$ as subsets of $\Omega^\bullet(TM \times \alg g)$. Abusing notation we may ignore the $~$'s of the mapping to simply state that, for $\alpha \in \Omega^r(TM)$ and $\beta \in C^\infty(M)\otimes_{\bb K} \bigwedge^s \alg g^*$, $r, s \in \ZZ_{\geq 0}$, \ptext{$\alpha \wedge \beta \in \Omega^{r+s}(TM \times \alg g)$}. 
\end{remark}

\begin{definition}
Let
\begin{align}
    d: \Omega^\bullet(TM) \otimes \bigwedge^\bullet \alg g^* &\to \Omega^{\bullet+1}(TM) \otimes \bigwedge^\bullet \alg g^*\\
    \alpha \otimes 1 &\mapsto d\alpha \otimes 1
\end{align}
be the usual de Rham differential on $\Omega^\bullet(TM)$ applied to the first factor, and
\begin{align*}
    s: \Omega^\bullet(TM) \otimes \bigwedge^\bullet \alg g^* &\to \Omega^\bullet(TM) \otimes \bigwedge^{\bullet+1} \alg g^* \\
    1 \otimes \beta &\mapsto 1 \otimes s\beta, 
\end{align*} where, given $\eta_i \in C^\infty(M, \alg g)$,
\begin{multline}
    s\beta(\sect \eta_1, \cdots, \sect \eta_{p+1}) := \sum_{1 \leq i < j \leq p+1} (-1)^{i+j} \beta([\sect \eta_i, \sect \eta_j], \sect \eta_1, \cdots, \overset{\vee}{\sect \eta_i}, \cdots, \overset{\vee}{\sect \eta_j}, \cdots \sect \eta_{p+1})
\end{multline}
be \emph{the Chevalley-Eilenbert differential on $\bigwedge \alg g^*$}, applied to the second factor. On $\Omega^\bullet(TM) \otimes \bigwedge^\bullet \alg g^*$ define the $C^\infty(M)$-linear endomorphism $\lbtext{\hat d_{triv}}$:
\begin{align}
    \hat d_{triv}: \Omega^r(TM) \otimes \bigwedge^s \alg g^* &\to \left( \Omega^\bullet(TM) \otimes \bigwedge^\bullet \alg g^*\right)^{r+s+1} \\
    \alpha \otimes \beta = \alpha \wedge \beta &\mapsto  \hat d_{triv} (\alpha \wedge \beta) := d\alpha \wedge \beta + (-1)^|\alpha| \alpha \wedge s\beta \\
    \alpha \otimes 1 &\mapsto \hat d_{triv}\alpha = d\alpha \in \Omega^{r+1}(TM) \otimes 1 \\
    1 \otimes \beta &\mapsto \hat d_{triv}\beta = s\beta \in 1 \otimes \bigwedge^{s+1} \alg g^*.
\end{align} $\hat d_{triv}$ is sometimes also denoted by $d+s$ in the literature.

\end{definition}

\begin{theorem}\label{theoIsoScalarFormsTLA}
\rtext{$(\Omega^\bullet(TM) \otimes \bigwedge^\bullet, \wedge, \hat d_{triv})$ is a (bi)graded (commutative) differential algebra and graded differential module over $\Omega^\bullet(TM \times \alg g)$, isomorphic to the space $(\Omega^\bullet(TM \times \alg g), \wedge, \hat d_{TM \times \alg g})$} of $\bb K$-valued forms on the trivial Lie algebroid $TM \times \alg g$.
\end{theorem}

\begin{proof}
To see:

Graded module: done already, pretty much by definition of wedge, when restricting to the ``subspace'' $\Omega^\bullet(TM)$ in the first entry.

$\hat d_{triv}$ is a differential of module and algebra: pretty much by definition

Is isomorphism of graded modules: Respects the differentials DO!\todo{Done in my handwritten notes.}

Is isomorphism of graded algebras: Respects the wedge product: by definition of the wedge on the new space as defined in the remark following theorem \ref{propIsoScalarFormsTLA}.
\end{proof}

\ptext{Due to this isomorphism, from now on we will call $(\Omega^\bullet(TM) \otimes \bigwedge^\bullet, \wedge, \hat d_{triv})$ the space of scalar valued forms on the trivial Lie algebroid $TM \times \alg g$, and $\Omega^\bullet(TM)$ and $C^\infty(M, \bigwedge^\bullet \alg g*)$ as subspaces.}

\linea

We now apply a similar result to the $L$-valued forms on the trivial Lie algebroid $A = TM \times \alg g$.

\begin{theorem}
The $C^\infty(M)$-module $\Omega(TM \times \alg g, M \times \alg g)$ of $M \times \alg g$-valued forms on the trivial Lie algebroid bundle $TM \times \alg g$, is isomorphic to $\Omega^\bullet(TM) \otimes \left(\bigwedge^\bullet \alg g^* \otimes \alg g\right)$, and this isomorphism is compatible with the grading, where the set of homogeneous elements of degree $p \in \ZZ_{\geq 0}$ is $\left( \Omega^\bullet(TM) \otimes \left(\bigwedge^\bullet \alg g^* \otimes \alg g\right) \right)^p = \sum_{r + s = p} \Omega^r(TM) \otimes \left(\bigwedge^s \alg g^* \otimes \alg g\right)$. 

This isomorphism induces a $C^\infty(M)$-multilinear mapping
\begin{equation}
    \wedge: \Omega^\bullet(TM) \times \left(\bigwedge^s \alg g^* \otimes \alg g\right) \to \Omega^\bullet(TM) \otimes \left(\bigwedge^s \alg g^* \otimes \alg g\right).
\end{equation}

Furthermore, defining the $C^\infty(M)$-linear endomorphism $\hat d$ on this space, 
\begin{align}
    \hat d: \Omega^r(TM) \otimes \left(\bigwedge^s \alg g^* \otimes \alg g\right) &\to \left( \Omega^\bullet(TM) \otimes \left(\bigwedge^\bullet \alg g^* \otimes \alg g\right)\right)^{r+s+1} \\
    \alpha \otimes \beta = \alpha \wedge \beta &\mapsto  \hat d (\alpha \wedge \beta) := d\alpha \wedge \beta + (-1)^|\alpha| \alpha \wedge s'\beta \\
    \alpha \otimes 1 &\mapsto \hat d\alpha = d\alpha \in \Omega^{r+1}(TM) \otimes 1 \\
    1 \otimes \beta &\mapsto \hat d\beta = s'\beta \in 1 \otimes \bigwedge^{s+1} \alg g^*.
\end{align}
where $d$ is the deRhan cohomology differential applied to the first factor, and, similarly, $s'$ is the \emph{the Chevalley-Eilenbert differential on the $\alg g$-valued forms $\bigwedge \alg g^* \otimes \alg g$} applied to the second factor. This last mapping satisfies
\begin{align*}
    s': \Omega^\bullet(TM) \otimes \left(\bigwedge^\bullet \alg g^* \otimes \alg g\right) &\to \Omega^\bullet(TM) \otimes \left(\bigwedge^{\bullet+1} \alg g^* \otimes \alg g\right),
\end{align*} when applied to functions $\eta_i \in C^\infty(M, \alg g)$, gives the $\alg g$-valued function
\begin{multline}
    s'\omega(\sect \eta_1, \cdots, \sect \eta_{p+1}) := \sum_{i = 0}^{p+1} (-1)^i [g_i, \omega(\sect \eta_1, \cdots, \overset{\vee}{\sect \eta_i}, \cdots, \sect \eta_{p+1})] \\
    \sum_{1 \leq i < j \leq p+1} (-1)^{i+j} \omega([\sect \eta_i, \sect \eta_j], \sect \eta_1, \cdots, \overset{\vee}{\sect \eta_i}, \cdots, \overset{\vee}{\sect \eta_j}, \cdots \sect \eta_{p+1}).
\end{multline}

Finally, \rtext{$\left( \Omega^\bullet(TM) \otimes \left(\bigwedge^\bullet \alg g^* \otimes \alg g\right), \hat d \right)$ is a graded differential module over $\Omega^\bullet(M)$ under the $\wedge$ product, isomorphic to $(\Omega^\bullet(TM \times \alg g, M \times \alg g), \hat d)$}.
\end{theorem}

\ptext{Due to this isomorphism, from now on we will call $(\Omega^\bullet(TM) \otimes \left(\bigwedge^\bullet \alg g^* \otimes \alg g\right), \hat d)$ the space of $\alg g$-valued forms on the trivial Lie algebroid $TM \times \alg g$, and $\Omega^\bullet(TM)$, $\Omega^\bullet(TM)\otimes \alg g$, $C^\infty(M, \bigwedge^\bullet \alg g*)$ and $C^\infty(M, \bigwedge^\bullet \alg g*)\otimes \alg g$ as subspaces.}

\begin{proof}
To see:

See proofs of the previous 2 results.

\end{proof}

TODO: typical element of both differential algebras given a basis dual to a basis of the Lie algebra.
%%%%%%%%%%%%%%%%%%%%%%%%%%%%%%%%%%%%%%%%%%%%%%%%%%%%%%%%%%%%%%%%%%%%%%%%%%%%%%%%%%%%
%%%%%%%%%%%%%%%%%%%%%%%%%%%%%%%%%%%%%%%%%%%%%%%%%%%%%%%%%%%%%%%%%%%%%%%%%%%%%%%%%%%%
%%%%%%%%%%%%%%%%%%%%%%%%%%%%%%%%%%%%%%%%%%%%%%%%%%%%%%%%%%%%%%%%%%%%%%%%%%%%%%%%%%%%
%%%%%%%%%%%%%%%%%%%%%%%%%%%%%%%%%%%%%%%%%%%%%%%%%%%%%%%%%%%%%%%%%%%%%%%%%%%%%%%%%%%%
\section{Local Description of ($\bb K$ and $L$ valued) forms on Transitive Lie Algebroids}

\begin{itemize}
    
\item Local trivialization of a form $\omega \mapsto \omega_{loc}$

\item $d_{TLA}\omega_{loc} = (d \omega)_{loc}$

\item $\hat \alpha^i_j$ for both real and $L$ valued forms.

\item Respects $d$. So $\hat \alpha$ is isomorphism of differential graded algebras.

\item $G^i_j$, matrix representation of $\hat \alpha^i_j$. Real valued, and $L$-valued:
    \begin{align}
        \omega^i_{\mu_1 \cdots a_1 \cdots a_s} &= G\cdots G \alpha^i_j(\omega^j_{\mu_1 \cdots b_1}) \\
        \omega^i_{\mu_1 \cdots a_1 \cdots a_s} &= G\cdots G \omega^j_{\mu_1 \cdots b_1}
    \end{align}

\item For Atiyah... for example, how does $G$ look. $\omega^i_{loc} = s_i^* \hat \omega$
    
\end{itemize}



Let $0 \to L \xrightarrow{j} A \xrightarrow{a} TM \to 0$ be an Atiyah sequenece of the transitive Lie algebroid $A$ over $M$. Let $\{(U_i, \psi_i: U_i \times \alg g \to L|_{U_i}, \Theta^i: TU_i \to A|_{U_i})\}_{i \in I}$ be a Lie algebroid atlas for $A$.

\begin{definition}
Let $\omega \in \Omega^q(A)$.
    \begin{itemize}
    
    \item For each $i \in I$ define the local $q$-form $\omega^i_{loc}$ by
    \begin{align}
        \omega^i_{loc} = \omega \circ \Theta^i  \quad \in \Omega^q(TU_i\times \alg g)
    \end{align}
    
    \item A \emph{family of trivializations of $\omega$} is a set $\{\omega^i_{loc} \in \Omega^q(TU_i \times \alg g)\}$.
    
    \end{itemize}

\end{definition}

\begin{definition}
Let $\omega \in \Omega^q(A, L)$.
    \begin{itemize}
    
    \item For each $i \in I$ define the local $q$-form $\omega^i_{loc}$ by
    \begin{align}
        \omega^i_{loc} = \psi_i^{-1} \circ \omega \circ \Theta^i  \quad \in \Omega^q(TU_i\times \alg g, U_i \times \alg g) \equiv \Omega^q_{TLA}(U_i, \alg g)
    \end{align}
    
    \item A \emph{family of trivializations of $\omega$} is a set $\{\omega^i_{loc} \in \Omega^q_{TLA}(U_i, \alg g)\}$.
    
    \end{itemize}

\end{definition}

\begin{proposition}
Let $U$ be a Lie algebroid trivializing neighborhood of $A$ \textbf{that also trivializes $M$}, with associated trivializing maps $\psi: U \times \alg g \to L|_{U}$, $\Theta: TU \to A_{U}$, and let $\varphi = (x^1, \dots, x^m): U \to \RR^m$ be a coordinate map of $M$, and let $\set{E_a}_{a= 1, \dots, n}$ be a basis for $\alg g$. Then 
\begin{multline}
    \sectoid A_1, \dots, \sectoid A_m, \sectoid A_{m+1}, \dots, \sectoid A_{m+n} =\\
    \{S\left(\pder{x^1}\right), \dots, S\left(\pder{x^m}\right), S(\stilde{E_1}), \dots, S(\stilde{E_m}))\} \subset \Gamma_U(A)
\end{multline}
is a local frame for $A$.
\end{proposition}

\begin{proof}
For every $m \in U$, $S$ is fiberwise a vector space isomorphism between $T_m M \oplus \alg g$ and $A_m$, so the result follows since $\{\pder{x^1}_m, \dots, \pder{x^m}_m, E_1, \dots, E_m\}$ is a basis of $T_m M \oplus \alg g$.
\end{proof}

\begin{theorem}
Let $U$ be a trivializing neighborhood of $A$ that also trivializes $M$. The map
\begin{eqnsplit}
\cdot_{loc} : (\Omega^\bullet_U(A), \wedge, \hat d_A) &\to (\Omega^\bullet(TU \oplus (U \times \alg g)), \wedge, \hat d_{TU \times \alg g})
\end{eqnsplit}
is an isomorphism of differential graded algebras. In particular
\begin{equation}
    (\hat d_A \omega)_{loc} = \hat d_{TU \times \alg g} \omega_{loc}.
\end{equation}
\end{theorem}
\begin{proof}
They are isomorphic as $C^\infty(U)$-modules for each $\Omega^p_U$: easy

They are isomorphic as $\bb K$-algebras: easy

The differential ``commutes'' with loc: using the ``Koszul'' formula for the differential, it is necessary to use both that $S$ respects the anchor and the Lie bracket, i.e. \textbf{that $S$ is a Lie algebroid morphism}.
\end{proof}

\begin{theorem}
Let $U$ be a trivializing neighborhood of $A$. The map
\begin{eqnsplit}
\cdot_{loc} : (\Omega^\bullet_U(A, L), \wedge, \hat d) &\to (\Omega^\bullet(TU \oplus (U \times \alg g), U \times \alg g), \wedge, \hat d_{TLA})
\end{eqnsplit}
is an isomorphism of differential graded algebras. In particular
\begin{equation}
    (\hat d \omega)_{loc} = \hat d_{TLA} \omega_{loc}.
\end{equation}
\end{theorem}
\begin{proof}
The proof is identical to that of the previous theorem, except for the last part. To see that the differential commutes it is neceessary to show that the induced representation of $TLA$ on $U \times \alg g$ via $ad: A|_U \to D_{Der}(U \times \alg g)$ is the $ad$ representation. This will follow from the fact that $\tilde psi$ respects the Lie bracket, i.e. $\psi$ is a trivialization of a LAB.
\end{proof}
\linea 

Let $\omega \in \Omega^q(A, L)$ and let $\{\sectoid X_k\} \subset \Gamma(A)$ for $k = 1, \dots, q$. Let each $\sectoid X_k$ have a family of trivializations $\{\sect X_k \oplus \stilde \eta_k^i \in \Gamma(TU_i) \oplus C^\infty(U_i, \alg g)\}$, then, in $U_{ij} = U_i \inter U_j \neq \empty$
\begin{align*}
    \omega^i_{loc}(X_1 \oplus \stilde \eta_1^i, \cdots, X_q \oplus \stilde \eta_q^i) = \alpha^i_j \circ \omega^j_{loc}(X_1 \oplus \stilde \eta_1^j, \cdots, X_q \oplus \stilde \eta_q^j)
\end{align*}
which can be expressed as
\begin{align}
    \omega^i_{loc} = \alpha^i_j \circ \omega^j_{loc} \circ s^j_i
\end{align}

\begin{definition}
    \begin{itemize}
    
    \item Let $\omega \in \Omega^q(A, L)$. Define
    \begin{eqnsplit}
    \hat \alpha^i_j: \Omega^q_{TLA}(U_{ij}, \alg g) &\to \Omega^q_{TLA}(U_{ij}, \alg g)\\
                    \omega^j_{loc} &\mapsto \omega^i_{loc} = \alpha^i_j \circ \omega^j_{loc} \circ s^j_i
    \end{eqnsplit}
    
    \item Let $\omega \in \Omega^q(A)$. Define
    \begin{eqnsplit}
    \hat \alpha^i_j: \Omega^q(TU_{ij}\times \alg g) &\to \Omega^q(TU_{ij}\times \alg g)\\
                    \omega^j_{loc} &\mapsto \omega^i_{loc} = \omega^j_{loc} \circ s^j_i
    \end{eqnsplit}
    
\end{itemize}

\end{definition}

\begin{theorem}
$\alpha^i_j: \Omega^q_{TLA}(U_{ij}, \alg g) \to \Omega^q_{TLA}(U_{ij}, \alg g)$ and $\alpha^i_j: \Omega^q(TU_{ij}\times \alg g) \to \Omega^q(TU_{ij}\times \alg g)$ are isomorphisms of differential graded algebras.
\end{theorem}
\begin{proof}
Recall that, since a differential $d$ in a differential algebra is a local operator, $(d\omega)|_U = d|_U \omega|_U$; we will omit the $|_U$ besides the differentials for ease of reading.

We only show the result for the first $\hat \alpha^i_j$ map, since the other one follows from an identical calculation.

The interesting part is the commutation of $\hat \alpha^i_j$ with the differential, i.e.
\begin{equation*}
    \hat d_{TLA} \hat \alpha^i_j(\omega^j_{loc}|_{U_{ij}}) = \hat \alpha^i_j(\hat d_{TLA} (\omega^i_{loc}|_{U_{ij}})),
\end{equation*} which follows from the following calculation:
\begin{align*}
    \hat d_{TLA} \hat \alpha^i_j(\omega^j_{loc}|_{U_{ij}})
    &= \hat d_{TLA} (\omega^i_{loc}|_{U_{ij}}) \\
    &= (\hat d \omega)^i_{loc}|_{U_{ij}} & \text{since $\cdot_{loc}$ is isomorphism of differential algebras}\\
    &= \hat \alpha^i_j((\hat d \omega)^j_{loc}|_{U_{ij}}) \\
    &= \hat \alpha^i_j(\hat d_{TLA} (\omega^j_{loc}|_{U_{ij}})).
\end{align*}
\end{proof}
\linea 

Let $\set{E_a}_{1 \leq a \leq n}$ be a basis of the Lie algebra $\alg g$, and let $\set{\theta^a}$ be its dual basis.

\begin{definition}
    ${G_i^j}_a^b(m) = \theta^b \circ \alpha^i_{j, m}(E_a)$
\end{definition}

\begin{theorem}
Let$U_i, U_j \subset M$ trivializing neighborhoods for the transitive Lie algebroid $A$ with $U_{ij} \neq \empty$ and let $\phi = (x^1, \dots, x^m) \to \RR^m$ be a coordinate map for $M$. Then, for real valued forms on $A$:
\begin{align}
    \omega^i_{\mu_1 \cdots \mu_r a_1 \dots a_s} = {G^i_j}^{b_1}_{a_1} \cdots {G^i_j}^{b_s}_{a_s} \omega^i_{\mu_1 \cdots \mu_r b_1 \dots b_s}
\end{align}

Similarly, for $L$-valued forms on $A$:
\begin{align}
    \omega^i_{\mu_1 \cdots \mu_r a_1 \dots a_s} = {G^i_j}^{b_1}_{a_1} \cdots {G^i_j}^{b_s}_{a_s} \alpha^i_j(\omega^i_{\mu_1 \cdots \mu_r b_1 \dots b_s})
\end{align}
\end{theorem}
\begin{proof}

\end{proof}
\begin{remark}
Notice that a single coordinate map of $U_{ij} \subset M$ is used in the theorem, explaining wh
\end{remark}

TODO: how does G look for the Atiyah Lie algebroid
%%%%%%%%%%%%%%%%%%%%%%%%%%%%%%%%%%%%%%%%%%%%%%%%%%%%%%%%%%%%%%%%%%%%%%%%%%%%%%%%%%%%
%%%%%%%%%%%%%%%%%%%%%%%%%%%%%%%%%%%%%%%%%%%%%%%%%%%%%%%%%%%%%%%%%%%%%%%%%%%%%%%%%%%%
%%%%%%%%%%%%%%%%%%%%%%%%%%%%%%%%%%%%%%%%%%%%%%%%%%%%%%%%%%%%%%%%%%%%%%%%%%%%%%%%%%%%
%%%%%%%%%%%%%%%%%%%%%%%%%%%%%%%%%%%%%%%%%%%%%%%%%%%%%%%%%%%%%%%%%%%%%%%%%%%%%%%%%%%%
\section{Gauge group, Infinitesimal Gauge transformations and Infinitesimal gauge action on forms}

(Lazzarini 2012) Perhaps this is not the best place, but try, specially to understand which are definitions and which are somehow forced definitions.

%%%%%%%%%%%%%%%%%%%%%%%%%%%%%%%%%%%%%%%%%%%%%%%%%%%%%%%%%%%%%%%%%%%%%%%%%%%%%%%%%%%%
%%%%%%%%%%%%%%%%%%%%%%%%%%%%%%%%%%%%%%%%%%%%%%%%%%%%%%%%%%%%%%%%%%%%%%%%%%%%%%%%%%%%
%%%%%%%%%%%%%%%%%%%%%%%%%%%%%%%%%%%%%%%%%%%%%%%%%%%%%%%%%%%%%%%%%%%%%%%%%%%%%%%%%%%%
%%%%%%%%%%%%%%%%%%%%%%%%%%%%%%%%%%%%%%%%%%%%%%%%%%%%%%%%%%%%%%%%%%%%%%%%%%%%%%%%%%%%
\section{Atiyah Lie Algebroid alternative Description?}

There seem to be $2$, related, alternative descriptions (Lazzarini2012):
    
    \begin{itemize}
        
    \item $(\Omega(A, L), \hat d)$ a $(\Omega_{TLA}(P, \mathfrak g)_{equ}, \hat d_{TLA})$,
    
    \item and $(R, Ad)$-equivariant forms in $(\Omega(P) \times \mathfrak g, d)$. Denoted by: \[ (\Omega_{Lie}(P, \mathfrak g), \hat d) \]
        
    \end{itemize}

Recall: Space of $k$-forms on $M$ with values in $P \times V/G$ $\longleftarrow$ Space of $G$-equivariant and \emph{horizontal} $V$-valued $k$-forms on $P$.
