Fernandes remarks that it is thanks to the Lie braket that it is possible to make this kind of Cartan calculi.

Why is it necessary? What does this mean? What does it allow?

What is the difference, if any, with connection theory?

Relation between exterior derivative and usual covariant derivative:
    \begin{itemize}
    
    \item Finer than $d$: requiring $TM$ necessarily, because it is based on a connection $TM \to \mathcal D(E)$
    
    \item Coarser than $d$: does not require a representation, a connection doesn't have to be a homomorphism of Lie algebras.
        
    \end{itemize}

\section{Differential Structures}
\section{Trivial Lie Algebroid}
\section{Local Description of (real and $L$ valued) forms on Transitive Lie Algebroids}
\section{Atiyah Lie Algebroid alternative Description?}
\section{Cartan operations?}

