Fernandes remarks that it is thanks to the Lie braket that it is possible to make this kind of Cartan calculi.

Why is it necessary? What does this mean? What does it allow?

What is the difference, if any, with connection theory?

Relation between exterior derivative and usual covariant derivative:
    \begin{itemize}
    
    \item Finer than $d$: requiring $TM$ necessarily, because it is based on a connection $TM \to \mathcal D(E)$
    
    \item Coarser than $d$: does not require a representation, a connection doesn't have to be a homomorphism of Lie algebras.
        
    \end{itemize}

Why relevant to me:
    \begin{itemize}
        
    \item Vector calculus?
    
    \item What can be integrated. Inner integration generates usual (vector bundle valued) forms on the base manifold.
    
    \item Algebroid connections are ``normalized'' elements $ \omega \in \Omega^1(A, L)$, and their curvature is $d\omega \in \Omega^2(A, L)$. Generalized connections are precisely the elements of $\Omega^1(A, L)$.
    
    \item When talking about matter fields, a generalized connection $\hat \omega$ induces covariant derivatives $\hat \nabla^E: \Gamma(E) \to \mathbf{\Omega^1(A, E)}$, so that $\hat \nabla^E \phi \in \Omega^1(A, E)$.
    
    \item $\Omega^\bullet(A)$ is used to do cohomology on Lie algebroids
    
    \item In each $\Omega^p(A, L)$ the natural scalar product $(\omega, \eta) = \cl{\omega, \star \eta} = \int_A h(\omega, \star \eta)$, for $h$ in inner metric, is what will be used to define the gauge action:
        
        \begin{itemize}
            
        \item For $\hat \omega \in \Omega^1(A, L)$, define $S_{Gauge}[\hat \omega] = (\hat R, \hat R)$
        
        \item For $\phi \in \Gamma(E)$ and $\hat \omega$ as above, $S_{Matter}[\phi, \hat \omega] := (\hat \nabla^E \phi, \hat \nabla^E \phi)$
        
        \end{itemize}
        
    \end{itemize}

TODO\todo{check this}: the third ``important example'' introduced in Lazzarini is important to me? $A$ connections on $E$ are used to define finite gauge transformations, which somehow induce the infinitesimal gauge transformation definition on generalized connections? If not for this, $A$-connections, and so this examples, are of no relevance to me.

%%%%%%%%%%%%%%%%%%%%%%%%%%%%%%%%%%%%%%%%%%%%%%%%%%%%%%%%%%%%%%%%%%%%%%%%%%%%%%%%%%%%
%%%%%%%%%%%%%%%%%%%%%%%%%%%%%%%%%%%%%%%%%%%%%%%%%%%%%%%%%%%%%%%%%%%%%%%%%%%%%%%%%%%%
%%%%%%%%%%%%%%%%%%%%%%%%%%%%%%%%%%%%%%%%%%%%%%%%%%%%%%%%%%%%%%%%%%%%%%%%%%%%%%%%%%%%
%%%%%%%%%%%%%%%%%%%%%%%%%%%%%%%%%%%%%%%%%%%%%%%%%%%%%%%%%%%%%%%%%%%%%%%%%%%%%%%%%%%%
\section{Differential Structures and Cartan Operations}

%%%%%%%%%%%%%%%%%%%%%%%%%%%%%%%%%%%%%%%%%%%%%%%%%%%%%%%%%%%%%%%%%%%%%%%%%%%%%%%%%%%%
%%%%%%%%%%%%%%%%%%%%%%%%%%%%%%%%%%%%%%%%%%%%%%%%%%%%%%%%%%%%%%%%%%%%%%%%%%%%%%%%%%%%
\subsection{Differential Algebras}

\begin{itemize}

\item Alt

\item $\Omega$

\item $d$. Is antiderivation

\item $(\Omega, d)$ Graded differential algebras. Differential calculus (over a commutative algebra?) ?
    
    \begin{itemize}
    
    \item $(\Omega(A), \hat d_A)$ is graded commutative differential algebra
    
    \item $(\Omega(A, L), \hat d)$ is graded differential Lie algebra.
    
    \end{itemize}

\item Respect further structure... of $L$?

\end{itemize}

\linea 

Throughout this section, let $A$ be a Lie algebroid over the manifold $M$ with anchor $a$.

\begin{definition}
Let $E$ be a vector bundle over $M$. 

    \begin{itemize}
    
    \item Let \emph{$\Alt^0(A, E)$}$:= E$.
    
    \item For $n \in \ZZ_{\geq 1}$, let \emph{$\Alt^n(A, E)$} be the natural embedding, as alternating maps, of the hom-bundle $\Hom(\bigwedge^n A, E)$ into the vector bundle of $\Hom^n(A, E)$ over $M$. Its fiber at $m \in M$ is the vector space of alternating linear transformations $\{\omega: \underbrace{A_m \times \cdots \times A_m}_{\text{$n$ times}} \to E_m \st \text{$\omega$ is $\RR$-linear and alternating}\}$.
    
    \item Let \emph{$\Alt^\bullet(A, E)$}$:= \bigoplus^\infty_{n = 0} \Alt^n(A, E)$.
    % OJO: i can't call this forms, since forms are either C^\infty M multilinear, or complete vector bundle morphisms!
    
    %\item For $n \in \ZZ_{\geq 0}$, let \emph{$\Alt^n(A)$} and \emph{$Alt^\bullet (A)$} be the application of the previous definitions to the trivial vector bundle $E = M \times \RR$.
        
    \end{itemize}
    
\end{definition}

Notice that $\Alt^\bullet(A, E)$ is a vector bundle with finite dimensional fibers since, due to the antisymmetry property, $\Alt^n(A, E) = 0$ for all values of $n$ greater than some fixed value $N \in \ZZ > 0$.

\begin{definition}
Let $E$ be a vector bundle over $M$ on which there is a representation $\phi: A \to \alg D(E)$ of $A$.

    \begin{itemize}
    
    \item For $n \in \ZZ_{\geq 0}$, define the $C^\infty(M)$-module \emph{$\Omega^n(A, E)$}$:= \Gamma\Alt^n(A, E)$. Its elements are called \emph{$E$-valued $n$-forms on $A$}.
    
    \item Define the $C^\infty(M)$-module $\emph{\Omega^\bullet(A, E)} := \bigoplus^\infty_{n = 0} \Omega^n(A, E)$. Its elements are simply called \emph{$E$-valued forms on $A$}.
    
    %\item \item For $n \in \ZZ_{\geq 0}$, let \emph{$\Omega^n(A)$}$:= \Gamma\Alt^n(A)$. 
    
    %\item Define the $C^\infty(M)$-module $\emph{\Omega^\bullet(A)} := \bigoplus^\infty_{n = 0} \Omega^n(A)$.
        
    \end{itemize}
    
\end{definition}

\begin{remark}
Due to the bijective correspondence between $C^\infty(M)$-linear maps between sections of vector bundles and vector bundle morphisms, $\Omega^n(A, E)$ may be considered the space of $C^\infty(M)$-multilinear antisymmetric maps from $\Gamma(A)^n$ to $\Gamma(E) = \Omega^0(A, E)$. We will refer as \ptext{$E$-valued $n$-forms on $A$} to both elements
\begin{align}
    \omega &: \underbrace{A \oplus \cdots \oplus A}_{\text{$n$ times}} \to E & \text{alternating $\RR$-vector bundle morphism}\\
    \omega &: \underbrace{\Gamma(A) \times \cdots \times \Gamma(A)}_{\text{$n$ times}} \to \Gamma(E) & \text{alternating $C^\infty(M)$-multilinear map}.
\end{align}
    
    %\item $\Omega^n(A)$ may be considered the space of $C^\infty(M)$-multilinear antisymmetric maps from $\Gamma(A)^n$ to $C^\infty(M)$.
    
\end{remark}

\begin{theorem}
    \begin{itemize}
    
    \item $\Alt^n(A, E)$ and $E \otimes \bigwedge^n A^*$ are isomorphic vector bundles.
    
    %\item $\Alt^n(A)$ and $\bigwedge^n A^*$ are isomorphic vector bundles.
    
    \item $\Omega^n(A, E)$ and $\Gamma(E) \otimes \Gamma\Alt^n A^*$ are isomorphic $C^\infty(M)$-modules.
    
    %\item $\Omega^n(A)$ and $\Gamma\Alt^nA^*$ are isomorphic $C^\infty(M)$-modules.
    
    \end{itemize}
\end{theorem}

\begin{proof}
$\Alt^n(A, E) \cong \Hom(\bigwedge^n A, E)$ by definition, and $\Hom(\bigwedge^n A, E) \cong E \otimes (\bigwedge^n A)* \cong E \otimes \bigwedge^n A*$.

Since $\Gamma$ is a strong monoidal function\todo{can I make this more clear?}, it distributes over the tensor product, and so the result follows from the first part of the theorem.
\end{proof}

% Only define things for A, E, and THEN do define the one without E
\begin{definition}
Let $E$ be a Lie algebra bundle over $M$, i.e. a vector bundle such that each fiber is an algebra and such that there exists an atlas compatible with the algebra multiplication, on which $A$ is represented. Define on $\Omega^\bullet(A, E)$ the $C^\infty(M)$-bilinear operation operation, called \emph{the wedge product of LAB-valued forms on $A$}
\begin{eqnsplit*}
\wedge : \Omega^\bullet(A, E) &\times \Omega^\bullet(A, E) \to \Omega\bullet(A, E)
\end{eqnsplit*}
as the linear extension of the maps, for any $p, q \in \ZZ_{\geq 0}$
\begin{eqnsplit*}
\wedge : \Omega^p(A, E) \times \Omega^q(A, E) &\to \Omega^{p+q}(A, E)
\end{eqnsplit*}
\begin{eqnsplit}
(\omega \wedge \eta)(\sectoid X_1, \dots, \sectoid X_{p+q}) &:= \\
\frac{1}{p!q!} \sum_{\sigma \in S_{p+q}} &(-1)^{\sigma} \omega(\oid X_{\sigma(1)}, \cdots, \oid X_{\sigma(p)}) \bullet \eta(\oid X_{\sigma(p+1)}, \cdots, \oid X_{\sigma(p+q)}).
\end{eqnsplit}
for any $p$-form $\omega$, $q$-form $\eta$ and any $\sectoid X_1, \dots, \sectoid X_{p+q} \in \Gamma(A)$, where $\bullet$ is the algebra multiplication induced on $\Gamma(E)$ by the fiberwise algebra multiplication in $E$.
\end{definition}

\begin{definition}
Let $E$ be an algebra bundle over $M$ on which there is a representation $\phi: A \to \alg D(E)$ of $A$. Define \emph{the differential of $E$-valued forms on $A$} by the Koszul formula:
\begin{eqnsplit*}
\hat d_\phi: \Omega^{\bullet}(A, E) &\to \Omega^{\bullet+1}(A, E)
\end{eqnsplit*}
\begin{multline}
(\hat d_\phi \omega)(\sectoid X_1, \dots, \sectoid X_{p+1}) = \sum_{i=1}^{p+1} (-1)^{i+1} \phi(\sectoid X_i)\cdot \omega(\sectoid X_1, \cdots, \overset{\vee}{\sectoid X_i}, \cdots, \sectoid X_{p+1}) \\
 + \sum_{1 \leq i < j \leq p+1} (-1)^{i+j}\omega([\sectoid X_i, \sectoid X_j], \sectoid X_1, \cdots, \overset{\vee}{\sectoid X_i}, \cdots, \overset{\vee}{\sectoid X_j}, \cdots, \sectoid X_{p+1})
\end{multline}
for any $p \in \ZZ_{\geq 0}$, $p$-form $\omega$, and any $\sectoid X_1, \dots, \sectoid X_{p+1} \in \Gamma(A)$.
\end{definition}

\begin{definition}
Let $\mathcal A = \oplus_{n = 0}^\infty \mathcal A^n$ be a graded algebra (over $\RR$).
    \begin{itemize}
    
    \item An element $\omega \in \mathcal A^n$ is called a \emph{homogeneous element of degree $n$}, for $n \in \ZZ_{\geq 0}$, and its degree is denoted by $|\omega|$.
    
    \item An \emph{antiderivation on $\mathcal A$} is an $\RR$-linear map $d: \mathcal A \to \mathcal A$ such that, if $\omega$ is homogeneous, 
    \begin{align*}
        d(\omega \, \eta) = (d\omega)\tau + (-1)^{|\omega|} \omega (d\tau)
    \end{align*}
    
    \item An antiderivation $d: \mathcal A \to \mathcal A$ is \emph{of degree $m$} if, for all homogeneous elements $\omega \in \mathcal A$,
    \begin{align*}
        |d\omega| = |\omega| + m.
    \end{align*}
    If the antiderivation is of degree $1$ or $-1$, it is called a \emph{differential on $\mathcal A$}. The same definitions apply for $M = \oplus_{n = 0}^\infty M^n$ graded module over the graded ring $R = \oplus_{n = 0}^\infty R^n$, i.e. an $R$-module such that $R^p \cdot M^q \subset M^{p+q}$.
    
    \item Let $d: \mathcal A \to \mathcal A$ be a derivation. $(A, d)$ is a \emph{differential graded algebra} if $d \circ d = 0$. 
    
    \item Similarly, let $M = \oplus_{n = 0}^\infty M^n$ be a graded module and $d: M \to M$ a differential on $M$. $(M, d)$ is called a \emph{differential graded module} if $d \circ d = 0$
    
    \item Let $(\mathcal A, \bullet, d)$ and $(\mathcal A', \bullet', d')$ be differential graded algebras. A \emph{morphism of differential graded algebras} is a graded algebra morphism $\phi: \mathcal A \to \mathcal A'$ such that $d' \circ \phi = \phi \circ d$.
    
    \end{itemize}
    
\end{definition}

\begin{theorem}
Let $E$ be an algebra bundle over $M$ on which there is a representation $\phi: A \to \alg D(E)$ of $A$. Then $(\Omega^\bullet(A, E), \wedge, \hat d_\phi)$ is a differential graded algebra.
\end{theorem}

\begin{proof}
To see:

Associativity... just say it is, alhthough, is there an easy way to see it?

$d^2 = 0$. Apparently this follows from the fact that $\phi$ respects the Lie algebroid bracket.

Graded Leibniz
\end{proof}

The following instances of $(\Omega^\bullet(A, E), \wedge, \hat d_\phi)$ will be important to us in future chapters.

\begin{definition}
On the trivial vector bundle $M \times \RR$, let $A$ act by the trivial action $\phi^0: A \to \alg D(E)$, i.e. considering $\Gamma(M \times \RR)$ as $C^\infty(M)$ and letting the representation be the anchor. %$\phi^0(\sectoid X)f = a(\sectoid X)(f)$, for any $\sectoid X \in \Gamma(A)$ and $f \in C^\infty(M)$.
Define \[(\Omega^\bullet(A), \wedge, \hat d_A)\] to be the graded differential algebra be $(\Omega^\bullet(A, M \times \RR), \wedge, \hat d_{\phi^0})$, and call its elements \emph{the real valued forms on $A$}.
\end{definition}

\begin{theorem}
$(\Omega^\bullet(A), \wedge, \hat d_A)$ is a graded-commutative differential graded algebra.
\end{theorem}

\begin{proof}
To see:

$xy = (-1)^{|x||y|}yx$. This follows from $\RR$ being commutative.
\end{proof}


\begin{definition}
Let $0 \to L \xrightarrow{j} A \xrightarrow{a} TM \to 0$ be a transitive Lie algebroid sequence.
Define \[(\Omega^\bullet(A, L), \wedge, \hat d)\] to be the graded differential algebra be $(\Omega^\bullet(A, L), \wedge, \hat d_{ad})$, where $ad: A \to \alg D(L)$, $\oid X \mapsto j^{-1}[\oid X, \cdot]$ is the adjoint action defined in \ref{defnAdjointAct}.
\end{definition}

\begin{theorem}
$(\Omega^\bullet(A), \wedge, \hat d_A)$ is a Lie-graded differential graded algebra.
\end{theorem}

\begin{proof}
To see:

\end{proof}

\begin{proposition}
$(\Omega^\bullet(A, L), \wedge, \hat d)$ is a differential graded module over the graded ring $(\Omega^\bullet(A), \wedge)$, through the linear extension of the action:
\begin{align*}
    \wedge: \Omega^\bullet(A) \times \Omega^\bullet(A, L) &\to \Omega^\bullet(A, L)
\end{align*}
\begin{eqnsplit}
(\omega \wedge \eta)(\sectoid X_1, \dots, \sectoid X_{p+q}) &:= \\
\frac{1}{p!q!} \sum_{\sigma \in S_{p+q}} &(-1)^{\sigma} 
\underbrace{\omega(\oid X_{\sigma(1)}, \cdots, \oid X_{\sigma(p)})}_{\in \,\Gamma(M \times \RR)} 
\cdot 
\underbrace{\eta(\oid X_{\sigma(p+1)}, \cdots, \oid X_{\sigma(p+q)})}_{\in \, \Gamma(L)}
\end{eqnsplit}
for any real valued $p$-form $\omega$, $L$-valued $q$-form $\eta$ and any $\sectoid X_1, \dots, \sectoid X_{p+q} \in \Gamma(A)$.
\end{proposition}

\begin{proof}
To see:

$\Omega^p(A) \wedge \Omega^q(A, L) \subset \Omega^{p+q}(A, L)$.
\end{proof}
%%%%%%%%%%%%%%%%%%%%%%%%%%%%%%%%%%%%%%%%%%%%%%%%%%%%%%%%%%%%%%%%%%%%%%%%%%%%%%%%%%%%
%%%%%%%%%%%%%%%%%%%%%%%%%%%%%%%%%%%%%%%%%%%%%%%%%%%%%%%%%%%%%%%%%%%%%%%%%%%%%%%%%%%%
\subsection{Cartan Operations}

% \begin{itemize}

% \item $i, L$

% \item Horizontal, invariant and basic forms.

% \end{itemize}

\begin{definition}
Let $E$ be an algebra bundle over $M$, let $\phi: A \to \alg D(E)$ be a representation of $A$, and let $B$ be a Lie algebroid over $M$. A \emph{Cartan operation on $(\Omega^\bullet(A, E), \wedge, \hat d_\phi)$} is a triple $(B, i, L)$ such that: for any $\sectoid X \in B$, for any $p \geq 1$ there is a map $i_{\sectoid X}: \Omega^p(A, E) \to \Omega^{p-1}(A, E)$ such that, defining 
\begin{align}
L_{\sectoid X} := \hat d_\phi i_{\sectoid X} + i_{\sectoid X}\hat d_\phi,
\end{align}
the following relations hold for any $\sectoid X, \sectoid Y \in B$ and $f \in C^\infty(M)$:
\begin{align}
    i_{f \sectoid X} = f i_{\sectoid X} & i_{\sectoid X} i_{\sectoid Y} + i_{\sectoid Y} i_{\sectoid X} = 0 \\
    [L_{\sectoid X}, i_{\sectoid Y}] = i_{[\sectoid X, \sectoid Y]} & [L_{\sectoid X}, L_{\sectoid Y}] = L_{[\sectoid X, \sectoid Y]}
\end{align}
\end{definition}

\begin{example}
$B = A$ defines one.
\end{example}

\begin{example}
$B = L$ also defines one.
\end{example}

\begin{definition}
Let $(B, i, L)$ be a Cartan operation on $(\Omega^\bullet(A, E), \wedge, \hat d_\phi)$.
    
    \begin{itemize}
        
    \item Define $\Omega^\bullet(A, E)_{Hor}$ to be the graded subspace of \emph{horizontal elements in $\Omega^\bullet(A, E)$} characterized by 
    \[
    i_{\sectoid X} \omega = 0?
    \]
    
    \item Define $\Omega^\bullet(A, E)_{Inv}$ to be the graded subspace of \emph{invariant elements in $\Omega^\bullet(A, E)$} characterized by 
    \[
    L_{\sectoid X} \omega = \omega?
    \]
    
    \item Define $\Omega^\bullet(A, E)_{Basic} = \Omega^\bullet(A, E)_{Hor} \inter \Omega^\bullet(A, E)_{Inv}$ to be the graded subspace of \emph{basic or tensorial elements in $\Omega^\bullet(A, E)$}.
        
    \end{itemize}
\end{definition}
%%%%%%%%%%%%%%%%%%%%%%%%%%%%%%%%%%%%%%%%%%%%%%%%%%%%%%%%%%%%%%%%%%%%%%%%%%%%%%%%%%%%
%%%%%%%%%%%%%%%%%%%%%%%%%%%%%%%%%%%%%%%%%%%%%%%%%%%%%%%%%%%%%%%%%%%%%%%%%%%%%%%%%%%%
%%%%%%%%%%%%%%%%%%%%%%%%%%%%%%%%%%%%%%%%%%%%%%%%%%%%%%%%%%%%%%%%%%%%%%%%%%%%%%%%%%%%
%%%%%%%%%%%%%%%%%%%%%%%%%%%%%%%%%%%%%%%%%%%%%%%%%%%%%%%%%%%%%%%%%%%%%%%%%%%%%%%%%%%%
\section{Trivial Lie Algebroid}

To understand better:
\begin{itemize}
    
\item $(\Omega(A), \hat d_A)$ as $(\Omega(M)\otimes \bigwedge g^*, d + s)$.

\item $(\Omega(A, L), \hat d)$ as $(\Omega(M)\otimes \bigwedge g^* \otimes \mathfrak g, d + s')$ : \[(\Omega_{TLA}(M, \mathfrak g), \hat d_{TLA})\]

\item Generic element in $\Omega^n$, given dual basis $\theta = \set{\theta^a}$ of a basis of $\alg g$ $\set{E_a}$
\begin{align*}
    \omega = \sum_{p + s = n} \omega^\theta_{\mu_1 \cdots a_1 \cdots a_s} dx^{\mu_1} \wedge \cdots \theta^{a_s}
\end{align*}
    
\end{itemize}



%%%%%%%%%%%%%%%%%%%%%%%%%%%%%%%%%%%%%%%%%%%%%%%%%%%%%%%%%%%%%%%%%%%%%%%%%%%%%%%%%%%%
%%%%%%%%%%%%%%%%%%%%%%%%%%%%%%%%%%%%%%%%%%%%%%%%%%%%%%%%%%%%%%%%%%%%%%%%%%%%%%%%%%%%
%%%%%%%%%%%%%%%%%%%%%%%%%%%%%%%%%%%%%%%%%%%%%%%%%%%%%%%%%%%%%%%%%%%%%%%%%%%%%%%%%%%%
%%%%%%%%%%%%%%%%%%%%%%%%%%%%%%%%%%%%%%%%%%%%%%%%%%%%%%%%%%%%%%%%%%%%%%%%%%%%%%%%%%%%
\section{Local Description of (real and $L$ valued) forms on Transitive Lie Algebroids}

\begin{itemize}
    
\item Local trivialization of a form $\omega \mapsto \omega_{loc}$

\item $d_{TLA}\omega_{loc} = (d \omega)_{loc}$

\item $\hat \alpha^i_j$ for both real and $L$ valued forms.

\item Respects $d$. So $\hat \alpha$ is isomorphism of differential graded algebras.

\item $G^i_j$, matrix representation of $\hat \alpha^i_j$. Real valued, and $L$-valued:
    \begin{align}
        \omega^i_{\mu_1 \cdots a_1 \cdots a_s} &= G\cdots G \alpha^i_j(\omega^j_{\mu_1 \cdots b_1}) \\
        \omega^i_{\mu_1 \cdots a_1 \cdots a_s} &= G\cdots G \omega^j_{\mu_1 \cdots b_1}
    \end{align}

\item For Atiyah... for example, how does $G$ look. $\omega^i_{loc} = s_i^* \hat \omega$
    
\end{itemize}

%%%%%%%%%%%%%%%%%%%%%%%%%%%%%%%%%%%%%%%%%%%%%%%%%%%%%%%%%%%%%%%%%%%%%%%%%%%%%%%%%%%%
%%%%%%%%%%%%%%%%%%%%%%%%%%%%%%%%%%%%%%%%%%%%%%%%%%%%%%%%%%%%%%%%%%%%%%%%%%%%%%%%%%%%
%%%%%%%%%%%%%%%%%%%%%%%%%%%%%%%%%%%%%%%%%%%%%%%%%%%%%%%%%%%%%%%%%%%%%%%%%%%%%%%%%%%%
%%%%%%%%%%%%%%%%%%%%%%%%%%%%%%%%%%%%%%%%%%%%%%%%%%%%%%%%%%%%%%%%%%%%%%%%%%%%%%%%%%%%
\section{Gauge group, Infinitesimal Gauge transformations and Infinitesimal gauge action on forms}

(Lazzarini 2012) Perhaps this is not the best place, but try, specially to understand which are definitions and which are somehow forced definitions.

%%%%%%%%%%%%%%%%%%%%%%%%%%%%%%%%%%%%%%%%%%%%%%%%%%%%%%%%%%%%%%%%%%%%%%%%%%%%%%%%%%%%
%%%%%%%%%%%%%%%%%%%%%%%%%%%%%%%%%%%%%%%%%%%%%%%%%%%%%%%%%%%%%%%%%%%%%%%%%%%%%%%%%%%%
%%%%%%%%%%%%%%%%%%%%%%%%%%%%%%%%%%%%%%%%%%%%%%%%%%%%%%%%%%%%%%%%%%%%%%%%%%%%%%%%%%%%
%%%%%%%%%%%%%%%%%%%%%%%%%%%%%%%%%%%%%%%%%%%%%%%%%%%%%%%%%%%%%%%%%%%%%%%%%%%%%%%%%%%%
\section{Atiyah Lie Algebroid alternative Description?}

There seem to be $2$, related, alternative descriptions (Lazzarini2012):
    
    \begin{itemize}
        
    \item $(\Omega(A, L), \hat d)$ a $(\Omega_{TLA}(P, \mathfrak g)_{equ}, \hat d_{TLA})$,
    
    \item and $(R, Ad)$-equivariant forms in $(\Omega(P) \times \mathfrak g, d)$. Denoted by: \[ (\Omega_{Lie}(P, \mathfrak g), \hat d) \]
        
    \end{itemize}

