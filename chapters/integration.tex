% Relevant to me:

% $\int_A := \int_{inner} \int_{outer}$, defined for both forms $\Omega(A, L)$ and $\Omega(A)$ (for the decomposition? Or when is that used?). Used for below:

%     \begin{itemize}
    
%     \item For $\hat \omega \in \Omega^1(A, L)$, define $S_{Gauge}[\hat \omega] = (\hat R, \hat R) = \int_A h(\hat R, \star \hat R)$.
    
%     \item For $\phi \in \Gamma(E)$ and $\hat \omega$ as above, \[S_{Matter}[\phi, \hat \omega] := (\hat \nabla^E \phi, \hat \nabla^E \phi)\]
    
%     \end{itemize}
    
% That means the star is needed for: 
%     \begin{itemize}
        
%     \item $\star \hat R$, where $\hat R \in \Omega^2(A, L)$
    
%     \item $\star \hat \nabla^E \phi$ for $\hat \nabla^E \phi \in \Omega^1(A, E)$
    
%     \item Some other forms of the same type that are generated by the decomposition
        
%     \end{itemize}

{
\color{gray}
    Outline:
    
    \begin{itemize}
    
    \item Inner Metrics and Metrics on Representation Vector Bundle of Transitive Lie Algebroids
    
        \begin{itemize}
            
        \item $h^E$ on $E$ symmetric bilinear form.
        
            \begin{itemize}
                
            \item e.g. Defn: inner metric
            
            \item $h^E_{uv}$
                
            \end{itemize}
        
        \item $h^E$ on $\Omega^\bullet (A, E)$.
            
            \begin{itemize}
                
            \item $h^E(\alpha, \beta) = \alpha^u \wedge \beta^v h_{uv}$ (or, slightly more generally, multiplicatio by $0$ form gets out with $h$)
                
            \end{itemize}
            
        \item $\phi_L$-compatible $E$-metric: $h^E(\phi_L(\eta) \mu_1, \mu_2) + h^E(\mu_1, \phi_L(\eta)\mu_2) = 1$
        
            \begin{itemize}
                
            \item e.g. Defn: Killing metric
                
            \end{itemize}
        
        \item (Needed?) Locally constant metric. Compatible with $d$'s.
        
        \end{itemize}
    
    \item Symmetric Bilinear forms on Transitive Lie Algebroids
    
        \begin{itemize}
            
        \item $\hat g \to h$. Defn: Inner part
        
            \begin{itemize}
                
            \item Defn: Inner nondegenerate metric on $A$
                
            \end{itemize}
        
        \item $g \to \hat g$. Null inner part
        
        Let $\nabla$ be an ordinary connection on $A$. Then:
        
        \item $h \to \hat g$
        
        \item $\hat g \to g$
            
        \item Theorem: Inner nondegenerate bilsymform iff $(g, h, \nabla)$ with $h$ nondeg. (i.e. actual metric) and $\nabla$ ordinary connection. Proof: Adaptation of Riesz representatio theorem (of Hilbert spaces, say).
        
        Such $\hat g \equiv (g, h, \nabla)$ assumed from now on.
        
        \item Local:
        
            \begin{itemize}
                
            \item $\hat g = g_{\mu \nu} dx^\mu dx|\nu + h_{ab} \alg a^a \alg a^b$
            
            \item $h_i = G \, G h_j$
                
            \end{itemize}
        
        \end{itemize}
    
    \item Inner Orientability and Orientability
    
        \begin{itemize}
            
        \item Defn: inner orientable transitive LAoid. Suppose basis such that $|G| \geq 0$
        
        From now on $\hat g \equiv (g, h, \nabla)$ (inner n.d.) and inner orientable
        
        \item Defn: Inner volume form $\omega_{h, \alg a}$. Exists globally since it transforms well.
        
        \item Inner integration:
        
            \begin{itemize}
                
            \item (needed?) Of $L$-valued forms
            
            \item of $\bb K$-valued forms
            
            \item It is the factor of $\sqrt{|h|} \epsilon^1 \wedge \cdots \wedge \epsilon^n$. In particular, it doesn't actually depend on $\nabla$ (nor $g$), only on $h$.
                
            \end{itemize}
        
        --------------
        
        \item Defn: Orientable. From now on $\hat g \equiv (g, h, \nabla)$ (inner n.d.) and orientable.
        
        \item Defn: Integration on $A$: $\Omega^\bullet(A) \to \bb K$ ONLY (or at least all what we need)
        
        \item Defn: $\langle , \rangle \to \bb K := \int _A h^E(\cdot, \cdot)$ of $E$-valued forms given $h^E$ metric. Nondeg. since $h^E$ nondeg.
        
        \end{itemize}
    
    \item Horizontal Metrics and Hodge-$*$ Operator
    
    From now on $\hat g \equiv (g, h, \nabla)$ and orientable and nondegenerate i.e. $\hat g$ a metric on $A$ and $A$ orientable.
    
        \begin{itemize}
            
        \item Defn: volume form. Never $0$.
        
        \item Defn: $\hat g^{-1}_{h^E}(\alpha, \beta) \in C^\infty(M)$ for $\alpha, \beta \in  \Omega^p(A)$. Then:
        
        \item Defn: $\hat g^{-1}_{h^E}(\alpha, \beta) \in C^\infty(M)$ for $\alpha, \beta \in \Omega^p(A, E)$ given $E$-metric (i.e. nondeg.) $h^E$.
        
            \begin{itemize}
                
            \item Locally: $\hat g^{-1}_{h^E}(\alpha, \beta) = (h^E)^{ab} g^{\mu_1 \nu_2} \cdots h^{a_s b_s} \alpha^a_{\mu_1 \dots a_s} \beta^b_{\nu_1 \dots b_s} $ 
            
            \item Then define $\beta^\# = \beta_a^{\mu_1 \dots a_s} \epsilon^a \nabla_{\mu_1}\wedge \cdots \wedge (-E_{a_s}) \in \Omega^{p*}_{U_i}(A, E)$ where $\beta_a^{\mu_1 \dots a_s} := (h^E)^{ab} g^{\mu_1 \nu_2} \cdots h^{a_s b_s} \beta^b_{\nu_1 \dots b_s}$. Pretty sure it transforms well TODO. Then $\hat g^{-1}_{h^E}(\alpha, \beta) = \alpha^a_{\mu_1 \dots a_s} \beta_a^{\mu_1 \dots a_s}$.
                
            \end{itemize}
        
        \item Hodge-$*$ of $E$-valued forms: Defined such that $h^E(\alpha, *\beta) = \hat g^{-1}_{h^E}(\alpha, \beta) vol$
        
            \begin{itemize}
                
            \item In particular, for $\bb K$-valued: $\alpha \wedge *\beta =\hat g^{-1}(\alpha, \beta) vol$
                
            \end{itemize}
        Given by formula $2.6$: $\cdots$. Easy to transforms well, so it is globally defined.
        
        \item Defn: $(,): \Omega^p(A, E) \times \Omega^p(A, E) \to \bb K := \langle \cdot , * \cdot \rangle = \int A h^E(\cdot, * \cdot)$.
        
        \item Theorem: In a trivialization of $A$:
        \begin{equation}
            (\alpha, \beta) = (-1)^n \int_M \sqrt{|g|} dx^1 \wedge \cdots \wedge dx^m \sum_{r + s = p} (-1)^{s(m-s)} (m-r)! (n-s)! \times \alpha^a_{\mu_1 \dots a_s} \beta_a^{\mu_1 \dots a_s}.
        \end{equation}
            
        \end{itemize}
        
    \end{itemize}

}

\section{Metrics on Representation Vector Bundles and Inner Metrics}
%%%%%%%%%%%%%%%%%%%%%%%%%%%%%%%%%%%%%%%%%%%%%%%%%%%%%%%%%%%%%%%%%%%%%%
%%%%%%%%%%%%%%%%%%%%%%%%%%%%%%%%%%%%%%%%%%%%%%%%%%%%%%%%%%%%%%%%%%%%%%


\section{Symmetric Bilinear Forms}
%%%%%%%%%%%%%%%%%%%%%%%%%%%%%%%%%%%%%%%%%%%%%%%%%%%%%%%%%%%%%%%%%%%%%%
%%%%%%%%%%%%%%%%%%%%%%%%%%%%%%%%%%%%%%%%%%%%%%%%%%%%%%%%%%%%%%%%%%%%%%



\section{Inner Orientability and Orientability}
%%%%%%%%%%%%%%%%%%%%%%%%%%%%%%%%%%%%%%%%%%%%%%%%%%%%%%%%%%%%%%%%%%%%%%
%%%%%%%%%%%%%%%%%%%%%%%%%%%%%%%%%%%%%%%%%%%%%%%%%%%%%%%%%%%%%%%%%%%%%%

\section{Horizontal Metrics and Hodge-$*$ Operator}
%%%%%%%%%%%%%%%%%%%%%%%%%%%%%%%%%%%%%%%%%%%%%%%%%%%%%%%%%%%%%%%%%%%%%%
%%%%%%%%%%%%%%%%%%%%%%%%%%%%%%%%%%%%%%%%%%%%%%%%%%%%%%%%%%%%%%%%%%%%%%