\section{Traditional Gauge Transformations}
%%%%%%%%%%%%%%%%%%%%%%%%%%%%%%%%%%%%%%%%%%%%%%%%%%%%%%
%%%%%%%%%%%%%%%%%%%%%%%%%%%%%%%%%%%%%%%%%%%%%%%%%%%%%%

In this chapter we will finally formulate a gauge theory on a transitive Lie algebroid, by means of defining gauge invariant action functionals.

Throughout this section let $A$, $\alg g$, $E$, representation $\phi$, basis for these, connected? \todo{complete}.

{\color{gray}
\subsection{Vector Bundles}
%%%%%%%%%%%%%%%%%%%%%%%%%%%%%%%%%%%%%%%%%%%%%%%%%%%%%%

Let $E$ be an arbitrary vector bundle over $M$.

\begin{definition}
\emph{The gauge group of $E$}, denoted by $\mathcal G(E)$ is the group of (vertical) vector bundles automorphisms of $E$.
\end{definition}

Notice that $\mathcal G(E)$ is the subset of $\Gamma\End(E)$ of invertible endomorphisms of $E$.

{
\color{gray}
\begin{definition}
\emph{The gauge algebra of $E$} is the (infinite dimensional) Lie algebra $\Gamma\End(E)$ of endomorphisms of $E$.
\end{definition}
}

\subsection{Principal Bundles}
%%%%%%%%%%%%%%%%%%%%%%%%%%%%%%%%%%%%%%%%%%%%%%%%%%%%%%

\subsection{Vector Bundles Associated to a Principal Bundle}
%%%%%%%%%%%%%%%%%%%%%%%%%%%%%%%%%%%%%%%%%%%%%%%%%%%%%%
}

We have finally introduced all the necessary ingredients to define gauge theories. The traditional gauge theories arise from imposing that the dynamics of a matter field, i.e. sections of a vector bundle, be left invariant under space dependent changes of ``internal reference frame'' associated to the action of a structure group $G$, i.e. under gauge transformations, and this is achieved by the introduction of a connection on an associated principal bundle, also called a \emph{gauge potential}, with certain dynamics. Fournel, et al. in \cite{} propose the formulation of gauge theories on transitive Lie algebroids from a lagrangian approach through the action functional $\mathcal S$ of the theory, which decompomposes into the matter action $\mathcal S_{matter}$ and the gauge action $\mathcal S_{matter}$. The complete framework for the formulation is composed of an orientable transitive Lie algebroid $A$, over a base manifold $M$, equipped with a metric $\hat g \equiv (g, h, \tilde \nabla)$ with adjoint Lie algebroid $L$ on which $h$ is a Killing metric, and a vector bundle $E$ on which there is a representation $\phi: A \to \alg D(E)$ and a metric $h^E$ compatible with $\phi$; this notation will be used throughout the rest of the chapter.

Traditional gauge theories start with a principal bundle $P$  over a manifold $M$ with structure group $G$, where $\alg g$ is its Lie algebra, and an associated vector bundle $E$ with typical fiber $V$. \emph{The gauge group} $\mathcal G(P) \ni f$ of $P$ is the group of (vertical) automorphisms of $P$, that respect the group action; the group $C^\infty_G(P, G) \ni g$ of $G$-invariant functions, where $G$ acts on itself through the adjoint action, is (anti)isomorphic to the $\mathcal G(P)$, via the relation $f(p) = R_{g(p)}$. $f\in \mathcal G(P)$ acts on vector valued forms on $P$ through the pullback, this is called a (finite) gauge transformation; in particular, it acts on $G$-invariant forms of $\Omega^0(TP, P \times V)$, i.e. on $\Gamma(E)$, the space of \emph{matter fields}; on the subset of $\Omega^1(TP, P \times \alg g)$ made of principal connection forms $w$, producing once again a connection; and on their curvatures, basic forms in $\Omega^2(TP, O \times \alg g)$, where the curvature of $f^*w$ is $f^*R$ where $R$ is the curvature of $w$.

The infinite dimensional Lie algebra $C^\infty_G(P, \alg g)$, isomorphic to the sections of $P \times \alg g/G$, is called \emph{the gauge Lie algebra} of $P$, since there is an exponential map $\text{Exp}: C^\infty_G(P, \alg g) \to C^\infty_G(G, G)$, and hence an exponential map $\text{exp}: C^\infty_G(P, \alg g) \to \mathcal G(P)$. This allows to define the \emph{infinitesimal gauge action} of elements of $\eta \in C^\infty_G(P, \alg g)$ on forms $\alpha$ on $P$ as
\begin{equation}
    \der{t}[t=0]{\text{exp}(t\eta)} \alpha.
\end{equation}
In particular, from the fact that if $f = R_g \in \mathcal G(P)$, then the pushforward, for $\ppal X \in T_pP$, $f_* \ppal X = (L_{g(p) *}^{-1} g_*(\ppal X))^*_{f(p)} + R_{g(p)*}(X)$, we can easily conclude that the infinitesimal gauge action of $\eta$ on a matter field $\mu$, on a connection $w$ and on a curvature form $R$ is, respectively:
\begin{align}
    -\eta \cdot & \mu, &
    d\eta &+ w \wedge \eta, &
    - \eta \wedge  R;
\end{align}
the wedge product used is the one on the differential graded Lie algebra of $\alg g$-valued forms, which, for the last equation, is nothing more than the action of $\alg g$ on $\alg g$ induced by $G$-action on $\alg g$, i.e. the Lie bracket, just as $\eta \cdot$ is the action of $\alg g$ on $V$ induced by that of $G$. 

Throughout this chapter we will be dealing with forms on differential graded Lie algebras, and we will use the symbol $\wedge$ instead of $\wedge^{[,]}$ to denote the multiplication within this algebra, unless confusion may arise.

\section{Gauge Transformations associated to Transitive Lie Algebroids}
%%%%%%%%%%%%%%%%%%%%%%%%%%%%%%%%%%%%%%%%%%%%%%%%%%%%%%
%%%%%%%%%%%%%%%%%%%%%%%%%%%%%%%%%%%%%%%%%%%%%%%%%%%%%%

For this framework of gauge theories on transitive Lie algebroids, Lazzarini, et al. in \cite{2012} define the gauge algebra in such a way that it coincides with the traditional definion in the case that $A$ is the Atiyah Lie algebroid associated to a principal bundle: as the space of sections of an adjoint Lie algebroid.

{\color{gray}
Throughout this section, let $\hat \nabla^E: A \to \alg D(E)$ be an $A$-connection on $E$, with associated $A$-connection form $\hat \omega^E \in \Omega^1(A, \End(E))$ with respect to the representation $\phi$, and let its curvature form be $\hat R^E \in \Omega^2(A, L)$. 
The following is a straightforward generalization of the concept of gauge transformation of a vector bundle connection.\todo{If this isn't done before, then mention that this coincides with the traditional definition if $A = TM$}

\begin{definition}
Let $g$ be an element of $\mathcal G(E)$. \emph{The gauge transformation of the $A$-connection $\hat \nabla^E$ with respect to $g$} is the $A$-connection $\hat \nabla^{E, g} := g^{-1} \comp \hat \nabla^E \comp g$. \emph{The gauge transformation of the curvature of $\hat \nabla^E$} is the curvature of $\hat \nabla^{E,g}$.
\end{definition}

\begin{proposition}
Let $g \in \mathcal G(E)$. The $A$-connection form associated to $\hat \nabla^{E, g}$ is given by
\begin{equation*}
    \hat \omega^{E, g}  := g^{-1} \comp \hat \omega^E \comp g + g^{-1} \comp \hat d_E g,
\end{equation*}
and its curvature form by
\begin{equation*}
    \hat R^{E, g} = g^{-1} \comp  \hat R^E \comp g.
\end{equation*}

\end{proposition}
\begin{proof}
Recalling the an $A$-connection $\hat \nabla^E$ and its associated form $\hat \omega^E$ are related by $\hat \nabla^E = \phi + \hat \omega^E$, we have that
\begin{align*}
    \hat \omega^{E, g} &= \hat \nabla^{E, g} - \phi \\
        &= g^{-1} \comp \hat \nabla^E \comp g  - \phi\\
        &= g^{-1} \comp \phi \comp g + g^{-1} \comp  \hat \omega^E \comp g  - \phi
\end{align*}
\todo{The $\circ$ can't be removed, since then wedge product would be understood, which means that a commutator appears}
Let $\sectoid X \in \Gamma(A)$ and $\mu \in \Gamma(E) be arbitrary$, then
\begin{align*}
    g^{-1} \comp \phi(\sectoid X) \comp g (\mu) 
        &= g^{-1} \comp \phi (g(\mu)) \\
        &= g^{-1}\comp \tilde \phi(\sectoid X)(g)\mu + g^{-1} \comp g \comp \phi(\sectoid X)\mu\\
        &= g^{-1} \comp [\phi(\sectoid X), g]\mu + \phi(\sectoid X) \mu\\
        &= g^{-1} \hat d_E g(\sectoid X) \mu + \phi(\sectoid X) \mu.
\end{align*}
The last calculation shows that $g^{-1} \comp \phi \comp g = \phi + g^{-1} \comp \hat d_E g$, which, combined with the previous calculation, show the desired equation for $\hat \omega^{E, g}$.

\todo{For curvature}
\end{proof}

{
\color{red}
\begin{proposition}
Let $T \in \End(E)$. The infinitesimal gauge action of $T$ on the $A$-connection form $\hat \omega^E$ is
\begin{equation}
    \hat d_E T + \hat \omega \wedge T,
\end{equation}
on $\hat \nabla^E$ it is
\begin{equation}
    \hat \nabla^E \wedge T
\end{equation}
and on the curvature it is
\begin{equation}
    \hat R^E \wedge T.
\end{equation}
\end{proposition}


This is the connection between the ``traditional'' definitions, and the infinitesimal definitions given below for gauge transformations, so I would like to put the prove at least for the $A = TM$ case, i.e. ordinary vector bundle connections.
}
}
\linea



\begin{definition}
Given a Lie algebroid sequence $0 \to L \to A \to TM \to 0$, \emph{the gauge algebra of $A$} is the (infinite dimensional) Lie algebra $\Gamma(L)$.
\end{definition}

\begin{definition}\label{definitionInfinitesimalGaugeActionAndTransformationForAlgebroidFormsAndCurvature}
Let $\hat \omega \in \Omega^1(A, L)$ be a connection for on $A$, and let $\eta \in \Gamma(L)$. \emph{The infinitesimal gauge action on $\hat \omega$ with respect to $\eta$} is 
\begin{equation}
    \hat d \eta + \hat \omega \wedge \eta
\end{equation}
and \emph{the infinitesimal gauge transformation of $\hat \omega$} is the connection
\begin{equation}
    \hat \omega^\eta := \hat \omega + \hat d \eta + \hat \omega \wedge \eta.
\end{equation}
\emph{The infinitesimal gauge action on the curvature with respect to $\eta$} is defined to be
\begin{equation}
    \hat R \wedge \eta
\end{equation}
and its \emph{infinitesimal gauge transformation} is
\begin{equation}
    \hat R^\eta = \hat R + \hat R \wedge \eta
\end{equation}
\end{definition}

The curvature of $\hat \omega^\eta$ is $\hat R + \hat d \hat \omega \wedge \eta + \frac{1}{2} (\hat \omega \wedge \hat \omega) \wedge \eta + [\frac{1}{2} \hat d \eta \wedge \hat d \eta + \hat d \eta \wedge (\hat \omega \wedge \eta) + \frac{1}{2} (\hat \omega \wedge \eta) \wedge (\hat \omega \wedge \eta)]$, meaning that it is equal to $\hat R^\eta$ plus second order terms in $\eta$.

\begin{proposition}\label{propositionInfinitesimalGaugeTransformationOfAlgebroidEndomorphismOfConnection}
The algebroid morphism corresponding to the infinitesimal gauge transformation $\hat \omega^\eta$ of a connection form $\hat \omega$ with respect to $\eta \in \Gamma(L)$ is 
\begin{equation}
    \hat \nabla^\eta := \hat \nabla + [\hat \nabla, j\eta];
\end{equation}
the reduced kernel endomorphism corresponding to $\hat \omega$ is
\begin{equation}
    \tau^\eta := \tau + [\tau, \eta].
\end{equation}
These functions will be called \emph{the infinitesimal gauge transformations} of $\hat \nabla$ and $\tau$, respectively, with respect to $\eta$.
\end{proposition}
\begin{proof}
$\hat \nabla^\eta$ is defined to be, for every $\oid X \in A$
\begin{align*}
    \hat \nabla^\eta_{\oid X} &=
        \oid X + j\hat \omega^\eta(\oid X) \\
        &= \oid X + j\hat \omega(\oid X) + j \comp (\hat d \eta + \hat \omega \wedge \eta)(\oid X) \\
        &= \hat \nabla_{\oid X} + [\oid X, j\eta] + [j\hat \omega (\oid X), j\eta]\\
        &= \hat \nabla_{\oid X} + [\hat \nabla_{\oid X}, \eta].
\end{align*}
Similarly, given any $\theta \in L$,
\begin{align*}
    \tau^\eta (\theta)
        &= \hat \omega^\eta \comp j(\theta) + id_L(\theta) \\
        &= (\hat \omega \comp j + id_L)(\theta) + \hat d \eta (j \theta) + \hat \omega \wedge \eta(j \theta)\\
        &= \tau + [\theta, \eta] + [\hat \omega \comp j(\theta), \eta] \\
        &= \tau + [Id_L(\theta) + \hat \omega \comp j(\theta) , \eta]\\
        &= (\tau + [\tau, \eta])(\theta);
\end{align*}
the desired results follow from this calculations.
\end{proof}


\lin

Throughout the rest of the chapter, let $\phi$ be the representation of $A$ on $E$, with vertical part $\phi_L: L \to \End(E)$.

\begin{definition}\label{definitionInfinitesimalGaugeActionAndTransformationAConnection}
Let $\mu \in \Gamma(E)$, $\hat \nabla^E: A \to \alg D(E)$ be an $A$-connection on $E$, and $\eta \in \Gamma(L)$. Then, \emph{the infinitesimal gauge transformation of $\mu$ with respect to $\eta$} is the section
\begin{equation}
    \mu^\eta := \mu - \phi_L(\eta)\mu;
\end{equation}
\emph{the infinitesimal gauge transformation of $\hat \nabla^E$} is the $A$-connection
\begin{equation}
    \hat \nabla^{E, \eta} := \hat \nabla^E + [\hat \nabla^E, j\phi_L(\eta)];
\end{equation}
\emph{the infinitesimal gauge transformation of the curvature $\hat R^E$}$\in \Omega^2(A, \End(L))$ is
\begin{equation}
    \hat R^E \wedge \eta.
\end{equation}
\end{definition}

Following a calculation identical to that in the first part of the proof of proposition \ref{propositionInfinitesimalGaugeTransformationOfAlgebroidEndomorphismOfConnection}, but in the inverser order, we conclude that, with respect to any representation $\phi: A \to \alg D(E)$, the $\End(E)$-valued form associated to $\hat \nabla^{E, \eta}$ is
\begin{equation}
    \hat \omega^{E, \eta} := \hat \omega^E + \hat d_E(\phi_L(\eta)) + \hat \omega^E \wedge (\phi_L(\eta)).
\end{equation}

\begin{proposition}
Given a connection form $\hat \omega \in \Omega^1(A, L)$ and an element $\nabla \in \Gamma(L)$, let $\hat \nabla^E$ the $A$ be the $A$-connection produced by $\hat \omega$ on $E$ with respect to $\phi$. Then the infinitesimal transformation of $\hat \nabla^E$ coincides with the $A$-connection produced by $\hat \omega^{E, \eta}$.
\end{proposition}
\begin{proof}
The $A$-connection produced by $\hat \omega^{E, \eta}$ is
\begin{align}
    \phi(\hat \nabla^{\eta}) 
        &= \phi(\hat \nabla + [\nabla, j\eta]) \\
        &= \phi(\hat \nabla) + [\phi(\hat \nabla), \phi_L(\eta)]\\
        &= \hat \nabla^E + [\hat \nabla^E, \phi_L(\eta)\\
        &= \hat \nabla^{E, \eta}.
\end{align}
\end{proof}

\section{Gauge Action Functional}
%%%%%%%%%%%%%%%%%%%%%%%%%%%%%%%%%%%%%%%%%%%%%%%%%%%%%%
%%%%%%%%%%%%%%%%%%%%%%%%%%%%%%%%%%%%%%%%%%%%%%%%%%%%%%

\section{Matter Action Functional}
%%%%%%%%%%%%%%%%%%%%%%%%%%%%%%%%%%%%%%%%%%%%%%%%%%%%%%
%%%%%%%%%%%%%%%%%%%%%%%%%%%%%%%%%%%%%%%%%%%%%%%%%%%%%%