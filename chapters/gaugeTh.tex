We have finally introduced all the necessary ingredients to define gauge theories. The traditional gauge theories arise from imposing that the dynamics of a matter field, i.e. sections of a vector bundle, be left invariant under space dependent changes of ``internal reference frame'' associated to the action of a structure group $G$, i.e. under gauge transformations, and this is achieved by the introduction of a connection on an associated principal bundle, also called a \emph{gauge potential}, with certain dynamics. Fournel, et al. in \cite{Fournel2013} propose the formulation of gauge theories on transitive Lie algebroids from a lagrangian approach through the action functional $\mathcal S$ of the theory, which decompomposes into the matter action $\mathcal S_{matter}$ and the gauge action $\mathcal S_{matter}$. 

The complete framework for the formulation is made up of an orientable transitive Lie algebroid $A$, over a base manifold $M$, equipped with a metric $\hat g \equiv (g, h, \tilde \nabla)$ with adjoint Lie algebroid $L$ on which $h$ is a Killing metric, and a vector bundle $E$ on which there is a representation $\phi: A \to \alg D(E)$ and a metric $h^E$ compatible with $\phi$.

\section{Traditional Gauge Transformations}

%%%%%%%%%%%%%%%%%%%%%%%%%%%%%%%%%%%%%%%%%%%%%%%%%%%%%%
%%%%%%%%%%%%%%%%%%%%%%%%%%%%%%%%%%%%%%%%%%%%%%%%%%%%%%


% In this chapter we will finally formulate a gauge theory on a transitive Lie algebroid, by means of defining gauge invariant action functionals.

% Throughout this section let $A$, $\alg g$, $E$, representation $\phi$, basis for these, connected? \todo{complete}.

% {\color{gray}
% \subsection{Vector Bundles}
% %%%%%%%%%%%%%%%%%%%%%%%%%%%%%%%%%%%%%%%%%%%%%%%%%%%%%%

% Let $E$ be an arbitrary vector bundle over $M$.

% \begin{definition}
% \emph{The gauge group of $E$}, denoted by $\mathcal G(E)$ is the group of (vertical) vector bundles automorphisms of $E$.
% \end{definition}

% Notice that $\mathcal G(E)$ is the subset of $\Gamma\End(E)$ of invertible endomorphisms of $E$.

% {
% \color{gray}
% \begin{definition}
% \emph{The gauge algebra of $E$} is the (infinite dimensional) Lie algebra $\Gamma\End(E)$ of endomorphisms of $E$.
% \end{definition}
% }

% \subsection{Principal Bundles}
% %%%%%%%%%%%%%%%%%%%%%%%%%%%%%%%%%%%%%%%%%%%%%%%%%%%%%%

% \subsection{Vector Bundles Associated to a Principal Bundle}
% %%%%%%%%%%%%%%%%%%%%%%%%%%%%%%%%%%%%%%%%%%%%%%%%%%%%%%
% }


Traditional gauge theories start with a principal bundle $P$  over a manifold $M$ with structure group $G$, where $\alg g$ is its Lie algebra, and an associated vector bundle $E$ with typical fiber $V$. \emph{The gauge group} $\mathcal G(P) \ni f$ of $P$ is the group of (vertical) automorphisms of $P$, that respect the group action; the group $C^\infty_G(P, G) \ni g$ of $G$-invariant functions, where $G$ acts on itself through the adjoint action, is (anti)isomorphic to the $\mathcal G(P)$, via the relation $f(p) = R_{g(p)}$. $f\in \mathcal G(P)$ acts on vector valued forms on $P$ through the pullback, this is called a (finite) gauge transformation; in particular, it acts on $G$-invariant forms of $\Omega^0(TP, P \times V)$, i.e. on $\Gamma(E)$, the space of \emph{matter fields}; on the subset of $\Omega^1(TP, P \times \alg g)$ made of principal connection forms $w$, producing once again a connection; and on their curvatures, basic forms in $\Omega^2(TP, O \times \alg g)$, where the curvature of $f^*w$ is $f^*R$ where $R$ is the curvature of $w$.

The infinite dimensional Lie algebra $C^\infty_G(P, \alg g)$, isomorphic to the sections of $P \times \alg g/G$, is called \emph{the gauge Lie algebra} of $P$, since there is an exponential map $\text{Exp}: C^\infty_G(P, \alg g) \to C^\infty_G(G, G)$, and hence an exponential map $\text{exp}: C^\infty_G(P, \alg g) \to \mathcal G(P)$. This allows to define the \emph{infinitesimal gauge action} of elements of $\eta \in C^\infty_G(P, \alg g)$ on forms $\alpha$ on $P$ as
\begin{equation}
    \der{t}[t=0]{\text{exp}(t\eta)} \alpha.
\end{equation}
In particular, from the fact that if $f = R_g \in \mathcal G(P)$, then the pushforward, for $\ppal X \in T_pP$, $f_* \ppal X = (L_{g(p) *}^{-1} g_*(\ppal X))^*_{f(p)} + R_{g(p)*}(X)$, we can easily conclude that the infinitesimal gauge action of $\eta$ on a matter field $\mu$, on a connection $w$ and on a curvature form $R$ is, respectively:
\begin{align}
    -\eta \cdot & \mu, &
    d\eta &+ w \wedge \eta, &
    - \eta \wedge  R;
\end{align}
the wedge product used is the one on the differential graded Lie algebra of $\alg g$-valued forms, which, for the last equation, is nothing more than the action of $\alg g$ on $\alg g$ induced by $G$-action on $\alg g$, i.e. the Lie bracket, just as $\eta \cdot$ is the action of $\alg g$ on $V$ induced by that of $G$. 

Throughout this chapter we will be dealing with forms on differential graded Lie algebras, and we will use the symbol $\wedge$ instead of $\wedge^{[,]}$ to denote the multiplication within this algebra, unless confusion may arise.

\section{Gauge Transformations associated to Transitive Lie Algebroids}
%%%%%%%%%%%%%%%%%%%%%%%%%%%%%%%%%%%%%%%%%%%%%%%%%%%%%%
%%%%%%%%%%%%%%%%%%%%%%%%%%%%%%%%%%%%%%%%%%%%%%%%%%%%%%

For this framework of gauge theories on transitive Lie algebroids, Lazzarini, et al. in \cite{Lazzarini2012} define the gauge algebra in such a way that it coincides with the traditional definition in the case that $A$ is the Atiyah Lie algebroid associated to a principal bundle: as the space of sections of an adjoint Lie algebroid.

% {\color{gray}
% Throughout this section, let $\hat \nabla^E: A \to \alg D(E)$ be an $A$-connection on $E$, with associated $A$-connection form $\hat \omega^E \in \Omega^1(A, \End(E))$ with respect to the representation $\phi$, and let its curvature form be $\hat R^E \in \Omega^2(A, L)$. 
% The following is a straightforward generalization of the concept of gauge transformation of a vector bundle connection.\todo{If this isn't done before, then mention that this coincides with the traditional definition if $A = TM$}

% \begin{definition}
% Let $g$ be an element of $\mathcal G(E)$. \emph{The gauge transformation of the $A$-connection $\hat \nabla^E$ with respect to $g$} is the $A$-connection $\hat \nabla^{E, g} := g^{-1} \comp \hat \nabla^E \comp g$. \emph{The gauge transformation of the curvature of $\hat \nabla^E$} is the curvature of $\hat \nabla^{E,g}$.
% \end{definition}

% \begin{proposition}
% Let $g \in \mathcal G(E)$. The $A$-connection form associated to $\hat \nabla^{E, g}$ is given by
% \begin{equation*}
%     \hat \omega^{E, g}  := g^{-1} \comp \hat \omega^E \comp g + g^{-1} \comp \hat d_E g,
% \end{equation*}
% and its curvature form by
% \begin{equation*}
%     \hat R^{E, g} = g^{-1} \comp  \hat R^E \comp g.
% \end{equation*}

% \end{proposition}
% \begin{proof}
% Recalling the an $A$-connection $\hat \nabla^E$ and its associated form $\hat \omega^E$ are related by $\hat \nabla^E = \phi + \hat \omega^E$, we have that
% \begin{align*}
%     \hat \omega^{E, g} &= \hat \nabla^{E, g} - \phi \\
%         &= g^{-1} \comp \hat \nabla^E \comp g  - \phi\\
%         &= g^{-1} \comp \phi \comp g + g^{-1} \comp  \hat \omega^E \comp g  - \phi
% \end{align*}
% \todo{The $\circ$ can't be removed, since then wedge product would be understood, which means that a commutator appears}
% Let $\sectoid X \in \Gamma(A)$ and $\mu \in \Gamma(E) be arbitrary$, then
% \begin{align*}
%     g^{-1} \comp \phi(\sectoid X) \comp g (\mu) 
%         &= g^{-1} \comp \phi (g(\mu)) \\
%         &= g^{-1}\comp \tilde \phi(\sectoid X)(g)\mu + g^{-1} \comp g \comp \phi(\sectoid X)\mu\\
%         &= g^{-1} \comp [\phi(\sectoid X), g]\mu + \phi(\sectoid X) \mu\\
%         &= g^{-1} \hat d_E g(\sectoid X) \mu + \phi(\sectoid X) \mu.
% \end{align*}
% The last calculation shows that $g^{-1} \comp \phi \comp g = \phi + g^{-1} \comp \hat d_E g$, which, combined with the previous calculation, show the desired equation for $\hat \omega^{E, g}$.

% \todo{For curvature}
% \end{proof}

% {
% \color{red}
% \begin{proposition}
% Let $T \in \End(E)$. The infinitesimal gauge action of $T$ on the $A$-connection form $\hat \omega^E$ is
% \begin{equation}
%     \hat d_E T + \hat \omega \wedge T,
% \end{equation}
% on $\hat \nabla^E$ it is
% \begin{equation}
%     \hat \nabla^E \wedge T
% \end{equation}
% and on the curvature it is
% \begin{equation}
%     \hat R^E \wedge T.
% \end{equation}
% \end{proposition}


% This is the connection between the ``traditional'' definitions, and the infinitesimal definitions given below for gauge transformations, so I would like to put the prove at least for the $A = TM$ case, i.e. ordinary vector bundle connections.
% }
% }

Let $0 \to L \xrightarrow{j} A \xrightarrow{a} TM \to 0$ be a Lie algebroid sequence for the transitive Lie algebroid $A$.

\begin{definition}
\emph{The gauge algebra of $A$} is the (infinite dimensional) Lie algebra $\Gamma(L)$.
\end{definition}

\begin{definition}\label{definitionInfinitesimalGaugeActionAndTransformationForAlgebroidFormsAndCurvature}
Let $\hat \omega \in \Omega^1(A, L)$ be a connection for on $A$, and let $\eta \in \Gamma(L)$. \emph{The infinitesimal gauge action on $\hat \omega$ with respect to $\eta$} is 
\begin{equation}
    \hat d \eta + \hat \omega \wedge \eta
\end{equation}
and \emph{the infinitesimal gauge transformation of $\hat \omega$} is the connection
\begin{equation}
    \hat \omega^\eta := \hat \omega + \hat d \eta + \hat \omega \wedge \eta.
\end{equation}
\emph{The infinitesimal gauge action on the curvature with respect to $\eta$} is defined to be
\begin{equation}
    \hat R \wedge \eta
\end{equation}
and its \emph{infinitesimal gauge transformation} is
\begin{equation}
    \hat R^\eta = \hat R + \hat R \wedge \eta
\end{equation}
\end{definition}

The curvature of $\hat \omega^\eta$ is 
$$\hat R + \hat d \hat \omega \wedge \eta + \frac{1}{2} (\hat \omega \wedge \hat \omega) \wedge \eta + [\frac{1}{2} \hat d \eta \wedge \hat d \eta + \hat d \eta \wedge (\hat \omega \wedge \eta) + \frac{1}{2} (\hat \omega \wedge \eta) \wedge (\hat \omega \wedge \eta)],$$ 
meaning that \textit{it is equal to $\hat R^\eta$ plus second order terms in $\eta$}: for arbitrary $\oid X, \oid Y \in A$, and ignoring the immersion $j$, 
\begin{multline}\frac{1}{2} \hat d \eta \wedge \hat d \eta + \hat d \eta \wedge (\hat \omega \wedge \eta) + \frac{1}{2} (\hat \omega \wedge \eta) \wedge (\hat \omega \wedge \eta)](\oid X, \oid Y) 
\\= [[\oid X, \eta], [\oid Y, \eta]] + [[\oid X, \eta], [\tilde \omega(\oid Y), \eta]] + [[\oid Y, \eta], [\tilde \omega(\oid X), \eta]] + [[\tilde \omega(X), \eta], [\tilde \omega(Y), \eta]];
\end{multline}
these are all terms that involve twice the product with $\eta$, hence we call them second order terms in $\eta$.

\begin{proposition}\label{propositionInfinitesimalGaugeTransformationOfAlgebroidEndomorphismOfConnection}
The algebroid morphism corresponding to the infinitesimal gauge transformation $\hat \omega^\eta$ of a connection form $\hat \omega$ with respect to $\eta \in \Gamma(L)$ is 
\begin{equation}
    \hat \nabla^\eta := \hat \nabla + [\hat \nabla, j\eta];
\end{equation}
the reduced kernel endomorphism corresponding to $\hat \omega$ is
\begin{equation}
    \tau^\eta := \tau + [\tau, \eta].
\end{equation}
These functions will be called \emph{the infinitesimal gauge transformations} of $\hat \nabla$ and $\tau$, respectively, with respect to $\eta$.
\end{proposition}
\begin{proof}
$\hat \nabla^\eta$ is defined to be, for every $\oid X \in A$
\begin{align*}
    \hat \nabla^\eta_{\oid X} &=
        \oid X + j\hat \omega^\eta(\oid X) \\
        &= \oid X + j\hat \omega(\oid X) + j \comp (\hat d \eta + \hat \omega \wedge \eta)(\oid X) \\
        &= \hat \nabla_{\oid X} + [\oid X, j\eta] + [j\hat \omega (\oid X), j\eta]\\
        &= \hat \nabla_{\oid X} + [\hat \nabla_{\oid X}, \eta].
\end{align*}
Similarly, given any $\theta \in L$,
\begin{align*}
    \tau^\eta (\theta)
        &= \hat \omega^\eta \comp j(\theta) + id_L(\theta) \\
        &= (\hat \omega \comp j + id_L)(\theta) + \hat d \eta (j \theta) + \hat \omega \wedge \eta(j \theta)\\
        &= \tau + [\theta, \eta] + [\hat \omega \comp j(\theta), \eta] \\
        &= \tau + [Id_L(\theta) + \hat \omega \comp j(\theta) , \eta]\\
        &= (\tau + [\tau, \eta])(\theta);
\end{align*}
the desired results follow from this calculations.
\end{proof}


\lin

Throughout the rest of the chapter, let $\phi$ be a representation of $A$ on $E$, with vertical part $\phi_L: L \to \End(E)$.

\begin{definition}\label{definitionInfinitesimalGaugeActionAndTransformationAConnection}
Let $\mu \in \Gamma(E)$, $\hat \nabla^E: A \to \alg D(E)$ be an $A$-connection on $E$, and $\eta \in \Gamma(L)$. Then, \emph{the infinitesimal gauge transformation of $\mu$ with respect to $\eta$} is the section
\begin{equation}
    \mu^\eta := \mu - \phi_L(\eta)\mu;
\end{equation}
\emph{the infinitesimal gauge transformation of $\hat \nabla^E$} is the $A$-connection
\begin{equation}
    \hat \nabla^{E, \eta} := \hat \nabla^E + [\hat \nabla^E, j\phi_L(\eta)];
\end{equation}
\emph{the infinitesimal gauge transformation of the curvature $\hat R^E$}$\in \Omega^2(A, \End(L))$ is
\begin{equation}
    \hat R^E \wedge \eta.
\end{equation}
\end{definition}

Following a calculation identical to that in the first part of the proof of proposition \ref{propositionInfinitesimalGaugeTransformationOfAlgebroidEndomorphismOfConnection}, but in the inverser order, we conclude that, with respect to any representation $\phi: A \to \alg D(E)$, the $\End(E)$-valued form associated to $\hat \nabla^{E, \eta}$ is
\begin{equation}
    \hat \omega^{E, \eta} := \hat \omega^E + \hat d_E(\phi_L(\eta)) + \hat \omega^E \wedge (\phi_L(\eta)).
\end{equation}

\begin{proposition}\label{propositionGaugeActionTransformationAConnectionEqualToCommute}
Given a connection form $\hat \omega \in \Omega^1(A, L)$ and an element $\nabla \in \Gamma(L)$, let $\hat \nabla^E$ the $A$ be the $A$-connection produced by $\hat \omega$ on $E$ with respect to $\phi$. Then the infinitesimal transformation of $\hat \nabla^E$ coincides with the $A$-connection produced by $\hat \omega^{E, \eta}$.
\end{proposition}
\begin{proof}
The $A$-connection produced by $\hat \omega^{E, \eta}$ is
\begin{align*}
    \phi(\hat \nabla^{\eta}) 
        &= \phi(\hat \nabla + [\nabla, j\eta]) \\
        &= \phi(\hat \nabla) + [\phi(\hat \nabla), \phi_L(\eta)]\\
        &= \hat \nabla^E + [\hat \nabla^E, \phi_L(\eta)\\
        &= \hat \nabla^{E, \eta}.
\end{align*}
\end{proof}

\section{Gauge Action Functional}
%%%%%%%%%%%%%%%%%%%%%%%%%%%%%%%%%%%%%%%%%%%%%%%%%%%%%%
%%%%%%%%%%%%%%%%%%%%%%%%%%%%%%%%%%%%%%%%%%%%%%%%%%%%%%

Let $A$ be an orientable transitive Lie algebroid over a base manifold $M$ equipped with a metric $\hat g \equiv (g, h, \tilde \nabla)$ with adjoint Lie algebroid $L$ on which $h$ is a Killing metric, $g$ is a metric on $M$ and $\tilde \nabla$ is an ordinary connection on $A$, called the background connection, with associated $1$-form $\tilde \omega$. This assumptions will be carrier throughout the rest of the chapter.

\begin{definition}\label{definitionGaugeActionLagrangian}
\emph{The gauge lagrangian density} given a connection form $\hat \omega \in \Omega^1(A, L)$ with curvature $\hat R$ is defined as
\begin{equation}
    \mathcal L_{gauge}[\hat \omega] := \int_{inner} h(\hat R, *\hat R);
\end{equation}
\emph{the gauge action functional} is defined as 
\begin{align}
    \mathcal S_{gauge}[\hat \omega]& := (\hat R, \hat R),
\end{align}
i.e. as the integral over $M$ of the Lagrangian density.
\end{definition}

\begin{lemma}\label{lemmaIntegrationFOrmGaugeActionTHeoryhRRisInvariantGaugeTransforamtions}
Let $\eta \in \Gamma(L)$. For any connection with curvature form $\hat R$, the form $h(\hat R, \hat R)$ is invariant under infinitesimal gauge transformations up to first order terms in $\eta$.
\end{lemma}
\begin{proof}
Since $\eta$ is a $0$-form, it can be shown that $*(\eta \wedge \hat R) = \eta \wedge *\hat R$; also, since $h$ is a Killing metric, $h(\eta \wedge \hat R, \alpha) + h(\hat R, \eta \wedge \alpha) = 0$ for any $L$-valued form $\alpha$, hence
\begin{align*}
    h(\hat R^\eta, \hat R^\eta) 
        &=h( \hat R - \eta \wedge \hat R, *(\hat R - \eta \wedge \hat R)) \\
        &= h(\hat R, *\hat R) - [h(\eta \wedge \hat R, *\hat R) + h(\hat R, \eta \wedge *\hat R)] + h(\eta \wedge \hat R, \eta \wedge *\hat R)\\
        &= h(\hat R, *\hat R) + h(\eta \wedge \hat R, \eta \wedge *\hat R),
\end{align*}
since $h$ is yet another product of forms, which locally looks like in equation \eqref{equationLocalhProductOfMetrics}, the last term is of second order in $\eta$, and so the statement is proven.
\end{proof}

\begin{proposition}
The gauge lagrangian density $\mathcal L_{gauge}$ is invariant under infinitesimal gauge transformations of the connection $\hat \omega$ with respect to any $\eta \in \Gamma(L)$, up to first order terms in $\eta$.
\end{proposition}
\begin{proof}
This is a corollary of lemma \ref{lemmaIntegrationFOrmGaugeActionTHeoryhRRisInvariantGaugeTransforamtions}, since $\mathcal L_{gauge} = \int_{inner} h(\hat R, *\hat R)$ is the equal to the factor of $h(\hat R, *\hat R)$ that accompanies the inner volume form.
\end{proof}

Due to proposition \ref{propositionRelationHodgeStarAndInverseMetricVolume} for the Hodge-*, and the decomposition of the curvature given in theorem \ref{theoremDecompositionOfAlgebroidFormIntoNice2Forms}, in a trivializing neighborhoood $U_i$ of $A$ and $M$ the lagrangian density applied to the connection $\hat \omega$ with local decomposition $\hat \omega_i = \hat A_i - \epsilon + \tau_i $ is
\begin{eqnsplit}\label{equationGaugeLagrangianDecompositionLocallyInverseMetric}
    \mathcal L_{gauge}[\hat A_i, \tau_i] &= 
        g^{\mu_1 \nu_1} g^{\mu_2 \nu_2} h_{cd} (\hat F_i)^c_{\mu_1 \mu_2} (\hat F_i)^d_{\nu_1 \nu_2} 
        +  
        g^{\mu \nu} h^{a b} h_{cd} (\mathcal D \tau_i)^c_{\mu, a} (\mathcal D \tau_i)^d_{\nu, b} \\
        &\hfill 
        + h^{a_1 b_1} g^{a_2 b_2} h_{cd} (W_i)^c_{a_1 a_2} (W_i)^d_{b_1 b_2} \\
        &= (\hat F_i)^c_{\mu \nu} (\hat F_i)_c^{\mu \nu} 
        + (\mathcal D \tau)^c_{\mu, a} (\mathcal D \tau)_c^{\mu, a} 
        + (W_i)^c_{ab} (W_i)_c^{ab}.
\end{eqnsplit}
This shows that the gauge action itself can be decomposed as
\begin{equation}\label{equationDecompositionOfGaugeActionFunctionalGlobal}
    S_{gauge}[\hat \omega] = (a^* \hat F, a^* \hat F) + ([a^* \mathcal D \tau, \tilde \omega], [a^* \mathcal D \tau, \tilde \omega]) + (\tilde \omega^* R_\tau, \tilde \omega^* R_\tau).
\end{equation}

In case that $\hat \omega$ is an ordinary connection form on $A$, $a^* \hat F = \hat R$ and so $\hat F \in \Omega^2(TM, L)$ is the traditional curvature of the ordinary connection, also called the \emph{field strength}; since $\tau = 0$, the second and third term of the action functional annihilate and the left term coincides, in the case that $A$ is an Atiyah Lie algebroid associated to a principal bundle, with the traditional action functional of the gauge potential.

Recall that in a trivializing neighborhood $U_i$ of $A$ on which there are coordinates $\{x^\mu\}_{\mu = 1, \dots, m}$, with respect to a basis $\{E_a\}_{a = 1, \dots, n}$ of $\alg g$ dual $\{e^a\}_a$, with structure constants $C^a_{bc} E_a = [E_b, E_c]$. 
% Then, 
% \begin{align*}
%     \hat (R_i)_{\mu \nu} &= (R_i)(\partial_\mu, \partial_\nu \\
%         &= (dA + )
% \end{align*}
\begin{align}
    (\hat F_i)^a_{\mu \nu} &= R^a_{\mu \nu} - \tau^a_b \tilde R^b_{\mu \nu}\\
    (\mathcal D \tau_i)^b_{\mu, a} &= \partial_\mu \tau^b_a + (A_i)^c_\mu (\tau_i)^d_c C^b_{cd} - (\tilde A_i)_\mu^d C^c_{da} \tau^b_c\\
    (W_i)^c_{ab} &= \tau^b_a \tau^e_b C^c_{de} - C^d_{ab}\tau^c_d.
\end{align}

So far no method to extramize the action to extract the equations of motion of the system has been stated, but from its resemblance to the traditional Yang-Mills gauge theories, specially when looking at a local trivialization of the lagrangian density, we might expect the following very rough interpretation of the decomposition \eqref{equationDecompositionOfGaugeActionFunctionalGlobal} of the action functional:
\begin{enumerate}
    \item The first term describes the dynamics of the connection. 
    %We we might call the globally defined object $\tilde A$ the gauge potential; as seen in \eqref{}, when the Lie algebroid is the one associated to a principal bundle, between local trivializations this $A$ object transforms as a principal connection form.
    
    \item The second term contains both the Kinetic energy of the tau fields, and A coupling of the $A$ fields with the $\tau$ fields in terms quadratic in $\hat A$, which give rise to mass terms for the $A$ fields in the Lagrangian.
    
    \item The third term is the potential term for the $\tau$ fields.
\end{enumerate}

However, notice that new $\tilde A_i$ fields appear in the first and second term (which annihilates when $\tau = 0$) due to the background ordinary connection that emerges as a consequence of a choice of metric on the algebroid, and their contribution to the action disappears whenever $\hat \omega$ is an ordinary connection. Also, the contribution of $\tau$ to the first term disappears if the background connection is flat, leaving the first term devoid of contributions by $\tau$, only as a term that resembles the kinetic term of the ``gauge fields'' $A$ in the traditional gauge theories. Also notice that $\hat F$ is not the ``curvature'' of the global object defined by the trivializations $\hat A_i$, nor do the $\tau$ contribution to the first term disappear when expanded, meaning that there is no clear choice for what the ``gauge fields'' should be; however, the full contribution of $\tau$ does disappear in some cases on the first term, leaving again a kinetic term for the ``gauge fields'' $\hat A$.

\section{Matter Action Functional}
%%%%%%%%%%%%%%%%%%%%%%%%%%%%%%%%%%%%%%%%%%%%%%%%%%%%%%
%%%%%%%%%%%%%%%%%%%%%%%%%%%%%%%%%%%%%%%%%%%%%%%%%%%%%%

Now, in addition to the assumptions on $A$ stated at the beginning of the previous section for the matter part of the gauge theory, let $E$ be a vector bundle $E$ on which there is a representation $\phi: A \to \alg D(E)$ and a metric $h^E$ compatible with $\phi$. The sections of $E$ will be called \emph{matter fields} of the gauge theory.

\begin{definition}\label{definitionMatterActionLagrangian}
Given a connection form $\hat \omega \in \Omega^1(A, L)$ which produces the $A$-connection $\hat \nabla^E$, and a matter field $\mu$, \emph{the matter lagrangian density} is defined as
\begin{equation}
    \mathcal L_{matter}[\mu, \hat \omega] := \int_{inner} h^E(\hat \nabla \mu, *\hat \nabla \mu);
\end{equation}
where $\hat \nabla^E \mu$ is an elements of $\Omega^1(A,E)$; \emph{the gauge action functional} is defined as 
\begin{align}
    \mathcal S_{gauge}[\hat \omega]& := (\hat \nabla^E \mu, \hat \nabla^E \mu).
\end{align}
\end{definition}

\begin{lemma}
Let $\eta \in \Gamma(L)$. Given $\hat \omega \in \Omega^1(A, L)$ which induces the $A$-connection $\hat \nabla^E$, and given any $\mu \in \Gamma(E)$, $h^E(\hat \nabla^E \mu, \hat \nabla^E \mu)$ is invariant under infinitesimal gauge transforations up to first order terms in $\phi_L(\eta)$ (where composition of endomorphisms of $E$ plays the role of multiplication).
\end{lemma}

\begin{proof}
Recall from proposition \ref{propositionGaugeActionTransformationAConnectionEqualToCommute} that acting with $\eta$ on $\hat \omega$ equates to acting with $\phi_L(\eta)$ on $\hat \nabla^E$, so, conserving only first order terms in $\phi_L(\eta)$, the infinitesimal gauge transformation on $\hat \nabla^E \mu$ is:
\begin{eqnsplit}
    \hat \nabla^{E, \eta} \mu^\eta &=
        (\hat \nabla^E + [\hat \nabla^E, \phi_L(\eta)])(\mu - \phi_L(\eta)\mu) \\
        &= \hat \nabla^E \mu + [\hat \nabla^E, \phi_L(\eta)]\mu - \hat \nabla^E(\phi_L(\eta) \mu) \\
        &=\hat \nabla^E \mu + \phi_L \comp \hat \nabla^E \mu.
\end{eqnsplit}
Hence, conserving only first order terms in $\phi_L(\eta)$:
\begin{align*}
    h^E((\hat \nabla^E \mu)^\eta, &*(\hat \nabla^E \mu)^\eta) 
        = h^E(\hat \nabla^E \mu + \phi_L \comp \hat \nabla^E \mu, *(\hat \nabla^E \mu + \phi_L \comp \hat \nabla^E \mu))\\
        &= h^E(\hat \nabla^E \mu, \hat \nabla^E \mu) + h^E(\phi_L(\eta) \comp \hat \nabla^E \mu, * \hat \nabla^E \mu) + h^E(\hat \nabla^E \mu, \phi_L(\eta) \comp *\hat \nabla^E \mu)\\
        &= h^E(\hat \nabla^E \mu, \hat \nabla^E \mu);
\end{align*}
the last step follows from the fact that $h^E$ is compatible with $\phi$.
\end{proof}

From this lemma it follows that:

\begin{proposition}
The matter lagrangian density $\mathcal L_{matter}$ is invariant under infinitesimal gauge transformations of the connection $\hat \omega$ with respect to any $\eta \in \Gamma(L)$, up to first order terms in $\phi_L(\eta)$.
\end{proposition}

\begin{proposition}\label{propositionDecompositionFormActionMatterAConnection}
Let $\hat \omega \in \Omega^1(A, L)$ be a connection form in $A$ that produces the $A$-connection $\hat \nabla^E$, and let $\nabla: TM \to A$ be the connection associated to the ordinary connection form $\omega = \hat \omega + \tau \comp \tilde \omega$. For any $\mu \in \Gamma(E)$:
\begin{equation}
    \hat \nabla^E \mu = a^* \phi(\nabla)\mu - (\phi_L(\tau) \mu)\comp \tilde \omega,
\end{equation}
i.e. for any $\oid X \in A$, $\hat \nabla^E_{\oid X} \mu = \phi(\hat \nabla_{a(\oid X)})\mu - \phi_L \comp \tau \comp \tilde \omega(\oid X) \mu$.
\end{proposition}
\begin{proof}
This decomposition is induced by the ordinary connection $\omega = \hat \omega + \tau \comp \tilde \omega$, which allows to write any $\oid X \in A$ as $\oid X = \phi(\nabla_{a(\oid X)}) - \phi_L \comp \omega(\oid X)$. Then 
\begin{align*}
    \hat \nabla^E_{\oid X} 
      &= \phi(\oid X) + \phi_L \comp \hat \omega(\oid X)\\
      &= \phi(\nabla_{a(\oid X)}) - \phi_L \comp \omega(\oid X) + \phi_L \comp (\omega - \tau \comp \tilde \omega)(\oid X)\\
      &= \phi(\hat \nabla_{a(\oid X)}) - \phi_L \comp \tau \comp \tilde \omega(\oid X).
\end{align*}
\end{proof}

Let $\{(U_i, \psi_i: U_i \times \alg g \to L|_{U_i}, \nabla^{0, i}: TU_i \to A|_{U_i})\}_{i \in I}$ be a Lie algebroid atlas for $A$, where $\alg g$ is a Lie algebra with basis $\{E_a\}_{a=1, \dots, n}$ and associated dual basis $\{\epsilon^a\}$. Assume that $\{e_u\}_{u = 1, \dots, t}$ is a basis of the fiber $V$ of $E$, and that the matter field $\mu \in \Gamma(E)$ is locally realized as $(\mu_i)^u e_u$ with $\mu_i \in C^\infty(U_i, V)$ for each $i \in I$; also suppose that there are coordinates $\{x^\mu: U_i \to \RR\}$over $U_i \subset M$. Recall the decomposition \eqref{} of a produced $A$-connection; then it can be easily seen that the decomposition of $\hat \nabla^E \mu$ of proposition \ref{propositionDecompositionFormActionMatterAConnection} is such that the local trivializations over $U_i$ of each term are:
\begin{align}\label{equationExplicitTrivializationsOfDecompositionOfTermsOfMatterGaugeTermsAConnection}
    a^* \phi(\nabla)\mu_i &= [\partial_\nu (\mu_i)^u + (B_i)^u_{\nu, v} (\mu_i)^v + (\phi_{L, i})^u_{b, v} (\hat A_i)_\mu^b (\mu_i)^v ]e_u dx^\nu\\
    (\phi_L(\tau) \mu)\comp \tilde \omega_i &= [(\phi_{L, i})_{b, v}^u \tau^b_a \,(\mu_i)^v] e_u \alg a_i^a,
\end{align}
where $B_i \in \Omega^1(TU_i, U_i \times \End(V))$ is the Maurer-Cartan form that, together with $\phi_{L, i} \in C^\infty(U_i, End(V))$, trivializes the representation $\phi$ (recall that $B_i = 0$ and $\phi_{L, i} = \pi$ is the group representation on $V$ for the group induced representations of example \ref{exampleLocalTrivializationOfGroupInducedRepresentationOfAtiyahLieAlgebroidAction}); and $\alg a_i^a$ is the component with respect to $E_a \in \alg g$ of the background connection $\tilde \omega$. Thus, thanks to proposition \ref{propositionRelationHodgeStarAndInverseMetricVolume} about the Hodge-* operator, the matter lagrangian density has the formula
\begin{eqnsplit}\label{equationTrivializationMatterLagrangian}
    \mathcal L_{matter}
      &= g^{\nu_1 \nu_2} h^E_{u_1 u_2}  (a^* \phi(\nabla)\mu_i)_{\nu_2}^{u_2}
      + h^{a_1 a_2} h^E_{u_1 u_2} ((\phi_L(\tau) \mu)\comp \tilde \omega_i)_{a_1}^{u_1} ((\phi_L(\tau) \mu)\comp \tilde \omega_i)_{a_2}^{u_2}\\
      &= (a^*\phi(\nabla)\mu_i)_{\nu}^{u} (a^*\phi(\nabla)\mu_i)^{\nu}_{u} 
      + ((\phi_L(\tau) \mu)\comp \tilde \omega_i)_{a}^{u} ((\phi_L(\tau) \mu)\comp \tilde \omega_i)^{a}_{u}, 
\end{eqnsplit}
and the matter gauge functional has the decomposition
\begin{equation}\label{equationDecompositionGlobalMatterActionFUnctional}
    \mathcal S_{matter}[\mu, \hat \omega] = (a^* \phi(\nabla)\mu, a^* \phi(\nabla)\mu) 
    + ((\phi_L(\tau) \mu)\comp \tilde \omega, (\phi_L(\tau) \mu)\comp \tilde \omega).
\end{equation}

When $A$ is the Atiyah Lie algebroid associated to a principal bundle, when the representation is the group induced representation, and when the connection $\hat \omega$ is ordinary, we are left only with the first term, which then coincides with the matter action of a Yang-Mills gauge theory, containing the Kinetic term of the matter fields and the interaction term from the coupling of the gauge fields $A = \hat A$ with the matter field. In general, the second term of the action functional comes from the extra dimensions in which the $A$-connection allows ``covariant derivatives'' to be taken, and in the lagrangian density it becomes a quadratic term for the matter field, i.e. a ``mass term''.