\section{Traditional Gauge Transformations}
%%%%%%%%%%%%%%%%%%%%%%%%%%%%%%%%%%%%%%%%%%%%%%%%%%%%%%
%%%%%%%%%%%%%%%%%%%%%%%%%%%%%%%%%%%%%%%%%%%%%%%%%%%%%%

In this chapter we will finally formulate a gauge theory on a transitive Lie algebroid, by means of defining gauge invariant action functionals.

Throughout this section let $A$, $\alg g$, $E$, representation $\phi$, basis for these, connected? \todo{complete}.

\subsection{Vector Bundles}
%%%%%%%%%%%%%%%%%%%%%%%%%%%%%%%%%%%%%%%%%%%%%%%%%%%%%%

Let $E$ be an arbitrary vector bundle over $M$.

\begin{definition}
\emph{The gauge group of $E$}, denoted by $\alg G(E)$ is the group of (vertical) vector bundles automorphisms of $E$.
\end{definition}

Notice that $\alg G(E)$ is the subset of $\Gamma\End(E)$ of invertible endomorphisms of $E$.

{
\color{gray}
\begin{definition}
\emph{The gauge algebra of $E$} is the (infinite dimensional) Lie algebra $\Gamma\End(E)$ of endomorphisms of $E$.
\end{definition}
}

\subsection{Principal Bundles}
%%%%%%%%%%%%%%%%%%%%%%%%%%%%%%%%%%%%%%%%%%%%%%%%%%%%%%

\subsection{Vector Bundles Associated to a Principal Bundle}
%%%%%%%%%%%%%%%%%%%%%%%%%%%%%%%%%%%%%%%%%%%%%%%%%%%%%%


\section{Gauge Transformations associated to Transitive Lie Algebroids}
%%%%%%%%%%%%%%%%%%%%%%%%%%%%%%%%%%%%%%%%%%%%%%%%%%%%%%
%%%%%%%%%%%%%%%%%%%%%%%%%%%%%%%%%%%%%%%%%%%%%%%%%%%%%%

In this section, $\hat \nabla^E: A \to \alg D(E)$ will be an $A$-connection on $E$, with associated $A$-connection form $\hat \omega^E \in \Omega^1(A, \End(E))$ with respect to the representation $\phi$, and let its curvature form be $\hat R^E \in \Omega^2(A, L)$. The following is a straightforward generalization of the concept of gauge transformation of a vector bundle connection.\todo{If this isn't done before, then mention that this coincides with the traditional definition if $A = TM$}

\begin{definition}
Let $g$ be an element of $\alg G(E)$. \emph{The gauge transformation of the $A$-connection $\hat \nabla^E$ with respect to $g$} is the $A$-connection $\hat \nabla^{E, g} := g^{-1} \comp \hat \nabla^E \comp g$. \emph{The gauge transformation of the curvature of $\hat \nabla^E$} is the curvature of $\hat \nabla^{E,g}$.
\end{definition}

\begin{proposition}
Let $g \in \alg G(E)$. The $A$-connection form associated to $\hat \nabla^{E, g}$ is given by
\begin{equation*}
    \hat \omega^{E, g}  := g^{-1} \comp \hat \omega^E \comp g + g^{-1} \comp \hat d_E g,
\end{equation*}
and its curvature form by
\begin{equation*}
    \hat R^{E, g} = g^{-1} \comp  \hat R^E \comp g.
\end{equation*}
Here, we have omitted the wedge product of the $0$-forms $g, g^{-1} \in \Omega^0(A, \End(L))$ with other forms.
\end{proposition}
\begin{proof}
Recalling the an $A$-connection $\hat \nabla^E$ and its associated form $\hat \omega^E$ are related by $\hat \nabla^E = \phi + \hat \omega^E$, we have that
\begin{align*}
    \hat \omega^{E, g} &= \hat \nabla^{E, g} - \phi \\
        &= g^{-1} \comp \hat \nabla^E \comp g  - \phi\\
        &= g^{-1} \comp \phi \comp g + g^{-1} \comp  \hat \omega^E \comp g  - \phi
\end{align*}
\todo{The $\circ$ can't be removed, since then wedge product would be understood, which means that a commutator appears}
Let $\sectoid X \in \Gamma(A)$ and $\mu \in \Gamma(E) be arbitrary$, then
\begin{align*}
    g^{-1} \comp \phi(\sectoid X) \comp g (\mu) 
        &= g^{-1} \comp \phi (g(\mu)) \\
        &= g^{-1}\comp \tilde \phi(\sectoid X)(g)\mu + g^{-1} \comp g \comp \phi(\sectoid X)\mu\\
        &= g^{-1} \comp [\phi(\sectoid X), g]\mu + \phi(\sectoid X) \mu\\
        &= g^{-1} \hat d_E g(\sectoid X) \mu + \phi(\sectoid X) \mu.
\end{align*}
The last calculation shows that $g^{-1} \comp \phi \comp g = \phi + g^{-1} \comp \hat d_E g$, which, combined with the previous calculation, show the desired equation for $\hat \omega^{E, g}$.

\todo{For curvature}
\end{proof}

{
\color{red}
\begin{proposition}
Let $T \in \End(E)$. The infinitesimal gauge action of $T$ on the $A$-connection form $\hat \omega^E$ is
\begin{equation}
    \hat d_E T + \hat \omega \wedge T,
\end{equation}
on $\hat \nabla^E$ it is
\begin{equation}
    \hat \nabla^E \wedge T
\end{equation}
and on the curvature it is
\begin{equation}
    \hat R^E \wedge T.
\end{equation}
\end{proposition}

This is the connection between the ``traditional'' definitions, and the infinitesimal definitions given below for gauge transformations, so I would like to put the prove at least for the $A = TM$ case, i.e. ordinary vector bundle connections.
}

\linea



\begin{definition}
Given a Lie algebroid sequence $0 \to L \to A \to TM \to 0$, \emph{the gauge algebra of $A$} is the (infinite dimensional) Lie algebra $\Gamma(L)$.
\end{definition}

Let $\hat \omega \in \Omega^1(A, L)$ be an ordinary connection

\begin{definition}
Let $\eta \in \Gamma(L)$. \emph{The infinitesimal gauge action on $\hat \omega$} is the connection
\begin{equation}
    \hat d \eta + \hat \omega \wedge \eta
\end{equation}
and \textbf{the infinitesimal gauge transformation} is
\begin{equation}
    \hat \omega^\eta := \hat \omega + \hat d \eta + \hat \omega \wedge \eta.
\end{equation}
\emph{The infinitesimal gauge action on the curvature} is defined to be
\begin{equation}
    \hat R \wedge \eta
\end{equation}
and its \emph{infinitesimal gauge transformation} is
\begin{equation}
    \hat R^\eta = \hat R + \hat R \wedge \eta
\end{equation}
\end{definition}

Notice that this is the curvature of $\hat omega^\eta$ without the second order terms in $\eta$.

\begin{proposition}
The algebroid morphism corresponding to $\hat \omega^\eta$ is 
\begin{equation}
    \hat \nabla^\eta = \hat \nabla + [\hat \nabla, \eta]
\end{equation}
\end{proposition}

\begin{proposition}
The infinitesimal gauge transformation of the A-connection induced by a generalized connection IS THE SAME AS the induced A-connection of the transformed generalized connection.
\end{proposition}

\section{Gauge Action Functional}
%%%%%%%%%%%%%%%%%%%%%%%%%%%%%%%%%%%%%%%%%%%%%%%%%%%%%%
%%%%%%%%%%%%%%%%%%%%%%%%%%%%%%%%%%%%%%%%%%%%%%%%%%%%%%

\section{Matter Action Functional}
%%%%%%%%%%%%%%%%%%%%%%%%%%%%%%%%%%%%%%%%%%%%%%%%%%%%%%
%%%%%%%%%%%%%%%%%%%%%%%%%%%%%%%%%%%%%%%%%%%%%%%%%%%%%%