\section{Gauge Transformations and Infinitesimal Gauge Transformations}

\subsection{Gauge Transformations in Associated Lie Algebroid}

\subsection{Infinitesimal Gauge Transformations for Connections in Lie Algebroid}

\subsection{Infinitesimal Gauge Transformations for Generalized Connections in Lie Algebroid}

\section{Gauge Invariant Action Functional}

\section{Minimal Coupling to Matter Fields}

\section{Total Action and Examples}

\section{Basics}

    \begin{itemize}
            
        \item Description of the principal bundles
        
        \item Local frame for the trivializations of the Atiyah Lie algebroid given coordinates of $M$ and trivialization of $P$.
        
        \item Local version of $\alg D(E)$? 
        
        \item Local version of a representation. Intersection? Adjoint representation. Trivial representation.
        
        \item (Opc.) I know a local frame in $U_C$ that doesn't come from a Lie algebroid trivialization.
        
        \end{itemize}

        
\subsection{Description of the principal}

The Hopf bundle may be described as the principal bundle $S^3(S^2, S^1)$ over $S^2$ and where the right action of $e^{i \alpha} \in S^1 \subset \CC$ on $(z^1, z^2) \in S^3 \subset \CC^2$ is given by coordinate-wise multiplication: $(z^1, z^2) z := (z^1 z, z^2 z)$. 

The Hopf bundle as $S^3(S^2, S^1)$ and as $S^3(\CC P^1, U(1))$. Perhaps using using the general setting $\FF P^{n-1} \cong S^{dn-1}/S^{d-1}$, al menos en el caso $n = 2$ que basta para mis dos ejemplos.

Atlas for the base spaces:
    Atlas for $S^1, S^2, S^4$ of pg. 1. 
    
        In $S^2$ there are $2$ useful charts: polar coordinates, which will give us another perspective and intuition. Since the transition maps between vector fields is the identity, in $U_C$ I can define the local frame of $TS^2$ $\partial_\phi, \partial_\theta \in \Gamma_{U_C}$.
    
    Atlas for $\FF P^{n-1}$ when $n = 2$ of pg. 2.

Diffeomorphism of the base spaces compatible with these atlases and explicit formulas in pg. 3.

Description of the principal bundles ($n = 2$) as $S^{2d - 1}\FF P^1, SP)$ of pg. 5.

    Action
    
    Principal bundle trivialization / local sections given the previously defined atlas of $\FF P^1$.
    
    Transition functions
    
Description of the principal bundles as $S^{2d-1}(S^d, S^1)$ using the previously defined atlas for $S^d$ of pg. 5 and 6.
    Also for the $2$ polar charts of $S^2$.
    
    Relation with the previous description

Description of the more general principal bundles over $S^2$ and $S^4$
    Transition functions
    
    Local sections

\subsection{Description of the corresponding Atiyah Lie algebroid}

\subsubsection{Global Description of Hopf $S^3$ using $S^2$ as the base space}

* Notation $\underline{\ppal X}$ also valid, as an alternative to $\cl{\ppal X}$

$\underline{\partial_\theta}, \underline{\partial_{\xi^1}}, \underline{\partial_{\xi^2}} \in \Gamma_{U_C} (TS^3/S^1)$ is a local frame of $A$ coming from a local frame of $TS^3$ in $\mathcal U_C = \pi^{-1}(U_C)$.

Bracket and anchor of this local frame.

Having this local frame, a section in $U_C$ can be written with functions $f(\phi, \theta)$ as coordinates, and their equivalence to $\Gamma_{\mathcal U_C}(TS^3)$ is clearly given by \dots. I ONLY CARE, AS FAR AS I HAVE OBSERVED, OF UNDERSTANDING SECTIONS OF THE ALGEBROID.

\subsubsection{Local Descriptions and Change of Coordinates in some very general setting}

Two local trivializations of these Atiyah bundles coming from the trivialization of the principal bundle and the base manifold in the two neighborhoods $U_S$ and $U_N$ if we are taking $S^d$ as base manifold, or $U_2$ and $U_1$ if we are taking $\FF P^1$ instead.

These trivializations have global frames $\set{\partial_\mu, \epsilon}$ which become local frames of $A|_U$.

For Hopf $S^3$: $\nabla_S$, $\psi_S$, $S_S$ can be described more precisely, last page from batch from august 18, to given an intuitive meaning, as well as the analogues for $N$ for $S^2$ in terms of the known local frame and viceversa (the page I have used exhaustively from august 18). 

In general, all the information encoded in the change of coordinate maps: $\alpha^i_j$, $l^i_j$, $\chi^i_j$, $S^i_j$: depends on the transition functions \dbtext{do this}.



\subsubsection{Local description of $\mathcal D(E)$}

\subsubsection{Local description of a representation}