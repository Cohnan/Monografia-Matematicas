\section{Ordinary Gauge Theory}

\subsection{Motivation 1: Schrodinger equation and Dirac Equation}

Link: \href{https://quantummechanics.ucsd.edu/ph130a/130_notes/node296.html}{Gauge symmetry in Schrodinger}

In this section we present a heuristic and non-formal motivation to the main ingredients of gauge theories: principal bundles, associated bundles and connections.

\subsection{Why Principal Bundles?}

(one answer: encode ``internal'' (which I still don't know if it can always be considered nonphysical or nonobservable) symmetry for ALL associated vector bundles... although the theories I know seem to care only about 1 matter field?)

Many physical theories have at its core a differential equations that must be satisfied by a set of ``functions'' in spacetime, or fields. 

For example, 

\begin{itemize}
    \item In classical electrodynamics electric charges generate electric and magnetic fields, which in turn dictate the movement of the charges. Thus, given a source of electric charge described by its scalar field of charge density $\rho$ and its current vector field $\vec J$, the vector fields $\vec E$ and $\vec B$ must satisfy Maxwell Equations, and the charge density changes according to Newton's second law underlying the Lorent'z force \[f = \rho E + J \times B\]: these are a total of $4$ first order and $1$ second order coupled partial differential equations involving all $4$ fields. These equations can be written in terms of the field $A$, in which case $E = \partial A$, $B = \times A$. Now they become second order.
    
    \item In Quantum Mechanics: Schrodinger's equation for the ``wave function''.
    
    \item In Quantum Electrodynamics, the electromagnetic $4$-vector field $A$ must satisfy the Yang-Mills equations \[\partial_\nu F^{\nu \mu} = e \bar \psi \gamma^\mu \psi\], and the electron field $\psi$ satisfies Dirac's equation \[i \gamma^\mu \partial_\mu \psi - m\psi = e \gamma^\mu (A_\mu + B\mu) \psi\]\complete{that is only lorentz force, per unit volume I believe, I'm not sure which is the equation of motion}. These are, again, coupled second order partial differential equations.
    
    \item Electroweak Theory. 
\end{itemize}

Some ``physical fields'', i.e. satisfying a set of equations proposed in physics, can be transformed globally by the action of a group and the resulting field is, once again, a physical field. For example, if a field $\psi$ were to satisfy the Schrodinger's equation/Dirac equation (a possible special-relativistic version of Schrodinger's equation) $i \hbar \gamma^\mu \partial_\mu \psi - mc \psi = 0$, then the field $\psi ' = e^{i \theta} \psi$ also satisfies this equation. This corresponds to the action of an element in $U(1)$ on $\psi$. However, this equation is no longer satisfied by the transformed field if we allow $\theta$ to depend on spacetime, because derivatives of $\theta$ now appear as unwanted terms. However, quantum theory (talking about Schrodinger's equation) tells us that physical observables like position, linear and angular momentum and energy only depend on the induced probability density $f(x) = |\psi(x)|^2$, so not only a a globally gauge transformed wave equation produces the same experimental results, but also the locally gauge transformed one, which changes the phase of the wave equation in a position dependent way, should be a ``physical field'' as it has the same physical content. \unsure{I think it is not true that we can not differentiate one field from the other once the $A$ fields appear, so I'm confused} Alternative: However, physical theories like QED or in general the Standard Model of Particles can be seen to arise mathematically if we force this kind of ``local gauge transformation'' to produce physical fields. In these theories, ``gauge theories'', the solution comes from changing slightly the differential equations by adding new fields that ``interact minimally'' with the matter fields and which have to obey certain gauge transformation rules in order to cancel the unwanted extra terms \unsure{how thorough should I be with this? Should I deduce the transformation properties? Here? Perhaps do it in full detail with QED?}. In our example of Dirac's equation we \complete{example: QED} \dbtext{Can Classical EM and the EM field $A$ arise from this procedure: start with equation satisfied by }
 (\dbtext{how can this be seen in the equations for Schrodinger? why is the physical thing the probability density, which isn't what appears in the ``equation of motion'': Schrodingers equation?})

This liberty to change (smoothly) at every point the value of the matter field and still getting a physical matter field may be understood as changing an ``internal'' coordinate or degree of freedom of the field with value in the pointwise acting group. This denomination of ``internal'' to this additional degree of freedom is justified in the relevant physical theories by the interesting property that no ``observable'' property of these matter fields is affected by its pointwise value, in other words, the physical content of a matter field and any of its gauge transformations is exactly the same, so, in some sense, we can not have access to the value of this degree of freedom \dbtext{later this internality is also observed in the ``field strength'' or curvature} \unsure{is this (the invisibility in both $\psi$ and $F$, or really in what is physical, of this degree of freedom) true for any gauge theory, or only for abelian gauge theories}. 

 AQUI VOY
 
 Furthermore, implicit coordinate choice... principal bundle.
 
 Matter field as associated vector bundle section because group acts on it pointwise in an equivariant way.
 
 Additional terms can be seen to arise from modifying derivatives -> covariant derivative induced from connection in principal bundle. Why covariant in this case? THis gauge transformation can be understood as a mere \rbox{``change of coordinates''}, so the covariant derivative asserts that changing coordinates doesn't change the fact that this field is a physical one.
 

``Real'' motion occurs in total space, but its projection into spacetime is all we can see?

If that is not taken into account -> Path lifting -> Connection

Locally this connection appears as fields. IN FACT SOME OF THE FUNDAMENTAL FIELDS ARISE PRECISELY AS CONNECTIONS.

Examples:
\begin{itemize}
    \item $E$, $B$ are first derivatives of the connection $A$.
    
    \item Photons in QED are the connection
    
    \item Z and W bosons are the connection
\end{itemize}

\dbbox{What guarantees that the emerging $A$ field has as equation of motion of lagrangian the EM one.} i.e. where does the equation of motion come from

\subsection{Why Associated Bundles?}

Principal connections are important to tell us a way to define covariant derivatives in EVERY associated vector bundle, whose sections are matter fields.

\subsection{Why Connections?}

The previous sections shows us that:

Connections arise in a physical setting when, once a vector valued field satisfies a desired physical equation, it is imposed that a position dependent transformation by a group $G$ of this field generates also a physical field. This is achieved by replacing the coordinate directional derivatives by ``covariant derivatives'' to guarantee that this ``change of coordinates'' doesn't affect the underlying physics. However, the introduction of covariant derivatives imply the introduction of ``fields with certain transformation properties'', which can be seen to be precisely the transformation properties of a principal bundle connection.


\dbbox{What suggests that the lagrangian of this new conenction field should be $tr(F^2)?$} Perhaps something more than ``that's how it is for classical EM''.

\subsection{Motivation 2: Magnetic Monopole}

\subsection{Space of Local Symmetries}

\subsection{Particle Fields and Implementation of Symmetries}

\subsection{Minimal Coupling: covariant derivative}

\subsection{Gauge and Infinitesimal Gauge Transformations}

\subsection{Action Functional}

\subsection{Higgs Mechanism}

%%%%%%%%%%%%%%%%%%%%%%%%%%%%%%%%%%%%%%%%%%%%%%%%%%%%%%%%%%%%%%%%%%%%%%%%%%%%%%
\section{How to think/Why (transitive) Lie Algebroids?}




\subsection{Why Lie Algebroids?}
Throughout this section $P$ or $P(\pi, M, G)$ will be a principal bundle, with a principal connection form $w : TP \to P \times \algeb g$. \unsure{Forma correcta de llamar a este mapa, pues no es realmente una forma}\unsure{Estoy motivando la noción de algebroides de Lie y curvatura desde el concepto de levantamiento de caminos, a pesar de que esta noción de levantar caminos dejará de ser posible con la generalización de conexión propuesta en los artículos de Lazzarini, Fournel, et. al.} 

So far we have seen that principal bundles over the base space $M$ and their associated vector bundles are a way to formalize gauge theory, where the additional internal degrees of freedom necessary to have a full description of the physical phenomena and their consequences can be seen in bigger spaces, called fibrations, over the base space $M$. To extrapolate what is happenning in the total spaces based on what we can see, a principal connection has been used. 

To motivate the concept of Lie algebroids we will \rtext{develop the Atiyah sequence of a principal bundle as an infinitesimal version of the principal bundle which is minimal, in some sense, to talk about connections and path lifting, hence allowing covariant derivatives for ``matter fields'', i.e. sections of associated vector bundles}. Although important definitions naturally arise here, they will be presented once again in chapter \ref{chp:basicLie} in the general setting of Lie algebroid theory.


\subsubsection{Path Lifting}

The Dirac monopole motivated the introduction of a principal bundle over the base space $S^3$ as the space where a path OF in the base space would be lifted to explain . However, a principal connection doesn't offer a unique horizontal lift of a path, so we will see how the Atiyah Lie algebroid $\frac{TP}{G}$ arises naturally as the space where this lift is indeed unique.

\begin{definition} [Horizontal Lift of a Path]
Let $\gamma: I \to M$ be a (smooth) path in $M$. A path $\tilde{\gamma}: I \to P$ is called a \emph{lift of the path $\gamma$} if $\gamma = \pi \comp \tilde{\gamma}$. Given a principal connection $w$ in $P$, a lift is called \emph{horizontal} if furthermore $w_p(\tilde{\gamma}'(t_0)) = 0$ for each $t_0 \in I$ and $p = \Tilde{\gamma}(t_0)$.
\end{definition}

\begin{lemma} [Uniqueness of Horizontal Lifts through a point]
If $\Tilde{\gamma}_1, \Tilde{\gamma}_2:I \to P$ are two horizontal lifts of the path $\gamma:I \to M$ in $P$, and if they share a point $p = \Tilde{\gamma}_1(t_0) = \Tilde{\gamma}_2(t_0)$, then $\Tilde{\gamma}_1 = \Tilde{\gamma}_2$. \info{See Ehresmann connection in Wikipedia. It seems this is deduced from the rank-nullity theorem}
\end{lemma}


It may seem that the possible horizontal lifts is far from unique, as many different '`initial points'' may be chosen, however the group action shows the close relation between them all.

\begin{theorem}\label{thm:pathRed}
If $\tilde{\gamma}_1: I \to P$ and $\tilde{\gamma}_2: I \to P$ are two horizontal liftings of the path $\gamma:I \to M$, then there is an element $g \in G$ such that $\tilde{\gamma}_1 (t) = \tilde{\gamma}_2 \cdot g$ for all $t \in I$. Conversely, if $\tilde{\gamma}_1$ is a horizontal lifting of $\gamma$, then $\Tilde{\gamma}\cdot g :I \to P$ is also a horizontal path lift of $\gamma$ for any $g\in G$.
\end{theorem}

Thus, to lift a path from $M$ to $P$ horizontally (given a connection), the group action is somehow redundant. To better understand what is happening it is necessary to have an \dgtext{explicit view at the phenomena occurring infinitesimally}, as the differential structure has an important role in determining which paths in $P$ can be considered lifts of a path in $M$, in addition to being necessary to determine what horizontal means. 

%We first notice the tangent spaces allow us to see how the differentiability of the path has a role restricting the possible paths in $P$ that can be considered lifts of $\gamma$. %For every $t \in I$, we can apply the chain rule Where lifting a path $\gamma:I \to M$ to $P$, for each $m = \gamma(t_0) \in M$ the vector $X_m$ tangent to $\gamma$ in $m$ gets lifted to a tangent vector in the horizontal space of $m$, but the exact vector in $\pi_*^{-1}(X_m) \subset H_mP$ that it gets assigned to depends on which horizontal lift of $\gamma$ we choose.

\begin{proposition}
A path $\tilde{\gamma}:I \to P$ is a lift in $P$ of $\gamma:I \to M$ if and only if $\gamma'(t) = \pi_*(\tilde{\gamma}'(t)) \in T_{\gamma(t)}M$ for each $t \in I$.
\end{proposition}

\begin{proof}
Given $t \in I$, $\pi_*(\tilde{\gamma}'(t)) \in T_{\gamma(t)}M$ tells us that $\tilde{\gamma}'(t) \in T_p M$ for some $p \in P$ in the fiber of $\gamma(t)$, i.e. $\gamma = \pi \circ \tilde{\gamma}$. Conversely, if $\tilde{\gamma}$ is a path lift of $\gamma$, $\gamma = \pi \circ \tilde{\gamma}$, and so by the chain rule $\gamma'(t) = \pi_*(\tilde{\gamma}'(t))$.
\end{proof}

This motivates us to define the following ``infinitesimal alternative'' of a path. We will denote the interval $[-1, 1] \subset R$ by $I$.

\begin{definition} [Tangent Path, Associated Tangent Path]
Let $B$ be a ($C^2$/smooth)-manifold and $\rho: TB \to B$ be the projectio map of its tangent bundle. A \emph{tangent path in $B$} is a ($C^1$/smooth) map $\overline{\gamma}:I \to TB$ such that $\rho \comp \overline{\gamma} : I \to B$ is a ($C^2$/smooth) map in $B$. If $\gamma: I \to B$ is a ($C^2$/smooth) path in $B$, we call the ($C^1$/smooth) path $\gamma'(t): I \to TB$ the \emph{tangent path associated to $\gamma$}. \unsure{I'm not sure if there is any value in being explicit with the differentiability conditions necessary, or if perhaps I should have everything be smooth}  
\end{definition}

This definitions show that tangent paths in a differentiable manifold are an alternative, ``infinitesimal'', way of talking about differentiable paths. This is a first step toward rewriting the path lifting procedure in an ``infinitesimal language'', which will be the setting where the uniqueness of a path lift can be stated. 

\begin{definition}[Horizontal Tangent Path]
Given a tangent path in $M$  $\gamma': I \to TM$, we say it is a \emph{horizontal tangent path if} if $w \comp \overline{\gamma} = 0$.
\end{definition}

We see now how this inifitesimal language doesn't require additional spaces to talk about horizontality, as the previous definition of a horizontal lift did as it required talking about both $P$ and $TP$.

\begin{theorem}
Let $\gamma$. $\Tilde{\gamma}$ is a path lift of $\gamma$ through $\pi$ if and only if $\overline{\Tilde{\gamma}}$ is a path lift of $\overline{\gamma}$ through $\pi_*$. $\Tilde{\gamma}$ is a horizontal path lift of $\gamma$ through $\pi$ if and only if $\overline{\Tilde{\gamma}}$ is a horizontal tangent path lift of $\overline{\gamma}$ through $\pi_*$.
\end{theorem}

\rtext{Thus, (horizontal) path lifting can be understood infinitesimally as a way to assign a (horizontal) path in $TP$ to a path $\gamma'(t)$ in $TM$}. 

Now, to see once again the redundancy of the action we first understand how $G$ acts on $TP$.

\begin{proposition}
The map $R : G \times TP \to TP, (g, \mathfrak{X}) \mapsto R_{g*}(\mathfrak X)$ is a right Lie group action of $G$ induced by the right Lie action of $G$ on $P$.
\end{proposition}

The following corollary of \ref{thm:pathRed} shows again the redundancy of the group action, but now in infinitesimal languange.

%\begin{corollary}\label{cor:tpRed}
%Let $\gamma$. Suppose $\overline{\Tilde{\gamma}}_1$, $\overline{\Tilde{\gamma}}_2: I \to TP$ are two horizontal tangent path lifts of $\overline{\gamma}$. Thus there exists an element $g \in G$ for which, for all $t \in I$, $\overline{\Tilde{\gamma}}_2(t) = R_{g*}\overline{\Tilde{\gamma_1}}(t) \in T_{\gamma(t)}M$.
%\end{corollary}
\begin{corollary}
If $\overline{\tilde{\gamma}}_1, \overline{\tilde{\gamma}}_2: I \to TP$ are two horizontal liftings of the path $\overline{\gamma}:I \to TM$, then there is an element $g \in G$ such that $\overline{\tilde{\gamma}}_1 (t) = \overline{\tilde{\gamma}}_2 (t) \cdot g$ for all $t \in I$. Conversely, if $\overline{\tilde{\gamma}}$ is a horizontal lifting of $\overline{\gamma}$, then $\overline{\Tilde{\gamma}}\cdot g :I \to TP$ is also a horizontal path lift of $\overline{\gamma}$ for any $g\in G$.%Let $\gamma$. Any two horizontal tangent path lifts of $\overline{\gamma}:I \to TM$ are related Once we have chosen $\mathfrak{X}_{p_1} \in \pi_*^{-1}(X_m) \subset H_mP$, the whole horizontal lift of $\gamma$ is determined, and any other horizontal lift is generated by the action of some $g \in G$ on all of the tangent vectors in $TP$ that compose the horizontal lift, which in equivalent to determining a distinct element $\mathfrak{X}_{p_2} \in \pi_*^{-1}(X_m)$.
\end{corollary}

We see from the corollaries of theorem \ref{thm:pathRed} that the exact path in $P$ we choose as the horizontal lift of a $\gamma:I \to M$ is unimportant, as they are all generated from one another by the action of $G$, either in $P$ or in $TP$. So, \rtext{$TP$ is looking rather big for our purposes of lifting a path horizontally from $M$ to $P$}.

This suggests that horizontal path lifting is indeed unique, not just modulo the group action, if we consider the set $\frac{TP}{G}$ as the place where a tangent vector in $TM$ is lifted to. To formalize this we first need to give some appropriate structure to $\frac{TP}{G}$.

\begin{definition}
$\frac{TP}{G}$ vector bundle over $M$. $\Gamma(\frac{TP}{G})$ is a $C^\infty(M)-$module with well defined brackets. $a = \pi_{*}(<\cdot>)$ is a surjective vector bundle morphism that respects the brackets.  
\end{definition}

\begin{theorem}
Let $\gamma:I \to M$ and $w:TP \to P \times \algeb g$ a principal connection form in the principal bundle $P(\pi, M, G)$. Then there is a unique path lift $\tilde{\gamma}:I \to \frac{TP}{G}$.
\end{theorem}

However, we must not forget that the principal connection form which defines what \emph{horizontal} means is \improvement{I have to decide if a connection form is either a map from $TP$ or from sections of it} a map involving all of $TP$, $w:TP \to P\times \algeb g$. \dgtext{Fortunately, for the connection $TP$ is much more than needed in the same sense as before, as we'll see below.}

\subsubsection{Connection Form}

One of the defining properties of a principal connection form $w:TP \to P \times \algeb g$ is its (equi?)(in?)variance under the group action, and this is what will allow us to restrict its definition to $\frac{TP}{G}$. More explicitly, for any $\mathfrak X_p \in T_p P$, $w(R_{g*}\,\mathfrak{X_p}) = (R_g(p), Ad_{g^{-1}} w(\mathfrak{X_p})) \in P \times \algeb g$, which tells us that once the effect of $w$ is on $\mathfrak{X}_p \in TP$, the effect of $w$ of every element of its orbit in $TP$ is established.

However, to remove this redundancy in the definition of $w:TP \to P \times g$ it must be possible to identify unambiguously as a single point every element in $P \times \algeb g$ which is the image of every orbit in $TP$.

\begin{proposition}
The map $R:G \times (P \times \algeb g) \to P \times \algeb g$, $(g, (p, A)) \mapsto (R_g(p), Ad_{g^{-1}} A)$ is a Lie group action of $G$ on $P \times \algeb g$.
\end{proposition}

\begin{definition}
$\frac{P \times \algeb g}{G}$ is a vector bundle over $M$, and $\Gamma \left(\frac{P \times \algeb g}{G} \right)$ is a $C^\infty(M)-$module with well defined bracket.
\end{definition}

\begin{proposition}
Let $w:TP \to P \times \algeb g$ be a principal connection form. Every element $\mathfrak{Y}_q$ in the orbit of $\mathfrak{X}_p$, i.e. $\mathfrak{Y}_q = R_{g*} \mathfrak{X}_p$ for some $g\in G$ satisfies $w(\mathfrak{Y}_p) = \mathfrak{X}_p \cdot g$.
\end{proposition}

Finally, the next theorem is our first step to show that \ytext{all the information that a principal connection form has, including the meaning of ``horizontal'', can be encapsulated in a map $\omega: \frac{TP}{G} \to \frac{P \times \algeb g}{G}$, which makes use of this minimal infinitesimal version of the principal bundle and its adjoint bundle}.

\begin{theorem}
Let $w:TP \to P \times \algeb g$ be a principal connection form. The map $\omega: \frac{TP}{G} \to \frac{P \times \algeb g}{G}$, $\omega(<\mathfrak{X}_p>) = <w(\mathfrak{X}_p)>$ is a well defined vector bundle morphism. It can be seen as a $C^\infty(M)-$module morphism $\omega: \Gamma(\frac{TP}{G}) \to \Gamma(\frac{P \times \algeb g}{G})$, i.e. a $\frac{P \times \algeb g}{G}$-valued one form $\omega \in \Omega^1(\frac{TP}{G}, \frac{P \times \algeb g}{G})$.
\end{theorem}

\lgtext{However, we can not yet say that a connection reform and a principal connection form are equivalent yet, as there is another defininig property of a principal connection, which we will now formulate this in the language we are currently developing.}

\begin{definition}
Let $P(\pi, M, G)$. Given $A \in \algeb g$, we define \emph{the fundamental tangent vector associated to $A$ at $p$}, denoted by $A_p^P$, as the tangent vector $ \frac{d}{dt}|_{t = 0} p \cdot exp(tA) \in T_pP$. The \emph{fundamental vector field associated to $A$} is the vector field in $P$, denoted by $A^P$, defined pointwise as the fundamental tangent vector.\improvement{why is it indeed a smooth or $C^n$ section?}
\end{definition}

The following theorem allows us to see that the adjoint Lie algebroid of an Atiyah Lie algebroid are related via the fundamental tangent vectors.

\begin{lemma}
The map $i:\frac{P \times \algeb g}{G} \to \frac{TP}{G}$, $<p, A> \mapsto <A_p^P>$ is an injective vector bundle morphism that respects the Lie bracket. Inclusion  map.
\end{lemma}

The following apparently useless definition will, nonetheless, be seen as an important property when talking about Lie algebroids.

\begin{definition}
The anchor of $\frac{P \times \algeb g}{G}$, defined by $a:\frac{P \times \algeb g}{G} \to TM$, $<p, A> \mapsto 0_{\pi(p)}$.
\end{definition}

\begin{proposition}
The anchor of $\frac{P \times \algeb g}{G}$ is a vector bundle morphism which respects the brackets and commutes with the anchors.
\end{proposition}

Now we can see the full extent of the inclusion map.

\begin{theorem}
The inclusion map $i:\frac{P \times \algeb g}{G} \to \frac{TP}{G}$ is a vector bundle morphism that respects the brackets and commutes with the anchor maps.
\end{theorem}

Finally we can formulate completely the full definition of a principal connection form in the current language. any $A \in \algeb g$.

\begin{proposition}
Let $A \in \algeb g$, and $w:TP \to P \times \algeb g$ be a principal connection form. The property $w(A^P_p) = (p, A)$ of the connection form induces the following property of the induced map $\omega:\frac{TP}{G} \to \frac{P \times \algeb g}{G}$: $\omega \comp i = Id_{\frac{P \times \algeb g}{G}}$.
\end{proposition}

\begin{definition} [Connection Reform]
A vector bundle morphism $\omega:\frac{TP}{G} \to \frac{P \times \algeb g}{G}$ that satisfies $\omega \comp i = Id_{\frac{P \times \algeb g}{G}}$ is called \emph{a connection reform}.
\end{definition}

\rtext{Finally}

\begin{theorem}
A principal connection form in $P$ induces naturally a connection reform in $\frac{TP}{G}$. Conversely, any connection reform in $\frac{TP}{G}$ induces naturally a principal connection form.
\end{theorem}


\subsubsection{Covariant Derivatives}

As all the information of a principal connection is incorporated in the connection reform, the usual covariant derivates defined on the associated vector bundles of the principal bundle. ALTHOUGH I MAY USE THIS OPPORTUNITY TO SEE THE CONNECTION AS EQUIVALENT TO THE TM -> TP/G MAP THAT APPEARS IN THE DEFINITION OF COVARIANT DERIVATIVES AND THAT WILL BE USED AS THE INITIAL DEFINITION OF A CONNECTION IN AN ATIYAH LIE ALGEBROID


The gauge potentials, and so the covariant derivatives WHERE (RTA: in every associated bundle) can also be formulate in this infinitesimal language. \complete{Otra razon alternativa a path lifting}

Intuition: trivial principal bundle

\begin{proposition}
$\frac{P \times \algeb g}{G}$ can be injected as vector bundle (in particular as manifold: immersion) in $\frac{TP}{G}$
\end{proposition}

\begin{theorem}
A map $\omega:\frac{TP}{G} \to \frac{P \times \algeb g}{G}$ is equivalent to a principal connection (in a canonical way).
\end{theorem}

\begin{definition}
Anchor
\end{definition}

\begin{proposition}
$\frac{P \times \algeb g}{G}$ corresponds to the kernel of the anchor.
\end{proposition}

De alguna manera debo motivar \complete{De alguna manera debo motivar la introduccion del corchete en ambos lados: difeomorfismos? Algebra de Lie?}

\begin{definition}
Corchete en $\frac{TP}{P}$
\end{definition}

\begin{definition}
Corchete en $\frac{P \times \algeb g}{G}$
\end{definition}

\begin{proposition}
El corchete conmuta con la inclusion.
\end{proposition}



%%%%%%%%%%%%%%%%%%%%%%%%%%%%%%%%%%%%%%%%%%%%%%%%%%%%%%%%%%%%%%%%%%%%%%%%%%%%%%
\section{Examples: Electroweak Interaction; Monopole; QED in $S^1$}


