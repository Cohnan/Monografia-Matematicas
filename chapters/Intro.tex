% La dinámica de las partículas elementales y sus interacciones son modeladas por teorías gauge cuantizadas, un tipo de teorías de campos. Éstas incluyen a la Electrodinámica Cuántica, a la Interacción Electrodébil, a la Cromodinámica Cuántica y el Modelo Estándar de la Física de Partículas, la cual describe a tres de las cuatro fuerzas fundamentales conocidas y clasifica a todas las partículas elementales conocidas. El modelo matemático usualmente utilizado para la formulación de las teorías gauge, previo a su cuantización, es el de la geometría diferencial de haces principales y haces vectoriales. En este proyecto se estudia una posible generalización de la formulación de teorías gauge en el lenguaje de algebroides de Lie transitivos a través de funcionales de acción, de la cual la formulación estándar se deriva como el caso particular en que el algebroide subyacente es el algebroide de Lie de Atiyah asociado a un haz principal. La generalización presentada introduce naturalmente nuevos campos $\tau$ que inducen en el lagrangiano nuevos términos de acomple de estos campos $\tau$ tanto con los campos gauge, como con los campos de materia, sin tener que recurrir a mecanismos externos a la teoría como de rompiento de simetría para el surgimiento de términos de masa de los campos.

The dynamics of the elementary particles of physics and their interactions are modeled by quantized gauge theories, a type of field theories. These include Quantum Electrodynamics, the Electroweak Interaction, Quantum Chromodynamics and the Standard Model of particle physics, which describe three of the four known forces and which classify all observed fundamental particles. The mathematical framework usually used for the formulation of gauge theories, prior to quantization, is that of the differential geometry of principal fiber bundles and vector bundles. Based on \cite{Fournel2013, Fournel2011, Lazzarini2012}, in this document we study the use of the structure of transitive Lie algebroids as the mathematical framework for a possible generalization of the \textit{formulation of a gauge theory through an action functional}: an integral of a differential form on the algebroid that enters as a parameter. From this, the standard formulation is derived as a particular case in which the underlying algebroid is the Atiyah Lie algebroid associated to a corresponding principal bundle. The formulation here presented naturally introduces new fields $\tau$ that induce the appearance of new coupling terms between these $\tau$ fields with both the gauge fields, as well as with the matter fields, implying, at least on the Atiyah Lie algebroid of a principal bundle, the presence of mass terms for these fields without having to appeal to mechanisms external to the theory like spontaneous symmetry braking.


%En el primer capítulo se repasan los fundamentos sobre álgebroides de Lie
%sobre una variedad M
% , haces vectoriales $A$ con un campo de corchetes de Lie $[\,,\,]$ y un morfismo de haces $a:A \to TM$ llamado ancla. Se hace especial énfasis en los algebroides de Lie transitivos, aquellos cuya ancla es sobreyectiva fibra a fibra, y que por lo tanto son suma directa del haz tangente del espacio base con un haz de álgebras de Lie, llamado un algebroide de Lie adjunto. El ejemplo que motiva el estudio de estos algebroides es el algebroide de Lie de Atiyah $TP/G$ asociado a un haz un principal $P$ con grupo de estructura $G$, el cual tiene como algebroide adjunto a $P \times \mathfrak g/G$ donde $\mathfrak g = \text{Lie}(G)$; las conexiones de un haz principal están en correspondencia biyectiva con los morfismos de haz vectorial entre $TP/G$ y $P \times \mathfrak g/G$. El algebroide de Lie transitivo de derivaciones $\mathfrak D(E)$ de un haz vectorial $E$ permite la expresión del concepto de conexión en $E$, y de la generalización de la noción de haz vectorial asociado a un haz principal a través del concepto de representación $\phi: A \to \mathfrak D(E)$ de un algebroide de Lie $A$ en $E$. Una propiedad de los algebroides de Lie transitivos que facilita su manipulación, especialmente para quienes están familiarizados con el punto de vista de las teorías gauge utilizado en la literatura de la física, es el hecho de que localmente son isomorfos a los algebroides de Lie triviales, de la forma $TM\oplus (M \times \mathfrak g)$ con $M$ una variedad, lo cual permite la descripción de los algebroides de Lie transitivos a partir de descripciones locales en abiertos $U_i$ y dos funciones de pegado en cada intersección no vacía de abiertos $U_{ij}$; para $TP/G$, con $G$ un grupo de Lie matricial, estas dos funciones son las familiares $\alpha^i_j = g^{-1}_{ij}\cdot \, \cdot g_{ij}$ y $\chi^i_j = g_{ij}^{-1} dg_{ij}$, con $g_{ij}$ la función de transición de $P$.

In chapter \ref{chp:basicLie}, following Mackenzie \cite{Mackenzie2005}, we review the needed fundamentals about Lie algebroids, vector bundles $A$ over a manifold $M$ with a Lie bracket field $[\cdot, \cdot ]$ and a vector bundle morphism $a:A \to TM$ called the anchor. Special emphasis is made on transitive Lie algebroids, those with fiberwise surjective anchor, meaning that they are the direct sum of the tangent bundle of the base space with a Lie algebra bundle called an adjoint Lie algebroid of $A$. The example that motivates the study of these algebroids is the Atiyah Lie algebroid $TP/G$ associated to a principal bundle $P$ with structure group $G$, which has $P \times \alg g/G$ as adjoint Lie algebroid, where $\alg g = \text{Lie}(G)$; the principal bundle connections of $P$ are in a bijective correspondence with the vector bundle morphisms from $TP/G$ to $P \times \alg g/G$. The transitive Lie algebroid $\alg D(E)$ of a vector bundle enables the formulation of the concept of a connection on $E$, as well as a generalization of the concept of a vector bundle associated to a principal bundle through the concept of a Lie algebroid representation $\phi: A \to \alg D(E)$ on $E$. A property of transitive Lie algebroids that simplifies their manipulation, specially for those familiar with the point of view used on gauge theories in the physics literature, is the fact that locally they are isomorphic to trivial Lie algebroids, of the form $TM \oplus (M \times \alg g)$ with $M$ the base manifold. This allows the description of transitive Lie algebroids from a family of local descriptions identical to that of the traditional gauge theories over open sets $U_i \subset M$, plus two pasting functions on each non-empty intersection $U_i \cup U_j$; for $TP/G$, with $G$ a matrix Lie group, this two pasting functions are the familiar $\alpha^i_j = g_{ij} \,\,\cdot \,\, g_{ij}^{-1}$ and $\chi^i_j = g_{ij} dg_{ij}^{-1}$, with $g_{ij}$ transition functions of $P$. In this chapter an important effort is made to understand the local trivializations of the different concepts introduced throughout the chapter, since this point of view provides a way to build piece by piece all the necessary ingredients for the formulation of gauge theories on transitive Lie algebroids in concrete examples.

In chapter \ref{chp:diffStruc} we define spaces of differential forms on a Lie algebroid based on \cite{Lazzarini2012}, enabling the definition on the following chapters of connection forms and of integration on transitive Lie algebroids. Given a representation vector bundle $E$ of an algebroid $A$, a differential form is a multilinear, alternating vector bundle morphism taking values in $A$ and returning values on $E$; the space of such $E$-valued forms is denoted by $\Omega^\bullet(A, E)$, and on it a differential $\hat d_\phi$ and a wedge product $\wedge$ will be defined. Section \ref{ChFormsSectionTheory} of this chapter is devoted to understand the structure of this spaces, and hence the properties that its elements and operations posses, providing a way to manipulate forms to obtain both theoretical and practical results. Since a transitive Lie algebroid is locally trivial, we first study in section \ref{ChFormsSectionTrivial} important spaces of differential forms on trivial Lie algebroids, and then in section \ref{ChFormsSectionLocaltransitive} we conclude that it is indeed correct to study global forms on a transitive Lie algebroid $A$ from the family of forms on trivial Lie algebroids that a trivialization of $A$ provide. The majority of the chapter in focused on laying the theoretical background that justifies the manipulation of forms that is observed in the guiding articles.

%En el capítulo $2$ se definen espacios de formas diferenciales sobre los algebroides de Lie, dando las bases para la definición en los siguientes capítulos de las formas de conexión e integración en algebroides de Lie transitivos. Dada una representación $\phi: A \to \mathfrak D(E)$ de un algebroide de Lie $A$ sobre un haz vectorial $E$, una forma diferencial de grado $n \in \mathbb Z_{\geq 0}$, elemento de $\Omega^\bullet(A, E)$, es un morfismo de haces multilineal alternante que toma valores en $A$ y retorna valores en $E$. El álgebra diferencial graduada $(\Omega^\bullet(A), \wedge, \hat d_A)$ de formas diferenciales con valores escalares, que incluirá a la forma de volumen de ciertos algebroides, permitirá definir a $(\Omega^\bullet(A, E), \hat d_\phi)$ como un complejo diferencial sobre el anillo graduado $(\Omega^\bullet(A), \wedge)$. Cuando $E$ es además un haz de álgebras, el producto $\bullet$ en las fibras de $E$ induce un producto $\wedge^\bullet$ entre las formas diferenciales con valores en $E$, haciendo de $(\Omega^\bullet(A, E), \wedge^\bullet, \hat d_\phi)$ un álgebra diferencial graduada, y en particular en álgebra de Lie diferencial graduada cuando $E$ es un haz de álgebras de Lie, como lo son el haz $\text{End}(E)$ o los haces adjuntos de un algebroide de Lie transitivo. Cuando $A$ es transitivo, la trivialización local del algebroide da lugar a la localización de las formas diferenciales, donde esta trivialización es una operación que respeta tanto a los diferenciales como a los productos cuña. En cada trivialización local sobre $U$ abierto de la variedad base que también trivialice a $E$, el diferencial toma la forma $\hat d_\phi = d + s'$, donde $d$ es el diferencial usual de $TU$ y $s'$ es un diferencial asociado al diferencial de Chevalley-Eilenbert en el álgebra exterior del álgebra de Lie, fibra del haz adjunto a $A$.

Both the connections of a principal bundle $P$, and the connections of a vector bundle may be seen as sections of the anchor of the associated transitive Lie algebroids $TP/G$ and $\alg D(E)$, and this is called an ordinary connection on the Lie algebroids \cite{Mackenzie2005}. In chapter \ref{chp:connections} we define two notions of connection given a Lie algebroid $A$, each one generalizing one of the previously mentioned kinds of connection \cite{Lazzarini2012}. Generalizing principal bundle connections, the space of \textit{generalized connections} on a transitive Lie algebroids is defined as the space of adjoint Lie algebroid-forms on $A$. Generalizing vector bundle connections on $E$, also called covariant derivatives, the $A$-connections are introduced, which, roughly, allow the derivation of sections of $E$ with respect to directions not only tangent to the base manifold, but with respect to the ``generalized directions'' of $A$. Just as principal connections induce covariant derivatives on the associated vector bundles, a generalized connection on the transitive Lie algebroid $A$ will induced an $A$-connection given a representation $\phi:A \to \alg D(E)$ on $E$; under a trivialization of $A$ and $E$ and the base manifold $M$ the produced $A$-connection decomposes in two types of derivations of sections of $E$: tangent $\hat \nabla^{E, loc}_{\partial_\mu} = \partial_\mu + B_\mu + A^b_\mu \phi^{loc}(E_b)$, and vertical $\hat \nabla^{E, loc}_{E_a} = - \tau^b_a \phi^{loc} (E_b)$, where the $\tau$ fields indicate the deviation of the connection on $A$ from being an ordinary connection, $E_a, E_b$ being the elements of a basis of $\mathfrak g$ and $B$ is a Maurer-Cartan form coming from the trivialization of the representation $\phi$. In particular, the extra degrees of freedom in which covariant derivates can be taken become a coupling between the sections of $E$, or ``matter fields'', with the tau fields. 

%Tanto las conexiones de un haz principal $P$, como las conexiones de un haz vectorial $E$ pueden ser vistas como secciones del ancla de los algebroides de Lie transitivos asociados, también llamadas conexiones ordinarias. En el capítulo $3$ se definen dos nociones de conexión dado un algebroide de Lie transitivo $A$, cada una generalizando una de los dos tipos de conexión previamente mencionadas. Generalizando a las conexiones principales, el conjunto de formas de conexión generalizadas en $A$ es $\Omega^1(A, L)$, donde $L$ es cualquier algebroide de Lie adjunto a $A$; cada una tiene asociada un endomorfismo $\nabla: A \to A$ de haces, que será una proyección horizontal si la conexión es ordinaria, y la $2$-forma de curvatura expresa la falla de $\nabla$ de respetar el corchete. Generalizando a las conexiones, o derivadas covariantes, de haces vectoriales se definen las $A$-conexiones de $E$ como morfismos de haces vectoriales $\hat \nabla^E: A \to \mathfrak D(E)$ que respetan el ancla, con su curvatura definida como la falla de $\hat \nabla^E$ de respetar el corchete. Así como una conexión de un haz principal define una derivada covariante en los haces vectoriales asociados, sobre los haces vectoriales $E$ de representación de un algebroide de Lie transitivo $A$ tenemos el concepto, central en este proyecto, de la $A$-conexión en $E$ producida por una conexión generalizada $\hat \omega$ en $A$; si la representación es $\phi: A \to \mathfrak D(E)$, la $A$-conexión producida tiene la descomposición $\hat \nabla^E = \phi + \hat \omega^E$ con $\hat \omega^E = \phi \circ \hat \omega \in \Omega^1(A, \text{End}(E))$. En una vecindad que trivialice a $A$, la forma de conexión tiene la trivialización $\hat \omega_{loc} = A \oplus (-\mathbb 1 + \tau)$ con $A \in \Omega^1(U)\otimes \mathfrak g$ y $\tau \in C^\infty(U, \text{End}(\mathfrak g))$, donde $\tau = 0$ si y solo si la conexión es ordinaria. Si $U$ además trivializa a $E$ con fibra $V$ y hay coordenadas ${x^\mu}$, una $A$-conexión producida por $\hat \omega$ se descompone en dos tipos de derivaciones de $C^\infty(U, V)$: tangentes $\hat \nabla^{E, loc}_{\partial_\mu} = \partial_\mu + \alpha_\mu + A^b_\mu \phi^{loc}(E_b)$ y verticales $\hat \nabla^{E, loc}_{E_a} = - \tau^b_a \phi^{loc} (E_b)$, donde $E_a, E_b$ son elementos de una base de $\mathfrak g$ y $\alpha$ es una $1$-forma de Maurer-Cartan que proviene de la trivialización de la representación $\phi$; en particular, los grados de libertad adicionales en que se pueden tomar ``derivadas covariantes'' se convierten en un acople de las secciones de $E$ con campos $\tau^b_a$ que no están presentes en el formalismo usual de las teorías gauge.

Fournel et. al, in \cite{Fournel2013}, develop the framework to build the action functional of a gauge theory as an inner product of differential forms on $A$ defined in chapter \ref{chp:integration} as a composition of metrics and integrals applied to forms. A metric on $A$ will induce a notion of horizontallity on $A$, i.e. an ordinary connection, which then defines a differential form that we will call a volume form, that can be decomposed as a volume form for inner integration, or inner volume form, and a volume form on the base manifold. Having metrics on representation vector bundles $E$ enables a multiplication of $E$-valued forms to get a scalar-valued form that may be integrated over the transitive Lie algebroid, resulting in an inner product on the space of $E$-valued forms.



% Los últimos ingredientes para construir el funcional de acción de una teoría gauge son métricas en los haces involucrados e integrales sobre algebroides de Lie transitivos. Una métrica $\hat g$ en el haz vectorial $A$ es equivalente a una tripla $(g, h, \nabla)$, donde $g$ es una métrica en la variedad base, $h$ es una métrica en un algebroide de Lie adjunto $L$ y $\nabla$ es una conexión ordinaria que determina el isomorfismo de haces $A \cong TM \oplus L$, con $1$-forma asociada $\mathfrak a$. Si $L$ es orientable, y suponiendo que $\{E_a\}_{a = 1, \dots, n}$ es una base de la fibra típica $\mathfrak g$ de $L$, en una vecindad $U_i$ que trivialice a $A$, $\mathfrak a$ se descompone como $\mathfrak a_i^a E_a$ y la forma $(-1)^{n}\sqrt{|h_i|} \mathfrak a_i^1 \wedge \cdots \mathfrak a_i^{n}$ se transforma correctamente entre vecindades, definiendo una forma global $\omega_{h,\mathfrak a} \in \Omega^{n}(A)$ llamada la forma de volumen interna, la cual permite definir la integración interna $\int_{inner}$ de formas arbitrarias en $A$ como la operación que extrae el factor que acompaña a $\omega_{h,\mathfrak a}$. Si $A$ es orientable, la integración sobre $A$ de formas con valores escalares $\int_A$ se define como la composición de la integración interna con la integración en la variedad base; asociada a esta integración está la forma de volumen en $A$ $\omega^{Vol}$, con trivialización local sobre $U_i$ igual a $(-1)^{n}\sqrt{|h_i| |g_i|} dx^1\wedge\cdots \wedge dx^m \wedge \mathfrak a_i^1 \wedge \cdots \mathfrak a_i^{n}$, si $\{x^\mu\}_{\mu = 1, \dots, m}$ son coordenadas en $U_i$. Dada una métrica $h^E$ en un haz vectorial $E$ de representación de $A$, se induce el producto de formas $h^E: \Omega^p(A, E)\otimes \Omega^q(A, E) \to \Omega^{p+q}(A)$ y la métrica inversa $\hat g^{-1}_{h^E}: \Omega^p(A, E)\otimes \Omega^p(A, E) \to C^\infty(M, \mathbb R)$ para todo $p, q \in \mathbb Z_{\geq 0}$, a partir de las cuales se define el operador Hodge-$*$ de una forma $E$-valuada $\beta$ como aquel para el que se satisface $h_E(\alpha, *\beta) = \hat g_{h^E}^{-1}(\alpha, \beta) \omega^{Vol}$ para toda forma $\alpha$. El operador Hodge-$*$ permite la definición de un producto interno entre formas homogéneas $\alpha, \beta \in \Omega^p(A, E)$ como $(\alpha, \beta) = \int_A h^E(\alpha, *\beta)$, del cual el funcional de acción es ejemplo.

Finally, on chapter \ref{chp:gaugeTh} a gauge theory will be defined on a transitive Lie algebroid $A$ with a metric whose vertical part is Killing, together with a representation vector bundle $E$ on which there is a metric compatible with the representation. With the current language only a Lie algebra of infinitesimal gauge transformation is defined in a way very much analogous to the traditional theory: as the space of sections of the adjoint Lie algebroid. The action functional of the theory is an operator which takes as input a connection on $A$ and a matter field and it is defined as the sum of the norm of certain differential forms; the Lagrangian of the theory can be found by taking only the inner integrals defining the norm. These action functionals will be invariant under the action of the infinitesimal gauge transformations. The resulting theory when applied to the Atiyah Lie algebroid associated to a principal bundle is identical to a standard Yang-Mills theory when the connections are restricted to be ordinary, associated to $\tau$ fields identically $0$, but additional terms appear on the Lagrangian in the general case when $\tau \neq 0$, and these terms describe the dynamics of the tau fields and their quadratic interaction with the connection and the matter field. In the general case, since all transitive Lie algebroids are locally trivial, and hence locally isomorphic to the Atiyah Lie algebroids of a principal bundle, the resulting Lagrangian has the same kind of decomposition, although no interpretation can yet be made since no further study on the extremization of the gauge functional or the quantization of the theory has been made so far.

Throughout the document two particular families of transitive Lie algebroids are studied in order to be able to apply in concrete cases the theory that has been put forward. These are the families of Atiyah Lie algebroids associated to the principal $S^1$-bundles over $S^2$ and $S^3$-bundles over $S^4$. The local version of some concepts was studied by us in order to have ``ready to use'' formulas for the action of a gauge theory, in particular for the Lagrangian density of the matter action, once a simple set of maps have been established.

%Finalmente, una teoría gauge se define en el capítulo $5$ a partir de un algebroide de Lie transitivo $A$ orientable con métrica $\hat g \equiv (g, h, \tilde \nabla)$ y un haz vectorial $E$ de representación de $A$ con métrica $h^E$, a través del funcional de acción $\mathcal S = \mathcal S_{gauge} + \mathcal S_{matter}$ de la teoría. El funcional gauge $\mathcal S_{gauge}$ se aplica a una conexión generalizada en $A$ $\hat \omega$ con curvatura $\hat R$, y se define como $S_\text{gauge}[\hat \omega] := (\hat R, \hat R)$. La segunda parte del funcional de acción, el funcional de materia $S_{matter}[\hat \omega, \mu] = (\hat \nabla^E \mu, \hat \nabla^E \mu)$ se aplica a campos de materia $\mu$, i.e. secciones del haz vectorial de representación, donde $\hat \nabla^E$ es la $A$-conexión en $E$ producida por $\hat \omega$. Cada una de estos funcionales es una integral sobre $A$, y truncando esta integral a nivel de la integral interna se obtienen las densidades lagrangianas $\mathcal L_{gauge}$ y $\mathcal L_{matter}$ correspondientes. El funcional de acción de una teoría gauge es invariante ante transformaciones gauge infinitesimales, siempre y cuando las métricas $h$ y $h^E$ satisfagan una condición de compatibilidad con la representación correspondiente, donde se entiende a $\Gamma(L)$ como el álgebra de transformaciones gauge infinitesimales de la teoría, con $L$ un algebroide adjunto a $A$. Un elemento $\eta \in \Gamma(L)$ actúa sobre la conexión generando $\hat \omega^\eta = \hat \omega + \hat d\eta + \hat \omega \wedge \eta$, y sobre la $A$-conexión generando $\hat \nabla^{E, \eta} = \hat \nabla^E + [\hat \nabla^E, \eta]$. Finalmente, el funcional de acción tiene una descomposición de la forma $\mathcal S[\hat \omega, \mu] = (a^* \hat F, a^* \hat F) + ((a^* \mathcal D \tau) \circ \tilde \omega, (a^* \mathcal D \tau) \circ \tilde \omega) + (R_\tau \circ \tilde \omega, R_\tau \circ \tilde \omega) + (a^* \phi(\nabla)\cdot \mu, a^* \phi(\nabla)\cdot \mu) + ((\phi_L(\tau)\mu)\circ \tilde \omega, (\phi_L(\tau)\mu)\circ \tilde \omega)$; el segundo, tercer y quinto término desaparecen cuando $\hat \omega$ es una conexión ordinaria, i.e. cuando su $\tau$ asociado es $0$, dejando términos análogos a los de una teoría gauge usual; a partir del acople con $\tau$, el segundo término será un término de masa para los campos gauge, el cuarto término lo será para los campos de materia, y el tercer término describe la dinámica de los campos $\tau$.