%%%%%%%%%%%%%%%%%%%%%%%%%%%%%%%%%%%%%%%%%%%%%%%%%%%%%%%%%%%%%%%%%%%%%%%%%%%%%%%%%
\section{Crucial}
\begin{itemize}
    \item Why connections?:
    \begin{itemize}
        \item Horizontal and vertical: projection and lifting
        
        \item Path lifting
        
        \item Covariant derivative in associated vector bundles:
        
        \item \dbtext{Covariant derivative in associated bundle comes necessarily from connection in principal bundle?}
        
        \item Some physical fields which can not be globally defined turn out to be the local description of a connection. \dbtext{why? See monopole} \dbtext{Path lifting? into space which considers internal degrees of freedom}
        
        \item For \dbtext{widespread} physical theories, \rtext{``local'' invariance under a ``gauge group'' can be asserted }
    \end{itemize}
    
    \item How important is the condition of good behaviour of the injection of $L$ after the connection applied to a fundamental field? Is it equivalent to removing the second basic condition in the definition of a ppal bundle connection?
    
    \item \rtext{Bosons? Of spin 1?} Perhaps bosons simply because that's what worked when quantizing (although some justification should be possible, perhaps knowing the consequences i.e. cheating) but why spin 1?
    
    \item The ordinary curvature is:
    \begin{itemize}
        \item The covariant derivative of $\omega$ with respecto to $\omega$, which can also be written as
        \item $d\omega(X_1^h, X_2^h)$
        \item $d\omega + \frac{1}{2}[\omega,\omega]$ (this $[,]$ is the Lie bracket of forms, not simplified notation).
        \item $\Omega^\omega = \omega + \omega \wedge \omega$ if $G$ is a matrix group.
    \end{itemize}
    
    \item Curvatures satisfy $2$ equations: Bianchi (or homogeneous field strength equation) and Another one (in the case of EM field $\delta F = 0$, which gives the conservation of charge), which talks about conservation.
    
    \item Why is ``field strenth'', i.e. curvature, a $2$-form?
    
    \item Perhaps the nice part of this language is the local trivializations where we see that ordinary connections do not act in any way in the second part: $\omega(X \oplus \gamma) = A(X) - \gamma$.
    
    \item Does our generalization of Connection give rise to a \rtext{truly covariant} derivative in ``associated vector bundles''?
    \begin{itemize}
        \item Perhaps induced by the fact that the covariant derivative in an associated vector bundle is precisely the covariant derivative of $E$-valued forms on $P$ seeing $\Gamma(E)$ as $0$-forms?
        
        \item \rtext{The covariant derivative of forms may also be defined as $D^\omega(\Omega)(X_1, \dots, X_m) = d\Omega(X_1^h, \dots, X_m^h)$}. The curvature is precisely the covariant derivative of $\omega$ wrt to $\omega$
        
        \item I read that ``covariant derivatives'' are equivalent to ordinary connections in $P$.
        
        \item Also, that the space of $G$-equivariant and horizontal $W$-valued $k$-forms on $P$ is equivalent to Space of $k$-forms on $M$ with values in $P\times_G W \to M$
        
        \item The covariant derivative of a \emph{basic}(horizontal and invariant) $k$-form may be written as: $D^\omega \tau = d\tau + \omega \dot{\wedge} \tau$ (See defn 1.3.4 and Thm 3.1.5 Bleecker) 
    \end{itemize}
    
    
    \item Ordinary connections give rise to a sense of horizontality both in $P$ AND in the associated vector bundles.
    
    \item What meaning may be given to
    \begin{itemize}
        \item The connection, instead of vertical projection
        
        \item The curvature, instead of $d\omega(X_1^h, X_2^h)$ (we still have: failure of $\omega$ to be ... morphism of Lie algebras?), perhaps specially in the ``associated'' bundles. Measure of \dbbox{integrability?}
        
        \item To the $\Theta:A \to A$, $\mathfrak X \mapsto \mathfrak X + i \comp \omega(\mathfrak X)$ 
        
        \item To $\hat \omega$, $\hat \Theta$, $\hat R$, $\hat F$?
        
        \item $L$: still the ``vertical part''?
    \end{itemize}
    
    \item Can we keep
    \begin{itemize}
        \item the property ``$\omega^2 = \omega$ ''? So that there is still a projection associated (without preserving $\omega |_L = -id_L$... it isn't the same)
        
        \item Perhaps, instead, we try can preserve $\omega |_L = -id_L$, which allows us to see $L$ as the vertical part.
        
        \item Would it make sense to say $\omega_L = -2id_L$
    \end{itemize}
\end{itemize}

%%%%%%%%%%%%%%%%%%%%%%%%%%%%%%%%%%%%%%%%%%%%%%%%%%%%%%%%%%%%%%%%%%%%%%%%%%%%%%%%%
\section{Important, Calculation Related}

\begin{itemize}
    \item Exact definition of a \emph{Differential Algebra}
    
    \item Give $\partial_\mu + A_\mu$ alternative to these covariant derivatives in associated bundles!!
\end{itemize}

%%%%%%%%%%%%%%%%%%%%%%%%%%%%%%%%%%%%%%%%%%%%%%%%%%%%%%%%%%%%%%%%%%%%%%%%%%%%%%%%%
\section{Better Understand}

\begin{itemize}
    \item Intuitive meaning of the exterior derivative $d$.
\end{itemize}

%%%%%%%%%%%%%%%%%%%%%%%%%%%%%%%%%%%%%%%%%%%%%%%%%%%%%%%%%%%%%%%%%%%%%%%%%%%%%%%%%
\section{Calculations}

\subsection{Local Formulas}
\begin{itemize}
    \item $\psi^i$ in the Atiyah Lie algebroid preserves the bracket.
\end{itemize}

%%%%%%%%%%%%%%%%%%%%%%%%%%%%%%%%%%%%%%%%%%%%%%%%%%%%%%%%%%%%%%%%%%%%%%%%%%%%%%%%%
\section{To Add, Maybe}
\begin{itemize}
    \item Principal bundles as natural places where ``particle fields'' or ``wave functions'' take values that takes into account the implicit ``choice of gauge'' or of frame we do (like the zero-phase angle).
    
    \item Principal bundle as natural places where forces are geometrized, exactly like adding a time dimension geometrizes gravity. 
    
    \item \rbox{Some of these things may be added explicitly mentioning} why they (I) may be expected to be useful. These accomplishes 2 goals: 
        \begin{itemize}
        \item incorporation of material which I believe might be useful
        \item Increase the ammount of mathematical theory in this MATH thesis
        \end{itemize}
\end{itemize}