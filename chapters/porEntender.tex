%%%%%%%%%%%%%%%%%%%%%%%%%%%%%%%%%%%%%%%%%%%%%%%%%%%%%%%%%%%%%%%%%%%%%%%%%%%%%%%%%
\section{Crucial}
\begin{itemize}
    \item How important is the condition of good behaviour of the injection of $L$ after the connection applied to a fundamental field? Is it equivalent to removing the second basic condition in the definition of a ppal bundle connection?
    
    \item \rtext{Bosons?}
    
    \item Curvatures satisfy $2$ equations: Bianchi (or homogeneous field strength equation) and Another one (in the case of EM field $\delta F = 0$, which gives the conservation of charge), which talks about conservation.
    
    \item Perhaps the nice part of this language is the local trivializations where we see that ordinary connections do not act in any way in the second part: $\omega(X \oplus \gamma) = A(X) - \gamma$.
    
    \item Does our generalization of Connection give rise to a \rtext{truly covariant} derivative in ``associated vector bundles''?
    \begin{itemize}
        \item Perhaps induced by the fact that the covariant derivative in an associated vector bundle is precisely the covariant derivative of $E$-valued forms on $P$ seeing $\Gamma(E)$ as $0$-forms?
        
        \item I read that ``covariant derivatives'' are equivalent to ordinary connections in $P$.
        
        \item Also, that the space of $G$-equivariant and horizontal $W$-valued $k$-forms on $P$ is equivalent to Space of $k$-forms on $M$ with values in $P\times_G W \to M$
    \end{itemize}
    
    
    \item Ordinary connections give rise to a sense of horizontality both in $P$ AND in the associated vector bundles.
    
    \item What meaning may be given to
    \begin{itemize}
        \item The connection, instead of vertical projection
        
        \item The curvature, instead of $d\omega(X_1^h, X_2^h)$ (we still have: failure of $\omega$ to be ... morphism of Lie algebras?), perhaps specially in the ``associated'' bundles. Measure of \dbbox{integrability?}
        
        \item To the $\Theta:A \to A$, $\mathfrak X \mapsto \mathfrak X + i \comp \omega(\mathfrak X)$ 
        
        \item To $\hat \omega$, $\hat \Theta$, $\hat R$, $\hat F$?
        
        \item $L$: still the ``vertical part''?
    \end{itemize}
    
    \item Can we keep
    \begin{itemize}
        \item the property ``$\omega^2 = \omega$ ''? So that there is still a projection associated (without preserving $\omega |_L = -id_L$... it isn't the same)
        
        \item Perhaps, instead, we try can preserve $\omega |_L = -id_L$, which allows us to see $L$ as the vertical part.
        
        \item Would it make sense to say $\omega_L = -2id_L$
    \end{itemize}
\end{itemize}

%%%%%%%%%%%%%%%%%%%%%%%%%%%%%%%%%%%%%%%%%%%%%%%%%%%%%%%%%%%%%%%%%%%%%%%%%%%%%%%%%
\section{Important, Calculation Related}

\begin{itemize}
    \item Exact definition of a \emph{Differential Algebra}
    
    \item Give $\partial_\mu + A_\mu$ alternative to these covariant derivatives in associated bundles!!
\end{itemize}

%%%%%%%%%%%%%%%%%%%%%%%%%%%%%%%%%%%%%%%%%%%%%%%%%%%%%%%%%%%%%%%%%%%%%%%%%%%%%%%%%
\section{Better Understand}

\begin{itemize}
    \item Intuitive meaning of the exterior derivative $d$.
\end{itemize}

%%%%%%%%%%%%%%%%%%%%%%%%%%%%%%%%%%%%%%%%%%%%%%%%%%%%%%%%%%%%%%%%%%%%%%%%%%%%%%%%%
\section{Calculations}

\subsection{Local Formulas}
\begin{itemize}
    \item $\psi^i$ in the Atiyah Lie algebroid preserves the bracket.
\end{itemize}

%%%%%%%%%%%%%%%%%%%%%%%%%%%%%%%%%%%%%%%%%%%%%%%%%%%%%%%%%%%%%%%%%%%%%%%%%%%%%%%%%
\section{To Add, Maybe}
\begin{itemize}
    \item Principal bundles as natural places where ``particle fields'' or ``wave functions'' take values that takes into account the implicit ``choice of gauge'' or of frame we do (like the zero-phase angle).
    
    \item Principal bundle as natural places where forces are geometrized, exactly like adding a time dimension geometrizes gravity. 
\end{itemize}