%%%%%%%%%%%%%%%%%%%%%%%%%%%%%%%%%%%%%%%%%%%%%%%%%%%%%%%%%%%%%%%%%%%%%%%%%%%%%%%%%%%%
\subsection{Basic Definitions}

Let $P(M, G, \pi)$ be a principal connection, and $w : \Gamma(P) \to \Gamma(P \times \algeb g)$ be a connection from on $P$.\unsure{ME PREOCUPA QUE TODA MI MOTIVACION DEPENDE DE AGARRARME DE LEVANTAMIENTOS HORIZONTALES. Hay alguna otra forma de motivar el uso de $TP/G$?}

As was motivated in the introduction, principal bundles over the base space $M$ can be thought of as the full space where the physics is ocurring, although we can only see what happens in the base space. To extrapolate what is happenning in the total space based on what we can see, we need a principal connection. To motivate the concept of Lie algebroids we will develop the Atiyah sequence of a principal bundle as a sort of infinitesimal version of a principal bundle which is minimal in some way to talk about connections/path lifting.

We first notice a connection doesn't offer a unique lifting of a path $\gamma:I \to M$ to a path in $P$, as an additional point in $P$ (whose projection belongs to the graph of the path) is required as an initial condition to determine the exact lifted path. This ambiguity can be easily removed if we notice that each possible lifting of $\gamma$ is related to any other lifting by the action of the group in $G$.

\begin{definition} [Horizontal Lift of a Path]
Let $\gamma: I \to M$ be a (smooth) path in $M$. A path $\tilde{\gamma}: I \to P$ is called a \emph{horizontal lift of the path $\gamma$} if $\gamma = \pi \comp \tilde{\gamma}$ and $w_p(\frac{d}{dt}_{|t_0} \tilde{\gamma}(t)) = 0$ for each $t_0 \in I$ and $p = \Tilde{\gamma}(t_0)$.
\end{definition}

\begin{proposition} [Uniqueness of Horizontal Lifts through a point]
If $\Tilde{\gamma}_1, \Tilde{\gamma}_2:I \to P$ are two horizontal lifts of the path $\gamma:I \to M$ in $P$, and if they share a point $p = \Tilde{\gamma}_1(t_0) = \Tilde{\gamma}_2(t_0)$, then $\Tilde{\gamma}_1 = \Tilde{\gamma}_2$. \info{See Ehresmann connection in Wikipedia. It seems this is deduced from the rank-nullity theorem}
\end{proposition}

\begin{theorem}\label{thm:pathRed}
If $\tilde{\gamma}_1: I \to P$ and $\tilde{\gamma}_2: I \to P$ are two liftings of the path $\gamma:I \to M$, then there is an element $g \in G$ such that $\tilde{\gamma}_1 (t) = \tilde{\gamma}_2 \cdot g$ for all $t \in I$. Conversely, if $\tilde{\gamma}_1$ is a horizontal lifting of $\gamma$, then $\Tilde{\gamma}\cdot g :I \to P$ is also a horizontal path lift of $\gamma$ for any $g\in G$.
\end{theorem}

Thus, to lift a path from $M$ to $P$ horizontally, the group action is somehow redundant. Infinitesimally we see that when lifting a path $\gamma:I \to M$ to $P$, for each $m = \gamma(t_0) \in M$ the vector $X_m$ tangent to $\gamma$ in $m$ gets lifted to a tangent vector in the horizontal space of $m$, but the exact vector in $\pi_*^{-1}(X_m) \subset H_mP$ that it gets assigned to depends on which horizontal lift of $\gamma$ we choose.

\begin{corollary}\label{cor:tpRed}
Let $\gamma$. Suppose $\Tilde{\gamma}_1$, $\Tilde{\gamma}_2$. Thus there exists an element $g \in G$ for which, for all $t \in I$ $\Tilde{\gamma}_2'(t) = R_{g*}\Tilde{\gamma_1}'(t)$.
\end{corollary}


\begin{corollary}
Let $\gamma$. Once we have chosen $\mathfrak{X}_{p_1} \in \pi_*^{-1}(X_m) \subset H_mP$, the whole horizontal lift of $\gamma$ is determined, and any other horizontal lift is generated by choosing $g \in G$, which in equivalent to determining a distinct element $\mathfrak{X}_{p_2} \in \pi_*^{-1}(X_m)$.
\end{corollary}

However, we see from the corollaries of theorem \ref{thm:pathRed} that the exact path in $P$ we choose as the horizontal lift of a $\gamma:I \to M$ is unimportant, as they are all generated from one another by the action of $G$, either in $P$ or in $TP$. So, \rtxt{$TP$ is looking rather big for our purposes of lifting a path horizontally from $M$ to $P$}.

However, we can't discard $TP$ yet as the connection which defines what \emph{horizontal} means is \improvement{I have to decide if a connection form is either a map from $TP$ or from sections of it} a map starting at $TP$,  $w:TP \to P\times \algeb g$. Fortunately, for the connection $TP$ is much more than needed in the same sense as before: the group action shows redundancy between elements in $TP$. More precisely $w(\mathfrak{X}\cdot g) = w(\mathfrak{X}) \cdot g$, where the first action ...

All in all, we see that to define a horizontal lift of a path $\gamma$ to $P$ we can both define what horizontal is $P$ with a connection and lift a vector in $M$ to be a horizont

Intuition: trivial principal bundle

\begin{definition}
$\frac{TP}{G}$ vector bundle over $M$
\end{definition}

\begin{definition}
$\frac{P \times \algeb g}{G}$
\end{definition}

\begin{proposition}
$\frac{P \times \algeb g}{G}$ can be injected as vector bundle (in particular as manifold: immersion) in $\frac{TP}{G}$
\end{proposition}

\begin{theorem}
A map $\omega:\frac{TP}{G} \to \frac{P \times \algeb g}{G}$ is equivalent to a principal connection (in a canonical way).
\end{theorem}

\begin{definition}
Anchor
\end{definition}

\begin{proposition}
$\frac{P \times \algeb g}{G}$ corresponds to the kernel of the anchor.
\end{proposition}

De alguna manera debo motivar \complete{De alguna manera debo motivar la introduccion del corchete en ambos lados: difeomorfismos?}

\begin{definition}
Corchete en $\frac{TP}{P}$
\end{definition}

\begin{definition}
Corchete en $\frac{P \times \algeb g}{G}$
\end{definition}

\begin{proposition}
El corchete conmuta con la inclusion.
\end{proposition}




%%%%%%%%%%%%%%%%%%%%%%%%%%%%%%%%%%%%%%%%%%%%%%%%%%%%%%%%%%%%%%%%%%%%%%%%%%%%%%%%%%%%
\subsection{Representation of a Lie Algebroid}

%%%%%%%%%%%%%%%%%%%%%%%%%%%%%%%%%%%%%%%%%%%%%%%%%%%%%%%%%%%%%%%%%%%%%%%%%%%%%%%%%%%%
\subsection{The Atiyah Lie Algebroid of a Principal Bundle}

%%%%%%%%%%%%%%%%%%%%%%%%%%%%%%%%%%%%%%%%%%%%%%%%%%%%%%%%%%%%%%%%%%%%%%%%%%%%%%%%%%%%
\subsection{The Derivations Algebroid of a Vector Bundle}

%%%%%%%%%%%%%%%%%%%%%%%%%%%%%%%%%%%%%%%%%%%%%%%%%%%%%%%%%%%%%%%%%%%%%%%%%%%%%%%%%%%%
\section{Transitive Lie Algebroids and LABs}

%%%%%%%%%%%%%%%%%%%%%%%%%%%%%%%%%%%%%%%%%%%%%%%%%%%%%%%%%%%%%%%%%%%%%%%%%%%%%%%%%%%%
\section{Local Description of Transitive Lie Algebroids}

%%%%%%%%%%%%%%%%%%%%%%%%%%%%%%%%%%%%%%%%%%%%%%%%%%%%%%%%%%%%%%%%%%%%%%%%%%%%%%%%%%%%
\section{Actions of Lie Algebroids?}