%%%%%%%%%%%%%%%%%%%%%%%%%%%%%%%%%%%%%%%%%%%%%%%%%%%%%%%%%%%%%%%%%%%%%%%%%%%%%%%%%%%%
%%%%%%%%%%%%%%%%%%%%%%%%%%%%%%%%%%%%%%%%%%%%%%%%%%%%%%%%%%%%%%%%%%%%%%%%%%%%%%%%%%%%
INTRODUCTION \complete{ Hacer la introducción. Quizá ver pequeños comentrios de Mackenzie y de Lazzarini} In \ref{chp:intro} we saw how the basic elements in a gauge theory are a principal bundle, encoding the 


Throughout this section, $M$ will be a smooth manifold.

\section{Basic Definitions}

\begin{definition} [Lie Algebroid over $M$]\label{defnLieAlgoid}
A \emph{Lie algebroid over $M$} is a triple $(q:A \to M, a, [\cdot, \cdot ])$, where $q:A \to M$ is a vector bundle, 
\begin{itemize}
    \item $a:A \to TM$ is a vector bundle morphism called \emph{the anchor of $A$}, and
    \item \emph{the bracket of $A$} $[\cdot, \cdot ]: \Gamma(A) \times \Gamma(A) \to \Gamma(A)$ is a Lie algebra structure on the $\RR$-vector space $\Gamma(A)$ (an $\RR$-bilinear alternating map) \info{The bracket doesn't need to exist fiberwwise as a metric. e.g. the bracket of vector fields doesn't}
\end{itemize}  which satisfy the following compatibility conditions:

\begin{enumerate}
    \item The map induced by $a$ on the sections, which we call by the same name, $a:\Gamma(A) \to \Gamma(TM)$ is a \emph{Lie algebra morphism}, that is, for any $\oid X, \oid Y \in \Gamma(A)$ \[ a([\oid X, \oid Y])  = [a(\oid X), a(\oid Y)] \in  \Gamma(TM)\]
    
    \item (Leibniz identity) For any $\oid X, \oid Y \in \Gamma(A)$, $f \in C^\infty (M)$: \[ [\oid X, f\oid Y] = f[\oid X, \oid Y] + a(\oid X)(f)\, \oid Y \] where $a(\oid X)(f)$ means the usual derivative of an element of a function on $M$ in the direction of $a(\oid X) \in \Gamma(TM)$.
\end{enumerate}
\end{definition}

As it is usual, to simplify notation we will often say instead that \emph{$A$ is a Lie algebroid over the manifold $M$ with anchor $a$}, and the brackets will be denoted simply by $[,]$ for all Lie algebroids.

\begin{proposition}
$\Gamma A$ is a finite projective module over $M$. Furthermore, the map $a: \Gamma(A) \to \Gamma(TM)$ is $C^\infty(M)$-linear.
\end{proposition}
\begin{proof}
\begin{itemize}
    \item It is $C^\infty(M)$ module:
\end{itemize}
\end{proof}

The previous result remarks the algebraic perspective which may also be used to study this topic, in contrast to the topological perspective that has been taken in this document. However, this perspective is sometimes useful and will be used whenever clarity may be gained by its use.

For the rest of the chapter $A$ will refer to a Lie algebroid over a manifold $M$ with anchor $a$, unless otherwise stated.

\begin{definition} [(Base preserving) morphism of Lie algebroids] 
Given two algebroids over the manifold $M$, $A$ with anchor $a$ and $A'$ with anchor $a'$, \emph{a base preserving Lie algebroid morphism} $\phi: A \to A'$ is a vector bundle morphism such that $a' \comp \phi = a $, and such that $\phi[\oid X, \oid Y] = [\phi(\oid X), \phi(\oid Y)]$ for all $\oid X, \oid Y \in \Gamma(A)$. 
\end{definition}

\begin{definition} [Extension of Lie algebroids over $M$]
Let $A, A', A''$ be Lie algebroids over $M$. \emph{An extension of Lie algebroids over $M$} is a sequence of Lie algebroids over $M$ \[ 0 \to A' \to A \to A'' \to 0 \] which is exact as sequence of vector bundles over $M$.
\end{definition}

\rule{10cm}{1mm}

\begin{definition}[Direct sum of Lie algebroids]
Let $A, A'$ be Lie algebroids over $M$. 
\end{definition}

\begin{proposition}
Given an open subset $U$ of $M$, and given $\oid X, \oid Y : M \to A$ sections of the vector bundle $A$, the restriction to $U$ of $[\oid X, \oid Y] \in \Gamma(A)$ depends only on the restrictions of $\oid X$ and $\oid Y$ to $U$.
\end{proposition}
\begin{proof}

\end{proof}

The previous result allows us to restrict the Lie bracket of $A$ to the restriction vector bundle of $A$ to $U$, $A_U$.

\begin{definition}
Let $(q:A \to M, a: A \to M, [,]:\Gamma(A)\times \Gamma(A) \to \Gamma(A))$ be a Lie algebroid. The restriction of the Lie algebroid $A$ to $U \subset M$, is the restriction vector bundle $A$ to $U$ $A_U$ with projection map $q_U: A_U \to U$, together with the anchor $a_U:A_U \to TU$ and bracket $[,]:\Gamma(A_U)\times \Gamma(A_U) \to \Gamma(A_U)$ that result from the restrictions to $U$ of the anchor and bracket of $A$.
\end{definition}

\begin{definition}[General Morphism]

\end{definition}

\begin{definition}[Direct products]

\end{definition}

\begin{definition}[Pullbacks]

\end{definition}

\begin{definition}[Quotients]

\end{definition}

%%%%%%%%%%%%%%%%%%%%%%%%%%%%%%%%%%%%%%%%%%%%%%%%%%%%%%%%%%%%%%%%%%%%%%%%%%%%%%%%%%%%
%%%%%%%%%%%%%%%%%%%%%%%%%%%%%%%%%%%%%%%%%%%%%%%%%%%%%%%%%%%%%%%%%%%%%%%%%%%%%%%%%%%%
\section{Examples}

\subsection{The Tangent Bundle}

For any manifold $M$, its tangent bundle $TM$ with its projection $\pi:TM \to M$ and the commutator $[,]: \Gamma(TM)\times \Gamma(TM) \to \Gamma(TM)$ as bracket, together with the identity of $TM$ as anchor make $TM$ a Lie algebroid over $M$.

\complete{Come up with a way to refer to sections that differs from element of TM, or perhaps but mentioning it as notation}

%%%%%%%%%%%%%%%%%%%%%%%%%%%%%%%%%%%%%%%%%%%%%%%%%%%%%%%%%%%%%%%%%%%%%%%%%%%%%%%%%%%%
\subsection{Trivial Lie Algebroid}

Compatibility map

Let $M$ be a manifold, and $\algeb g$ a (real) Lie algebra with Lie bracket $[,]: \algeb g \times \algeb g$. 

Consider $p_1: M \times \algeb g \to M$, $(m, \eta) \mapsto m$ as the trivial vector bundle with fiber $\algeb g$ and base $M$. To refer to an element of this bundle we will usually use greek letters like $\eta, \theta, \kappa, \dots$, and to emphasize that such an element belongs to the fiber of $m \in M$ we will use subscripts, so $\eta_m \in M \times \algeb g$ is such that $p_1(\eta_m) = m$; . A section of this bundle will be denoted by capital greek letters like $H, \Theta, K, \dots$ and their image in the fiber of $m \in M$ will once again be denoted by the subscript $m$, so $H \in \Gamma(M \times \algeb g)$ and $H_m \in p_1^{-1}(m)$. $X(H)$, $H$ corresponds canonically to a function $M \to g$. (The sections of/This vector bundle) has a natural Lie structure given by the fiberwise Lie bracket of the Lie algebra $\algeb g$.

Now consider the internal direct sum over $M$ of the vector bundles $TM$ and $M \times \algeb g$; recall \unsure{Here I'm being explicit with basic stuff of vector bundle theory on purpose, but I'm not sure if this will stay} that the underlying set of this bundle is simply the subset of the cartesian product $TM \times (M \times \algeb g)$ given by $TM \oplus (M \times \algeb g) = \set{(X, \eta) \in TM \times (M \times \algeb g) \st \pi(X) = p_1(\eta)}$; we will call the projection of this vector bundle $\pi: X \oplus \eta \mapsto \pi(X) = p_1(\eta)$, abusing the notation; we write an element of this vector bundle as $X \oplus \eta$, where $X \in \pi^{-1}(m)$ and $\eta \in p_1^{-1}(m)$.

\begin{itemize}
    \item The anchor is given by the projection $a: TM \times (M \times \algeb g) \to TM$, $X \oplus \eta \mapsto X$. This map is a vector bundle map as it clearly respects the fibers and, on the fiber over each $m\in M$ it restricts to  a projection map $a_x:T_x M \times (\set{m}\times \algeb g)$ of vector spaces, which is a linear map.
    
    \item The bracket is given by 
    \[
        [X \oplus H, Y \oplus \Theta] = [X, Y] \oplus \{X(\Theta) - Y(H) + [H, \Theta]\}
    \]
    for all $X, Y \in \Gamma(TM)$, $H, Y \in \Gamma(M \times \algeb g)$. Notice that in the above formula the symbols $[,]$ are used in $3$ different instances, each one referring to the brackets of $3$ different spaces. $X(\Theta)$ makes reference to the derivative in direction of $X$ of the vector valued function $M \to \algeb g$ associated to $\Theta$, or, simply, to the Lie derivative of the section $\Theta \in \Gamma(M \times \algeb g)$, $\mathcal L_X \Theta$. Veamos que esto es una estructura de Lie en $\Gamma(TM \times (M \times \algeb g))$:
    \begin{itemize}
        \item Alternating:
        \item Leibniz:
        \item Jacobi:
    \end{itemize}
\end{itemize}

Let's check that the compatibility conditions between the anchor and the bracket are satisfied; let $X \oplus H, Y \oplus \Theta \in \Gamma(TM \oplus (M \times \algeb g))$, $f \in C^\infty(M)$:

\begin{itemize}
    \item $a([X \oplus H, Y \oplus \Theta]) = a(\, [X, Y] \oplus \{X(\Theta) - Y(H) + [H, \Theta] \,) = [X, Y] = [a(X \oplus H), a(Y \oplus \Theta)]$
    
    \item 
    \begin{align*}
        [X \oplus H, &f\cdot (Y \oplus \Theta)] =\\
        &= [X \oplus H, fY \oplus f\Theta] \\
        &=  [X, fY] \oplus \{X(f\Theta) - fY(H) + [H, f\Theta]\} \\
        &= (f[X, Y] + X(f)\,Y) \oplus \{X(f)\,\Theta + f\,X(\Theta) - fY(H) + f[H, \Theta]\} \\
        &=f([X, Y] \oplus \{ X(\Theta) -Y(H) + [H, \Theta]\} ) + (X(f) Y \oplus X(f)\Theta) \\
        &= f[X \oplus H, Y \oplus \Theta] + a(X \oplus H)f \, (Y \oplus \Theta)
    \end{align*}
\end{itemize}

The vector bundle $TM \times (M \times \algeb g)$ together with the previously defined anchor and bracket is called \emph{the trivial Lie algebroid on $M$ with structure algebra $\algeb g$}.

%%%%%%%%%%%%%%%%%%%%%%%%%%%%%%%%%%%%%%%%%%%%%%%%%%%%%%%%%%%%%%%%%%%%%%%%%%%%%%%%%%%%
\subsection{The Atiyah Lie Algebroid of a Principal Bundle}

Can be viewed as G-invariant P->g functions, TP 

$TP$ is a vector bundle over $P$, $G$ acts to the right. (+ $G$ acts by v.b. isomorphism + $TP$ is covered by equivariant charts i.e. the action of $g$ doesn't affect the vector in which $p$ or $pg$ correspond). By proposition 3.1.1. of Mackenzie 
\begin{itemize}
    \item $TP/G$ is a vector bundle over $M$,
    %\item the projection $TP \to TP/G$ is a surjective bundle morphism (and submersion)
    %\item and $TP$ is a pullback of $TP/G$ by the projection $\pi:P \to M$
\end{itemize}  

\subsubsection{Trivial Principal Bundle and its Atiyah Lie Algebroid}

I hope to see that this motivates the Lie algebroid bracket.

%%%%%%%%%%%%%%%%%%%%%%%%%%%%%%%%%%%%%%%%%%%%%%%%%%%%%%%%%%%%%%%%%%%%%%%%%%%%%%%%%%%%
\subsection{The Derivations Algebroid of a Vector Bundle}

Summary in the beginning Ch 3.3 Liner vector fields

%%%%%%%%%%%%%%%%%%%%%%%%%%%%%%%%%%%%%%%%%%%%%%%%%%%%%%%%%%%%%%%%%%%%%%%%%%%%%%%%%%%%
\subsection{Other Examples}

Foliations

Homogeneous

Extensions

Coming from Lie groupoids

%%%%%%%%%%%%%%%%%%%%%%%%%%%%%%%%%%%%%%%%%%%%%%%%%%%%%%%%%%%%%%%%%%%%%%%%%%%%%%%%%%%%
\section{Representations}

This is a generalization of associated vector bundles of a principal bundle. It is in representation lie algebroids where connections of this bundle induce covariant derivatives, just as principal bundle connections induce covariant derivatives in associated vector bundles.

Adjoint representation

Equivariant
%%%%%%%%%%%%%%%%%%%%%%%%%%%%%%%%%%%%%%%%%%%%%%%%%%%%%%%%%%%%%%%%%%%%%%%%%%%%%%%%%%%%
\section{LABs and Transitive Lie Algebroids}

Definitions comming from Lie algebra theory semisimple, nilpotent, abelian, reductive.

If $A$ is regular, the image of the anchor defines a foliation of the base manifold, \emph{the characteristic foliation of $A$}, which is transitive over each leaf of the foliation.

%%%%%%%%%%%%%%%%%%%%%%%%%%%%%%%%%%%%%%%%%%%%%%%%%%%%%%%%%%%%%%%%%%%%%%%%%%%%%%%%%%%%
\section{Local Description of Transitive Lie Algebroids}

%%%%%%%%%%%%%%%%%%%%%%%%%%%%%%%%%%%%%%%%%%%%%%%%%%%%%%%%%%%%%%%%%%%%%%%%%%%%%%%%%%%%
\section{Actions of Lie Algebroids?}\subsection{Trivial Lie Algebroid}
