%%%%%%%%%%%%%%%%%%%%%%%%%%%%%%%%%%%%%%%%%%%%%%%%%%%%%%%%%%%%%%%%%%%%%%%%%%%%%%%%%%%%
\subsection{Basic Definitions}

Let $P(M, G, \pi)$ be a principal connection, and $w : TP \to P \times \algeb g$ be a connection from on $P$.\unsure{ME PREOCUPA QUE TODA MI MOTIVACION DEPENDE DE AGARRARME DE LEVANTAMIENTOS HORIZONTALES. Hay alguna otra forma de motivar el uso de $TP/G$?}

\subsubsection{Path Lifting}

As was motivated in the introduction, principal bundles over the base space $M$ can be thought of as the full space where the \improvement{the physics? the full space?} physics is ocurring, although we can only see what happens in the base space. To extrapolate what is happenning in the total space based on what we can see, we need a principal connection. To motivate the concept of Lie algebroids we will develop the Atiyah sequence of a principal bundle as a sort of infinitesimal version of a principal bundle which is minimal in some way to talk about connections/path lifting.

We first notice a connection doesn't offer a unique lifting of a path $\gamma:I \to M$ to a path in $P$, as an additional point in $P$ (whose projection belongs to the graph of the path) is required as an initial condition to determine the exact lifted path. This ambiguity can be easily removed if we notice that each possible lifting of $\gamma$ is related to any other lifting by the action of the group in $G$.

\begin{definition} [Horizontal Lift of a Path]
Let $\gamma: I \to M$ be a (smooth) path in $M$. A path $\tilde{\gamma}: I \to P$ is called a \emph{lift of the path $\gamma$} if $\gamma = \pi \comp \tilde{\gamma}$. Given a principal connection $w$ in $P$, a lift is called \emph{horizontal} if furthermore $w_p(\frac{d}{dt}_{|t_0} \tilde{\gamma}(t)) = 0$ for each $t_0 \in I$ and $p = \Tilde{\gamma}(t_0)$.
\end{definition}

\begin{proposition} [Uniqueness of Horizontal Lifts through a point]
If $\Tilde{\gamma}_1, \Tilde{\gamma}_2:I \to P$ are two horizontal lifts of the path $\gamma:I \to M$ in $P$, and if they share a point $p = \Tilde{\gamma}_1(t_0) = \Tilde{\gamma}_2(t_0)$, then $\Tilde{\gamma}_1 = \Tilde{\gamma}_2$. \info{See Ehresmann connection in Wikipedia. It seems this is deduced from the rank-nullity theorem}
\end{proposition}

We would like to see now how the horizontal lifts are not only unique once an initial point in $P$ has been chosen, but also that the group action allows us to essentially identify any possible horizontal lifts independently of the chosen initial point.

\begin{theorem}\label{thm:pathRed}
If $\tilde{\gamma}_1: I \to P$ and $\tilde{\gamma}_2: I \to P$ are two liftings of the path $\gamma:I \to M$, then there is an element $g \in G$ such that $\tilde{\gamma}_1 (t) = \tilde{\gamma}_2 \cdot g$ for all $t \in I$. Conversely, if $\tilde{\gamma}_1$ is a horizontal lifting of $\gamma$, then $\Tilde{\gamma}\cdot g :I \to P$ is also a horizontal path lift of $\gamma$ for any $g\in G$.
\end{theorem}

\begin{corollary}\label{cor:tpRed}
Let $\gamma$. Suppose $\Tilde{\gamma}_1$, $\Tilde{\gamma}_2$. Thus there exists an element $g \in G$ for which, for all $t \in I$ $\Tilde{\gamma}_2'(t) = R_{g*}\Tilde{\gamma_1}'(t)$.
\end{corollary}

Thus, to lift a path from $M$ to $P$ horizontally (given a connection), the group action is somehow redundant. We would like now to emphasize how all of this, i.e. path lifting, connectios and group action, can be understood as infinitesimal phenomena. 

We first notice the tangent spaces allow us to see how the differentiability of the path has a role restricting the possible paths in $P$ that can be considered lifts of $\gamma$. %For every $t \in I$, we can apply the chain rule Where lifting a path $\gamma:I \to M$ to $P$, for each $m = \gamma(t_0) \in M$ the vector $X_m$ tangent to $\gamma$ in $m$ gets lifted to a tangent vector in the horizontal space of $m$, but the exact vector in $\pi_*^{-1}(X_m) \subset H_mP$ that it gets assigned to depends on which horizontal lift of $\gamma$ we choose.

\begin{proposition}
A path $\tilde{\gamma}:I \to P$ is a lift in $P$ of $\gamma:I \to M$ if and only if $\gamma'(t) = \pi_*(\tilde{\gamma}'(t)) \in T_{\gamma(t)}M$ for each $t \in I$.
\end{proposition}

\begin{proof}
Given $t \in I$, $\pi_*(\tilde{\gamma}'(t)) \in T_{\gamma(t)}M$ tells us that $\tilde{\gamma}'(t) \in T_p M$ for some $p \in P$ in the fiber of $\gamma(t)$, i.e. $\gamma = \pi \circ \tilde{\gamma}$. Conversely, if $\tilde{\gamma}$ is a path lift of $\gamma$, $\gamma = \pi \circ \tilde{\gamma}$, and so by the chain rule $\gamma'(t) = \pi_*(\tilde{\gamma}'(t))$.
\end{proof}

This motivates us to define the following:

\begin{definition}[Tangent path]
Given a $C^2$ path $\gamma: I \to M$, define the \emph{tangent path} $\overbar \gamma:I \to TM$ as the map defined by $\overbar \gamma(t) = \gamma'(t)$. 
\end{definition}

\begin{definition}[Horiontal Tangent Path]
A path in $\overbar{\gamma}$ in $TP$ is called horizontal if $w \circle \overbar{\gamma} = 0$.
\end{definition}

\begin{lemma}
Given $\gamma$, $\overbar \gamma$ is a $C^1$ path in $TM$ such that \gamma = \pi \circl \overbar{\gamma}.
\end{lemma}

\begin{theorem}
Let $\gamma$. $\Tilde{\gamma}$ is a path lift of $\gamma$ through $\pi$ if and only if $\overbar{\Tilde{\gamma}}$ is a path lift of $\overbar{\gamma}$ through $\pi_*$.
\end{theorem}

\rtext{Thus, path lifting can be understood infinitesimally as a way to assign a path in $TP$ given a path in $\gamma'(t) \in TM$}. 

Now, to see once again the redundancy of the action we first need to understand how $G$ acts on $TP$.

\begin{proposition}
The map $R : G \times TP \to TP, (g, \mathfrak{X}) \mapsto R_{g*}(\mathfrak X)$ is a right Lie group action of $G$ induced by the right Lie action of $G$ on $P$.
\end{proposition}

The following

\begin{corollary}
Let $\gamma$. Once we have chosen $\mathfrak{X}_{p_1} \in \pi_*^{-1}(X_m) \subset H_mP$, the whole horizontal lift of $\gamma$ is determined, and any other horizontal lift is generated by the action of some $g \in G$ on all of the tangent vectors in $TP$ that compose the horizontal lift, which in equivalent to determining a distinct element $\mathfrak{X}_{p_2} \in \pi_*^{-1}(X_m)$.
\end{corollary}

We see from the corollaries of theorem \ref{thm:pathRed} that the exact path in $P$ we choose as the horizontal lift of a $\gamma:I \to M$ is unimportant, as they are all generated from one another by the action of $G$, either in $P$ or in $TP$. So, \rtxt{$TP$ is looking rather big for our purposes of lifting a path horizontally from $M$ to $P$}.

This suggests that horizontal path lifting is indeed unique, not just modulo the group action, if we consider the set $\frac{TP}{G}$ as the place where a tangent vector in $TM$ is lifted to. To formalize this we first need to give some structure to $\frac{TP}{G}$.

\begin{definition}
$\frac{TP}{G}$ vector bundle over $M$. $\Gamma(\frac{TP}{G})$ has well defined brackets. $\pi_{*}$ is a vector bundle morphism that respects the brackets.
\end{definition}

\begin{theorem}
Let $\gamma:I \to M$ and $\omega:\frac{TP}{G} \to P \times \algeb g$ a principal connection in the principal bundle $P(\pi, M, G)$. Then there is a unique path lift $\tilde{\gamma}:I \to P$
\end{theorem}
However, we mustn't forget that the connection which defines what \emph{horizontal} means is \improvement{I have to decide if a connection form is either a map from $TP$ or from sections of it} a map starting at $TP$,  $w:TP \to P\times \algeb g$. Fortunately, for the connection $TP$ is much more than needed in the same sense as before: the group action shows redundancy between elements in $TP$. More precisely $w(\mathfrak{X}\cdot g) = w(\mathfrak{X}) \cdot g$, where the first action ...

All in all, we see that to define a horizontal lift a path $\gamma: I \to M$ it is enough to define a connection

\subsubsection{Covariant Derivatives}

The gauge potentials, and so the covariant derivatives WHERE can also be formulate in this infinitesimal language. \complete{Otra razon alternativa a path lifting}

Intuition: trivial principal bundle

\begin{definition}
$\frac{P \times \algeb g}{G}$
\end{definition}

\begin{proposition}
$\frac{P \times \algeb g}{G}$ can be injected as vector bundle (in particular as manifold: immersion) in $\frac{TP}{G}$
\end{proposition}

\begin{theorem}
A map $\omega:\frac{TP}{G} \to \frac{P \times \algeb g}{G}$ is equivalent to a principal connection (in a canonical way).
\end{theorem}

\begin{definition}
Anchor
\end{definition}

\begin{proposition}
$\frac{P \times \algeb g}{G}$ corresponds to the kernel of the anchor.
\end{proposition}

De alguna manera debo motivar \complete{De alguna manera debo motivar la introduccion del corchete en ambos lados: difeomorfismos?}

\begin{definition}
Corchete en $\frac{TP}{P}$
\end{definition}

\begin{definition}
Corchete en $\frac{P \times \algeb g}{G}$
\end{definition}

\begin{proposition}
El corchete conmuta con la inclusion.
\end{proposition}




%%%%%%%%%%%%%%%%%%%%%%%%%%%%%%%%%%%%%%%%%%%%%%%%%%%%%%%%%%%%%%%%%%%%%%%%%%%%%%%%%%%%
\subsection{Representation of a Lie Algebroid}

%%%%%%%%%%%%%%%%%%%%%%%%%%%%%%%%%%%%%%%%%%%%%%%%%%%%%%%%%%%%%%%%%%%%%%%%%%%%%%%%%%%%
\subsection{The Atiyah Lie Algebroid of a Principal Bundle}

%%%%%%%%%%%%%%%%%%%%%%%%%%%%%%%%%%%%%%%%%%%%%%%%%%%%%%%%%%%%%%%%%%%%%%%%%%%%%%%%%%%%
\subsection{The Derivations Algebroid of a Vector Bundle}

%%%%%%%%%%%%%%%%%%%%%%%%%%%%%%%%%%%%%%%%%%%%%%%%%%%%%%%%%%%%%%%%%%%%%%%%%%%%%%%%%%%%
\section{Transitive Lie Algebroids and LABs}

%%%%%%%%%%%%%%%%%%%%%%%%%%%%%%%%%%%%%%%%%%%%%%%%%%%%%%%%%%%%%%%%%%%%%%%%%%%%%%%%%%%%
\section{Local Description of Transitive Lie Algebroids}

%%%%%%%%%%%%%%%%%%%%%%%%%%%%%%%%%%%%%%%%%%%%%%%%%%%%%%%%%%%%%%%%%%%%%%%%%%%%%%%%%%%%
\section{Actions of Lie Algebroids?}