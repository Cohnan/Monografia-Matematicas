%%%%%%%%%%%%%%%%%%%%%%%%%%%%%%%%%%%%%%%%%%%%%%%%%%%%%%%%%%%%%%%%%%%%%%%%%%%%%%%%%%%%
\subsection{Basic Definitions}

Let $P(M, G, \pi)$ be a principal connection, and $w : TP \to P \times \algeb g$ be a connection on $P$.\unsure{ME PREOCUPA QUE TODA MI MOTIVACION DEPENDE DE AGARRARME DE LEVANTAMIENTOS HORIZONTALES. Hay alguna otra forma de motivar el uso de $TP/G$?}

\subsubsection{Path Lifting}

As was motivated in the introduction, principal bundles over the base space $M$ and their associated vector bundles can be thought of as the total spaces where \improvement{the physics? the full space?} particle physics occurs (before quantization), although we can only see what happens in the base space. To extrapolate what is happenning in the total spaces based on what we can see, we need a principal connection. To motivate the concept of Lie algebroids we will develop the Atiyah sequence of a principal bundle as an infinitesimal version of the principal bundle which is minimal in some way to talk about connections and path lifting, hence allow covariant derivatives for ``matter fields'', i.e. sections of associated vector bundles.

We first notice a connection doesn't offer a unique lifting of a path $\gamma:I \to M$ to a path in $P$, as an additional point in $P$ (whose projection belongs to the graph of the path) is required as an initial condition to determine the exact lifted path. This ambiguity can be easily removed if we notice that each possible lifting of $\gamma$ is related to any other lifting by the action of the group in $G$.

\begin{definition} [Horizontal Lift of a Path]
Let $\gamma: I \to M$ be a (smooth) path in $M$. A path $\tilde{\gamma}: I \to P$ is called a \emph{lift of the path $\gamma$} if $\gamma = \pi \comp \tilde{\gamma}$. Given a principal connection $w$ in $P$, a lift is called \emph{horizontal} if furthermore $w_p(\frac{d}{dt}_{|t_0} \tilde{\gamma}(t)) = 0$ for each $t_0 \in I$ and $p = \Tilde{\gamma}(t_0)$.
\end{definition}

\begin{proposition} [Uniqueness of Horizontal Lifts through a point]
If $\Tilde{\gamma}_1, \Tilde{\gamma}_2:I \to P$ are two horizontal lifts of the path $\gamma:I \to M$ in $P$, and if they share a point $p = \Tilde{\gamma}_1(t_0) = \Tilde{\gamma}_2(t_0)$, then $\Tilde{\gamma}_1 = \Tilde{\gamma}_2$. \info{See Ehresmann connection in Wikipedia. It seems this is deduced from the rank-nullity theorem}
\end{proposition}


We would like to see now how the horizontal lifts are not only unique once an initial point in $P$ has been chosen, but also that the group action allows us to essentially identify as one all possible horizontal lifts, independently of the chosen initial point.

\begin{theorem}\label{thm:pathRed}
If $\tilde{\gamma}_1: I \to P$ and $\tilde{\gamma}_2: I \to P$ are two horizontal liftings of the path $\gamma:I \to M$, then there is an element $g \in G$ such that $\tilde{\gamma}_1 (t) = \tilde{\gamma}_2 \cdot g$ for all $t \in I$. Conversely, if $\tilde{\gamma}_1$ is a horizontal lifting of $\gamma$, then $\Tilde{\gamma}\cdot g :I \to P$ is also a horizontal path lift of $\gamma$ for any $g\in G$.
\end{theorem}

Thus, to lift a path from $M$ to $P$ horizontally (given a connection), the group action is somehow redundant. To better understand what is happening it is necessary to have an \dgtxt{explicit view at the phenomena occurring infinitesimally}, as the differential structure has an important role in determining which paths in $P$ can be considered lifts of a path in $M$, in addition to being necessary to determine what horizontal means. 

%We first notice the tangent spaces allow us to see how the differentiability of the path has a role restricting the possible paths in $P$ that can be considered lifts of $\gamma$. %For every $t \in I$, we can apply the chain rule Where lifting a path $\gamma:I \to M$ to $P$, for each $m = \gamma(t_0) \in M$ the vector $X_m$ tangent to $\gamma$ in $m$ gets lifted to a tangent vector in the horizontal space of $m$, but the exact vector in $\pi_*^{-1}(X_m) \subset H_mP$ that it gets assigned to depends on which horizontal lift of $\gamma$ we choose.

\begin{proposition}
A path $\tilde{\gamma}:I \to P$ is a lift in $P$ of $\gamma:I \to M$ if and only if $\gamma'(t) = \pi_*(\tilde{\gamma}'(t)) \in T_{\gamma(t)}M$ for each $t \in I$.
\end{proposition}

\begin{proof}
Given $t \in I$, $\pi_*(\tilde{\gamma}'(t)) \in T_{\gamma(t)}M$ tells us that $\tilde{\gamma}'(t) \in T_p M$ for some $p \in P$ in the fiber of $\gamma(t)$, i.e. $\gamma = \pi \circ \tilde{\gamma}$. Conversely, if $\tilde{\gamma}$ is a path lift of $\gamma$, $\gamma = \pi \circ \tilde{\gamma}$, and so by the chain rule $\gamma'(t) = \pi_*(\tilde{\gamma}'(t))$.
\end{proof}

This motivates us to define the following:

\begin{definition}[Tangent Path, Horizontal Tangent Path]
Given a $C^2$ path $\gamma: I \to M$, define the \emph{tangent path} $\overline \gamma:I \to TM$ as the map defined by $\overline \gamma(t) = \gamma'(t)$. 
\end{definition}

\begin{definition}[Horizontal Tangent Path]
A path in $\overline{\gamma}$ in $TP$ is called horizontal if $w \comp \overline{\gamma} = 0$.
\end{definition}

\begin{lemma}
Given $\gamma$, $\overline \gamma$ is a $C^1$ path in $TM$ such that $\gamma = \pi_{*} \comp \overline{\gamma}$.
\end{lemma}

\begin{theorem}
Let $\gamma$. $\Tilde{\gamma}$ is a path lift of $\gamma$ through $\pi$ if and only if $\overline{\Tilde{\gamma}}$ is a path lift of $\overline{\gamma}$ through $\pi_*$. $\Tilde{\gamma}$ is a horizontal path lift of $\gamma$ through $\pi$ if and only if $\overline{\Tilde{\gamma}}$ is a horizontal tangent path lift of $\overline{\gamma}$ through $\pi_*$.
\end{theorem}

\rtxt{Thus, (horizontal) path lifting can be understood infinitesimally as a way to assign a (horizontal) path in $TP$ to a path $\gamma'(t)$ in $TM$}. 

Now, to see once again the redundancy of the action we first understand how $G$ acts on $TP$.

\begin{proposition}
The map $R : G \times TP \to TP, (g, \mathfrak{X}) \mapsto R_{g*}(\mathfrak X)$ is a right Lie group action of $G$ induced by the right Lie action of $G$ on $P$.
\end{proposition}

The following corollary of \ref{thm:pathRed} shows again the redundancy of the group action, but now in infinitesimal languange.

%\begin{corollary}\label{cor:tpRed}
%Let $\gamma$. Suppose $\overline{\Tilde{\gamma}}_1$, $\overline{\Tilde{\gamma}}_2: I \to TP$ are two horizontal tangent path lifts of $\overline{\gamma}$. Thus there exists an element $g \in G$ for which, for all $t \in I$, $\overline{\Tilde{\gamma}}_2(t) = R_{g*}\overline{\Tilde{\gamma_1}}(t) \in T_{\gamma(t)}M$.
%\end{corollary}
\begin{corollary}
If $\overline{\tilde{\gamma}}_1, \overline{\tilde{\gamma}}_2: I \to TP$ are two horizontal liftings of the path $\overline{\gamma}:I \to TM$, then there is an element $g \in G$ such that $\overline{\tilde{\gamma}}_1 (t) = \overline{\tilde{\gamma}}_2 (t) \cdot g$ for all $t \in I$. Conversely, if $\overline{\tilde{\gamma}}$ is a horizontal lifting of $\overline{\gamma}$, then $\overline{\Tilde{\gamma}}\cdot g :I \to TP$ is also a horizontal path lift of $\overline{\gamma}$ for any $g\in G$.%Let $\gamma$. Any two horizontal tangent path lifts of $\overline{\gamma}:I \to TM$ are related Once we have chosen $\mathfrak{X}_{p_1} \in \pi_*^{-1}(X_m) \subset H_mP$, the whole horizontal lift of $\gamma$ is determined, and any other horizontal lift is generated by the action of some $g \in G$ on all of the tangent vectors in $TP$ that compose the horizontal lift, which in equivalent to determining a distinct element $\mathfrak{X}_{p_2} \in \pi_*^{-1}(X_m)$.
\end{corollary}

We see from the corollaries of theorem \ref{thm:pathRed} that the exact path in $P$ we choose as the horizontal lift of a $\gamma:I \to M$ is unimportant, as they are all generated from one another by the action of $G$, either in $P$ or in $TP$. So, \rtxt{$TP$ is looking rather big for our purposes of lifting a path horizontally from $M$ to $P$}.

This suggests that horizontal path lifting is indeed unique, not just modulo the group action, if we consider the set $\frac{TP}{G}$ as the place where a tangent vector in $TM$ is lifted to. To formalize this we first need to give some appropriate structure to $\frac{TP}{G}$.

\begin{definition}
$\frac{TP}{G}$ vector bundle over $M$. $\Gamma(\frac{TP}{G})$ is a $C^\infty(M)-$module with well defined brackets. $a = \pi_{*}(<\cdot>)$ is a surjective vector bundle morphism that respects the brackets.  
\end{definition}

\begin{theorem}
Let $\gamma:I \to M$ and $w:TP \to P \times \algeb g$ a principal connection form in the principal bundle $P(\pi, M, G)$. Then there is a unique path lift $\tilde{\gamma}:I \to \frac{TP}{G}$.
\end{theorem}

However, we must not forget that the principal connection form which defines what \emph{horizontal} means is \improvement{I have to decide if a connection form is either a map from $TP$ or from sections of it} a map involving all of $TP$, $w:TP \to P\times \algeb g$. \dgtxt{Fortunately, for the connection $TP$ is much more than needed in the same sense as before, as we'll see below.}

\subsubsection{Connection Form}

One of the defining properties of a principal connection form $w:TP \to P \times \algeb g$ is its (equi?)(in?)variance under the group action, and this is what will allow us to restrict its definition to $\frac{TP}{G}$. More explicitly, for any $\mathfrak X_p \in T_p P$, $w(R_{g*}\,\mathfrak{X_p}) = (R_g(p), Ad_{g^{-1}} w(\mathfrak{X_p})) \in P \times \algeb g$, which tells us that once the effect of $w$ is on $\mathfrak{X}_p \in TP$, the effect of $w$ of every element of its orbit in $TP$ is established.

However, to remove this redundancy in the definition of $w:TP \to P \times g$ it must be possible to identify unambiguously as a single point every element in $P \times \algeb g$ which is the image of every orbit in $TP$.

\begin{proposition}
The map $R:G \times (P \times \algeb g) \to P \times \algeb g$, $(g, (p, A)) \mapsto (R_g(p), Ad_{g^{-1}} A)$ is a Lie group action of $G$ on $P \times \algeb g$.
\end{proposition}

\begin{definition}
$\frac{P \times \algeb g}{G}$ is a vector bundle over $M$, and $\Gamma \left(\frac{P \times \algeb g}{G} \right)$ is a $C^\infty(M)-$module with well defined bracket.
\end{definition}

\begin{proposition}
Let $w:TP \to P \times \algeb g$ be a principal connection form. Every element $\mathfrak{Y}_q$ in the orbit of $\mathfrak{X}_p$, i.e. $\mathfrak{Y}_q = R_{g*} \mathfrak{X}_p$ for some $g\in G$ satisfies $w(\mathfrak{Y}_p) = \mathfrak{X}_p \cdot g$.
\end{proposition}

Finally, the next theorem is our first step to show that \ytxt{all the information that a principal connection form has, including the meaning of ``horizontal'', can be encapsulated in a map $\omega: \frac{TP}{G} \to \frac{P \times \algeb g}{G}$, which makes use of this minimal infinitesimal version of the principal bundle and its adjoint bundle}.

\begin{theorem}
Let $w:TP \to P \times \algeb g$ be a principal connection form. The map $\omega: \frac{TP}{G} \to \frac{P \times \algeb g}{G}$, $\omega(<\mathfrak{X}_p>) = <w(\mathfrak{X}_p)>$ is a well defined vector bundle morphism. It can be seen as a $C^\infty(M)-$module morphism $\omega: \Gamma(\frac{TP}{G}) \to \Gamma(\frac{P \times \algeb g}{G})$, i.e. a $\frac{P \times \algeb g}{G}$-valued one form $\omega \in \Omega^1(\frac{TP}{G}, \frac{P \times \algeb g}{G})$.
\end{theorem}

\lgtxt{However, we can not yet say that a connection reform and a principal connection form are equivalent yet, as there is another defininig property of a principal connection, which we will now formulate this in the language we are currently developing.}

\begin{definition}
Let $P(\pi, M, G)$. Given $A \in \algeb g$, we define \emph{the fundamental tangent vector associated to $A$ at $p$}, denoted by $A_p^P$, as the tangent vector $ \frac{d}{dt}|_{t = 0} p \cdot exp(tA) \in T_pP$. The \emph{fundamental vector field associated to $A$} is the vector field in $P$, denoted by $A^P$, defined pointwise as the fundamental tangent vector.\improvement{why is it indeed a smooth or $C^n$ section?}
\end{definition}

The following theorem allows us to see that the adjoint Lie algebroid of an Atiyah Lie algebroid are related via the fundamental tangent vectors.

\begin{lemma}
The map $i:\frac{P \times \algeb g}{G} \to \frac{TP}{G}$, $<p, A> \mapsto <A_p^P>$ is an injective vector bundle morphism that respects the Lie bracket. Inclusion  map.
\end{lemma}

The following apparently useless definition will, nonetheless, be seen as an important property when talking about Lie algebroids.

\begin{definition}
The anchor of $\frac{P \times \algeb g}{G}$, defined by $a:\frac{P \times \algeb g}{G} \to TM$, $<p, A> \mapsto 0_{\pi(p)}$.
\end{definition}

\begin{proposition}
The anchor of $\frac{P \times \algeb g}{G}$ is a vector bundle morphism which respects the brackets and commutes with the anchors.
\end{proposition}

Now we can see the full extent of the inclusion map.

\begin{theorem}
The inclusion map $i:\frac{P \times \algeb g}{G} \to \frac{TP}{G}$ is a vector bundle morphism that respects the brackets and commutes with the anchor maps.
\end{theorem}

Finally we can formulate completely the full definition of a principal connection form in the current language. any $A \in \algeb g$.

\begin{proposition}
Let $A \in \algeb g$, and $w:TP \to P \times \algeb g$ be a principal connection form. The property $w(A^P_p) = (p, A)$ of the connection form induces the following property of the induced map $\omega:\frac{TP}{G} \to \frac{P \times \algeb g}{G}$: $\omega \comp i = Id_{\frac{P \times \algeb g}{G}}$.
\end{proposition}

\begin{definition} [Connection Reform]
A vector bundle morphism $\omega:\frac{TP}{G} \to \frac{P \times \algeb g}{G}$ that satisfies $\omega \comp i = Id_{\frac{P \times \algeb g}{G}}$ is called \emph{a connection reform}.
\end{definition}

\rtxt{Finally}

\begin{theorem}
A principal connection form in $P$ induces naturally a connection reform in $\frac{TP}{G}$. Conversely, any connection reform in $\frac{TP}{G}$ induces naturally a principal connection form.
\end{theorem}


\subsubsection{Covariant Derivatives}

As all the information of a principal connection is incorporated in the connection reform, the usual covariant derivates defined on the associated vector bundles of the principal bundle. ALTHOUGH I MAY USE THIS OPPORTUNITY TO SEE THE CONNECTION AS EQUIVALENT TO THE TM -> TP/G MAP THAT APPEARS IN THE DEFINITION OF COVARIANT DERIVATIVES AND THAT WILL BE USED AS THE INITIAL DEFINITION OF A CONNECTION IN AN ATIYAH LIE ALGEBROID


The gauge potentials, and so the covariant derivatives WHERE can also be formulate in this infinitesimal language. \complete{Otra razon alternativa a path lifting}

Intuition: trivial principal bundle



\begin{proposition}
$\frac{P \times \algeb g}{G}$ can be injected as vector bundle (in particular as manifold: immersion) in $\frac{TP}{G}$
\end{proposition}

\begin{theorem}
A map $\omega:\frac{TP}{G} \to \frac{P \times \algeb g}{G}$ is equivalent to a principal connection (in a canonical way).
\end{theorem}

\begin{definition}
Anchor
\end{definition}

\begin{proposition}
$\frac{P \times \algeb g}{G}$ corresponds to the kernel of the anchor.
\end{proposition}

De alguna manera debo motivar \complete{De alguna manera debo motivar la introduccion del corchete en ambos lados: difeomorfismos? Algebra de Lie?}

\begin{definition}
Corchete en $\frac{TP}{P}$
\end{definition}

\begin{definition}
Corchete en $\frac{P \times \algeb g}{G}$
\end{definition}

\begin{proposition}
El corchete conmuta con la inclusion.
\end{proposition}




%%%%%%%%%%%%%%%%%%%%%%%%%%%%%%%%%%%%%%%%%%%%%%%%%%%%%%%%%%%%%%%%%%%%%%%%%%%%%%%%%%%%
\subsection{Representation of a Lie Algebroid}

%%%%%%%%%%%%%%%%%%%%%%%%%%%%%%%%%%%%%%%%%%%%%%%%%%%%%%%%%%%%%%%%%%%%%%%%%%%%%%%%%%%%
\subsection{The Atiyah Lie Algebroid of a Principal Bundle}

%%%%%%%%%%%%%%%%%%%%%%%%%%%%%%%%%%%%%%%%%%%%%%%%%%%%%%%%%%%%%%%%%%%%%%%%%%%%%%%%%%%%
\subsection{The Derivations Algebroid of a Vector Bundle}

%%%%%%%%%%%%%%%%%%%%%%%%%%%%%%%%%%%%%%%%%%%%%%%%%%%%%%%%%%%%%%%%%%%%%%%%%%%%%%%%%%%%
\section{Transitive Lie Algebroids and LABs}

%%%%%%%%%%%%%%%%%%%%%%%%%%%%%%%%%%%%%%%%%%%%%%%%%%%%%%%%%%%%%%%%%%%%%%%%%%%%%%%%%%%%
\section{Local Description of Transitive Lie Algebroids}

%%%%%%%%%%%%%%%%%%%%%%%%%%%%%%%%%%%%%%%%%%%%%%%%%%%%%%%%%%%%%%%%%%%%%%%%%%%%%%%%%%%%
\section{Actions of Lie Algebroids?}