
\section{Ordinary Connections and Curvature in Transtive Lie Algebroids}

They always exist.

If the algebroid is regular.

\subsection{Equivalence for Atiyah Lie Algebroid to Principal Bundle Theory}
\subsection{Local Description of connection and curvature and ``Change of Gauge''}

\section{Mixed Local Basis of Forms given a Connection and $G^i_j$ Matrices}

\section{Covariant Derivatives of Forms}

\section{$A$-Connections / Covariant Derivatives in ``Generalized Associated/Representation Algebroids''}

This concept of $A$-connection is introduced by Fernandes (either this, or their notion of $A$ connection is completely different from this one, but still reproduces the ordinary covariant derivative if we have a $TM$-connection). In his 2001 paper he remarks some important things about them, like how parallel transport does not depen only on the base path, the holonomy of a flt $A$-connection may be non-discrete, etc (pg. 7)

Horizontal lift of $\mathfrak{X}$ in $e\in E$? Linear vector field associated to derivation?

%\section{Gauge Transformations and Infinitesimal Gauge Transformations}

\section{Generalized Connections}

Do they give rise to good ``covariant derivatives''? (In associated bundles) \ref{trulyCovariant}

The ordinary curvature is (see descriptions in \ref{chp:understand}, in \ref{ordinaryCurvatureIs}). It satisfies \ref{BianchiandInhomogeneous}. It gives sense of horizontality \ref{ordinaryHorizontality}.

Is our notion of curvature somehow redundant? \ref{curvatureRedundant}

What meaning may be given to \ref{whatMeaning}:

%\subsection{Infinitesimal Gauge Action}
\subsection{Local Description}
\subsection{Decomposition of a Generalized Connection and its Curvature}
