{
\color{gray}
Recall:
    \begin{itemize}
    
    \item Lifting of Paths gives lifting of Vector fields
    
    \item Sections of associated vector bundles are really $G$-equivariant $\psi(pg) = g^{-1} \psi(p)$ functions $P \to V$. In general, \dbtext{forms $\Omega^n(TM, E)$ are $V$ valued equivariant forms $\Omega_G(P, V)$?}, 
    
    and this may? be the same as considering the associated vector bundle with fiber $\bigwedge^n V$, but I don't think it matters to me right now.
    
        \begin{itemize}
        
        \item The action of $G$ on $V$ determines uniquely the associated vector bundle
        
        \item ``Conversely'', at least up to what I did by hand, having a single section of a vector bundle, and having a representation of $G$ on the fiber determines the cocycles, which in turns determines the principal bundle, so: ``the vector bundle $E$ with fiber $V$ + the representation of a group $\Longrightarrow$ determines the Principal Bundle''... so, \textbf{even if I start simply with a vector bundle and a group that acts in the fiber, I obtain a principal bundle}, and so I may see a seciton as an equivariant function $:P \to V$.
        
        \item This is interpreted as locally choosing a ``frame'' or local trivialization, so that I may, locally, consider the section as simply $:U \subset M \to V$, as it is perceived ``physically''.
            
        \end{itemize}
    
    \item \textbf{I want a ``covariant'' derivative of sections of $E$ to be something that sends $G$-equivariant funtions $:P \to V$ to $G$-equivariant functions},  given only $X \in TM$, because it is the manifold we see, on which $E$ is based. (In GR covariant means sending a tensor field to a tensor field)
    
    \item \textbf{This last point explains the $G$-equivariance of the principal connections!!} (see my example).
    
    \end{itemize}


---
The classical setting for Covariant derivate: 

    \begin{itemize}
    
    \item Given a vector bundle $E$ over $M$ and
    
    \item a (Lie algebroid) connection $\nabla: TM \to \mathcal D(E)$ 
        
        \begin{itemize}
        
        \item weaker than a Lie algebroid representation because it is not required to be a Lie algebra morphism in the sections, unless the curvature is $0$
        
        \end{itemize}
    
    \item the Covariant Derivative \emph{of vector (bundle) valued forms} $\Omega^n(TM, E)$, $D^\nabla: \Omega^\cdot \to \Omega^{\cdot+1}$, e.g. $0$-forms, which are precisely $\Gamma(E)$, the matter fields. (The covariant is satisfied by definition, at least in the sense of GR, where it means to send a tensor field into a tensor field)
    
    \item is given precisely by the formula of the formula of the exterior derivative: Koszul formula
    
    Take into account that instead of having a representation of $TM$ in $E$ (which would define an exterior derivative) we have a connection.
    \end{itemize}
    
In summary, a covariant derivative 
    \begin{itemize}
    
    \item Is something defined on vector bundle valued forms, just like the exterior derivative $d$ (as defined in Lazzarini, at least).
    
    \item Is coarser than a $d$ by only requiring a (Lie algebroid) connection, instead of a representation (so no Lie algebra morphism in the sections)
    
    \item Is finer than a $d$ by requiring as starting vector bundle $TM$, instead of any Lie algebroid $A$.
    
    \end{itemize}

-----



\section*{Ordinary Connections and Curvature in Transtive Lie Algebroids}

They always exist.

If the algebroid is regular.

The ordinary curvature is (see descriptions in \ref{chp:understand}, in \ref{ordinaryCurvatureIs}). It satisfies \ref{BianchiandInhomogeneous}. It gives sense of horizontality \ref{ordinaryHorizontality}.

Is our notion of curvature somehow redundant? \ref{curvatureRedundant}

What meaning may be given to \ref{whatMeaning}


\subsection*{Equivalence for Atiyah Lie Algebroid to Principal Bundle Theory}

Mackenzie does this in detail.

The general equivalence of $G$-equivariant/invariant forms on $TP$ and Lie algebroid forms has been established in Ch 2. THe normalized part is easy.

$\omega_{loc}(\cdot \oplus 0)$ is a local principal connection form.

\subsection*{Local Description of connection and curvature and ``Change of Gauge''}

Ordinary Lie algebroid connection, i.e. $\omega \circ j = - id_L$

\begin{align}
    \omega^j_{loc}(X \oplus \eta) 
    &= \omega^j_{loc}(X) - \eta !!\text{$\omega(S(\cdot \oplus 0))$ is a local principal connection}\\
    &= \underbrace{\alpha^j_i(\omega^i_{loc}(X)) + \chi^j_i(X)}_{\text{same as the ppal. connection $\omega(S(\cdot \oplus 0))$}}- \eta
\end{align}

Generalized connections: $\omega \circ j = \tau - id_L$

\begin{align}
\omega^j_{loc}(X \oplus \eta) 
 = \underbrace{\alpha^j_o)\omega^j_{loc})(X)) + \chi^j_i(X) - \eta}_{\text{same as before}} + \psi_j^{-1}\circ \tau (\psi_j(\eta) - l^j_i(X))\\
\end{align}

The extra bit can also be written as 
\begin{align}
    \psi_j^{-1}\circ \tau \circ j^{-1}S_i(\eta) \\
    \psi_j^{-1}\circ \tau \circ j^{-1}\psi_i(\eta^i) & \text{if $\eta$ is understood as $\eta = \eta^j$}
\end{align}

These all means that
\begin{proposition}
A family of local $1$-forms with the above transformation rules $\equiv$ a (generalized) Lie algebroid connection on $A$
\end{proposition}

\linea

\section*{Mixed Local Basis of Forms given a Connection and $G^i_j$ Matrices}

\section*{Covariant Derivatives of Forms}

\section{$A$-Connections / Covariant Derivatives in ``Generalized Associated/Representation Algebroids''}

This concept of $A$-connection is introduced by Fernandes (either this, or their notion of $A$ connection is completely different from this one, but still reproduces the ordinary covariant derivative if we have a $TM$-connection). In his 2001 paper he remarks some important things about them, like how parallel transport does not depen only on the base path, the holonomy of a flt $A$-connection may be non-discrete, etc (pg. 7)

Horizontal lift of $\mathfrak{X}$ in $e\in E$? Linear vector field associated to derivation?

%\section{Gauge Transformations and Infinitesimal Gauge Transformations}

\section*{Generalized Connections}

Do they give rise to good ``covariant derivatives''? (In associated bundles) \ref{trulyCovariant}



%\subsection{Infinitesimal Gauge Action}
\subsection{Local Description}
\subsection{Decomposition of a Generalized Connection and its Curvature}
}

{ \color{gray}
What we really want: 
matter fields 
on which groups acts <- “local” gauge transformation associated to a group: Representation v. bundles of a transitive Lie algebroid $A ~ TM x \alg g$
on which there is a notion of covariant derivative (momentum: w.r.t space, energy: w.r.t. time -> Time Evolution Equations), 
where the covariance means that $D_X\psi$ is indeed again a matter field 
that  transforms as one under a “local gauge transformation”:
pointwise action of the group: 
including: change of local principal bundle trivialization
making the Matter Action invariant under local gauge transformations
And where the covariant derivatives induce the existence of observable Gauge Fields like light
whose Gauge Action is again invariant under local gauge transformations
Hence,
Principal bundle introduced as the bigger space where the X in TP has a natural and consistent notion of derivative: generalized by representation of A=TP/G on E
Where the group and its action on the vector fields plays an important role in the representation
SEE HOW DOES THE TRADITIONAL ACTION OF TP ON E LOOKS AS A REPRESENTATION OF TP/G, and notice how the group action intervenes
Generalized by: the lie algebra “of the structure group” is the vertical part of every transitive Lie algebroid, so a representation must take into account this infinitesimal gauge transformations
The “local gauge transformations” induced on E related to the group are generated by the infinitesimal gauge transformations $L~C^\infty(M, g)$ of A
And from where each connection induces a covariant (under gauge transformations) derivative on all associated matter fields: generalized covariant derivatives: $A$-connections
Furthermore! The new directions on which we may take covariant derivatives ARE PRECISELY THE COUPLINGS WITH THE TAU (“Higgs”) FIELD.

What I don’t need:
The notion of horizontality in $TP/G$: we care about 
the gauge fields (and tau fields) that the connections generate; and
the notion of covariant derivative that it induces on associated matter fields coming from horizontal vector fields will be replaced by 
representation to talk about the natural action, and of 
A-connections to introduce the direction preferred by the gauge fields
The principal bundle: 
the structure group as generator of gauge transformations is still there; 

the natural notion of TP as derivatives is generalized by the notion of representation (so the horizontality is again unnecessary)
Takeaway: even though I lose the view of connections as a notion of horizontality that is equivalent which helps make intuitive sense of connections as vertical projections and of the covariant derivatives as horizontal derivatives, its generalization purely in terms of forms completely suffices to incorporate the ingredients that are important for Gauge Theory, and allow the natural introduction of new fields tau that couple with the rest of the fields to make mass terms appear in the lagrangians.


    \begin{itemize}
    
    \item Connections on Transitive Lie Algebroid
    
        \begin{itemize}
        
        \item Definition: Lie algebroid connection: section
        
        \item Theorem: Ordinary connection as normalized $1$-form
            
            \begin{itemize}
    
            \item Example: Principal bundle connection.
            
            \item Example: GENERALIZES: Covariant Derivatives of Vector Bundles
            
                \begin{itemize}
                    
                \item A + representation -> Induced Covariant Derivative
                
                \item Associated $\Omega^1(TM, End(E))$ form.
                
                Gauge fields (See Bleecker or Naber for precise definitions)
                    
                \end{itemize}
            
            \end{itemize}
        
        \item Definition: Generalized connection on Transitive $A$. Definition $\tau$.
        
        \item Proposition: Generalized connection equivalent to $\hat \Theta$ $C^\infty$ that respects the anchor. 
        
        \item Ordinary connection IFF $\hat \Theta \comp j = 0$. If that is the case, then $\hat \theta^2 = \hat \theta$ is the horizontal projection of the horizontal lift $\nabla$.
        
        From the properties $\nabla$ horizontal lift, $\hat \omega$ vertical projection onto horizontal and vertical subbundles of $A$. We lose the intuition of horizontality, but we will still have a generalized notion of covariant derivatives which will give rise to the additional field that may couple with the gauge fields and the matter to give them mass. If it were desired, $\hat \theta$ is still a projection iff $\hat \omega \circ j|_{Im \hat \,\omega} = - 1|_{Im \, \hat \omega}$, e.g. $\hat \omega = 0$, then $\tau = -1_L$ and $\hat \Theta$ is a projection onto all of $A$.
        
        Still valid interpretation: gauge potential, given of gauge fields that fix ``local'' gauge invariance.
        
        \item Ordinary connection from background connection
            
        \item Curvature    
            
            \begin{itemize}
                
            \item Nabla version
            
            \item Form version
            
            \item If ordinary connection, we can recover usual notion of curvature (of ppal bundle connection): nabla version
            
            \item Bianchi? Only for ordinary? ``HOmogeneous field equation''
                
            \end{itemize}
            
        \item Algebraic curvature of $\tau$
        
        \item Decomposition given background connection
        
        \end{itemize}
        
    \item Local Trivializations of Generalized Connections on $A$
    
        \begin{itemize}
            
        \item Local trivialization of connection: $A - \epsilon + \tau$
        
            \begin{itemize}
                
            \item Change of coordinates
                
            \end{itemize}
        
        \item \textbf{Mixed local basis}
        
        \item Local trivialization of curvature given background connection.
            
        \end{itemize}
        
        
    \item EXAMPLES:
    
        \begin{itemize}
            
        \item Complex Hopf
        
            \begin{itemize}
            \item Monopole connection, local trivializations 
            \end{itemize}
        
        \item $P^k$ over $S^2$
        
            \begin{itemize}
                
            \item Monopole connection. Different unit magnetic charge? In the north looks like so...
            
            \item More general connection from $U_S$: completely arbitrary phi and theta components \& $x^1 x^2$ components

            \item It transforms like so: in phi theta to phi theta; in x to y.
            
            \item \lbtext{Nablas in local trivialization associated to general connection}

            \end{itemize}
            
        \item Quaternonic Hopf
        
            \begin{itemize}
            
            \item Instanton connection, local trivializations / it transforms like so

            \item More general ordinary connection from local trivialization in $U_S$. It transforms like so in x to y coordinates.

            \end{itemize}
        
        \item $P^k$ over $S^4$
        
            \begin{itemize}
            
            \item Instanton connection in $U_S$. Local trivialization in $U_N$
            
            \item Nablas in local trivialization associated to general connection
                
            \end{itemize}
        
        \end{itemize}
        
    \item A-Connections on E
    
        \begin{itemize}
            
        \item Definition: $A$-Connection on $E$: Generalized covariant derivative
        
            \begin{itemize}
            \item Example? Perhaps later? Coming from generalized connection
            \end{itemize}
        
        \item Theorem: If representation. $A$-connection on $E$ <-> $1$-form $\Omega^1(A, End(E))$
        
        \item Theorem: Induced $A$-connections from generalized connections on $A$ (and representation) 2
        
        \item Local trivializations
        
            \begin{itemize}
            
            \item Of nabla for ordinary connection
            
            \item Of one induced from (generalized) connection on transitive $A$
                
            \end{itemize}
            
        \item EXAMPLES:
        
            \begin{itemize}
                
            \item $P^k$ over $S^2$, trivial representation
            
            \item $P^k$ over $S^4$, trivial representation
            
            \end{itemize}
        
        \item Definition: Curvature of $A$-connection on $E$
        
        \item Proposition: Curvature of $A$-connection on $E$ in terms of forms. Bianchi identity.
            
        \end{itemize}
    
    \end{itemize}
}

\section{Connections on Transitive Lie Algebroids}
%%%%%%%%%%%%%%%%%%%%%%%%%%%%%%%%%%%%%%%%%%%%%%%%%%%%%%%%%%%%%%%%%
%%%%%%%%%%%%%%%%%%%%%%%%%%%%%%%%%%%%%%%%%%%%%%%%%%%%%%%%%%%%%%%%%
%%%%%%%%%%%%%%%%%%%%%%%%%%%%%%%%%%%%%%%%%%%%%%%%%%%%%%%%%%%%%%%%%
%%%%%%%%%%%%%%%%%%%%%%%%%%%%%%%%%%%%%%%%%%%%%%%%%%%%%%%%%%%%%%%%%
Throughout this section consider $A$ to be a transitive Lie algebroid over the manifold $M$, and $0 \to L \xrightarrow{j} A \xrightarrow{a} TM \to 0$ be a Lie algebroid sequence of $A$; suppose all the vector bundles on this section have vector spaces over $\bb K$ as typical fibers, where $\bb K$ is one of $\RR$ or $\CC$.

\subsection{Ordinary Transitive Lie Algebroid Connections}
%%%%%%%%%%%%%%%%%%%%%%%%%%%%%%%%%%%%%%%%%%%%%%%%%%%%%%%%%%%%%%%%%
%%%%%%%%%%%%%%%%%%%%%%%%%%%%%%%%%%%%%%%%%%%%%%%%%%%%%%%%%%%%%%%%%



\begin{definition}
\emph{An ordinary connection on $A$} is a right splitting $\nabla: TM \to A$ (i.e. a section of $a$) as vector bundles of the short exact sequence
\begin{equation}\label{equationDefinitionConnectionRightSplittingNablaSectionAnchorShortExactSequence}
    \begin{tikzcd}
    0 \arrow{r} & L \arrow{r}{j} & A \arrow{r}{a} & TM \arrow{r} \arrow[bend left]{l}{\nabla} & 0.
    \end{tikzcd}
\end{equation}
For any $X \in TM$, the application of the connection is written as $\nabla_X \in A$.
\end{definition}

\begin{theorem}\label{theoremOrdinaryConnectionEquialenceRelatoinNablasandNormalizedForms}
Let $\nabla:TM \to A$ be an ordinary connection on $A$. There is a unique $L$-valued, surjective $1$-form $\omega \in \Omega^1(A, L)$, called \textit{the connection form}, such that
\begin{equation} \label{equationRelationNablaOmegaOfOrdinaryConnection}
    \nabla_X = \sectoid X + j \comp \omega(\sectoid X),
\end{equation}
for all $\oid X \in A$ with $X = a(\oid X)$. In particular, this form is \textit{normalized}, meaning that
\begin{equation}\label{equationNormalizedOneFormOrdinaryConnectionOnTransitiveAlgebroid}
    \omega \circ j = - 1_{L};
\end{equation}
that is, $-\omega$ is a left splitting (i.e. a retract of $j$) of \eqref{equationDefinitionConnectionRightSplittingNablaSectionAnchorShortExactSequence} as a short exact sequence of vector bundles. Conversely, equation \eqref{equationRelationNablaOmegaOfOrdinaryConnection} defines an equivalence between normalized $L$-valued one forms and ordinary connections.
\end{theorem}

\begin{proof}
For any $\oid X \in A$, with $X = a(\oid X)$, $a(\nabla_X - \oid X) \in ker(a)$, so we can define the one form 
\begin{equation}\label{equationPreciseDefinitionOrdinaryConnectionFormUnique}
    \omega(\oid X) := j^{-1}(\nabla_X - \oid X);
\end{equation}
since the Lie algebra bundle morphism $j$ is injective, this $\omega$ is the only vector bundle morphism that satisfies \eqref{equationRelationNablaOmegaOfOrdinaryConnection}.

For any $l \in L$, $\omega(j(l)) = j^{-1}(- j(l)) = - j \comp j^{-1}(l) = -1$, proving the first equation, and therefore the surjectivity of $\omega$.

Now, to see that equation \eqref{equationRelationNablaOmegaOfOrdinaryConnection} produces a well defined connection from the normalized $1$-form $\omega$, let $\oid X_1, \oid X_2 \in A$ be such that $a(\oid X_1) = a(\oid X_2) = X$. Since $a(\oid X_2 - \oid X_1) = 0$, let $l \in L$ be such that $j(l) = \oid X_2 - \oid X_1$; in fact, since $\omega$ is normalized, $l = -\omega(\oid X_2 - \oid X_1)$. Then $\nabla_X$ is well defined since
\begin{align*}
    \oid X_2 + j \comp \omega(\oid X_2) 
        &= \oid X_1 + j(l) + j \comp \omega(\oid X_1 + j(l))\\
        &= \oid X_1 + j \comp \omega(\oid X_1) + j(l) + j \comp w \comp j(l)\\
        &= \oid X_1 + j \comp \omega(\oid X_1).
\end{align*}
\end{proof}


\begin{theorem}
On any transitive Lie algebroid there is at least one ordinary connection.
\end{theorem}
\begin{proof}
This is simply a restatement of the (standard) fact that the short exact sequence \eqref{equationDefinitionConnectionRightSplittingNablaSectionAnchorShortExactSequence} in the category of vector bundles is split, which is equivalent to the statement that \begin{equation}
    A \cong TM \oplus L
\end{equation}
as vector bundles, where $\nabla: TM \to A$ is then an embedding and $\omega: A \to L$ a projection. To prove the last equation we follow a standard partition of unity argument: let $\{U_\alpha\}$ be an open cover that trivializes $TM$, $L$ and $A$ simultaneously; then, on each $U \in \{U_\alpha\}$, there is a vector bundle isomorphism $f_\alpha: A|_U \to TM|_U \oplus L|U$. Now take a partition of unity $\{\rho_\alpha:U_\alpha \to [0,1]\}$ subordinate to it; then $\sum_\alpha \rho_\alpha f_\alpha : A \to TM \oplus L$ is a vector bundle isomorphism.

%Although we may now see this to be true thanks to the existence of a Lie algebroid atlas for $0 \to L \to A \to a \to a$ stated in theorem \ref{algebroidAtlasExists}, recall that the proof of that statement was not within the scope of this text. 
\end{proof}

\begin{example}\label{example}
The following example shows how \textbf{the generalized notion of an ordinary connection on a transitive Lie algebroid coincides with that of connection on a principal bundle, when the transitive Lie algebroid in question is the Atiyah Lie algebroid associated to the principal bundle}. Let $G \to P \xrightarrow{\pi} M$ be a principal bundle over the manifold $M$ with structure Lie group $G$, with $\alg g = Lie(G)$. On a principal bundle the notion of connection has multiple equivalent definitions, including:
    \begin{itemize}
    
    \item A $G$-invariant subbundle $HP$ of $TP$ (under the pushforward of the right action $R$ of $G$ on $P$), called the \textit{horizontal subbundle}, such that $TP = HP \oplus VP$ where $VP = ker(\pi_*)$. 
    
    \item A $G$-equivariant Lie algebra valued form $w: TP \to P \times \alg g$ such that $w(\eta^*) = \eta$ where $\eta^*$ is the fundamental vertical vector field associated to $\eta \in \alg g$.
    
    \end{itemize}
    
These two definitions are related to each other by $HP = ker(w)$. From the first definition it follows that for each $p \in P$ there is a section or $\pi_*$, called \textit{the horizontal lift} $\tilde \nabla^p: T_{\pi(p)}M \to H_pP \subset T_p P$ with the property that $\tilde \nabla ^{pg}_X = R_{g, *}(\tilde \nabla^p_X) \in T_{pg} P$ for all $g \in G$ and $X \in T_{\pi(p)}M$, and, hence, a $G$-equivariant \textit{horizontal projection} onto $HP$ 
\begin{eqnsplit}
H: T_pP &\to T_pP\\ 
\ppal X &\mapsto \tilde \nabla^p_{\pi_*(\ppal X)} \in HP.
\end{eqnsplit}
From the second definition we can define the $G$-equivariant \textit{vertical projection} onto $VP$
\begin{eqnsplit}
    V: T_pP &\to T_pP\\ \ppal X &\mapsto w(\ppal X)^*_p \in VP.
\end{eqnsplit}
Both $H$ and $V$ are indeed projections since $H^2 = H$ and $V^2 = V$, and so any any $\ppal X \in T_p P$ with $\pi_*(\ppal X) = X$ can be decomposed as
\begin{equation} \label{equationDecompositionTangentElementOfTPVerticalHorizontalDirectSumSubbundles}
    \ppal X = \tilde \nabla^p_X \oplus w(\ppal X)^*_p \in HP \oplus VP.
\end{equation}


The $G$-invariance of the horizontal lift implies that it induces a well defined vector bundle map
\begin{eqnsplit}
    \nabla: TM &\to TP/G \\
    X &\mapsto \cl{\tilde \nabla^p_X} = \cl{H(\ppal X)},
\end{eqnsplit}
where $p$ is any element in the fiber of $\pi(X) \in M$ and $\ppal X$ is any tangent vector of $P$ such that $\pi_*(\ppal X) = x$; furthermore, $\nabla$ is a section of the anchor $\pi_*^G$ of the Atiyah Lie algebroid $TP/G$, meaning that $\nabla$ is an ordinary connection on the Atiyah Lie algebroid associated to the principal bundle $P$. Notice that $\nabla$ can be seen as a horizontal lift where the horizontal subbundle of $TP/G$ is the class of $HP$ under the group action; furthermore, this horizontal lift is simpler to use since it is globally defined as a vector bundle morphism, and no specification of a $p \in P$ is needed. 

Similary, the $G$-equivariance of $w$ implies that it induces a well defined element of $\Omega^1(TP/G, P \times \alg g/G)$:
\begin{eqnsplit}
    \omega: TP/G &\to P \times \alg g/G \\
        \cl{\ppal X} &\mapsto \cl{w(\ppal X)};
\end{eqnsplit}
the normalization condition of $\omega$ is then simply a restatement of the condition $w(\eta^*) = \eta$ where $\eta^*_p = \der{t}[t=0]{p \cdot \exp(t \eta)} = - \der{t}[t=0]{p \cdot \exp(-t \eta)}$ for all $\eta \in \alg g$. From the decomposition \eqref{equationDecompositionTangentElementOfTPVerticalHorizontalDirectSumSubbundles} of every $\ppal X \in T_pP$ it follows that
\begin{align*}
    \tilde \nabla_{X} - j \comp \omega(\cl{\ppal X}) 
        &=  \cl{\tilde \nabla^p_X} - j\cl{w(\ppal X)} \\
        &=  \cl{\tilde \nabla^p_X} - \cl{\der{t}[t=0]{\exp(-t w(\ppal X))}},& \text{definition of $j$} \\
        &= \cl{ \tilde \nabla^p_X + w(\ppal X)^*_p}\\
        &= \cl{\ppal X} & \text{decomposition  \eqref{equationDecompositionTangentElementOfTPVerticalHorizontalDirectSumSubbundles}};
\end{align*}
hence, the ordinary connection $\nabla$ on $TP/G$ is associated, as specified by theorem \ref{theoremOrdinaryConnectionEquialenceRelatoinNablasandNormalizedForms}, to the normalized $1$-form $\omega$.
\end{example}

\begin{example}
Let $E$ be a vector bundle over $M$. We will now see how the notion of connection on $E$, also called a covariant derivative on $E$, coincides with that of ordinary connection on the transitive Lie algebroid $\alg D(E)$ with associated sequence $0 \to End(E) \to \alg D(E) \to TM \to 0$. A connection on the vector bundle $E$ is an $\bb K$-linear map
\begin{eqnsplit}
    \nabla^E: \Gamma(E) &\to \Omega^1(TM, E)
\end{eqnsplit}
such that, for all $X \in TM$, $\mu \in \Gamma(E)$ and $f \in C^\infty(M)$, the following Leibniz rule is satisfied:
\begin{equation}
    \nabla^E_X \mu (f \mu) = X(f) \mu + f \nabla^E_X \mu.
\end{equation}

Notice that we may reorder the different components entering the connection to rewrite
%may restate this definition using the vector bundle $\alg D(E)$ whose sections are the derivations of $E$; in that case, we can say instead that $\nabla$ is a connection on the vector bundle $E$ if and only if it is a vector bundle map
$\nabla$ as a vector bundle morphism
\begin{eqnsplit}
    \nabla^E : TM &\to \alg D(E)\\
        X &\mapsto \nabla_X,
\end{eqnsplit}
since the Leibniz rule is what defines the elements of $\Gamma(\alg D(E))$ within the $\bb K$-linear maps $\Gamma(E) \to \Gamma(E)$; furthermore, this Leibniz rule shows that $a(\nabla_X) = X$, where $a$ is the anchor of $\alg D(E)$. The same arguments show that such an anchor preserving vector bundle morphism $:TM \to \alg D(E)$ defines a vector bundle connection of $E$. Hence, \textbf{ordinary connections of the transitive Lie algebroid $\alg D(E)$ and vector bundle connections of $E$ are the same notion}.

Now, let $G \to P \to M$ be a principal bundle, and let us use the notation of the previous example. Suppose that $E$ is associated to $P$ and that it has the vector space $V$ as typical fiber. A connection on the principal bundle $P$ induces a connection on $E = P \times V / G$ as follows: let $X \in T_pM$ and let $\tilde \mu \in C_G^\infty(P, V)$ be the $G$-equivariant function associated to the section $\mu \in \Gamma(E)$; then define the connection $\nabla^E$ by
\begin{eqnsplit}
    \tilde{\nabla^E_X(\mu)}: p &\mapsto \tilde \nabla^p_X(\tilde \mu) \in C_G^\infty(P, V).
\end{eqnsplit}
If $\ppal X$ is any element in $T_p P$ such that $\pi_*(\ppal X) = X$ it is then true that
\begin{eqnsplit}\label{equationsInducedConnectionCovariantDerivativeOnAssociatedVectorBundleByPrincipal}
    \nabla^E_X : \mu \mapsto& \underline{H(\ppal X)(\tilde \mu)}\\
    &= \underline{(\ppal X - V(\ppal X))(\Tilde{\mu})}
\end{eqnsplit}
where the underline is the $C^\infty(M)$-module isomorphism $:C^\infty_G(P, V) \to \Gamma(E)$, inverse to the tilde map. 

This connection on $E$ has a simple rewriting in the language of Lie algebroids. First recall that the map $\mu \mapsto \underline{\ppal X (\tilde{\mu}})$ is precisely $\phi(\cl{\ppal X}) \in \alg D(E)$ where 
\begin{equation}
    \phi: TP/G \to \alg D(E)
\end{equation} 
is the Lie algebroid representation of $TP/G$ in $E$ defined in \ref{example...}. Hence, we may rewrite the first equation in \eqref{equationsInducedConnectionCovariantDerivativeOnAssociatedVectorBundleByPrincipal} as the composition
\begin{align}
    \nabla^E : TM \xrightarrow{\nabla} A \xrightarrow{\phi} \alg D(E) && \nabla_X^E = \phi(\nabla_X)
\end{align}
since $\cl{H(\ppal X)} \equiv \nabla_{\pi^*_G\cl{\ppal X}}$. Notice, however, that other covariant derivatives on $E$ may be induced by an ordinary connection on $TP/G$ is another Lie algebroid representation $\phi$ is used. \textbf{In conclusion, the covariant derivative on vector bundles associated to principal bundles induced by principal bundle connections is an example of a $\alg D(E)$ ordinary connection induced by a Lie algebroid representation of $TP/G$ on $E$ and an ordinary connection on $TP/G$}.
\end{example}

\begin{definition}
Let $E$ be a vector bundle over $M$ and let $\phi: A \to \alg D(E)$ be a representation of $A$. Given an ordinary connection $\nabla: TM \to A$ on $A$, $\nabla^E: TM \to \alg D(E)$, $\nabla^E := \phi \circ \nabla$ is an ordinary connection on $\alg D(E)$ called \emph{the produced connection on $E$ induced by $\phi$ and $\nabla$}.
\end{definition}

Using the notation of the previous definition, and recalling that for a representation $\phi$ of a transitive Lie algebroid sequence $L \to A \to M$, $\phi \circ j$ is denoted by $\phi_L$, the equation $\nabla_X = \oid X - j \circ \omega(\oid X)$ relating $\nabla$ and its associated normalized connection $1$-form $\omega$ allows us to rewrite the produced connection on $E$ as
\begin{equation}\label{equationInducedProducedConnectionByNormalizedOrdinaryLieAlgebroidConnectionInTermsOfFormAssociatedRepresentationVector}
    \nabla^E_X = \phi(\oid X) - \phi_L \circ \omega(\oid X)
\end{equation}
for any $\oid X \in TP/G$ such that $\pi_*^G(\oid X) = X$. Then, 
\begin{equation}
    \phi_L \circ \omega \in \Omega^1(A, \End(E))
\end{equation} 
is a $1$-form associated to the ordinary connection $\nabla^E$; notice that the equation \eqref{equationInducedProducedConnectionByNormalizedOrdinaryLieAlgebroidConnectionInTermsOfFormAssociatedRepresentationVector} is well defined for $X$ precisely because $\omega$ is normalized, since then the right hand sides vanishes when $\oid X \in Im(j)$. We will now relax the normalization condition on transitive Lie algebroids when defining connections, in which case the ``covariant derivatives'' that will be induced on vector bundles will be taken in any direction $\oid X \in A$; these are called $A$-connections on $E$ and will be defined in chapter \ref{}.

\linea

\subsection{Generalized Connections}
%%%%%%%%%%%%%%%%%%%%%%%%%%%%%%%%%%%%%%%%%%%%%%%%%%%%%%%%%%%%%%%%%
%%%%%%%%%%%%%%%%%%%%%%%%%%%%%%%%%%%%%%%%%%%%%%%%%%%%%%%%%%%%%%%%%

\begin{definition}\label{definitionGeneralizedConnectionFOrmOnTransitiveLieAlgebroidAndTauReducedKernelEndomorphism}
Given a transitive Lie algebroid sequence $L \hookrightarrow A \twoheadrightarrow TM$ for the transitive Lie algebroid $A$, any $L$-valued $1$-form $\hat \omega \in \Omega^1(A, L)$ is called \emph{a (generalized) connection form on $A$}. Given $\hat \omega$, define \emph{the reduced kernel endomorphism $\tau \in End(L) = \Omega^1(L, L)$} ($L$ represented on itself through the bracket, i.e. the restriction to $L$ of the adjoint representation $ad:A \to \alg D(L)$) 
 by
\begin{equation}
    \tau := \omega \comp j + 1_L.
\end{equation}
\end{definition}

Notice that, although changing the chosen adjoint Lie algebroid $L$ changes the strict definition of what the connection forms on $A$ are, since every adjoint Lie algebroid is isomorphic to $ker(a)$ the distinction is not important, it is just a matter of choice of which adjoint Lie algebroid to work with. Also notice that a connection form on $A$ is ordinary, i.e. $\hat \omega$ is normalized, if and only if $\tau = 0$.

\begin{proposition}
Let $\hat \omega$ be a connection form on $A$. Then 
\begin{equation}\label{equationRelationBetweenGeneralizedConnectionFOrmAndGeneralizedNablaHatTransitive}
    \hat \nabla_{\oid X} := \oid X + \hat \omega \comp j(\oid X)
\end{equation}
is an anchor preserving vector bundle morphism. Conversely, any such map defines a generalized connection form $\hat \omega$ related to $\hat \nabla$ by the above equation. Furthermore, $\hat \nabla$ is associated to an ordinary connection on $A$ if and only if $\hat \nabla \comp j = 0$, in which case $\nabla_X := \hat \nabla_{\oid X} + j \comp \hat \omega(\oid X)$ is well defined and an ordinary connection.
\end{proposition}

\begin{proof}
The equivalence between $\hat \omega$ and $\hat \nabla$ is clear from equation \eqref{equationRelationBetweenGeneralizedConnectionFOrmAndGeneralizedNablaHatTransitive}, since $j$, being injective, has a left inverse.

If $\hat \omega$ is an ordinary connection form, then $\hat \omega \comp j = -1_L$, hence, for any $l \in L$ $\hat \nabla(j(l)) = j(l) + j \comp \hat \omega \comp j(l) = j(l) + j(-l) = 0$. Conversely, if $\hat \theta \comp j= 0$, then for any $\oid X_1, \oid X_2 \in A$ such that $a(\oid X_1) = a(\oid X_2) = X$ it is true that $\hat \nabla_{\oid X_1} = \hat \nabla_{\oid X_2}$ and so the map $\nabla_X := \hat \nabla_{\oid X_1}$ is well defined for all $X \in TM$; since $a(\nabla_X) = a(\hat \nabla_{\oid X_1}) = X$, $\nabla$ is an ordinary connection on $A$.
\end{proof}

\begin{remark}
Just as in the case of an Atiyah Lie algebroid shown in example \ref{}, an ordinary connection $\nabla$ on $A$ is equivalent to the definition of a subbundle $Im(\nabla)$ on $A$, called \emph{the horizontal subbundle}, that is isomorphic to $TM$ and that defines an isomorphism $A \cong TM \oplus L$; in that case the associated $\hat \nabla: A \to A$ is the projection onto this subbundle and $\hat \nabla^2 = \hat \nabla$. With the generalized connections on $A$ such an equivalent definition in terms of a horizontal subbundle is lost. However, a formula analogous to \eqref{equationInducedProducedConnectionByNormalizedOrdinaryLieAlgebroidConnectionInTermsOfFormAssociatedRepresentationVector} will still allow the definition of ``covariant derivatives'' on representation vector bundles $E$ as we will see in section \ref{}; furthermore, the additional directions will give rise to a coupling of the matter fields, i.e. sections of $E$, with new fields that enable the appearance of new mass terms in the Lagrangian of a gauge theory based on transitive Lie algebroids.

If we were to desire for $\hat \nabla$ to be a projection onto a subbundle, i.e. that $\hat \nabla^2 = \hat \nabla$, from equation \eqref{equationInducedProducedConnectionByNormalizedOrdinaryLieAlgebroidConnectionInTermsOfFormAssociatedRepresentationVector} it follows that a necessary and sufficient condition is 
\begin{equation}\label{equationEquivalentNecessarySufficientGeneralizedConnectionBeDefineHorizontalProjectionSubbundle}
    \omega \comp j|_{Im(\hat \omega)} = - 1_{Im(\hat \omega)}.
\end{equation}
For example, $\hat \omega = 0$ is a generalized connection form on $A$ and $\hat \nabla = 1_A$, so $\hat \nabla^2 = \hat \nabla$ trivially, and the induced ``horizontal'' subbundle of $A$ is $A$ itself.
\end{remark}

The following proposition allows us to obtain an ordinary connection from a generalized connection, if we posses a ``background'' ordinary connection.

\begin{proposition}
Let $\hat \omega$ be a connection form on $A$ with reduced kernel endomorphism $\tau$, and let $\tilde \omega$ be an ordinary connection form on $A$. Then,
\begin{equation}
    \omega = \hat \omega + \tau \circ \tilde \omega
\end{equation}
is an ordinary connection form on $A$.
\end{proposition}
\begin{proof}
We see that $\omega$ is normalized from:
\begin{align*}
    \omega \circ j &= \hat \omega \circ j + \tau \circ \tilde \omega \circ j\\
        &= (\tau - 1_L) + \tau \circ (-1_L) \\
        &= -1_L;
\end{align*}
the second line follows from the definition of $\tau$ and the normalization of $\tilde \omega$.
\end{proof}

\linea

\subsection{Curvature}
%%%%%%%%%%%%%%%%%%%%%%%%%%%%%%%%%%%%%%%%%%%%%%%%%%%%%%%%%%%%%%%%%
%%%%%%%%%%%%%%%%%%%%%%%%%%%%%%%%%%%%%%%%%%%%%%%%%%%%%%%%%%%%%%%%%
\begin{definition}
Let $\hat \omega$ be a connection form on $A$. \emph{The curvature form of $\hat \omega$} is the $L$-valued form $\hat R$
\begin{equation}
    \hat R := \hat d \hat \omega + \frac{1}{2} \hat \omega \wedge^{[,]} \hat \omega \in \Omega^2(A, L),
\end{equation}
where $\wedge^{[,]}$ is the wedge product in the Lie-graded differential Lie algebra $(\Omega^\bullet(A, L), \wedge^{[, ]}, \hat d)$ defined in \ref{definitionWedgeProductAlgebraEvaluedwithEValued}; i.e. for any $\oid X, \oid Y \in \Gamma(A)$
\begin{equation}
    \hat R(\oid X, \oid Y) = \hat d \hat \omega (\oid X, \oid Y) + [\hat \omega(\oid X), \hat \omega(\oid Y)].
\end{equation}

\emph{The algebraic curvature of the reduced kernel endormorphism} $\tau \in End(L)$ of $\hat \omega$ is
\begin{equation}
    R_\tau(\eta, \theta) := [\tau(\eta), \tau(\theta)] - \tau[\eta, \theta]. 
\end{equation}
\end{definition}

\begin{proposition}
Let $\hat \omega$ be a connection form on $A$ with reduced kernel endomorphism $\tau$. Then the curvature $2$-form of $\hat \omega$ can be expressed as
\begin{equation}
    \hat R(\oid X, \oid Y) = j^{-1}([\hat \nabla_{\oid X}, \hat \nabla_{\oid Y}] - [\hat \nabla_{\oid X}, \hat \nabla_{\oid Y}]),
\end{equation}
for all $\oid X, \oid Y \in \Gamma(A)$.
\end{proposition}

\begin{remark}
From the previous proposition we see that the connection induced by $\hat \omega$ is flat, i.e. has $0$ curvature, if and only if the endomorphism $\hat \nabla: A \to A$ is a Lie algebra bundle morphism, i.e. $\hat \nabla$ is a Lie algebroid morphism. Similarly, $\tau$ is flat, i.e. its algebraic curvature is $0$ if and only if the endomorphism $\tau: L \to L$ is a Lie algebra bundle morphism. 
\end{remark}

If $\hat \omega$ is an ordinary connection form, then $j^* \hat R = 0$ since $hat \nabla \comp j = 0$, and so it the curvature $2$-form $\hat R$ defines an element $R \in \Omega^2(TM, L)$ called \emph{the curvature of the ordinary connection on $A$}. When $\hat \omega$ is an ordinary connection on $A = TP/G$ for some $G \to P \to M$ principal bundle, then $R \in \Omega^2(TM, P \times \alg g/G)$ is the form that corresponds to the basic\footnote{} $2$-form that is usually defined to be the curvature of the connection on $P$; when $\hat \omega$ is an ordinary connection on $A = \alg D(E)$ for some vector bundle $E$ over $M$, $R \in \Omega^2(TM, End(E))$ is precisely what is called the curvature of the corresponding connection on $E$.

\begin{proposition}
Let $\hat \omega$ be a connection form on $A$. Then the curvature form $\hat R \in \Omega^2(A, L)$ satisfies the Bianchi identity:
\begin{equation}
    \hat d R + \hat \omega \wedge^{[,]} \hat R = 0.
\end{equation}
\end{proposition}
\begin{proof}
Recall from proposition \ref{} that $(\Omega^\bullet(A, L), \wedge^{[,]}, \hat d)$ is a differential graded Lie algebra, hence:
\begin{align*}
    \hat d \hat  R + \hat \omega \wedge^{[,]} \hat R 
        &= \hat d(\hat d \hat \omega + \frac{1}{2}\hat \omega \wedge^{[,]} \hat \omega) + \hat \omega \wedge^{[,]} (\hat d \hat \omega + \frac{1}{2} \hat \omega \wedge^{[,]} \hat \omega) \\
        &= (0 + \frac{1}{2} \hat d \hat \omega \wedge^{[,]} \hat \omega - \frac{1}{2} \hat \omega \wedge^{[,]} \hat d \hat \omega) + (\hat \omega \wedge^{[,]} \hat d \hat \omega + \frac{1}{2} \hat \omega \wedge^{[,]} (\hat \omega \wedge^{[,]} \hat \omega)) \\
        &= (- \hat \omega \wedge^{[,]} \hat d \hat \omega) + (\hat \omega \wedge^{[,]} \hat d \hat \omega + 0) \\
        &= 0.
\end{align*}
The second line follows from applying the graded Leibniz rule \eqref{} satisfies by $\hat d$; the third line follows from applying the anticommutativity \eqref{} in the first parenthesis, and the graded Jacobi identity \ref{} to the second parenthesis.
\end{proof}

\begin{definition}\label{definitionIngredientsCurvatureDecompositionConnection}
Let $\hat \omega$ be an connection form on $A$ with reduced kernel endomorphism $\tau \in End(L)$, let $\tilde \omega$ be an ordinary connection form, call it a \emph{background connection}, and let $\omega = \hat \omega + \tau \comp \tilde \omega$ be the induced ordinary connection form. For $\tilde \omega$ and $\omega$ let $\tilde \nabla$ and $\nabla: TM \to A$ be the associated connections and $\tilde R, R \in \Omega^2(TM, L)$ be the curvatures of $\tilde \omega$ and $\omega$.

    \begin{itemize}
        
    \item Define $\hat F = R - \tau \comp \tilde R \in \Omega^2(TM, L)$. Notice that $a^* \hat F \in \Omega^2(A, L)$.
    
    \item Let $\mathcal D \tau \in \Omega^1(TM, End(L))$ defined by
        \begin{equation}
            (\mathcal D_X \tau)(\eta) := [\nabla_X, \tau(\eta)] - \tau[\tilde \nabla_X, \eta].
        \end{equation}
        Now, define $(a^* \mathcal D \tau)\comp \omega (\oid X, \oid Y) := (\mathcal D_{a(\oid X)})(\tilde \omega \oid Y) - (\mathcal D_{a(\oid Y)})(\tilde \omega \oid X)$. This is an element of $\Omega^2(A, L)$.
        
    \end{itemize}
\end{definition}

\begin{theorem}
Using the notation of the previous definition, the curvature form of the connection form $\hat \omega$ may be decomposed as
\begin{equation}
    \hat R = a^* \hat F - (a^* \mathcal D \tau)\comp \omega + \tilde \omega ^* R_\tau.
\end{equation}
This is called the \emph{decomposition of the curvature form of $\hat \omega$ given the background connection $\tilde \omega$}.
\end{theorem}
\begin{proof}
TODO \todo{}.
\end{proof}

\section{Local Trivialization of Generalized Connection Forms}
%%%%%%%%%%%%%%%%%%%%%%%%%%%%%%%%%%%%%%%%%%%%%%%%%%%%%%%%%%%%%%%%%
%%%%%%%%%%%%%%%%%%%%%%%%%%%%%%%%%%%%%%%%%%%%%%%%%%%%%%%%%%%%%%%%%
%%%%%%%%%%%%%%%%%%%%%%%%%%%%%%%%%%%%%%%%%%%%%%%%%%%%%%%%%%%%%%%%%
%%%%%%%%%%%%%%%%%%%%%%%%%%%%%%%%%%%%%%%%%%%%%%%%%%%%%%%%%%%%%%%%%

\subsection{Connection Forms}
%%%%%%%%%%%%%%%%%%%%%%%%%%%%%%%%%%%%%%%%%%%%%%%%%%%%%%%%%%%%%%%%%
%%%%%%%%%%%%%%%%%%%%%%%%%%%%%%%%%%%%%%%%%%%%%%%%%%%%%%%%%%%%%%%%%
Throughout this section consider $A$ to be a transitive Lie algebroid over the manifold $M$, and $0 \to L \xrightarrow{j} A \xrightarrow{a} TM \to 0$ be a Lie algebroid sequence of $A$; suppose all the vector bundles on this section have vector spaces over $\bb K$ as typical fibers, where $\bb K$ is one of $\RR$ or $\CC$. Also, suppose that $\{\psi^i, \nabla^{0, i}\}_{i in I}$ is a transitive Lie algebroid atlas of $A$. Additionally, let $\hat \omega \in \Omega^1(A, L)$ be a connection for on $A$ with reduced kernel endomorphism $\tau \in \End(L)$; then, the local trivialization of $\hat \omega$ over $U_i$ is $\hat \omega_i \in \Omega^1()$. As usual with local trivializations, we prefer to adopt the module perspective.

\begin{theorem}\label{theoremLocalTrivializationDecompositionGenralizedConnectionForm}
Over any $U_i$, $i \in I$, the local trivialization of the connection form $\hat \omega$ may be decomposed as the following element of $\Omega^1(TU_i, U_i \times \alg g) \oplus_{C^\infty(U_i)} \Omega^1(U_i \times \alg g, U_i \times \alg g)$:
\begin{eqnsplit}
    \hat \omega_i =  A_i \oplus (- \epsilon + \tau_i) %\quad\in &\Omega^1(TU_i \times \alg g, U_i \times \alg g)
\end{eqnsplit}
where 
    \begin{itemize}
    
    \item $\tau_i:= (\psi^i)^{-1} \comp \tau \comp \psi^i \in C^\infty(U_i, End(\alg g)) \cong \Omega^1(U_i \times \alg g, U_i \times \alg g)$ is called \emph{the local trivialization of $\tau$} over $U_i$.
    
    \item $\epsilon = Id_{C^\infty(U_i, U_i \times \alg g)} \in \Omega^1(U_i \times \alg g, U_i \times \alg g)$.
    
    \item $A_i= \psi_i^{-1} \comp \hat \omega \comp \nabla^{0, i} \in \Omega^1(TU_i, U_i \times \alg g)$ is called \emph{the local tangent connection form of $\hat \omega$}.
    
    \end{itemize}

Furthermore, $A_i - \epsilon$ is an ordinary connection form on the trivial Lie algebroid $TU_i \times \alg g$, and $\omega_i$ is a generalized connection form on $TU_i\times \alg g$.
\end{theorem}

We may also write 
\begin{equation}
    \hat \omega_i = A_i - \epsilon + \tau_i,
\end{equation}
understanding $A$ as $A \comp pr_1$, where $pr_1: \Gamma(TU_i)\oplus C^\infty(U_i, \alg g) \to \Gamma(TU_i)$ is the projection to the first coordinate, and understanding $\epsilon, \tau_i$ as the ones from the propositions precomposed with $pr_2$.

\begin{proof}
\begin{align*}
    \hat \omega(X \oplus \eta) 
        &= \psi_i^{-1} \comp \hat \omega (\nabla^{0, i}_X + j\psi_i(\eta)) \\
        &= \psi_i^{-1} \comp \hat \omega (\nabla^{0, i}_X) + \psi_i^{-1} \comp \hat \omega \comp j (\psi_i(\eta))\\
        &= A_i(X) + \psi_i^{-1} \comp (\tau - 1_L) (\psi_i(\eta))\\
        &= A_i(X) - \eta + \psi_i^{-1} \comp \tau \comp psi_i(\eta) \\
        &= A_i(X) - \epsilon(\eta) + \tau_i(\eta).
\end{align*}

That $A_i - \epsilon$ is an ordinary connection follows simply from the fact that, for every $\eta \in C^\infty(U_i, \alg g)$, $(A_i - \epsilon)(\eta) = - \epsilon(\eta) = -\eta$. The last statement is a rewriting of $\hat \omega_i$ being a $1$-form.
\end{proof}

\begin{proposition}
Let $\hat \omega$ be a connection form on $A$ with local trivialization $\hat \omega = A_i - \epsilon + \tau_i$, decomposed as proposed in theorem \ref{theoremLocalTrivializationDecompositionGenralizedConnectionForm}, on each $U_i$, $i \in I$. Then, for all $X \in \Gamma(TM)$ and all $U_i, U_j$ such that $U_{ij} = U_i \inter U_j \neq \empty$,

    \begin{itemize}
    
    \item The local tangent connection form $A_i$ has the following transformation law:
        \begin{equation}
                A_j(X) = \alpha^i_j(\omega_i(X)) + \chi^j_i(X).
        \end{equation}
    
    \item The local trivialization $\tau_i$ of $\tau$ has the following transformation law:
        \begin{equation}
                \tau_j(\eta) = \tau_i(\eta) - \tau_j \comp \chi^j_i(X).
        \end{equation}\todo{}
    
    \end{itemize}
\end{proposition}

Notice that if $\hat \omega$ is an ordinary connection form, a family of local trivializations $\{\omega_i \in \Omega^1(TU_i \times \alg g, U_i \times \alg g)\}_{i \in I} $ of $\hat \omega$ is completely specified by its associated local tangent connection forms $\{A_i \in \Omega^1(TU_i, U_i \times \alg g)\}$, since then $\hat \omega_i = A_i - \epsilon$. Furthermore, notice that if $A$ is an Atiyah Lie algebroid associated to a principal bundle, the transformation law of the $A_i$'s is precisely that of the local connection $1$-forms of a principal connection \ref{}, hence the family $\{A_i\}_{i \in I}$ where the $A_i$'s transform as indicated in the previous proposition does specify a principal connection of $P$. In general, \textbf{a family $\{A_i, \tau_i\}_{i \in I}$ with the transformation laws indicated above define a connection form on the transitive Lie algebroid $A$}.

\begin{definition}\label{definitionLocalTrivializationOfHatNablaEndormorphismAnchorPreservingGeneralizedConnection}
Let $\hat \nabla \in End(A)$ be the anchor preserving endomorphism associated to the connection form $\hat \omega$. \emph{The local trivialization $\hat \nabla^i$ of $\hat \nabla$ over $U_i$} is the map
\begin{eqnsplit}
    \hat \nabla^i = S_i^{-1} \comp \hat \nabla \comp &= S_i \quad \in \, End(TU_i \times \alg g).
\end{eqnsplit}

If $\hat \omega$ is normalized, \emph{the local trivialization $\nabla^i$ over $U_i$} of the associated map $\nabla: TM \to A$ is
\begin{eqnsplit}
    \nabla^i = S_i^{-1} \comp \nabla  \quad : TU_i \to TU_i \times \alg g.
\end{eqnsplit}
\end{definition}

For any $X \oplus \eta \in \Gamma(TU_i)\oplus C^\infty(U_i, \alg g)$, it follows from the definitions that
\begin{eqnsplit}\label{equationLocalTrivializationOfGeneralizedConnectionHatNablaVersionEndomorphism}
    \hat \nabla^i_{X \oplus \eta} &= X\oplus \eta + \hat \omega_i(X \oplus \eta)\\
        &= X \oplus (A_i(X) - \tau(\eta)).
\end{eqnsplit}

If $\hat \omega$ were an ordinary connection form, then $\tau = 0$ and the local trivialization of $\nabla: TM \to A$ satisfies, for any $X \in \Gamma(TU_i)$:
\begin{eqnsplit}\label{equationLocalTrivializationOfOrdinaryConnectionNablaVersion}
    \nabla^i_{X} &= X \oplus A_i(X)\\
        &= \hat \nabla^i_{X \oplus \eta},
\end{eqnsplit}
for all $\eta \in C^\infty(U_i, \alg g)$.

\lin

Let $\{E_a\}_{a = 1, \dots, n}$ be a basis for $\alg g$. Let us also denote by $E_a$, $a = 1, \dots, n$, the corresponding constant functions in $C^\infty(U_i, \alg g) = \Omega^0(TU_i \times \alg g, U_i \times \alg g)$ for all $i \in I$. Recall our convention of not writing the wedge product symbol if one of the factors is a a $0$-form, like $E_a$.

\begin{definition}
    Let $a, b \in 1, \dots, n$. \emph{The components of $A_i$, $\epsilon$ and $\tau_i$ with respect to the basis $\{E_a\}_{a = 1, \dots, n}$} of $\alg g$ are, respectively: 
    
        \begin{itemize}
        
        \item $(A_i)^a \in \Omega^1(TU_i)$ be such that
            \begin{equation}
                A_i = \sum_{a = 1}^n (A_i)^a E_a,
            \end{equation}
        and if a set of coordinates $x^\mu: U_i \to \RR$ on $M$, $\mu = 1, \dots, m$, is clear from the context, we will denote by
        \begin{equation}
            (A_i)^a_\mu \equiv (A_i)^a(\partial_{x^\mu}) \in C^\infty(U_i);
        \end{equation}
        
        \item $\epsilon^a \in \Omega^1(U_i \times \alg g)$ such that
            \begin{equation}
                \epsilon = \sum_{a = 1}^n \epsilon^a E_a,
            \end{equation}
        i.e. each $\epsilon^a$ can be seen as the constant function in $C^\infty(U_i, \alg g^*)$ used in theorem \ref{TheoremDecompOfVectorValuedFormsTLAEpsilonsDxs}, taking the constant value $\epsilon^a \in \alg g^*$ that satisfies $\epsilon^a(E_b) = \delta^a_b$;
        
        \item $(\tau_i)^a_b \in C^\infty(U_i)$ such that
            \begin{equation}
                \tau_i = \sum_{a, b = 1}^n (\tau_i)^a_b E_b \epsilon^b,
            \end{equation}
        i.e. $\tau_i(E_b) = \sum_{a = 1}^n (\tau_i)^a_b E_a$.
        \end{itemize}
\end{definition}



\subsection{Mixed Local Basis}
%%%%%%%%%%%%%%%%%%%%%%%%%%%%%%%%%%%%%%%%%%%%%%%%%%%%%%%%%%%%%%%%%
%%%%%%%%%%%%%%%%%%%%%%%%%%%%%%%%%%%%%%%%%%%%%%%%%%%%%%%%%%%%%%%%%
In this section let $\omega \in \Omega(A, L)$ be an ordinary connection form with local trivializations $\omega_i = A_i - \epsilon$ over each $U_i$, and associated ordinary connection $\nabla: TM \to A$.

Let $\{E_a\}_{a=1, \dots, n}$ be a basis for $\alg g$ with dual basis $\{\epsilon^a\}_{a=1, \dots, n}$. Notice that, for any $i, j \in I$, $\hat \alpha^i_j(\epsilon^a)(X) = \epsilon^a(s^j_i(X)) = \epsilon^a(\chi^j_i(X)) \neq 0$; hence, $\hat \alpha^i_j(\epsilon^a)$ is a linear combination of forms that must include forms outside $\{\epsilon^b\}_{b = 1, \dots, n}$. In this section we develop new $1$-forms that will allow the decomposition of any form on $A$ into linear combination of products of $1$-forms, where each product of $1$-form transforms into one of the same kind under $\hat \alpha^i_j$.

\begin{definition}
    Let $a = 1, \dots, n$. The local $1$-forms, 
    \begin{equation}
        \alg a_i^a := (A_i)^a - \epsilon^a \in \Omega^1(TU_i \times \alg g)
    \end{equation}
    are called \emph{the mixed local basis of the inner part of $\Omega^1(TU_i \times \alg g, U_i \times \alg g)$ with respect to the ordinary connection form $\omega$ and the basis $\{E_a\}_{a= 1, \dots, n}$} of $\alg g$.
\end{definition}

For the rest of the section suppose that in $U_i$ the tangent bundle is trivialized with coordinates $x^\mu: U_i \to \RR$, $\mu = 1, \dots, m$, with associated local vector fields $\partial_\mu \equiv \partial_{x^\mu}$. Denote by $\nabla_{\mu} := \nabla_{\partial_{x^\mu}} \in \Gamma(TU_i \times \alg g)$. Notice that, from the trivialization of $\nabla$ equation \eqref{equationLocalTrivializationOfOrdinaryConnectionNablaVersion} it follows that
\begin{eqnsplit}
    \nabla^i_\mu = \partial_\mu \oplus A_i(\partial_\mu). 
\end{eqnsplit}

\begin{proposition}
The set 
\begin{equation}
    \{\nabla^i_\mu, -E_a\}_{\mu = 1, \dots, m; a = 1, \dots, n} \subset \Gamma(TU_i \times \alg g)
\end{equation} 
is a global frame of the transitive Lie algebroid $TU_i \times \alg g$ dual to the the frame
\begin{equation}
    \{dx^\mu, \alg a^a\}_{\mu = 1, \dots, m; a = 1, \dots, n} \subset \Gamma((TU_i\times \alg g)^*) = \Omega^1(TU_i \times \alg g).
\end{equation}
\end{proposition}

\begin{proof}
The linear independence of $\{\nabla^i_\mu, -E_a\}$ follows from that, if we try to write a $\nabla^i_\mu$ as a linear combination $\sum_{\nu \neq \mu} c^\nu \nabla^i_\nu + \sum_a d^a E_a$, applying the anchor reduces the argument to the fact that $\{\partial_\mu\}$ is linearly independent; similarly, writing $E_a$ as a linear combination $\sum_{\mu} c^\mu \nabla^i_\nu + \sum_{b \neq a} d^b E_b$ and applying $\epsilon$ reduces the argument to $\{E_b\}$ being linearly independent.

Now to see that they are indeed dual to each other. Let $\mu = 1, \dots, m$, $a = 1, \dots, n$, and recall that $A - \epsilon = \alg a^a E_a$ is an ordinary connection form, hence normalized:
\begin{align*}
    dx^\mu&(\partial_\nu \oplus( A_i(\partial_\nu))) = dx^\mu(\partial_\nu) = \delta^\mu_\nu\\
    dx^\mu&(0 \oplus (-E_a)) = 0\\
    \alg a^a&(\partial_\mu + A_i(\partial_\mu)) = A(\partial_\mu) + \alg a^a(+\oplus A_i(\partial_\mu)) \\
    &\quad = A(\partial_\mu) - A(\partial_\mu) = 0\\
    \alg a^a&(0\oplus (-E_b)) = -\epsilon^a(-E_b) = \delta^a_b.
\end{align*}

Since $\{dx^\mu, \alg a^a\}$ is dual to the global frame $\{\nabla^i_\mu, -E_a\}$, then it must be a global frame of the dual bundle.
\end{proof}

The following proposition now follows, giving an alternative decomposition of forms to the one given in theorem \ref{remarkLocalDecompositionOfFormsOnTransitiveLieAlgebroidsIfTrivializationOfM}:

\begin{proposition}\label{TheoremDecompOfVectorValuedFormsTLAMixedLocalBasisDxs}
Let $E$ be a vector bundle over $M$ that trivializes over $U_i$ as $U_i \times V$, and let $\{e_u\}_{r = 1, \dots, h}$ be a basis of the vector space $V$, and let $A$ be represented on $E$. Let $\beta$ be an $E$-valued $p$-form on $A$, $p \in \ZZ_{\geq 0}$. Viewing each $e_u$ as a constant section of $E$, i.e. $e_u \in \Omega^0(TM \times \alg g, E)$, each local trivialization of $\beta$ may be decomposed as follows
\begin{equation}
    \beta_i = \sum_{r + s = p} \left(\beta_i \right)^u_{\mu_1 \cdots \mu_r, a_1 \cdots a_s}\, e_u \, dx^{\mu_1} \wedge \cdots \wedge dx^{\mu_r} \wedge \alg a_i^{a_1} \wedge \cdots \wedge \alg a_i^{a_s}.
\end{equation} 
with each $\left(\beta_i\right)^a_{\mu_1 \cdots \mu_r, a_1 \cdots a_s} \in C^\infty(M, V)$, $a = 1, \dots, n$, are called \emph{the scalar components of $\beta_i$ with respect to the local mixed basis}.
The convention of ignoring the $\wedge$ symbol when one of the factors is a $0$-form has been used.
\end{proposition}

Let the matrix valued function $G^i_j$ on $U_{ij} \neq \empty$
    \begin{equation}
        {G_j^i}_a^b = \epsilon^b \circ \alpha^i_j(E_a)
    \end{equation}
be {the matrix representation of the transition function $\alpha^i_j$ from $U_j$ to $U_i$ of $L$ with respect to the basis $\{E_a\}_{a= 1,\ dots, n}$} of $\alg g$.

\begin{theorem}
Let $\mu_1, \dots, \mu_r \in \{1, \dots, m\}$ and $\{a_1, \dots, a_s\} \in \{1, \dots n\}$. Then the following homogeneous transformation law under change of trivializations is satisfied:
\begin{eqnsplit}
    \hat \alpha^i_j(dx^{\mu_1} \wedge \cdots \wedge dx^{\mu_r} \wedge \alg a_j^{a_1} \wedge \cdots \wedge \alg a_j^{a_s}) &=\\
    %\sum_{b_1, \dots, b_s \in \{1, \dots, n\}}
    {G_i^j}_{b_1}^{a_1} \cdots {G_i^j}_{b_1}^{a_1} &dx^{\mu_1} \wedge \cdots \wedge dx^{\mu_r} \wedge \alg a_i^{b_1} \wedge \cdots \wedge \alg a_i^{b_s};
\end{eqnsplit}
where Einstein's summation convention has been used.
In particular, let $E$ be a vector bundle over $M$ that trivializes over $U_i$ and $U_j$ as $U_i \times V$ and $U_j \times V$, $U_{ij} \neq \empty$ and with transition function from $U_j$ to $U_i$ $\alpha^i_j:U_{ij} \to End(V)$; let $\beta$ be an $E$-valued $p$-form on $A$, $p \in \ZZ_{\geq 0}$; for $E$-valued forms on $A$, its components with respect to the mixed local basis transform as follows
\begin{align}
    (\beta_i)_{\mu_1 \cdots \mu_r a_1 \dots a_s} = {G^j_i}^{b_1}_{a_1} \cdots {G^j_i}^{b_s}_{a_s} \alpha^i_j((\beta_j)_{\mu_1 \cdots \mu_r b_1 \dots b_s})
\end{align}
\end{theorem}
\begin{proof}
Since $\hat \alpha^i_j$ is a morphism of differential graded modules over the ring of scalar valued forms on $TU_{ij} \times \alg g$, it respects the wedge product and so the proof reduces to demonstrating that 
\begin{equation}
    \hat \alpha^j_i(\alg a^a_i) = {G^i_j}^{b}_{a} \alg a_j^b.
\end{equation}
Since $\alg a_i = \alg a_i^a E_a$ and $\alg a_j$ are local trivializations of the connection form $\omega$, we know that $\hat \alpha^i_j(\alg a_j) = \alg a_i$, but
\begin{align*}
    \hat \alpha^i_j(\alg a_j) &= \alpha^i_j \comp \alg a_j \comp s^i_j\\
        &= \alpha^i_j[(\alg a_j^b \comp s^j_i) E_b]\\
        &= {G^i_j}^{a}_{b} (\alg a_j^b \comp s^j_i) E_a\\
        &= \alg a_i;
\end{align*}
applying $\hat \alpha^j_i = (\hat \alpha^i_j)^{-1}$ on both sides gives the desired equation.
\end{proof}
\begin{remark}
Notice that a single coordinate map of $U_{ij} \subset M$ is used in the theorem, explaining wh
\end{remark}

\subsection{Curvature Forms}
%%%%%%%%%%%%%%%%%%%%%%%%%%%%%%%%%%%%%%%%%%%%%%%%%%%%%%%%%%%%%%%%%
%%%%%%%%%%%%%%%%%%%%%%%%%%%%%%%%%%%%%%%%%%%%%%%%%%%%%%%%%%%%%%%%%

Let $\hat R \in \Omega^2(A, L)$ be the curvature form of the connection form $\hat \omega$, and recall the maps defined in \ref{definitionIngredientsCurvatureDecompositionConnection} for the background connection form $\tilde \omega$ and the induced ordinary connection form $\omega$. Let $\{E_a\}_{a = 1, \dots, n}$ be a basis of $\alg g$.

\begin{definition}

    \begin{itemize}
    
    \item Over $U_i$, \emph{the local trivialization of the algebraic curvature of $\tau$}, the reduced kernel endomorphism of $\hat \omega$, is $C^\infty(U_i)$-multilinear antisymmetric map
    \begin{equation}
        (R_\tau)_i = \psi_i^{-1} \comp \psi_i^*R_\tau : C^\infty(U_i, \alg g) \times C^\infty(U_i, \alg g) \to C^\infty(U_i, \alg g).
    \end{equation}
    Denote by $(W_i)_{ab}^c \in C^\infty(U_i)$ the functions that satisfy
    \begin{equation}
        (R_\tau)_i(E_a, E_b) = (W_i)_{ab}^c E_c;
    \end{equation}
    call them \emph{the components of $R_\tau$ over $U_i$ with respect to the basis $\{E_a\}_{a = 1, \dots n}$}.
    
    \item Over $U_i$, \emph{the local trivialization of $\mathcal D\tau$} is the element 
    \begin{eqnsplit}
        (\mathcal D \tau)_i : \Gamma(TU_i) &\to C^\infty(U_i, End(\alg g^*))\\
        X &\mapsto \mathcal \psi_i^{-1} \comp \,\psi_i^*(\mathcal D_{\nabla^{0, i}_X} \tau)
    \end{eqnsplit}
    induced by the local trivialization of $A$ over $U_i$. 
    Let $x^\mu:U_i \to \RR$, $\mu = 1, \dots, m$ be coordinates on $U_i$ and denote by $(\mathcal D \tau_i)_{\mu, a}^b \in C^\infty(U_i)$ the functions that satisfy
    \begin{equation}
        (\mathcal D_{\partial_\mu} \tau )_i(E_a) = (\mathcal D \tau_i)_{\mu, a}^b E_b.
    \end{equation}
    \end{itemize}
\end{definition}

\begin{theorem}
Let $U_i \subset M$ be an open on which $x^\mu: U_i \to \RR$ are coordinates and the transitive Lie algebroid $A$ trivializes. Let $\tilde \omega\in \Omega^1(A, L)$ be an ordinary connection form on $A$ with local trivialization $\alg a_i$ over $U_i$, and call it a background connection on $A$. Finally, for a (generalized) connection form $\hat \omega$ on $A$ with reduced kernel endomorphism $\tau$, let $\omega = \hat \omega + \tau \comp \tilde \omega$ be the ordinary connection induced by $\hat \omega$ with respect to the background connection $\tilde \omega$, let $\hat R \in \Omega^2(A, L)$ be its curvature $2$-form, and $\hat F \in \omega^2(TM, L)$, $\mathcal D \tau \in \Omega^1(TM, End(L))$ and $R_\tau \in \Omega^2(L, L)$ be defined as in definition \ref{definitionIngredientsCurvatureDecompositionConnection}. Then, the local trivialization of $\hat R$ has the following decomposition in terms of the mixed local basis of the inner part of $\Omega^1(TU_i \times \alg g)$ with respect to the basis $\{E_a\}_{a = 1, \dots, n}$ of $\alg g$:
\begin{equation}
    \hat R_i = (\hat F_i)_{\mu \nu}^a E_a dx^\mu \wedge dx^\nu + 
                (\mathcal D \tau_i)^b_{\mu, a} E_b  dx^\mu \wedge (\alg a_i)^a +
                (W_i)^c_{ab} E_c (\alg a_i)^a \wedge (\alg a_i)^b
\end{equation}
where $a, b, c \in \{1, \dots a\}$, $\mu, \nu \in \{1, \dots, m\}$ and where Einstein's summation convention for the sum over repeated super- and sub-indices is used.
\end{theorem}
\begin{proof}
What we want to see is that the local trivialization over $U_i$ of each of the terms on the decomposition \ref{theoremDecompositionCurvatureGeneralizedConnection3Terms} coincides with each of the stated terms.

That $a^*F_i = \hat F^a_{\mu \nu} E_a dx^\mu \wedge dx^\nu$ is clear since $\hat F$ is a $2$-form on $TM$ and $a$ is the anchor.

To see that $(a^* \mathcal D \comp \tilde \omega)_i = (\mathcal D \tau_i)^b_{\mu, a} dx^\mu \wedge (\alg a_i)^a$, first recall that $\{\nabla^i_\mu, -E_a\} \subset \Gamma(TU_i \times \alg)$, where $\nabla^i$ is the local trivialization of the connection $\nabla:TM \to A$ associated to $\tilde \omega$, is the global frame of $TU_i \times \alg g$ dual to $\{dx^\mu, \alg a^a\}$. Then, the result follows from the following three calculations:
\begin{align*}
    (a^* \mathcal D \comp \tilde \omega)_i(\nabla^i_\mu, \nabla^i_\nu) 
        &= (\mathcal D_{\partial_\mu} \tau_i) (\tilde \omega_i(\nabla^i_\mu)) - (\mathcal D_{\partial_\nu} \tau_i) (\tilde \omega_i(\nabla^i_\nu))\\
        &= (\mathcal D_{\partial_\mu} \tau_i) (0) - (\mathcal D_{\partial_\nu} \tau_i)(0);
\end{align*}
\begin{align*}
    (a^* \mathcal D \comp \tilde \omega)_i(\nabla^i_\mu, -E_a) 
        &= (\mathcal D_{\partial_\mu} \tau_i) (\tilde \omega_i(-E_a)) - (\mathcal D_{0} \tau_i) (\tilde \omega_i(\nabla^i_\mu))\\
        &= (\mathcal D \tau_i)^b_{\mu, a} E_b;
\end{align*}
\begin{align*}
    (a^* \mathcal D \comp \tilde \omega)_i(-E_a, -E_b) 
        &= (\mathcal D_{0} \tau_i) (\tilde \omega_i(-E_b)) - (\mathcal D_{0} \tau_i) (\tilde \omega_i(-E_a))\\
        &= 0.
\end{align*}

Finally, that $(\tilde \omega^* R_\tau)_i = (W_i)^c_{ab} E_c (\alg a_i)^a \wedge (\alg a_i)^b$ follows from:
\begin{align*}
    (\tilde \omega^* R_\tau)_i(\nabla^i_\mu, \nabla^i_\nu) 
        &= (R_\tau)_i(\tilde \omega_i(\nabla^i_\mu), \tilde \omega_i(\nabla^i_\nu)) \\
        &= (R_\tau)_i(0, 0) \\
        &= 0;
\end{align*}
\begin{align*}
    (\tilde \omega^* R_\tau)_i(\nabla^i_\mu, -E_a) 
        &= (R_\tau)_i(\tilde \omega_i(\nabla^i_\mu), \tilde \omega_i(-E_a)) \\
        &= (R_\tau)_i(0, E_a) \\
        &= 0;
\end{align*}
\begin{align*}
    (\tilde \omega^* R_\tau)_i(-E_a, -E_b) 
        &= (R_\tau)_i(\tilde \omega_i(-E_a), \tilde \omega_i(-E_b)) \\
        &= (R_\tau)_i(E_a, E_b) \\
        &= (W_i)^c_{ab} E_c.
\end{align*}
\end{proof}


Denoting by $C^c_{ab} = \epsilon^c([E_a, E_b])$ the structure constants of $\alg g$ in the basis $\{E_a\}_{a = 1, \dots, n}$, it follows directly from the definition of $R_\tau$, in \ref{definitionIngredientsCurvatureDecompositionConnection}, that
\begin{equation}
    W^c_{ab} = \tau^b_a \tau^e_b C^c_{de} - C^d_{ab}\tau^c_d.
\end{equation}
Similarly, from the definition of $\mathcal D \tau$ if follows that
\begin{eqnsplit}
    (\mathcal D \tau_i)^b_{\mu, a} &= \epsilon^b ((\mathcal D_{\partial_\mu} \tau)(E_a))\\
        &= \epsilon^b (  [\partial_\mu + A_i(\partial_\mu), \tau_i(E_a)] - \tau_i[\partial_\mu + \tilde A_i(\partial_\mu), E_a]  )\\
        &= \epsilon^b (  [\partial_\mu + (A_i)^c_\mu E_c, \tau^d_a E_d] - \tau_i[\partial_\mu + (\tilde A_i)_\mu^d E_d, E_a ]  ) \\
        &= \epsilon^b (  \partial_\mu(\tau^b_a E_b) + (A_i)^c_\mu   \tau^d_a [E_c, E_d] - (\tilde A_i)_\mu^d \tau_i[E_d, E_a])\\
        &= \partial_\mu \tau^b_a + (A_i)^c_\mu (\tau_i)^d_c C^b_{cd} - (\tilde A_i)_\mu^d C^c_{da} \tau^b_c.
\end{eqnsplit}

\section{$A$-connections}
%%%%%%%%%%%%%%%%%%%%%%%%%%%%%%%%%%%%%%%%%%%%%%%%%%%%%%%%%%%%%%%%%
%%%%%%%%%%%%%%%%%%%%%%%%%%%%%%%%%%%%%%%%%%%%%%%%%%%%%%%%%%%%%%%%%
%%%%%%%%%%%%%%%%%%%%%%%%%%%%%%%%%%%%%%%%%%%%%%%%%%%%%%%%%%%%%%%%%
%%%%%%%%%%%%%%%%%%%%%%%%%%%%%%%%%%%%%%%%%%%%%%%%%%%%%%%%%%%%%%%%%

\todo{Do ``trivial'' example of representation of trivial Lie algebroid on trivial vector bundle on which $G$ acts: to this case reduces locally the representation of $TP/G$ on an associated vector bundle!}

In this section, let $A$ be any Lie algebroid over a manifold $M$, not necessarily transitive. $A$-connections will be yet another generalizations of connections, in this case of vector bundle connections or covariant derivatives, on which the directions in which we take the derivatives are not only the tangent directions of $TM$, but the ``generalized directions'' of $A$. Additionally, let $E$ be a vector bundle over $M$.

\begin{definition}
\emph{An $A$-connection on $E$} is an anchor preserving vector bundle map
\begin{equation}
    \hat \nabla^E: A \to \alg D(E).
\end{equation}
\emph{The curvature of $\hat \nabla^E$} is the $\End(E)$-valued $2$-form:
\begin{equation}
    \hat R^E(\oid X, \oid Y) := j^{-1}([\hat \nabla^E_{\oid X}, \hat \nabla^E_{\oid Y}] - \hat \nabla^E_{[\oid X, \oid Y]}).
\end{equation}
\end{definition}

\begin{example}
On any vector bundle $E$ over $M$, an $TM$-connection is simply an ordinary connection on $E$, which also coincides with a connection on the vector bundle $E$ as seen in example \ref{}.
\end{example}

\begin{example}
Let $A$ be a transitive Lie algebroid, and suppose that $\hat \nabla^E$ is an $A$-connection on $E$. If $\hat \nabla:A \to A$ is the vector bundle endomorphism associated to a connection $\hat \omega$ on $A$, then, the composition:
\begin{equation}
    \hat \nabla^{E, \hat \omega} : A \xrightarrow{\hat \nabla} A \xrightarrow{\hat \nabla^E} \alg D(E)
\end{equation}
is an anchor preserving vector bundle morphism, i.e. it is an $A$-connection on $E$.
\end{example}

An important subfamily of examples that will be of importance for us is the following. From now on we will assume that $0 \to L \xrightarrow{j} A \xrightarrow{a} TM \to 0$ is a transitive Lie algebroid sequence.

\begin{definition}
Let $\hat \omega$ be a connection $1$-form on $A$ with associated vector bundle endomorphism $\hat \nabla:A \to A$. Let $\phi: A \to \alg D(E)$ be a representation of $A$ on $E$. Then, \emph{the $A$-connection on $E$ produced by the connection $\hat \omega$} is the composition
\begin{equation}
    \hat \nabla^{E, \hat \omega} : A \xrightarrow{\hat \nabla} A \xrightarrow{\phi} \alg D(E).
\end{equation}
\end{definition}

\begin{theorem}
Let $\phi:A \to \alg D(E)$ be a representation. Then there is a bijective correspondence between $A$-connections $\hat \nabla^E$ on $E$ and $1$-forms $\hat \omega^E \in \Omega^1(A, \End(E))$ given by the relation:
\begin{equation}
    \hat \nabla^E_{\oid X} = \phi(\oid X) + j \comp \hat \omega^E(\oid X).
\end{equation}
Furthermore, the curvature of $\hat \nabla^E$ has the formula:\todo{have to define $\hat d_E$ from induced representation of $A$ on $End(E)$}
\begin{equation}
    \hat R^E = \hat d_E \hat \omega^E + \frac{1}{2} \hat \omega^E \wedge^{[,]} \hat \omega^E,
\end{equation}
and it satisfies the Bianchi identity:
\begin{equation}
    \hat d_E \hat R^E + \hat \omega^E \wedge^{[,]} \hat R^E = 0.
\end{equation}
\end{theorem}
\begin{proof}
Given an $A$-connection $\hat \nabla^E$, $a(\hat \nabla^E_{\oid X} - \phi(\oid X)) = a(\oid X) - a(\oid X) = 0$ for all $\oid X \in A$, so there is an element $\hat \omega^E(\oid X) \in End(E)$ (recall that $End(E) \xrightarrow[hook]{j} \alg D(E) \twoheadrightarrow TM$ is a Lie algebroid sequence for the transitive Lie algebroid $\alg D(E)$). That $\hat \omega^E: A \to End(E)$ is a vector bundle morphism, and hence a form, follows from the linearity of $\hat \nabla^E - \phi$. 

Conversely, given a $1$-form $\hat \omega^E(A, \End(E))$, for all $\oid X \in A$, $\phi(\oid X)$ and $\j \comp \hat \omega^E(\oid X)$ are elements of $\alg D(E)$, and so is their sum, so $\hat \nabla^E := \phi + j \comp \hat \omega^E: A \to \alg D(E)$ is a vector bundle morphism; Since $a(j \comp \hat \omega^E(\oid X)) = 0$, $\hat \nabla^E$ is anchor preserving.

Now, let $\oid X, \oid Y \in A$ be arbitrary:
\begin{align*}
    j \comp \hat R^E(\oid X, \oid Y) 
        &= [\hat \nabla^E_{\oid X}, \hat \nabla^E_{\oid Y}] - \hat \nabla^E_{[\oid x, \oid Y]}\\
        &= [\oid X + j \comp \hat \omega^E(\oid X), \oid Y + j \comp \hat \omega^E(\oid Y)] - ([\oid X, \oid Y] + j \comp \hat \omega^E[\oid X, \oid Y])\\
        &= [j \comp \hat \omega^E(\oid X), \oid Y] + [\oid X, j \comp \hat \omega^E(Y)] + [j \comp \hat \omega^E(\oid X), j \comp \hat \omega^E(\oid Y)] - j \comp \hat \omega^E[\oid X, \oid Y]\\
        &= [\oid X, j \comp \hat \omega^E(\oid Y)] - [\oid Y, j \comp \hat \omega^E(\oid X)] - j \comp \hat \omega^E[\oid X, \oid Y] + j[\hat \omega(\oid X), \hat \omega (\oid Y)]\\
        &= j(\hat d_E \omega^E(\oid X, \oid Y) + [\hat \omega^E(\oid X), \hat \omega^E(\oid Y)])\\
        &= j \comp (\hat d_E \hat \omega^E + \frac{1}{2} \hat \omega^E \wedge^{[,]} \hat \omega^E).
\end{align*}

That the curvature $2$-form satisfies the Bianchi identity follows from the previous formula for $\hat R^E$ and from the fact that $(\Omega^\bullet(A, \End(E)), \wedge^{[,]}, \hat d_E)$ is a differential graded Lie algebra\todo{prove this}, so a calculation identical to that followed for the curvature $2$-form of a connection on a transitive Lie algebroid, in the proof of \ref{}, can be applied.
\end{proof}

For an $A$-connection $\hat \nabla^{E, \, \hat \omega}$ produced by the connection $1$-form $\hat \omega \in \Omega^1(A, L)$, 
\begin{align*}
    \hat \nabla^{E, \hat \omega}_{\oid X} &= \phi(\hat \nabla_{\oid X})\\
        &= \phi(\oid X + j \comp \hat \omega(\oid X)) \\
        &= \phi(\oid X) + j \comp \phi_L \comp \hat \omega (\oid X)
\end{align*}
for all $\oid X \in A$. This means that the $End(E)$-valued $1$-form associated to the $A$-connection produced by $\hat \omega$ is $\phi_L \comp \hat \omega$.

\begin{definition}\label{definitionLocalTrivializationOfACOnnectionTransitiveLieAlgebroidComponents}
Let $\hat \nabla^E$ be an $A$-connection on $E$ ($A$ with Lie algebroid atlas), and let $E$ have local trivialization $\beta_i: U_i \times V\to E|_{U_i}$ over $U_i$. \emph{The local trivialization over $U_i$ of the $A$-connection $\hat \nabla^E$} is the $TU_i \times \alg g$-connection on $U_i \times V$ defined by 
\begin{equation}
    \hat \nabla^{E, i}_{X \oplus \eta} f := \beta_i^{-1}(\hat \nabla_{S_i(X \oplus \eta)}^E \beta_i(f)).
\end{equation}
% If $x^\mu: U_i \to \RR$ are coordinates on $U_i$, $\mu = 1, \dots, m$, \emph{the components of $\hat \nabla^{E}$} over $U_i$ with respect to the coordiates $x^\mu$ and basis $\{E_a\}_{a = 1, \dots, n}$ are the functions
% \begin{eqnsplit}
%     (\hat \nabla^{E, i})^b_{\mu, a} := \epsilon^b(\hat \nabla^{E, i}_{\partial_\mu \oplus E_a}) \in C^\infty()
% \end{eqnsplit}
\end{definition}



\begin{example}[Lie algebroid morphisms between Trivial Lie algebroid over $M$]
$\omega \in \Omega^1(TM, M \times \alg g_2)$ Maurer Cartan form (w.r.t. Lie composition), $\phi_+: M \times \alg g_1 \to M \times \alg g_2$ LAB morphism, Comp. Cond..
\end{example}


\begin{example}[The derivations Lie algebroid of a trivial vector bundle]
$TU_i \oplus (U_i \times \alg{gl}(V))$
\end{example}


\begin{example}[All representations of a TLA on a T.v.b.]
$\alpha_i: \Gamma(TU_i) \to C^\infty(U_i, \alg{gl}(V)$ Maurer Cartan, $\phi_{L, i} \in C^\infty(U_i, \Hom(\alg g. \alg{gl}(V)))$ LAB morphism.
\end{example}


\begin{example}[Group induced representation]\todo{Change of place}
Particular case: $\phi_L$ constant $\pi: \alg g \to End(V)$, and with $\alpha_i = 0$.

Let $G$ be a group with $\alg g$ as its Lie algebra, and let it act on the vector space $V$. Then, the map $\phi: TM \times \alg g \to \alg D(M \times V)$ such that, for all $X \oplus \eta \in TM \times \alg g$ and $f \in C^\infty(M, V)$:
\begin{eqnsplit}
    \phi(X \oplus \eta)(f) := X(f) + \eta \cdot f
\end{eqnsplit}
where $\eta \cdot f(m) := \der{t}[t=0]{\exp[t \eta] \cdot f(m)}$ is the action of $\eta \in \alg g$ on $V$ by the group action. This is called \emph{the $G$-induced representation of the trivial Lie algebroid on a trivial vector bundle}.
\end{example}


\begin{example}[Local trivialization of representation of $TP/G$ on $E = P \times V/G$]\label{localTrivializationOfRepresentationOfAtiyahTPGonAssociated}\todo{Change of place}
$U_i$ trivializes $E$ as $\beta_i: U_i \times V\to E|_{U_i}$

$\eta \in C^\infty(U_i, \alg g)$, $f \in C^\infty(U_i, V)$

$(\upsect{S_i(X \oplus \eta)})_{\sigma_i(m)} = \sigma_{i, *}(X) + \der{t}[t=0]{\sigma_i(m) \cdot \exp[-t \eta(m)]} \in T_{\sigma_i(m)} P$


$\upsect{\beta_i(f)}: \sigma_i(m)g \mapsto g^{-1} f(m)) \in C^\infty_G(P, V)$.


\begin{align*}
    \beta_i^{-1}[\phi\comp S_i(X \oplus \eta)(\beta_i(f))] (m)
        &= \upsect{\phi \comp S_i(X \oplus \eta)}_{\sigma_i(m)} (\upsect{\beta_i(f)})\\
        &= \upsect{\phi\cl{\sigma_{i, *}(X)}}_{\sigma_i(m)} (\upsect{\beta_i(f)}) + \upsect{\phi( j\psi_i(\eta))}_{\sigma_i(m)} (\upsect{\beta_i(f)})\\
        &= \sigma_{i, *}(X_m)(\upsect{\beta_i(f)}) + \der{t}[t=0]{\exp[t \eta(m)] f(m)}\\
        &= \der{t}[t = 0]{\sigma_i(\gamma(t))} (\upsect{\beta_i(f)}) + (\eta \cdot f)(m)\\
        &= \der{t}[t=0]{f(\gamma(t))} + (\eta \cdot f)(m)\\
        &= X_m(f) + (\eta \cdot f)(m),
\end{align*}
where $\gamma: I \to U_i$ is a path on $M$ such that $\gamma'(0) = X_m$. In conclusion, the local trivialization $\phi_i: TU_i \times \alg g \to \alg D(U_i \times V)$ of the representation $\phi: TP/G \to P \times V/G$ is
\begin{equation}\label{equationLocalRepresentationAtiyahLieAlgebroidOnAssociatedVectorBundleExpectedTrivialALgebroidRepresentationActionOfGroup}
    \phi_i(X \oplus \eta)(f) = X(f) + \eta \cdot f.
\end{equation}
\end{example}


\begin{example}[Family of Local trivializations of all representations Transitive Lie algebroids]
i.e. Find transformation law of $\alpha_i$ and $\phi_{L, i}$.
\end{example}


\begin{example}[All TLA-connections on T.v.b.]
$B: TU_i \to C^\infty(U_i, \alg{gl}(V))$ $C^\infty(U_i)$-linear, and $\nu \in C^\infty(U_i, \Hom(\alg g. \alg{gl}(V)))$.
\end{example}


\begin{example}[Local trivialization of arbitrary $A$-connection produced by connection on $A$]
$B_i = \alpha_i + \phi_L \comp A_i$, $\nu_i = - \phi_{L, i} \comp \tau_i$

\begin{equation} \label{equationLocalTrivializationOfGeneralProducedAConnectionArbitraryGeneralRepresentation}
    \hat \nabla^{E, i}_{X \oplus \eta} = X \oplus [\alpha_i(X) + \phi_{L, i} \comp A_i(X) - \phi_{L, i}\comp \tau_i(\eta)].
\end{equation}

Given basis $\{E_a\}_a$ with ${a, b = 1, \dots, n}$ of $\alg g$ dual to $\{\epsilon^a\}_a$, $\{e_u\}_{u}$ with $u, v = 1, \dots h$ of $V$ dual to $\{\tilde e^u\}$, and coordinates $\{x^\mu: U_i \to \RR\}_{\mu}$ with $\mu, \nu  = 1, \dots m$ of $M$:
\begin{align}
    \hat \nabla^{E, i}_\mu &:= \hat \nabla^{E_i}_{\partial_\mu \oplus 0} = \partial_\mu \oplus (\alpha_i)_{\mu, v}^u e_u \tilde e^v + (\phi_{L, i})^u_{b, v} (A_i)_{\mu, a}^b e_u \tilde e^v &
    \hat \nabla^{E, i}_a &:= \hat \nabla^{E_i}_{0 \oplus E_a} \ 
    cdots = - (\phi_{L, i})
\end{align}

\end{example}


\begin{example}[Trivialization of $A$-connection produced by connection on $A$ and gropu induced representation]
Let $U_i$ be a neighborhood on which the representation $\phi:A \to \alg D(E)$ trivializes as the $G$-induced representation of $TU_i\times \alg g$ on $U_i \times V$, where $G$ is a Lie group with $\alg g$ as Lie algebra, and $V$ is the typical fiber of $E$ over $U_i$.

Let $\hat \omega$ be a connection form on the transitive Lie algebroid $A$. The $A$-connection $\hat \nabla^E$ on $E$ produced by $\hat \omega$ trivializes \todo{define the local trivialization of a representation and of an $A$-connection} over $U_i$ as:
\begin{align*}
    \hat \nabla^{E, i}_{X \oplus \eta} f
        &= \phi_i(\hat \nabla^i_{X \oplus \eta}) f\\
        &= \phi_i( X \oplus (A_i(X) - \tau(\eta))) f\\
        &= X(f) + (A_i(X) - \tau(\eta))\cdot f.
\end{align*}
\end{example}




\section{Examples}

\subsection{$P^k$ over $S^2$}

The most general connection form on $TP^k/S^1$ has the local trivialization over $U_S$:
\begin{equation}
    \hat \omega_S = i\omega_{S; 1,}(\phi, \theta) dx^1 + i\omega_{S; 2,}(\phi, \theta) dx^2 + i\omega_{S; \, ,1}(\phi, \theta) Im
\end{equation}
where $ \in C^\infty(S^2)$; recall from \ref{} that this suffices to determine completely the form over all of $S^2$ since this $U_S$ covers all but one point of $S^2$. Its local trivialization over $U_N$ is
\begin{equation}
     \hat \omega_N = - kd\theta
\end{equation}

The local trivialization of the reduced kernel endomorphism of $\hat \omega$ is:
\begin{equation}
    \tau_S = h - 1.
\end{equation}

As already mentioned in \ref{}, the local trivialization of the curvature form of $\hat \omega$ is
\begin{equation}
    \hat R_S = \hat d \omega_S 
\end{equation}


\subsection{$P^k$ over $S^4$}

Thanks to all what we have proved, we know that the most general connection $\hat \omega \in \Omega^2(TP^k/S^3, P^k \times Im \HH/S^3)$ on $TP^k/S^4$ has the local trivialization over $U_S$
\begin{equation}
    \hat \omega_S = i \omega_S^1 + j \omega_S^2 + k \omega_S^3 - Im - Jm - Km + \tau_S,
\end{equation}
where $\omega_S^i = (\omega_S)^i_\mu dx^\mu$ are restrictions to $U_S$ of $1$-forms in $\Omega^(TS^7)$, hence the components of $\hat \omega$ over $U_S$ with respect to the basis $i, j, k$ of $Im \HH$ are actually functions $(\omega_S)^i_\mu \in C^\infty(S^2)$ defined over all of $S^2$; similarly, $\tau_S$ is any function $C^\infty(S^2, End(Im \HH))$.

If we use the matrix algebra $\alg{su}(2)$ as the representation of the Lie algebra,\todo{check isomorphism between $Im \HH$ and $su(2)$} then:
\begin{eqnsplit}
    \hat \omega_S 
    &= \begin{pmatrix} i\omega_S^3 & \omega_S^2 + i \omega_S^1 \\ \omega_S^2 - i \omega_S^1 & -i\omega_S^3  \end{pmatrix} - Im - Jm - Km
    + %\begin{pmatrix} i\tau^3 & \tau_S^2 + i \tau_S^1 \\ \tau_S^2 - i \tau_S^1 & -i\tau_S^3 \end{pmatrix} \\
    %&= \omega_S^i \cdot i\sigma_i - \epsilon + \tau_S^i \cdot i\sigma_i,
\end{eqnsplit}\todo{forma de expresar cualquier endomorfismo de $su(2)$ con matrices $2\times 2$}
$\hat \omega_S \in \Omega^1(TU_S \times Im \HH, U_S \times Im \HH)$

